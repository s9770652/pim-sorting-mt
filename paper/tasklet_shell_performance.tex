\subsection*{Evaluation of the Performance}
\label{sec:tasklet:shell:performance}
\addcontentsline{toc}{subsection}{\nameref{sec:tasklet:shell:performance}}

\pgfplotstablereadnamed{data/shell/two-tier/uint32/reverse.txt}{tableShellTwo_32reverse}
\pgfplotstablereadnamed{data/shell/two-tier/uint32/uniform.txt}{tableShellTwo_32uniform}
\pgfplotsinvokeforeach{7,...,17}{
	\pgfplotstablereadnamed{data/shell/h1=#1/uint32/reverse.txt}{tableShell#1_32reverse}
	\pgfplotstablereadnamed{data/shell/h1=#1/uint32/uniform.txt}{tableShell#1_32uniform}
}

Let us first focus on short inputs where only two passes with step sizes~\(\stepsizes = \paren{1, \stepsizes_1}\) suffice.
The previous results on \IS{} suggest that a fast \ShS{} should make use of~\(\stepsizes_1\) sentinel values.
\Cref{fig:shell:two-tier,fig:shell:two-tier_uint32,fig:shell:two-tier_uint64} show that, with the exception of the \ShS{} with step size \(\stepsizes_1 = 2\), the additional pass start to pay off at around 16 elements for both 32-bit and 64-bit values with the fully random input distributions, reaching a speed-up of around 115\%--120\% at 24 elements.
In case of the reverse sorted input, the speed-up is practically always positive even for very short inputs, reaching around 125\%--210\% at 24 elements.
On sorted and almost sorted inputs, \ShS{} exhibits a slow-down in the tested range of input lengths.

\begin{figure}
	\tikzsetnextfilename{shell_two-tier}
	\begin{tikzpicture}[plot]
		\begin{groupplot}[
			horizontal sep for labels,
			adaptive group=1 by 2,
			groupplot xlabel={Input Length \(n\)},
			xtick distance=3,
			minor xtick=data,
			title={},
			legend columns=5,
		]
			\nextgroupplot[ylabel=Cycles / \(n^2\), ymin=0, ymax=50, ytick distance=10]
			\pgfplotsset{legend to name=leg:shell:two-tier, legend entries={\(1\), \(...\), \(9\)}}
			\pgfplotsinvokeforeach{1,...,9}{
				\plotpernn[x filter/.expression={x > #1 ? x : nan}]{#1}{tableSmallSorts_32uniform}
			}
			%
			\nextgroupplot[ylabel=Speed-up, ymin=0.7, ymax=1.2, ytick distance=0.1, yticklabel style={/pgf/number format/.cd, precision=1, fixed, zerofill}]
			\pgfplotsset{cycle list shift=1}
			\pgfplotsinvokeforeach{2,...,9}{
				\plotspeedup[x filter/.expression={x > #1 ? x : nan}]{#1}{1}{tableSmallSorts_32uniform}
			}
		\end{groupplot}
	\end{tikzpicture}

	\hfil\pgfplotslegendfromname{leg:shell:two-tier}\hfil
	\caption{
		Comparison of \IS{} (1) and various two-tier \ShS*{} (2--9), whose step sizes \(\stepsizes_1\) are indicated by their labels.
		The speed-ups are with respect to the \IS{}.
	}
	\label{fig:shell:two-tier}
\end{figure}

\NewDocumentCommand{\shellscatter}{m m m}{
	\addplotnamedtable[select row={#1}, forget plot][x=µ_#2, y expr={6}]{tableShellTwo_#3};
	\ifnumless{#2}{7}{
		\addplotnamedtable[select row={#1}, forget plot][x=µ_#2, y expr={7}]{tableShell7_#3};
	}{}
	\ifnumless{#2}{8}{
		\addplotnamedtable[select row={#1}, forget plot][x=µ_#2, y expr={8}]{tableShell8_#3};
	}{}
	\ifnumless{#2}{9}{
		\addplotnamedtable[select row={#1}, forget plot][x=µ_#2, y expr={9}]{tableShell9_#3};
	}{}
	\addplotnamedtable[select row={#1}][x=µ_#2, y expr={10}, forget plot]{tableShell10_#3};
	\addplotnamedtable[select row={#1}][x=µ_#2, y expr={11}, forget plot]{tableShell11_#3};
	\addplotnamedtable[select row={#1}][x=µ_#2, y expr={12}, forget plot]{tableShell12_#3};
	\addplotnamedtable[select row={#1}][x=µ_#2, y expr={13}, forget plot]{tableShell13_#3};
	\addplotnamedtable[select row={#1}][x=µ_#2, y expr={14}, forget plot]{tableShell14_#3};
	\addplotnamedtable[select row={#1}][x=µ_#2, y expr={15}, forget plot]{tableShell15_#3};
	\ifnumgreater{#1}{16}{
		\addplotnamedtable[select row={#1}][x=µ_#2, y expr={16}, forget plot]{tableShell16_#3};
		\addplotnamedtable[select row={#1}][x=µ_#2, y expr={17}]{tableShell17_#3};
	}{
		\addplotnamedtable[select row={#1}, opacity=0][x=µ_#2, y expr={17}]{tableShell15_#3};
	}
}

\pgfplotsset{
	shell scatter plot/.style={
		adaptive group=3 by 2,
		groupplot xlabel={Runtime [\(10^4\) Cycles]},
		groupplot ylabel={\(\stepsizes_2\)},
		scaled x ticks=base 10:-4,
		xtick scale label code/.code={},  % removes exponent underneath the axis
		ytick={6, 7, 9, ..., 15},
		yticklabels={/, \(7\), \(9\), \(...\), \(17\)},
		minor ytick={8, 10, ..., 13},
		/tikz/only marks,
		cycle list shift=2,  % for sharing colours with the previous figure
		xmajorgrids=false,
		title/.add={}{Input Length \textit{n} =},
		legend columns=-1,
	},
}

\begin{figure}[p]
	\tikzsetnextfilename{shell_three-tier}
	\begin{tikzpicture}[plot]
		\newcommand{\type}{32uniform}
		\begin{groupplot}[shell scatter plot]
			\pgfplotsinvokeforeach{16,32,48,64,96,128}{
				\nextgroupplot[title/.add={}{ #1}]
				\pgfplotsforeachungrouped\h in {3,...,9}{
					\shellscatter{#1}{\h}{\type}
				}
				\ifnumgreater{#1}{17}{ \pgfplotsset{extra y ticks={17}, minor ytick={8, 10, ..., 16}} }{}
			}
			\pgfplotsset{legend to name=leg:shell:three-tier, legend entries={\(\stepsizes_1 = 3\), \(\stepsizes_1 = 4\), \(\stepsizes_1 = 5\), \(\stepsizes_1 = 6\), \(\stepsizes_1 = 7\), \(\stepsizes_1 = 8\), \(\stepsizes_1 = 9\)}}
		\end{groupplot}
	\end{tikzpicture}

	\hfil\pgfplotslegendfromname{leg:shell:three-tier}\hfil
	\caption{
		Mean runtimes of two-tier and three-tier \ShS*{}.
		The two-tier \ShS*{} are situated on the lowest file, which is labelled \enquote{/}.
		The three-tier \ShS*{} are situated on the files above, whose \(y\) values indicate the step size \(\stepsizes_2\).
		The coloured symbols encode the step sizes \(\stepsizes_1\).
	}
	\label{fig:shell:three-tier}
\end{figure}

When moving to greater input lengths (\cref{fig:shell:three-tier,fig:shell:three-tier_uint32reverse,fig:shell:three-tier_uint32uniform,fig:shell:three-tier_uint64reverse,fig:shell:three-tier_uint64uniform}), the differences in performance between the two-tier \ShS*{} become more pronounced;
%especially the ones with \(\stepsizes_1 = 3\) and \(\stepsizes_1 = 4\) fall off whereas the one with \(\stepsizes_1 = 6\) holds its ground quite well.
Between 48 and 64 elements, three passes get worthwhile to consider.
%Interestingly, many \ShS*{} with \(\stepsizes_2 = 4\) take the lead whilst the ones with \(\stepsizes_2 = 6\) are mid-table.
On the one hand, the findings are in accordance with the well-known ones by \citeauthor{10.1007/3-540-44669-9_12}~\cites{10.1007/3-540-44669-9_12}[cf.][]{skean2023optimization} who, for 128 elements, determined \(\stepsizes = \paren{1, 9}\) to be the optimal pair and \(\stepsizes = \paren{1, 4, 17}\) to be the optimal triplet, which is also the case in \cref{fig:shell:three-tier}.
On the other hand, the gain from doing three passes is far smaller:
\Citeauthor{10.1007/3-540-44669-9_12} calculated an average speed-up of 133\% over doing two passes, while it is only 116\% on a DPU.
In opposition to his findings, this also makes it unlikely that doing four passes would already net any gain at this input length.
This shows that past findings on non-uniform cost models cannot be applied one-to-one to DPUs.

But would pushing the limits of \ShS{} even be rewarding?
Greater input lengths require greater steps \Dash probably well into the three digits for \(n \approx 1000\) \cite{skean2023optimization,10.1007/3-540-44669-9_12} \Dash and those in turn require more sentinel values.
Implicit sentinel values could provide relief since the slow-down from implicitness should trend to zero for longer inputs, as was the case for \IS{}.
Still, finding the best step sizes for longer inputs requires a lot more work because the length and, thus, the number of reasonable combinations of step sizes become larger.
Unfortunately, longer optimal tuples cannot be constructed straightforwardly from shorter optimal ones, as seen in this section.

Its application, on the other hand, would likely be niche.
\ShS{} is outperformed by other algorithms presented hereafter, and those have no use for a \ShS{} adjusted to longer inputs.
Its only silver lining could be its in-place property (especially when relying solely on implicit sentinel values) combined with its medium speed, as discussed in \cref{sec:tasklet:conclusion}.
