\subsubsection*{Evaluation of the Performance}
\label{subsubsec:tasklet:shell:performance}

\pgfplotstablereadnamed{data/shell/two-tier/uint32/reverse.txt}{tableShellTwo_32reverse}
\pgfplotstablereadnamed{data/shell/two-tier/uint32/uniform.txt}{tableShellTwo_32uniform}
\pgfplotsinvokeforeach{7,...,17}{
	\pgfplotstablereadnamed{data/shell/h1=#1/uint32/reverse.txt}{tableShell#1_32reverse}
	\pgfplotstablereadnamed{data/shell/h1=#1/uint32/uniform.txt}{tableShell#1_32uniform}
}

Let us first focus on short input arrays where only two passes with step sizes~\(\stepsizes_1\) and \(1\) suffice.
The previous results on \IS{} suggest that the fastest \ShS{} should make use of~\(\stepsizes_1\) sentinel values.
\Cref{fig:shell:two-tier,fig:shell:two-tier_uint32,fig:shell:two-tier_uint64} show that, with the exception of the \ShS{} with step size \(\stepsizes_1 = 2\), the additional passes start to pay off at around 16 elements for both 32-bit and 64-bit values with the fully random input distributions, reaching a speed-up of around 15\% at around 24 elements.
In case of the reverse sorted input, the speed-up is practically always positive even for very short inputs, reaching between 50\% and 110\% at around 24 elements.

\begin{figure}
	\tikzsetnextfilename{shell_two-tier}
	\begin{tikzpicture}[plot]
		\begin{groupplot}[
			horizontal sep for labels,
			adaptive group=1 by 2,
			groupplot xlabel={Input Length \(n\)},
			xtick distance=3,
			minor xtick=data,
		]
			\nextgroupplot[ylabel=Cycles / \(n^2\), ymin=0, ymax=50, ytick distance=10, legend to name=leg:shell:two-tier]
			\legend{\(1\), \(...\), \(9\)}
			\pgfplotsinvokeforeach{1,...,9}{
				\plotpernn[x filter/.expression={x > #1 ? x : nan}]{#1}{tableSmallSorts_32uniform}
			}
			%
			\nextgroupplot[ylabel=Speed-up, ymin=0.7, ymax=1.2, ytick distance=0.1, yticklabel style={/pgf/number format/.cd, precision=1, fixed, zerofill}]
			\pgfplotsset{cycle list shift=1}
			\pgfplotsinvokeforeach{2,...,9}{
				\plotspeedup[x filter/.expression={x > #1 ? x : nan}]{#1}{1}{tableSmallSorts_32uniform}
			}
		\end{groupplot}
	\end{tikzpicture}

	\hfil\pgfplotslegendfromname{leg:shell:two-tier}\hfil
	\caption{
		Comparison of \IS{} (1) and various two-tier \ShS*{} (2--9), whose step sizes \(\stepsizes_1\) are indicated by the labels.
		The speed-ups are with respect to the \IS{}.
	}
	\label{fig:shell:two-tier}
\end{figure}

\NewDocumentCommand{\shellscatter}{m m m}{
	\addplotnamedtable[select row={#1}, forget plot][x=µ_#2, y expr={6}]{tableShellTwo_#3};
	\ifnumless{#2}{7}{
		\addplotnamedtable[select row={#1}, forget plot][x=µ_#2, y expr={7}]{tableShell7_#3};
	}{}
	\ifnumless{#2}{8}{
		\addplotnamedtable[select row={#1}, forget plot][x=µ_#2, y expr={8}]{tableShell8_#3};
	}{}
	\ifnumless{#2}{9}{
		\addplotnamedtable[select row={#1}, forget plot][x=µ_#2, y expr={9}]{tableShell9_#3};
	}{}
	\addplotnamedtable[select row={#1}][x=µ_#2, y expr={10}, forget plot]{tableShell10_#3};
	\addplotnamedtable[select row={#1}][x=µ_#2, y expr={11}, forget plot]{tableShell11_#3};
	\addplotnamedtable[select row={#1}][x=µ_#2, y expr={12}, forget plot]{tableShell12_#3};
	\addplotnamedtable[select row={#1}][x=µ_#2, y expr={13}, forget plot]{tableShell13_#3};
	\addplotnamedtable[select row={#1}][x=µ_#2, y expr={14}, forget plot]{tableShell14_#3};
	\addplotnamedtable[select row={#1}][x=µ_#2, y expr={15}, forget plot]{tableShell15_#3};
	\ifnumgreater{#1}{16}{
		\addplotnamedtable[select row={#1}][x=µ_#2, y expr={16}, forget plot]{tableShell16_#3};
		\addplotnamedtable[select row={#1}][x=µ_#2, y expr={17}]{tableShell17_#3};
	}{
		\addplotnamedtable[select row={#1},opacity=0][x=µ_#2, y expr={17}]{tableShell15_#3};
	}
}

\pgfplotsset{
	shell scatter plot/.style={
		adaptive group=3 by 2,
		groupplot xlabel={Mean [\(10^4\) Cycles]},
		groupplot ylabel={\(\stepsizes_1\)},
		scaled x ticks=base 10:-4,
		xtick scale label code/.code={},  % removes exponent underneath the axis
		ytick={6, 7, 9, ..., 17},
		yticklabels={/, \(7\), \(9\), \(...\), \(17\)},
		/tikz/only marks,
		cycle list shift=2,  % for sharing colours with the previous figure
		title={Input Length \textit{n} = },
		legend columns=-1,
	},
}

\begin{figure}[p]
	\tikzsetnextfilename{shell_three-tier}
	\begin{tikzpicture}[plot]
		\newcommand{\type}{32uniform}
		\begin{groupplot}[shell scatter plot]
			\pgfplotsinvokeforeach{16,32,48,64,96,128}{
				\nextgroupplot[title/.add={}{ #1}]
				\pgfplotsforeachungrouped\h in {3,...,9}{
					\shellscatter{#1}{\h}{\type}
				}
			}
			\pgfplotsset{legend to name=leg:shell:three-tier}
			\legend{\(3\), \(4\), \(5\), \(6\), \(7\), \(8\), \(9\)}
		\end{groupplot}
	\end{tikzpicture}

	\hfil\pgfplotslegendfromname{leg:shell:three-tier}\hfil
	\caption{
		Runtimes of \ShS*{} with two passes (/) and three passes (7--17).
		The coloured symbols encode the step size \(\stepsizes_1\) for two-tier \ShS*{} and the step size~\(\stepsizes_2\) for three-tier \ShS*{}.
		For the latter, the step size \(\stepsizes_1\) is noted on the y-axes.
	}
	\label{fig:shell:three-tier}
\end{figure}

When moving to greater input lengths (\cref{fig:shell:three-tier,fig:shell:three-tier_uint32reverse,fig:shell:three-tier_uint32uniform,fig:shell:three-tier_uint64reverse,fig:shell:three-tier_uint64uniform}), the differences in performance between the two-tier \ShS*{} become more pronounced;
%especially the ones with \(\stepsizes_1 = 3\) and \(\stepsizes_1 = 4\) fall off whereas the one with \(\stepsizes_1 = 6\) holds its ground quite well.
Between 48 and 64 elements, three passes get worthwhile to consider.
%Interestingly, many \ShS*{} with \(\stepsizes_2 = 4\) take the lead whilst the ones with \(\stepsizes_2 = 6\) are mid-table.
On the one hand, the results are in accordance with \citeauthor{10.1007/3-540-44669-9_12}~\cite{10.1007/3-540-44669-9_12} who, for 128 elements, determined \(\stepsizes = (1, 9)\) to be the optimal pair and \(\stepsizes = (17, 4, 1)\) to be the optimal triplet.
On the other hand, the gain from doing three passes is far smaller:
While \citeauthor{10.1007/3-540-44669-9_12} calculated an average speed-up of 33\% over doing two passes, while it is only 16\% on a DPU.
In opposition to his results, this also makes it unlikely that doing four passes would already net any gain at this input length.
Without access to \citeauthor{10.1007/3-540-44669-9_12}'s original code, giving a satisfactory explanation for the discrepancy is hard, however.

But would pushing the limits of \ShS{} even be rewarding?
Firstly, greater input lengths require greater steps \Dash well into the three digits for \(n \approx 1000\) \cite{skean2023optimization, 10.1007/3-540-44669-9_12} \Dash and those in turn require more sentinel values.
Implicit sentinel values could provide relief since the slow-down from implicitness should trend to zero for longer inputs, as was the case for \IS{}.
\todo[inline]{\enquote{%
		Secondly, there simply are better alternatives, namely \QS{} (which will be discussed in more detail in the next section).
		\Cref{fig:shell:against_others} shows that even though \ShS{} takes just a fraction of the time \IS{} takes \Dash apparently achieving a runtime between \(\bigomega{n \lb n}\) and \(\bigoh{n \lb^2 n}\) \Dash\negthinspace, \QS{} beats both from 20 elements onwards.
		Even \QS{}'s standard deviation of 1432 cycles at 128 elements is superior to \ShS{}'s 2670 cycles.
		Together with \cref{fig:shell:two-tier}, this means that \ShS{} is not worth using at all and will, consequently, not be improved upon in this thesis.%
}}
