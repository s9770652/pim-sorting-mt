\subsection{\texorpdfstring{\QS{}}{QuickSort}}
\label{subsec:tasklet:quick}

\pgfplotstableread{data/quick/fallback.txt}{\tableQuickFallback}
\pgfplotstableread{data/quick/pivot.txt}{\tableQuickPivot}

\QS{} uses partitioning to sort in an expected average runtime of \(\bigoh{n \log n}\):
A pivot element is chosen from the input array, then the input array gets scanned and elements bigger or smaller than the pivot are moved to the right or left of the pivot element, respectively.
Finally, \QS{} is used on the left and right partitions.


\paragraph{Base Cases}
When only a few elements remain in a partition, \QS{}'s overhead predominates such that \IS{} lends itself as fallback algorithm.
As \cref{fig:quick:fallback} demonstrates, the optimal threshold for switching the sorting algorithm is around 13 elements, netting a speed-up of 30\% and more over a \QS{} without fallback algorithm.
This low threshold also means that even a simple two-round \ShS{} is not worth considering.

\begin{figure}
	\begin{tikzpicture}[plot]
		\begin{groupplot}[
			width=0.4401\linewidth,
			group/group size=2 by 1,
			groupplot xlabel={Input Length \(n\)},
			groupplot ylabel={Speed-up},
			xmode=log,
			xtick={20, 32, 64, 128, 256, 512, 1024},
			xticklabels={\(20\), \(32\), \(64\), \(128\), \(256\), \(512\), \(1024\)},
			legend columns=-1,
		]
			\nextgroupplot[title={Over No Fallback\strut}, legend to name=leg:quick:fallback]
			\legend{\(10\), \(11\), \(...\), \(16\)}
			\pgfplotsinvokeforeach{10,...,16}{
				\plotspeedup{#1}{None}{\tableQuickFallback}
			}
			%
			\nextgroupplot[title={Over a Threshold of 13\strut}, /pgf/number format/.cd, precision=3, fixed zerofill=true]
			\pgfplotsinvokeforeach{10,...,16}{
				\ifnumequal{#1}{13}{
					\pgfplotsset{cycle list shift=1}
				}{
					\plotspeedup{#1}{13}{\tableQuickFallback}
				}
			}
		\end{groupplot}
	\end{tikzpicture}

	\hfil\pgfplotslegendfromname{leg:quick:fallback}\hfil
	\caption{
		Comparison of \QS*{} with different thresholds for the fallback to \IS{}, with a \QS{} without fallback algorithm and the fastest \QS{} with a threshold of 13 elements.
	}
	\label{fig:quick:fallback}
\end{figure}

Besides falling back to \IS{}, another base case is imaginable, namely terminating when the partition has a length of 1, 0, or even --1 elements.
Realistically speaking, this should not be necessary, because even though the extra check is done with just one additional instruction, it is a rare occurrence and the \IS{} would terminate after a few instructions anyway.
Yet, there are tremendous consequences for the runtime depending on the exact implementation of the base cases.
Since these are likely caused by the compiler, they are laid out in \cref{subsec:appendix:quick}.
\todo{zurückkehren nach Unterunterabschnitt}


\paragraph{Recursion vs. Iteration}
In theory, the question of whether an algorithm should be implemented recursively or iteratively comes down to convenience.
Due to the uniform cost of instructions, putting arguments automatically on the call stack or manually in an array essentially costs the same, as does jumping to the start of a loop and to the start of a function.
Furthermore, in case of \QS{}, the compiler turns tail-recursive calls into jumps back to the function start, so that one partition is sorted recursively and the other iteratively.
All this would suggest a recursive implementation with less code complexity.

In practice, it comes down to the compilation.
Selcouthly, even parts of the algorithms which are independent from the choice between recursion and iteration can be compiled differently, such that there are implementations where iteration is faster than recursion and the other way around.
Overall though, iterative implementations tend to be compiled better with superior register usage and less instructions used for the actual \QS{} algorithm.
The fastest implementation is indeed an iterative one, even if it beats the fastest recursive implementations \Dash outliers, admittedly \Dash by less than 4\%.
More details are given in \cref{subsubsec:tasklet:quick:compiler}.


\paragraph{Pivot Choice}
Another parameter to tune is the way in which the pivot is chosen.
The following were implemented and tested:
\begin{itemize}
	\item
	Using the \emph{last element} is the fastest way, requiring zero instructions.

	\item
	Choosing the \emph{middle element} is slower than choosing the last one, requiring a calculation of its address and swapping it with the last element so that it can act as sentinel value during partitioning.
	The upside is that it is more suited for sorted and nearly sorted inputs.

	\item
	Taking the \emph{median of three elements}, namely the first, middle, and last one, is even more computationally expensive but increases the chances of choosing a pivot that is neither particularly high nor particularly low.

	\item
	A \emph{random element} is most efficiently drawn using an xorshift random number generator and rejection sampling \cite{lukas_geis}.
\end{itemize}
Luckily, the pivot choice seldom has bearing on the overall compilation, making a comparison easier.
\todo{Stimmt nicht!}
The results are shown in \cref{fig:quick:pivot}.
Choosing the middle element is cheap enough for the runtime to be slowed down by a low single-digit percentage, and the increased pivot quality from choosing the median of three elements more than offsets the cost increase, thus making it the best choice.
At 1024 elements, the runtime with a random pivot is 10\% worse than with the median of three elements.
Since drawing the random index is more than thrice as costly as computing the middle index, a median of three random elements would likely yield even worse times, should one need randomisation.
Again, more details are given in \cref{subsubsec:tasklet:quick:compiler}.

\begin{figure}
	\begin{tikzpicture}[plot]
		\begin{groupplot}[
			width=0.4401\linewidth,
			group/group size=2 by 1,
			groupplot xlabel={Input Length \(n\)},
			xmode=log,
			xtick={20, 32, 64, 128, 256, 512, 1024},
			xticklabels={\(20\), \(32\), \(64\), \(128\), \(256\), \(512\), \(1024\)},
			legend columns=-1,
		]
			\nextgroupplot[ylabel=Cycles / \((n \lb n)\), ymin=55, ymax=70, legend to name=leg:quick:pivot]
			\legend{Last, Middle, Median of Three, Random}
			\plotpernlogn{End}{\tableQuickPivot}
			\plotpernlogn{Middle}{\tableQuickPivot}
			\plotpernlogn{MedianOfThree}{\tableQuickPivot}
			\plotpernlogn{Random}{\tableQuickPivot}
			%
			\nextgroupplot[ylabel=Speed-up, ymin=0.9, ymax=1.01, extra y ticks={1.01}, /pgf/number format/.cd, precision=2, fixed zerofill=true]
			\plotspeedup{End}{MedianOfThree}{\tableQuickPivot}
			\plotspeedup{Middle}{MedianOfThree}{\tableQuickPivot}
			\pgfplotsset{cycle list shift=1}
			\plotspeedup{Random}{MedianOfThree}{\tableQuickPivot}
		\end{groupplot}
	\end{tikzpicture}

	\hfil\pgfplotslegendfromname{leg:quick:pivot}\hfil
	\caption{
		Comparison of \QS{} with different pivot choices.
		The speed-ups are with respect to the \QS{} with the median of three as pivot choice.
	}
	\label{fig:quick:pivot}
\end{figure}


\paragraph{Prioritisation of Partitions}
After partitioning, in order to minimise the call stack, \QS{} should be used on the smaller of the two partitions first.
For code simplicity and to reduce the overhead, no such mechanism was implemented.
As shown in \cref{subsubsec:tasklet:quick:compiler}, the choice between always sorting the left-hand partition or the right-hand partition first can have tremendous effects nevertheless.



\subsection{\texorpdfstring{\QS{}}{QuickSort}}
\label{subsec:appendix:quick}

\pgfplotstableread{data/quick/recursive/no switched sides/uniform/end.txt}{\tableQuickRecNssUniEnd}
\pgfplotstableread{data/quick/recursive/no switched sides/uniform/middle.txt}{\tableQuickRecNssUniMiddle}
\pgfplotstableread{data/quick/recursive/no switched sides/uniform/median_of_three.txt}{\tableQuickRecNssUniMedian}
\pgfplotstableread{data/quick/recursive/no switched sides/uniform/random.txt}{\tableQuickRecNssUniRandom}
\pgfplotstableread{data/quick/recursive/switched sides/uniform/end.txt}{\tableQuickRecSsUniEnd}
\pgfplotstableread{data/quick/recursive/switched sides/uniform/middle.txt}{\tableQuickRecSsUniMiddle}
\pgfplotstableread{data/quick/recursive/switched sides/uniform/median_of_three.txt}{\tableQuickRecSsUniMedian}
\pgfplotstableread{data/quick/recursive/switched sides/uniform/random.txt}{\tableQuickRecSsUniRandom}
\pgfplotstableread{data/quick/iterative/no switched sides/uniform/end.txt}{\tableQuickIterNssUniEnd}
\pgfplotstableread{data/quick/iterative/no switched sides/uniform/middle.txt}{\tableQuickIterNssUniMiddle}
\pgfplotstableread{data/quick/iterative/no switched sides/uniform/median_of_three.txt}{\tableQuickIterNssUniMedian}
\pgfplotstableread{data/quick/iterative/no switched sides/uniform/random.txt}{\tableQuickIterNssUniRandom}
\pgfplotstableread{data/quick/iterative/switched sides/uniform/end.txt}{\tableQuickIterSsUniEnd}
\pgfplotstableread{data/quick/iterative/switched sides/uniform/middle.txt}{\tableQuickIterSsUniMiddle}
\pgfplotstableread{data/quick/iterative/switched sides/uniform/median_of_three.txt}{\tableQuickIterSsUniMedian}
\pgfplotstableread{data/quick/iterative/switched sides/uniform/random.txt}{\tableQuickIterSsUniRandom}

Two base cases exist:
\begin{enumerate*}
	\item
	If there are less than two elements left, terminate.

	\item
	If there are less than 14 elements left, call \IS{}.
\end{enumerate*}
The first base case is not strictly necessary as the second one covers it.
The fastest implementation with 622\,750 cycles on average for 1024 elements is as follows:
\begin{enumerate}
	\item\label[implementation]{imp:normal}
	If the partition has a length of 1 or less, terminate.
	If not and if the threshold is undercut, sort with \IS{}.
	Otherwise, sort with \QS{} and call \QS{} on both the left and right partition.
	\textcolor{red}{[Normal]}
\end{enumerate}
The following implementation, which avoids some function calls by reordering the checks, is ever so slightly slower at 625\,150 cycles:
\begin{enumerate}[resume]
	\item\label[implementation]{imp:triviality_before_call}
	If the threshold is undercut, sort with \IS{}.
	Otherwise, sort with \QS{}.
	Then check if the partitions have a length of 1 or less and call \QS{} on them if not.
	\textcolor{red}{[TrivialBC]}
\end{enumerate}
A look at the compilation reveals that only some jumps at the start and at the end of the function have changed.
It appears that the changed program flow causes one additional operation per call, since \QS{} gets called roughly 230 times and 230 × 11 cycles = 2530 cycles, which is the measured difference.
This one operation is already too much as partitions hardly have such short lengths and function calls are cheap.

But what if one were to get rid of all recursive calls on partitions below the threshold?
After all, they come up roughly 350 times.
Or what if one tried any other handling of the base cases?
All of the following implementations take between 710\,000 and 725\,000 cycles:
\begin{enumerate}[resume]
	\item\label[implementation]{imp:one_insertion}
	If the threshold is undercut, terminate.
	Otherwise, sort with \QS{} and call \QS{} on both the left and right partition.
	After all \QS*{} are done, sort the whole input array with \IS{}.
	\textcolor{red}{[OneInsertion]}

	\item\label[implementation]{imp:no_triviality}
	If the threshold is undercut, sort with \IS{}.
	Otherwise, sort with \QS{} and call \QS{} on both the left and right partition.
	\textcolor{red}{[NoTrivial]}

	\item\label[implementation]{imp:threshold_then_triviality}
	If the threshold is undercut, sort with \IS{}.
	If not, terminate if the partition has a length of 1 or less.
	Otherwise, sort with \QS{} and call \QS{} on both the left and right partition.
	\textcolor{red}{[ThreshThenTriv]}

	\item\label[implementation]{imp:threshold_before_call}
	Sort with \QS{}.
	Check if the partitions undercut the threshold and either call \IS{} or \QS{} on them.
	\textcolor{red}{[ThreshBC]}

	\item\label[implementation]{imp:threshold_and_triviality_before_call}
	Sort with \QS{}.
	Check if the partitions have a length of 1 or less or at least undercut the threshold and either call \IS{}, \QS{} or nothing on them.
	\textcolor{red}{[ThreshTrivBC]}

	\item\label[implementation]{imp:triviality_within_threshold}
	If the threshold is undercut, check whether the partition has a length of 1 or less and either terminate or sort with \IS{}.
	Otherwise, sort with \QS{} and call \QS{} on both the left and right partition.
	\textcolor{red}{[TrivInThresh]}
\end{enumerate}
In each of them, the compiler makes poorer use of the registers with the biggest impact on the loop which finds the next element to move to the right, increasing the length of an iteration from three instructions to four.

The phenomenon gets more complicated when one considers the iterative implementation.
In that case, \cref{imp:triviality_within_threshold} is the fastest with \cref{imp:triviality_before_call} coming in close second and all other implementations deteriorating to the similarly bad runtimes.

\todo[inline]{Verweis auf Begründung in Rek.\ gg.\ Iter.?}
\todo[inline]{Extra-Code ist teilweise verbuggt!}

\begin{figure}[p]
	\def\algos{Normal,TrivialBC,NoTrivial,OneInsertion,ThreshBC,ThreshTrivBC,ThreshThenTriv,TrivInThresh}
	\pgfplotsset{
		width=0.2308\linewidth,
		height=3.5cm,
		group/group size=4 by 2,
		group style={horizontal sep=1em, vertical sep=3.5em},  % For some reason, `group/horizontal sep` does not work.
		groupplot xlabel={Input Length \(n\)},
		groupplot ylabel={Cycles / \((n \lb n)\)},
		xmode=log,
		xtick={20, 64, 256, 1024},
		xticklabels={\(20\), \(64\), \(256\), \(1024\)},
		minor xtick={32, 128, 512},
		ymin=55,
		ymax=79,
		ytick={55,59,...,79},
		legend columns=-1,
	}
	\begin{subfigure}{\textwidth}
		\begin{tikzpicture}[plot]
			\begin{groupplot}
				\nextgroupplot[title={Last | Left First}, legend to name=leg:quick_implementations]
				\legend{\ref{imp:normal}, \ref{imp:triviality_before_call}, \ref{imp:no_triviality}, \ref{imp:one_insertion}, \ref{imp:threshold_before_call}, \ref{imp:threshold_and_triviality_before_call}, \ref{imp:threshold_then_triviality}, \ref{imp:triviality_within_threshold}}
				\expandafter\pgfplotsinvokeforeach\expandafter{\algos}{
					\plotpernlogn{#1}{\tableQuickRecNssUniEnd}
				}
				%
				\nextgroupplot[title={Middle | Left First}, yticklabels={}]
				\expandafter\pgfplotsinvokeforeach\expandafter{\algos}{
					\plotpernlogn{#1}{\tableQuickRecNssUniMiddle}
				}
				%
				\nextgroupplot[title={Median | Left First}, yticklabels={}]
				\expandafter\pgfplotsinvokeforeach\expandafter{\algos}{
					\plotpernlogn{#1}{\tableQuickRecNssUniMedian}
				}
				%
				\nextgroupplot[title={Random | Left First}, yticklabel pos=right]
				\expandafter\pgfplotsinvokeforeach\expandafter{\algos}{
					\plotpernlogn{#1}{\tableQuickRecNssUniRandom}
				}
				%
				\nextgroupplot[title={Last | Right First}]
				\expandafter\pgfplotsinvokeforeach\expandafter{\algos}{
					\plotpernlogn{#1}{\tableQuickRecSsUniEnd}
				}
				%
				\nextgroupplot[title={Middle | Right First}, yticklabels={}]
				\expandafter\pgfplotsinvokeforeach\expandafter{\algos}{
					\plotpernlogn{#1}{\tableQuickRecSsUniMiddle}
				}
				%
				\nextgroupplot[title={Median | Right First}, yticklabels={}]
				\expandafter\pgfplotsinvokeforeach\expandafter{\algos}{
					\plotpernlogn{#1}{\tableQuickRecSsUniMedian}
				}
				%
				\nextgroupplot[title={Random | Right First}, yticklabel pos=right]
				\expandafter\pgfplotsinvokeforeach\expandafter{\algos}{
					\plotpernlogn{#1}{\tableQuickRecSsUniRandom}
				}
			\end{groupplot}
		\end{tikzpicture}
		\caption{
			Recursive Implementation
		}
		\bigskip
	\end{subfigure}
	%
	\begin{subfigure}{\textwidth}
		\begin{tikzpicture}[plot]
			\begin{groupplot}
				\nextgroupplot[title={Last | Left First}]
				\expandafter\pgfplotsinvokeforeach\expandafter{\algos}{
					\plotpernlogn{#1}{\tableQuickIterNssUniEnd}
				}
				%
				\nextgroupplot[title={Middle | Left First}, yticklabels={}]
				\expandafter\pgfplotsinvokeforeach\expandafter{\algos}{
					\plotpernlogn{#1}{\tableQuickIterNssUniMiddle}
				}
				%
				\nextgroupplot[title={Median | Left First}, yticklabels={}]
				\expandafter\pgfplotsinvokeforeach\expandafter{\algos}{
					\plotpernlogn{#1}{\tableQuickIterNssUniMedian}
				}
				%
				\nextgroupplot[title={Random | Left First}, yticklabel pos=right]
				\expandafter\pgfplotsinvokeforeach\expandafter{\algos}{
					\plotpernlogn{#1}{\tableQuickIterNssUniRandom}
				}
				%
				\nextgroupplot[title={Last | Right First}]
				\expandafter\pgfplotsinvokeforeach\expandafter{\algos}{
					\plotpernlogn{#1}{\tableQuickIterSsUniEnd}
				}
				%
				\nextgroupplot[title={Middle | Right First}, yticklabels={}]
				\expandafter\pgfplotsinvokeforeach\expandafter{\algos}{
					\plotpernlogn{#1}{\tableQuickIterSsUniMiddle}
				}
				%
				\nextgroupplot[title={Median | Right First}, yticklabels={}]
				\expandafter\pgfplotsinvokeforeach\expandafter{\algos}{
					\plotpernlogn{#1}{\tableQuickIterSsUniMedian}
				}
				%
				\nextgroupplot[title={Random | Right First}, yticklabel pos=right]
				\expandafter\pgfplotsinvokeforeach\expandafter{\algos}{
					\plotpernlogn{#1}{\tableQuickIterSsUniRandom}
				}
			\end{groupplot}
		\end{tikzpicture}
		\caption{
			Iterative Implementation
		}
	\end{subfigure}

	\bigskip
	\hfil\pgfplotslegendfromname{leg:quick_implementations}\hfil
	\caption{
		Comparison of the different implementations (1--8) of \QS{} for all possible pivot choices.
		In the first rows, the left partitions are sorted before the right ones, while it is the reverse in the second rows.
	}
	\label{fig:quick_implementations}
\end{figure}

