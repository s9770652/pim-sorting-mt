\section{\texorpdfstring{\QS{}}{QuickSort}}
\label{sec:tasklet:quick}

\QS{} \cites{hoare1962quicksort}[88-91]{maurer1974datenstrukturen}[Chapter~2.2.6]{wirth1975algorithmen} uses partitioning to sort in an expected average runtime of \(\bigoh{n \log n}\) and a worst-case runtime of \(\bigoh{n^2}\):
A so-called \emph{pivot} element is chosen from the input array according to some method, then the whole input array gets scanned and elements greater or less than the pivot are moved to the right-hand or left-hand side of the array, that is the \emph{partitions}, respectively;
elements equal to the pivot are allowed to be in either partition.
Finally, \QS{} is used on the left-hand and right-hand partition.

The partitioning is implemented using a variant of \citeauthor{hoare1962quicksort}'s original scheme \cite{hoare1962quicksort}:
After the pivot \lstinline|p| is chosen, two pointers are set to either end of the array.
The left pointer \lstinline|i| moves rightwards until finding an element at least as great as the pivot (\lstinline|*i >= *p|).
Then, the right pointer \lstinline|j| moves leftwards until finding an element at most as great as the pivot (\lstinline|*j <= *p|).
Are the two elements found unequal, they are in the wrong order, so swapping them puts them in the right partitions.
Are they equal, swapping them anyway does not violate the order.
After swapping the elements, the pointers move onwards as described.
This process of repeated swaps continues until the pointers pass each other.
Where the right pointer \lstinline|j| came to rest marks the end of the left-hand partition, and where the left pointer \lstinline|i| did marks the beginning of the right-hand partition.

\QS{} does not sort in-place, as additional space of size \(\bigoh{\log n}\) is needed for a call stack.
Furthermore, \QS{} is not stable.


\paragraph{Sentinel Values}
\label{sec:tasklet:quick:imp:sentinels}
So as to avoid manifold bounds checks on the pointers, the partitioning presented above does not exactly follow \citeauthor{hoare1962quicksort}'s original scheme, where pointers halted only if \lstinline|*i > *p| and \lstinline|*j < *p|.
Instead, by halting if \lstinline|*i >= *p| and \lstinline|*j <= *p|, the pivot \lstinline|p| acts as sentinel value for both pointers as the halting condition is met there definitely.
This means that they cannot leave the array during their very first onwards movement.

If one pointer surpasses the pivot before the other reaches it, they act as additional sentinel values for each other.
If the left pointer \lstinline|i| reaches the right pointer \lstinline|j|, then all elements behind pointer \lstinline|j| are already in the correct partition, that is at least as great as the pivot.
An analogous argument can be made for when pointer \lstinline|j| meets pointer \lstinline|i|.
Of course, they can also halt directly on each other if the value they halted on is equal to the pivot.
In consequence, only one check is needed during partitioning, namely whether it holds \lstinline|j <= i| whenever both pointers halted.

A downside to this modification is that elements equal to the pivot are also swapped during partitioning.
But even on an input following Zipf's law, where many duplicates exist, this is a price worth paying.


\paragraph{Pivot Positioning}
By a further modification, one can find the final position of the pivot, so it needs not be touched anymore in the future.
After the pivot \lstinline|p| is chosen, it is swapped with the last element of the array.
Moreover, the right pointer \lstinline|j| begins at the second-last element.
Since the right-hand partition contains only elements at least as great as the pivot, the pivot must be the minimum of that partition.
Therefore, once the partitioning is over, the last element of the array, that is the pivot, can be swapped with the first element of the right-hand partition, that is the element at address \lstinline|i|.
The right-hand partition now begins at address \lstinline|i + 1| instead of \lstinline|i|.


\paragraph{Base Cases}
When only a few elements remain in a partition, \QS{}'s overhead predominates such that \IS{} lends itself as fallback algorithm.
Up to 40\% of the runtime is saved by falling back.
As seen in \cref{fig:quick:fallback}, the optimal threshold for switching the sorting algorithm is 18 elements for 32-bit integers on uniform inputs and likely similar on inputs following Zipf's or normal distributions.
For 64-bit integers, the optimal threshold is 17 elements.
Notwithstanding, we set 18 elements to be the default threshold for both data types to simplify matters since the impact is minuscule.
For sorted and almost sorted inputs, the threshold is higher since \IS{} performs well on them, so falling back earlier and, thus, ending the sorting process is better.
Because \QS{}'s two pointers invert large swaths of reverse sorted inputs while partitioning, the same is true for that input distribution even though it is \IS{}'s worst-case.
However, such input distributions should be catered for by a pattern-defeating \QS{} as laid out in \cref{sec:tasklet:conclusion}, hence the 18 elements as default threshold altogether.

To avoid unnecessary uses of \IS{}, another base case is imaginable, namely terminating when a partition contains at most one element.
There are tremendous consequences for the runtime depending on the exact implementation of the base cases, as shown later in \enquote{\nameref{sec:tasklet:quick:compilation}}.
\begin{figure}[t]
	\pgfplotstableset{
		create on use/n/.style={create col/copy column from table={data/quick/fallback/uint32/16.txt}{n}},
	}
	\pgfplotsinvokeforeach{14,15,16,17,18,19,20}{
		\pgfplotstableset{create on use/µ_#1_32/.style={create col/copy column from table={data/quick/fallback/uint32/#1.txt}{µ_TrivialBC}}}
		\pgfplotstableset{create on use/µ_#1_64/.style={create col/copy column from table={data/quick/fallback/uint64/#1.txt}{µ_TrivialBC}}}
	}
	\pgfplotstablenew[columns={n,µ_14_32,µ_15_32,µ_16_32,µ_17_32,µ_18_32,µ_19_32,µ_20_32,µ_14_64,µ_15_64,µ_16_64,µ_17_64,µ_18_64,µ_19_64,µ_20_64}]{\pgfplotstablegetrowsof{data/quick/fallback/uint32/16.txt}}{\tableQuickFallback}

	\tikzsetnextfilename{quick_fallback}
	\begin{tikzpicture}[plot]
		\begin{groupplot}[
			horizontal sep for labels,
			adaptive group=1 by 2,
			groupplot ylabel={Speed-up},
			x from 16 to 1024,
			ymin=0.994,
			ymax=1.002,
			yticklabel style={/pgf/number format/.cd, precision=3, fixed, zerofill},
		]
			\nextgroupplot[title/.add={}{32-bit}]
			\pgfplotsset{legend to name=leg:quick:fallback, legend entries={16,17,19,20}}
			\pgfplotsinvokeforeach{16,17,19,20}{
				\plotspeedup{#1_32}{18_32}{tableQuickFallback}
			}
			%
			\nextgroupplot[title/.add={}{64-bit}]
			\pgfplotsinvokeforeach{16,17,19,20}{
				\plotspeedup{#1_64}{18_64}{tableQuickFallback}
			}
		\end{groupplot}
	\end{tikzpicture}

	\hfil\pgfplotslegendfromname{leg:quick:fallback}\hfil
	\caption{
		Speed-ups of \QS*{} with different thresholds (16--20) for when to fall back to \IS{} over a threshold of 18 elements, conducted on uniform input distributions.
		Using \ShS{} is not beneficial because many partitions undercut the thresholds significantly.
	}
	\label{fig:quick:fallback}
\end{figure}


\paragraph{Recursion vs.\ Iteration}
In theory, the question of whether an algorithm should be implemented recursively or iteratively comes down to convenience.
Due to the uniform costs of instructions, jumping to the beginning of a loop or to the beginning of a function essentially costs the same, as does managing arguments automatically through the regular call stack and manually through a simulated one.
Furthermore, in case of \QS{}, the compiler turns tail-recursive calls into jumps back to the beginning of the function, so that one partition is sorted recursively and the other iteratively.
All this would suggest a recursive implementation due to the reduced maintenance.

In practice, it comes down to the compilation.
Even parts of the algorithms which are independent from the choice between recursion and iteration can be compiled differently, such that there are implementations where iteration is faster than recursion and the other way around.
Overall though, iterative implementations tend to be compiled better with superior register usage and less instructions used for the actual \QS{} algorithm.


\paragraph{Partition Prioritisation}
Always sorting the shorter partition first and putting the longer partition on the call stack guarantees that the problem size is at least halved each step, so that the call stack stores \(\bigoh{\log n}\) elements at most.
Unfortunately, this approach is detrimental to the quality of the compilation, which is why it is advisable to always prioritise the same side.
Whether the left-hand or the right-hand partition is sorted first should not make any difference for the runtime but even that changes the quality of the compilation;
in this thesis, the right-hand partitions are prioritised.


\paragraph{Pivot Choice}
Another parameter to tune is the way in which the pivot is chosen.
The following were implemented and tested:
\begin{itemize}
	\item
	Using the \emph{last element} is the fastest way to chose, requiring zero additional instructions.

	\item
	Taking the \emph{deterministic median} of three elements, namely the first, middle, and last one, is more computationally expensive since the position of the middle element must be calculated, the median be determined, and the pivot be swapped with the last element of the array.

	\item
	A \emph{random element} is most efficiently drawn using an xorshift random number generator and rejection sampling \cite{lukas_geis}.

	\item
	The \emph{random median} is a combination of the previous two methods, where the median of three random elements is taken.
	For simplicity, there is no check on whether an element is drawn twice or thrice.
	Since the partitions are rather long, this should happen seldom, anyhow.
\end{itemize}
A median increases the chances of choosing a pivot that is neither particularly high nor particularly low.
This leads to more balanced partitions such that the call stack is less likely to overflow and the base cases are reached faster.
But as long as one of the deterministic methods is used, it is possible to construct inputs where the runtime climbs up to \(\bigoh*{n^2}\)~\cite{erkiö1984worstcase}, for example when everything is moved to the same partition so that the problem size is reduced by only one element (namely the pivot) after each partitioning step.
This is especially problematic as it easily leads to an overflown call stack.

The random pivots circumvent this issue.
Whilst the pivots could, by ill luck, also lead to the same unbalanced partitions as the deterministic pivots, the worst-case expected runtime is \(\bigoh{n \log n}\) \cite{blum2011probabilistic}.
The median of medians \cite{blum1973median} guarantees a runtime of \(\bigoh{n \log n}\) but was not implemented because a \emph{performant} implementation would probably be quite complex and its benefit minuscule for this thesis.

\subsubsection*{Investigation of the Compilation}
\label{subsubsec:tasklet:quick:compilation}

\def\quickpivots{LAST,MEDIAN,RANDOM,MEDIAN_OF_RANDOM}
\expandafter\pgfplotsinvokeforeach\expandafter{\quickpivots}{
	\pgfplotstablereadnamed{data/quick/matrix/iterative/#1/shorter/uint32/uniform.txt}{tableQuickMatrixIt#1Shorter_32}
	\pgfplotstablereadnamed{data/quick/matrix/iterative/#1/left/uint32/uniform.txt}{tableQuickMatrixIt#1Left_32}
	\pgfplotstablereadnamed{data/quick/matrix/iterative/#1/right/uint32/uniform.txt}{tableQuickMatrixIt#1Right_32}

	\pgfplotstablereadnamed{data/quick/matrix/recursive/#1/shorter/uint32/uniform.txt}{tableQuickMatrixRec#1Shorter_32}
	\pgfplotstablereadnamed{data/quick/matrix/recursive/#1/left/uint32/uniform.txt}{tableQuickMatrixRec#1Left_32}
	\pgfplotstablereadnamed{data/quick/matrix/recursive/#1/right/uint32/uniform.txt}{tableQuickMatrixRec#1Right_32}
}
\def\quickalgos{Normal,TrivInThresh,NoTrivial,ThreshThenTriv,TrivialBC,ThreshBC,ThreshTrivBC,OneInsertion}

The quality of the compilation of \QS{} is highly erratic to such an extent that \Dash even with the same pivots! \Dash one implementation variant may see a reduction of 25\% from another one where none should be.
There are small details influencing the runtime, like storing the value of the pivot in a dedicated variable instead of accessing it through a pointer changing the runtime by a few percentage points in both directions, depending on the rest of implementation.
But as hinted at in the preceding part of this section, there are four major parameters to examine:
handling of the base cases, recursion/iteration, pivot choice, and partition prioritisation.
Before the findings are discussed, the first parameter shall be explained in more depth.

Besides falling back to \IS{} if 18 elements remain (\enquote{treshold undercut}), another base case is imaginable, namely a termination if at most one element remains (\enquote{trivial length}).
Realistically speaking, it should not be needed to check for trivial lengths because even though it is doable with just one additional instruction, such short partitions are rare and \IS{} would terminate after a few instructions anyway.
Nonetheless, its inclusion or exclusion can have significant impacts.
The following handlings were tested:
\begin{enumerate}[label=(\liningnums{\arabic*})]
	\item\label[implementation]{imp:normal}
	If the length is trivial, terminate immediately.
	If if the threshold is undercut, sort with \IS{} and terminate.
	Otherwise, sort with \QS{} and use \QS{} on both partitions.
%	\textcolor{red}{[Normal]}

	\item\label[implementation]{imp:triviality_within_threshold}
	If the threshold is undercut, check if the length is trivial and terminate immediately or sort with \IS{} and then terminate, respectively.
	Otherwise, sort with \QS{} and use \QS{} on both partitions.
%	\textcolor{red}{[TrivInThresh]}
	\begin{itemize}
		\item
		This handling significantly reduces the number of checks for trivial length.
	\end{itemize}

	\item\label[implementation]{imp:no_triviality}
	If the threshold is undercut, sort with \IS{} and terminate.
	Otherwise, sort with \QS{} and use \QS{} on both partitions.
%	\textcolor{red}{[NoTrivial]}
	\begin{itemize}
		\item
		This handling forgoes the check for a trivial length completely, at the cost of some unneeded \IS*{}.
	\end{itemize}

	\item\label[implementation]{imp:threshold_then_triviality}
	If the threshold is undercut, sort with \IS{} and terminate.
	If the length is trivial, terminate immediately.
	Otherwise, sort with \QS{} and use \QS{} on both partitions.
%	\textcolor{red}{[ThreshThenTriv]}
	\begin{itemize}
		\item
		This handling, while nonsensical from a logical point of view, gives the compiler an explicit guarantee that the partitioning loop does not end immediately.
	\end{itemize}

	\item\label[implementation]{imp:triviality_before_call}
	If the threshold is undercut, sort with \IS{} and terminate.
	Otherwise, sort with \QS{}.
	Then check for either partition if its length is trivial and use \QS{} if not.
%	\textcolor{red}{[TrivialBC]}
	\begin{itemize}
		\item
		This handling, as well as the next two, gets rid of some unneeded uses of \QS{}.
		In the recursive case, these handlings lose the property of being tail-recursive.
	\end{itemize}

	\item\label[implementation]{imp:threshold_before_call}
	Sort with \QS{}.
	Check for either partition if the threshold is undercut and use \IS{} or \QS{} on them, respectively.
%	\textcolor{red}{[ThreshBC]}

	\item\label[implementation]{imp:threshold_and_triviality_before_call}
	Sort with \QS{}.
	Check for either partition if its length is trivial or if the threshold is undercut and use \IS{}, \QS{}, or nothing on them, respectively.
%	\textcolor{red}{[ThreshTrivBC]}

	\item\label[implementation]{imp:one_insertion}
	If the threshold is undercut, terminate immediately.
	Otherwise, sort with \QS{} and use \QS{} on both partitions.
	After all \QS*{} are done, sort the whole input array with \IS{}.
%	\textcolor{red}{[OneInsertion]}
	\begin{itemize}
		\item
		This handling always does one pass of \IS{}.
		For example, the other handlings use \IS{} roughly 91 times on 1024 uniformly distributed elements.
	\end{itemize}
\end{enumerate}
The performances of all tested implementation for 32-bit integers are shown in \cref{fig:quick:implementations}.
The measurements were done on uniform input distributions so the deterministic pivots are, in expectation, of the same quality as the random ones.

\pgfplotsset{
	quick matrix/.style={
		height=2.567cm,
		horizontal sep for naught,
		vertical sep for naught,
		adaptive group=3 by 4,
		groupplot xlabel={Input Length \(n\)},
		groupplot ylabel={Cycles / \((n \lb n)\)},
		xmode=log,
		xtick={16, 64, 256, 1024},
		xticklabels={\(16\), \(64\), \(256\), \(1024\)},
		minor xtick={32, 128, 512},
		ymin=55,
		ymax=80,
		/tikz/mark repeat=2,
		legend columns=-1,
	}
}

\begin{figure}[p]
	\captionsetup[subfigure]{aboveskip=0mm,belowskip=1mm}
	\begin{subfigure}{\textwidth}
		\tikzsetnextfilename{quick_implementations_rec}
		\begin{tikzpicture}[plot]
			\begin{groupplot}[quick matrix]
				\nextgroupplot[title=Last, xticklabels={}]
				\pgfplotsset{legend to name=leg:quick:implementations, legend entries={\ref{imp:normal}, \ref{imp:triviality_within_threshold}, \ref{imp:no_triviality}, \ref{imp:threshold_then_triviality}, \ref{imp:triviality_before_call}, \ref{imp:threshold_before_call}, \ref{imp:threshold_and_triviality_before_call}, \ref{imp:one_insertion}}}
				\expandafter\pgfplotsinvokeforeach\expandafter{\quickalgos}{
					\plotpernlogn{#1}{tableQuickMatrixRecLASTLeft_32}
				}
				\nextgroupplot[title=Median, xticklabels={}, yticklabels={}]
				\expandafter\pgfplotsinvokeforeach\expandafter{\quickalgos}{
					\plotpernlogn{#1}{tableQuickMatrixRecMEDIANLeft_32}
				}
				\nextgroupplot[title=Random, xticklabels={}, yticklabels={}]
				\expandafter\pgfplotsinvokeforeach\expandafter{\quickalgos}{
					\plotpernlogn{#1}{tableQuickMatrixRecRANDOMLeft_32}
				}
				\nextgroupplot[title=Median (Random), xticklabels={}, yticklabel pos=right]
				\expandafter\pgfplotsinvokeforeach\expandafter{\quickalgos}{
					\plotpernlogn{#1}{tableQuickMatrixRecMEDIAN_OF_RANDOMLeft_32}
				}
				%
				\nextgroupplot[xticklabels={}]
				\expandafter\pgfplotsinvokeforeach\expandafter{\quickalgos}{
					\plotpernlogn{#1}{tableQuickMatrixRecLASTRight_32}
				}
				\nextgroupplot[xticklabels={}, yticklabels={}]
				\expandafter\pgfplotsinvokeforeach\expandafter{\quickalgos}{
					\plotpernlogn{#1}{tableQuickMatrixRecMEDIANRight_32}
				}
				\nextgroupplot[xticklabels={}, yticklabels={}]
				\expandafter\pgfplotsinvokeforeach\expandafter{\quickalgos}{
					\plotpernlogn{#1}{tableQuickMatrixRecRANDOMRight_32}
				}
				\nextgroupplot[xticklabels={}, yticklabel pos=right]
				\expandafter\pgfplotsinvokeforeach\expandafter{\quickalgos}{
					\plotpernlogn{#1}{tableQuickMatrixRecMEDIAN_OF_RANDOMRight_32}
				}
				%
				\nextgroupplot
				\expandafter\pgfplotsinvokeforeach\expandafter{\quickalgos}{
					\plotpernlogn{#1}{tableQuickMatrixRecLASTShorter_32}
				}
				\nextgroupplot[yticklabels={}]
				\expandafter\pgfplotsinvokeforeach\expandafter{\quickalgos}{
					\plotpernlogn{#1}{tableQuickMatrixRecMEDIANShorter_32}
				}
				\nextgroupplot[yticklabels={}]
				\expandafter\pgfplotsinvokeforeach\expandafter{\quickalgos}{
					\plotpernlogn{#1}{tableQuickMatrixRecRANDOMShorter_32}
				}
				\nextgroupplot[yticklabel pos=right]
				\expandafter\pgfplotsinvokeforeach\expandafter{\quickalgos}{
					\plotpernlogn{#1}{tableQuickMatrixRecMEDIAN_OF_RANDOMShorter_32}
				}
			\end{groupplot}
		\end{tikzpicture}
		\caption{
			Recursive Approach
		}
	\end{subfigure}
	\begin{subfigure}{\textwidth}
		\tikzsetnextfilename{quick_implementations_it}
		\begin{tikzpicture}[plot]
			\begin{groupplot}[quick matrix]
				\nextgroupplot[title=Last, xticklabels={}]
				\expandafter\pgfplotsinvokeforeach\expandafter{\quickalgos}{
					\plotpernlogn{#1}{tableQuickMatrixItLASTLeft_32}
				}
				\nextgroupplot[title=Median, xticklabels={}, yticklabels={}]
				\expandafter\pgfplotsinvokeforeach\expandafter{\quickalgos}{
					\plotpernlogn{#1}{tableQuickMatrixItMEDIANLeft_32}
				}
				\nextgroupplot[title=Random, xticklabels={}, yticklabels={}]
				\expandafter\pgfplotsinvokeforeach\expandafter{\quickalgos}{
					\plotpernlogn{#1}{tableQuickMatrixItRANDOMLeft_32}
				}
				\nextgroupplot[title=Median (Random), xticklabels={}, yticklabel pos=right]
				\expandafter\pgfplotsinvokeforeach\expandafter{\quickalgos}{
					\plotpernlogn{#1}{tableQuickMatrixItMEDIAN_OF_RANDOMLeft_32}
				}
				%
				\nextgroupplot[xticklabels={}]
				\expandafter\pgfplotsinvokeforeach\expandafter{\quickalgos}{
					\plotpernlogn{#1}{tableQuickMatrixItLASTRight_32}
				}
				\nextgroupplot[xticklabels={}, yticklabels={}]
				\expandafter\pgfplotsinvokeforeach\expandafter{\quickalgos}{
					\plotpernlogn{#1}{tableQuickMatrixItMEDIANRight_32}
				}
				\nextgroupplot[xticklabels={}, yticklabels={}]
				\expandafter\pgfplotsinvokeforeach\expandafter{\quickalgos}{
					\plotpernlogn{#1}{tableQuickMatrixItRANDOMRight_32}
				}
				\nextgroupplot[xticklabels={}, yticklabel pos=right]
				\expandafter\pgfplotsinvokeforeach\expandafter{\quickalgos}{
					\plotpernlogn{#1}{tableQuickMatrixItMEDIAN_OF_RANDOMRight_32}
				}
				%
				\nextgroupplot
				\expandafter\pgfplotsinvokeforeach\expandafter{\quickalgos}{
					\plotpernlogn{#1}{tableQuickMatrixItLASTShorter_32}
				}
				\nextgroupplot[yticklabels={}]
				\expandafter\pgfplotsinvokeforeach\expandafter{\quickalgos}{
					\plotpernlogn{#1}{tableQuickMatrixItMEDIANShorter_32}
				}
				\nextgroupplot[yticklabels={}]
				\expandafter\pgfplotsinvokeforeach\expandafter{\quickalgos}{
					\plotpernlogn{#1}{tableQuickMatrixItRANDOMShorter_32}
				}
				\nextgroupplot[yticklabel pos=right]
				\expandafter\pgfplotsinvokeforeach\expandafter{\quickalgos}{
					\plotpernlogn{#1}{tableQuickMatrixItMEDIAN_OF_RANDOMShorter_32}
				}
			\end{groupplot}
		\end{tikzpicture}
		\caption{
			Iterative Approach
		}
	\end{subfigure}

	\tikzexternaldisable
	\hfil\pgfplotslegendfromname{leg:quick:implementations}\hfil
	\tikzexternalenable
	\caption{
		Comparison of \crefrange{imp:normal}{imp:one_insertion} and different pivots.
		Left-hand partitions are prio\-ri\-tised in the first rows, right-hand ones in the second rows, and shorter ones in the third~rows.
	}
	\label{fig:quick:implementations}
\end{figure}

Even when ignoring the differences between specific handlings, the high fluctuations between the plots leap to the eye.
Plots within the same column share the same method to chose pivots, plots within the same row share the same prioritisation of partitions.
In general, it would be expected that plots within the same column are fairly similar, yet prioritising the shorter partition is almost universally associated with an increase in runtime.
When focussing on the top-performing implementations, the increase can reach more than 15\%.
There is no clear trend between the consistent prioritisations of either side, although the difference can be huge in individual cases as well.
However, recursion is more susceptible to the partition prioritisation than iteration.

The correlation of recursive and iterative performance is weak.
On one hand, there is, for example, \cref{imp:triviality_before_call} with deterministic medians and prioritisation of shorter partitions where the runtimes are essentially the same.
On the other hand, there is \cref{imp:one_insertion} with deterministic medians and prioritisation of right-hand partitions where recursion is slower by more than a third.
All in all, iterative implementations usually perform better, though, especially when focussing on the top-performing implementations of each pivot choice.

The ranking of the different handlings is rather incoherent.
\Cref{imp:triviality_before_call}, which does not call \QS{} on trivial partitions, decidedly fares the best out of all handlings, being amongst the top performers across all tested implementations.
\Cref{imp:threshold_before_call,imp:threshold_and_triviality_before_call}, which call \QS{} even less often than \cref{imp:triviality_before_call}, are the polar opposite and bring up the rear of the ranking every single time.
Recursive implementations of \cref{imp:triviality_within_threshold}, where triviality is checked for only if the threshold is undercut, are often worse than recursive implementations of \cref{imp:normal}, where triviality is always checked for, whilst it is the other way around for iterative implementations.
Interestingly, for all investigated implementations, the compiler is capable of eliminating the last possible recursive calls by turning them into jumps back to the function start, regardless of whether these were properly tail recursive or not.

These observations, however, only apply to 32-bit integers.
\Cref{fig:quick:implementations_64} shows the same measurements for 64-bit integers.
Whilst the general trend for pivots, partition prioritisation and recursion/iteration hold true, the rankings are vastly different.
\Cref{imp:triviality_before_call} is not undisputedly the best anymore.
\Cref{imp:threshold_before_call,imp:threshold_and_triviality_before_call} switch back and forth between being the worst and the best handlings.
Most notably, the two top-performing implementations using deterministic and random medians as pivots, respectively, are actually recursive.
Luckily, both use \cref{imp:triviality_before_call} so the only difference between the default configurations of 32-bit and 64-bit \QS{} is the usage of recursion/iteration.

What is causing these huge disparities?
There is a great variety in the compilations but some of the common occurrences are \dots{}
\begin{itemize}
	\item
	\dots{} one instruction more before (re-)starting to move the pointers, \dots{}

	\item
	\dots{} one instruction more while moving the left pointer by one element, \dots{}

	\item
	\dots{} one instruction more after the left pointer has stopped, \dots{}

	\item
	\dots{} more stores and loads when entering and leaving the function.
\end{itemize}
This focus on the left pointer \lstinline|i| is partially explainable by it being used to calculate the final position of the pivot \lstinline|p| and, thus, the inner boundaries of both new partitions:
When the left pointer \lstinline|i| stops, it holds \lstinline|*i >= p|.
Also, the left pointer either passed over the preceding element if \lstinline|*(i - 1) < p|, or it stopped there if \lstinline|*(i - 1) >= p|.
However, the right pointer \lstinline|j| stopped on some value fulfilling \lstinline|*j <= p|, so it holds \lstinline|*(i - 1) <= p| after swapping.
Either way, it holds \lstinline|*(i - 1) <= p|.
If the right pointer \lstinline|j| meets the left pointer \lstinline|i|, it either stops there immediately if \lstinline|*i = p|, or it stops at address \lstinline|i - 1| because of \lstinline|*(i - 1) <= p|.
In all cases, the pivot, which was moved to the right of the partition at the start, can now swap with pointer \lstinline|i|, and the addresses \lstinline|i - 1| and \lstinline|i + 1| form the end of the left-hand partition and the start of the right-hand partition, respectively.
We spot-checked implementations to see whether using the right pointer alone or both of them to calculate the boundaries could alleviate the problems but the results were mixed:
from betterment over indifference to worsening, everything was observable.


\subsection*{Evaluation of the Performance}
\label{sec:tasklet:quick:performance}
\addcontentsline{toc}{subsection}{\nameref{sec:tasklet:quick:performance}}

\pgfplotsinvokeforeach{sorted,reverse,almost,uniform,zipf,normal}{
	\pgfplotstablereadnamed{data/quick/matrix/iterative/Median_of_random/right/uint32/#1.txt}{tableQuickRand_32#1}
	\pgfplotstablereadnamed{data/quick/matrix/recursive/Median_of_random/right/uint64/#1.txt}{tableQuickRand_64#1}
}

Before turning to the performance of \QS{} on specific input distributions, the ratio between costs and benefits of the pivot selection methods shall be evaluated.
Looking again at \cref{fig:quick:implementations,fig:heap:runtime_uint64} shows that a median gets more beneficial, the longer the input becomes, achieving small pay-offs for the longest ones.
Moreover, the standard deviations of the runtimes, although not shown in the figures for reasons of clarity, are cut roughly in half.
Randomisation slows down noticeably, so random pivots are disadvantageous if the input is known to be fairly random.
However, the decrease remains in the single digits percentage-wise, supporting the findings by \citeauthor{lukas_geis}~\cite{lukas_geis} that drawing random numbers is quite cheap.
For this reason, the random median is used as default method throughout this thesis.

\begin{figure}
	\tikzsetnextfilename{quick_runtime}
	\begin{tikzpicture}[plot]
		\begin{groupplot}[
			adaptive group=1 by 2,
			groupplot ylabel={Cycles / \((n \lb n)\)},
			x from 16 to 1024,
			ytick distance=10,
		]
			\nextgroupplot[title/.add={}{32-bit}, ymin=30, ymax=80]
			\pgfplotsset{legend to name=leg:quick:runtime, legend entries={Sorted, Reverse S., Almost S., Uniform, Zipf's, Normal}}
			\pgfplotsinvokeforeach{sorted,reverse,almost,uniform,zipf,normal}{
				\plotpernlogn{TrivialBC}{tableQuickRand_32#1}
			}
			%
			\nextgroupplot[title/.add={}{64-bit}, ymin=40,ymax=90]
			\pgfplotsinvokeforeach{sorted,reverse,almost,uniform,zipf,normal}{
				\plotpernlogn{TrivialBC}{tableQuickRand_64#1}
			}
		\end{groupplot}
	\end{tikzpicture}

	\hfil\pgfplotslegendfromname{leg:quick:runtime}\hfil
	\caption{
		Mean runtime of \QS{} on all tested input distributions and data types.
	}
	\label{fig:quick:runtime}
\end{figure}

\Cref{fig:quick:runtime} shows the runtime of \QS{} in it default configuration, that is, with random medians.
\Cref{fig:quick:runtime_uint32,fig:quick:runtime_uint64} additionally contain the runtimes with deterministic medians as well as the standard deviations of the measurements.
The mean runtimes are rather close across all input distributions, a consequence of using random medians and of considering elements equal to the pivot as different.
In fact, it is \IS{} that primarily causes the discrepancies, as setting the threshold to one element proves.
This also explains why \QS{} performs so well on large inputs with Zipf's distribution:
This distributions generates many duplicates, which are put into the same partitions, so \IS{} performs many simple scans.

One might expect \QS{} to perform even better on sorted and reverse sorted input, since everything is either already in the correct position or because the two pointers quickly invert large swaths of the inputs.
However, a side effect of swapping the pivot twice can be that many elements are displaced by one position from where they should be in the sorted input.
Take reverse sorted inputs and the deterministic median as an example:
The element \(n/2\) is selected as pivot out of the elements \(n\), \(n/2\), and \(0\) and then gets swapped with the last element, that is, with \(0\).
Thereupon, the pointers invert the rest of the input such that the start of the input looks something like \(1, 2, \dots, n/2-1, 0, n/2, \dots\) after the first partitioning step.
Indeed, this pattern makes \QS{} with deterministic medians degrade and eventually overflow the call stack, which is why the respective plots in \cref{fig:quick:runtime_uint32,fig:quick:runtime_uint64} leave the charts.
An implementation without swapping the pivot promises better performance for such cases, but in exploratory ones, the performance on more random input distributions suffered drastically.

