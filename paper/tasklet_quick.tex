\subsection{\texorpdfstring{\QS{}}{QuickSort}}
\label{subsec:tasklet:quick}

%\pgfplotstablereadnamed{data/quick/fallback.txt}{tableQuickFallback}
%\pgfplotstablereadnamed{data/quick/pivot.txt}{tableQuickPivot}

\QS{} uses partitioning to sort in an expected average runtime of \(\bigoh{n \log n}\):
A pivot element is chosen from the input array, then the input array gets scanned and elements greater or lesser than the pivot are moved to the right or left of the pivot element, respectively.
Finally, \QS{} is used on the left and right partitions.
The \QS*{} implementations presented here are neither stable nor in-place.


\paragraph{Base Cases}
When only a few elements remain in a partition, \QS{}'s overhead predominates such that \IS{} lends itself as fallback algorithm.
As \cref{fig:quick:fallback} demonstrates, the optimal threshold for switching the sorting algorithm is around 13 elements, netting a speed-up of 30\% and more over a \QS{} without fallback algorithm.
This low threshold also means that even a simple two-round \ShS{} is not worth considering.

%\begin{figure}
%	\tikzsetnextfilename{quick_fallback}
%	\begin{tikzpicture}[plot]
%		\begin{groupplot}[
%			horizontal sep for labels,
%			adaptive group=1 by 2,
%			groupplot xlabel={Input Length \(n\)},
%			groupplot ylabel={Speed-up},
%			xmode=log,
%			xtick={20, 32, 64, 128, 256, 512, 1024},
%			xticklabels={\(20\), \(32\), \(64\), \(128\), \(256\), \(512\), \(1024\)},
%			legend columns=-1,
%		]
%			\nextgroupplot[title={Over No Fallback\strut}, legend to name=leg:quick:fallback]
%			\legend{\(10\), \(11\), \(...\), \(16\)}
%			\pgfplotsinvokeforeach{10,...,16}{
%				\plotspeedup{#1}{None}{tableQuickFallback}
%			}
%			%
%			\nextgroupplot[title={Over a Threshold of 13\strut}, /pgf/number format/.cd, precision=3, fixed zerofill=true]
%			\pgfplotsinvokeforeach{10,...,16}{
%				\ifnumequal{#1}{13}{
%					\pgfplotsset{cycle list shift=1}
%				}{
%					\plotspeedup{#1}{13}{tableQuickFallback}
%				}
%			}
%		\end{groupplot}
%	\end{tikzpicture}
%
%	\hfil\pgfplotslegendfromname{leg:quick:fallback}\hfil
%	\caption{
%		Comparison of \QS*{} with different thresholds for the fallback to \IS{}, with a \QS{} without fallback algorithm and the fastest \QS{} with a threshold of 13 elements.
%	}
%	\label{fig:quick:fallback}
%\end{figure}

Besides falling back to \IS{}, another base case is imaginable, namely terminating when the partition has a length of at most 1 elements.
Realistically speaking, checking for this should not be necessary, because even though the extra check is done with just one additional instruction, it occurs rarely, and the \IS{} would terminate after a few instructions anyway.
Yet, there are tremendous consequences for the runtime depending on the exact implementation of the base cases, as seen later.
\todo{zurückkehren}


\paragraph{Recursion vs. Iteration}
In theory, the question of whether a DPU algorithm should be implemented recursively or iteratively comes down to convenience.
Due to the uniform costs of instructions, putting arguments automatically on the call stack or manually in an array essentially costs the same, as does jumping to the start of a loop and to the start of a function.
Furthermore, in case of \QS{}, the compiler turns tail-recursive calls into jumps back to the function start, so that one partition is sorted recursively and the other iteratively.
All this would suggest a recursive implementation with less code complexity.

In practice, it comes down to the compilation.
Selcouthly, even parts of the algorithms which are independent from the choice between recursion and iteration can be compiled differently, such that there are implementations where iteration is faster than recursion and the other way around.
Overall though, iterative implementations tend to be compiled better with superior register usage and less instructions used for the actual \QS{} algorithm.
The fastest implementation is indeed an iterative one, even if it beats the fastest recursive implementations \Dash outliers, admittedly \Dash by less than 4\%.
\todo{zurückkehren}


\paragraph{Pivot Choice}
Another parameter to tune is the way in which the pivot is chosen.
The following were implemented and tested:
\begin{itemize}
	\item
	Using the \emph{last element} is the fastest way, requiring zero additional instructions.

%	\item
%	Choosing the \emph{middle element} is slower than choosing the last one, requiring a calculation of its address and swapping it with the last element so that it can act as sentinel value during partitioning.
%	The upside is that it is more suited for sorted and nearly sorted inputs.

	\item
	Taking the \emph{median of three elements}, namely the first, middle, and last one, is far more computationally expensive since the position of the middle element must be calculated, the median be determined, and the pivot be swapped with the last element of the array, where it acts as sentinel.
	The plus side is that this method increases the chances of choosing a pivot that is neither particularly high nor particularly low.
	This leads to more balanced partitions so the call stack is less likely to overflow and the base cases are reached faster.

	\item
	A \emph{random element} is most efficiently drawn using an xorshift random number generator and rejection sampling \cite{lukas_geis}.
	This takes some instructions but impedes worst-case inputs.

	\item
	Taking \emph{median of three random elements} is a combination of the previous two methods.
	For simplicity, there is no check on whether an element is drawn twice or even thrice.
	Since the partitions are rather long, this should happen seldom, though.
\end{itemize}
Luckily, the pivot choice seldom has bearing on the overall compilation, making a comparison easier.
\todo{Stimmt nicht!}
The results are shown in \cref{fig:quick:pivot}.
Choosing the middle element is cheap enough for the runtime to be slowed down by a low single-digit percentage, and the increased pivot quality from choosing the median of three elements more than offsets the cost increase, thus making it the best choice.
At 1024 elements, the runtime with a random pivot is 10\% worse than with the median of three elements.
Since drawing the random index is more than thrice as costly as computing the middle index, a median of three random elements would likely yield even worse times, should one need randomisation.
Again, more details are given in \cref{subsubsec:tasklet:quick:compilation}.
\todo{zurückkehren}

%\begin{figure}
%	\tikzsetnextfilename{quick_pivot}
%	\begin{tikzpicture}[plot]
%		\begin{groupplot}[
%			horizontal sep for labels,
%			adaptive group=1 by 2,
%			groupplot xlabel={Input Length \(n\)},
%			xmode=log,
%			xtick={20, 32, 64, 128, 256, 512, 1024},
%			xticklabels={\(20\), \(32\), \(64\), \(128\), \(256\), \(512\), \(1024\)},
%			legend columns=-1,
%		]
%			\nextgroupplot[ylabel=Cycles / \((n \lb n)\), ymin=55, ymax=70, legend to name=leg:quick:pivot]
%			\legend{Last, Middle, Median of Three, Random}
%			\plotpernlogn{End}{tableQuickPivot}
%			\plotpernlogn{Middle}{tableQuickPivot}
%			\plotpernlogn{MedianOfThree}{tableQuickPivot}
%			\plotpernlogn{Random}{tableQuickPivot}
%			%
%			\nextgroupplot[ylabel=Speed-up, ymin=0.9, ymax=1.01, extra y ticks={1.01}, /pgf/number format/.cd, precision=2, fixed zerofill=true]
%			\plotspeedup{End}{MedianOfThree}{tableQuickPivot}
%			\plotspeedup{Middle}{MedianOfThree}{tableQuickPivot}
%			\pgfplotsset{cycle list shift=1}
%			\plotspeedup{Random}{MedianOfThree}{tableQuickPivot}
%		\end{groupplot}
%	\end{tikzpicture}
%
%	\hfil\pgfplotslegendfromname{leg:quick:pivot}\hfil
%	\caption{
%		Comparison of \QS{} with different pivot choices.
%		The speed-ups are with respect to the \QS{} with the median of three as pivot choice.
%	}
%	\label{fig:quick:pivot}
%\end{figure}


\paragraph{Partition Prioritisation}
Whether the left-hand or the right-hand is put on the stack should not make any difference for the runtime.
However, always putting the longer partition on the stack guarantees that the problem size is at least halved each step, meaning the call stack stores \(\bigoh{log n}\) elements at most.
This last approach, as shown later, is linked to huge speed penalties, so always prioritising the same partition is actually used throughout this Thesis in general.
But even then, the choice between the two sides can have tremendous effects.
\todo{zurückkehren}



\subsubsection*{Investigation of the Compilation}
\label{subsubsec:tasklet:quick:compilation}

%\pgfplotstablereadnamed{data/quick/rec_vs_it.txt}{tableQuickRecVsIter}
%\pgfplotstablereadnamed{data/quick/recursive/no switched sides/uniform/end.txt}{tableQuickRecNssUniEnd}
%\pgfplotstablereadnamed{data/quick/recursive/no switched sides/uniform/middle.txt}{tableQuickRecNssUniMiddle}
%\pgfplotstablereadnamed{data/quick/recursive/no switched sides/uniform/median_of_three.txt}{tableQuickRecNssUniMedian}
%\pgfplotstablereadnamed{data/quick/recursive/no switched sides/uniform/random.txt}{tableQuickRecNssUniRandom}
%\pgfplotstablereadnamed{data/quick/recursive/switched sides/uniform/end.txt}{tableQuickRecSsUniEnd}
%\pgfplotstablereadnamed{data/quick/recursive/switched sides/uniform/middle.txt}{tableQuickRecSsUniMiddle}
%\pgfplotstablereadnamed{data/quick/recursive/switched sides/uniform/median_of_three.txt}{tableQuickRecSsUniMedian}
%\pgfplotstablereadnamed{data/quick/recursive/switched sides/uniform/random.txt}{tableQuickRecSsUniRandom}
%\pgfplotstablereadnamed{data/quick/iterative/no switched sides/uniform/end.txt}{tableQuickIterNssUniEnd}
%\pgfplotstablereadnamed{data/quick/iterative/no switched sides/uniform/middle.txt}{tableQuickIterNssUniMiddle}
%\pgfplotstablereadnamed{data/quick/iterative/no switched sides/uniform/median_of_three.txt}{tableQuickIterNssUniMedian}
%\pgfplotstablereadnamed{data/quick/iterative/no switched sides/uniform/random.txt}{tableQuickIterNssUniRandom}
%\pgfplotstablereadnamed{data/quick/iterative/switched sides/uniform/end.txt}{tableQuickIterSsUniEnd}
%\pgfplotstablereadnamed{data/quick/iterative/switched sides/uniform/middle.txt}{tableQuickIterSsUniMiddle}
%\pgfplotstablereadnamed{data/quick/iterative/switched sides/uniform/median_of_three.txt}{tableQuickIterSsUniMedian}
%\pgfplotstablereadnamed{data/quick/iterative/switched sides/uniform/random.txt}{tableQuickIterSsUniRandom}

The quality of the compilation and thus the real performance of \QS{} is erratic to such an extent that one implementation variant may see a speed-up of 25\% over another one even with the same pivot choice although virtually none would be expected.
As hinted in the preceding paragraphs, this raises the need for a benchmark suite with the following parameters:
base case handling, recursion/iteration, pivot choice, and partition prioritisation.
Before the results are discussed, the first parameter shall be explained in more depth.

Besides falling back to \IS{} if 13 elements remain (\enquote{treshold undercut}), another base case is imaginable, namely a full termination if 1, 0, or --1 elements remain (\enquote{trivial length}).
Theoretically, it should not be needed to check for trivial lengths because even though it is doable with just one additional instruction, such short partitions are rare and the \IS{} would terminate after a few instructions anyway.
Nonetheless, its inclusion or exclusion can have significant impacts.
The following \nameCrefs{imp:normal} were tested:
\begin{enumerate}
	\item\label[implementation]{imp:normal}
	If the length is trivial, terminate.
	If not and if the threshold is undercut, sort with \IS{}.
	Otherwise, sort with \QS{} and use \QS{} on both partitions.
%	\textcolor{red}{[Normal]}

	\item\label[implementation]{imp:triviality_within_threshold}
	If the threshold is undercut, check if the length is trivial and terminate or sort with \IS{}, respectively.
	Otherwise, sort with \QS{} and use \QS{} on both partitions.
%	\textcolor{red}{[TrivInThresh]}
	\begin{itemize}
		\item
		This \nameCref{imp:triviality_within_threshold} significantly reduces the number of checks for trivial length.
	\end{itemize}

	\item\label[implementation]{imp:no_triviality}
	If the threshold is undercut, sort with \IS{}.
	Otherwise, sort with \QS{} and use \QS{} on both partitions.
%	\textcolor{red}{[NoTrivial]}
	\begin{itemize}
		\item
		This \nameCref{imp:no_triviality} forgoes the check for a trivial length completely, at the cost of unneeded \IS*{}.
	\end{itemize}

	\item\label[implementation]{imp:threshold_then_triviality}
	If the threshold is undercut, sort with \IS{}.
	If not and if the length is trivial, terminate.
	Otherwise, sort with \QS{} and use \QS{} on both partitions.
%	\textcolor{red}{[ThreshThenTriv]}
	\begin{itemize}
		\item
		This \nameCref{imp:threshold_then_triviality}, while nonsensical from a logical point of view, gives the compiler an explicit guarantee that the partitioning loop does not end immediately.
	\end{itemize}

	\item\label[implementation]{imp:triviality_before_call}
	If the threshold is undercut, sort with \IS{}.
	Otherwise, sort with \QS{}.
	Then check for either partition if its length is trivial and use \QS{} if not.
%	\textcolor{red}{[TrivialBC]}
	\begin{itemize}
		\item
		This \nameCref{imp:triviality_before_call}, as well as the next two, gets rid of some unneeded uses of \QS{}.
		In the recursive case, these \nameCrefs{imp:triviality_before_call} lose the property of being tail-recursive.
	\end{itemize}

	\item\label[implementation]{imp:threshold_before_call}
	Sort with \QS{}.
	Check for either partition if the threshold is undercut and use \IS{} or \QS{}, respectively.
%	\textcolor{red}{[ThreshBC]}

	\item\label[implementation]{imp:threshold_and_triviality_before_call}
	Sort with \QS{}.
	Check for either partition if its length is trivial or if the threshold is undercut and use \IS{}, \QS{}, or nothing, respectively.
%	\textcolor{red}{[ThreshTrivBC]}

	\item\label[implementation]{imp:one_insertion}
	If the threshold is undercut, terminate.
	Otherwise, sort with \QS{} and use \QS{} on both partitions.
	After all \QS*{} are done, sort the whole input array with \IS{}.
%	\textcolor{red}{[OneInsertion]}
	\begin{itemize}
		\item
		This \nameCref{imp:one_insertion} always does one pass of \IS{}.
		For example, the other \nameCrefs{imp:normal} do roughly 90 at 1024 elements.
	\end{itemize}
\end{enumerate}

All results are shown in \cref{fig:quick:implementations}.
When using recursion, \cref{imp:normal,imp:triviality_before_call} perform the best, especially for longer inputs.
Their compilations are fundamentally the same, including the conversion of the second recursive call into a jump back to the function start.
All other \nameCrefs{imp:normal} fare vastly worse.
Common occurrences are \dots{}
\begin{itemize}
	\item
	\dots{} one more instruction in the loop finding the next element to move to the right, \dots{}

	\item
	\dots{} one more instruction after such an element has been found, \dots{}

	\item
	\dots{} more stores and loads when entering and leaving the function.
\end{itemize}

\todo[inline]{%
	Ich kann mir leider nicht alles erklären.
	Als Beispiel habe ich die Kompilate von \cref{imp:normal} / Recursive / Last für Left First und Right First hochgeladen (die verkürzten Varianten besitzen nur noch Befehle und Sprungmarken).
	Ersteres ist ja die schnellste rekursive Variante, während letzteres deutlich schlechter abschneidet.
	Dennoch sehe ich bei der langsameren Variante keinen fundamental anderen Algorithmus.
	Je Funktionsaufruf kommen ≈3 Extra-Aufrufe hinzu (bei insgesamt ≈104 rekursiven und ≈104 \enquote{Endaufrufen} bei 1024 Elementen), was fast 70\,000 Takte Unterschied nicht erklären kann.
}

%\begin{figure}[p]
%	\def\algos{Normal,TrivInThresh,NoTrivial,ThreshThenTriv,TrivialBC,ThreshBC,ThreshTrivBC,OneInsertion}
%	\pgfplotsset{
%		height=3.5cm,
%		horizontal sep for naught,
%		adaptive group=2 by 4,
%		groupplot xlabel={Input Length \(n\)},
%		groupplot ylabel={Cycles / \((n \lb n)\)},
%		xmode=log,
%		xtick={20, 64, 256, 1024},
%		xticklabels={\(20\), \(64\), \(256\), \(1024\)},
%		minor xtick={32, 128, 512},
%		ymin=55,
%		ymax=80,
%		ytick distance=5,
%		legend columns=-1,
%	}
%	\begin{subfigure}{\textwidth}
%		\tikzsetnextfilename{quick_implementations_rec}
%		\begin{tikzpicture}[plot]
%			\begin{groupplot}
%				\nextgroupplot[title={Last | Left First}, legend to name=leg:quick:implementations]
%				\legend{\ref{imp:normal}, \ref{imp:triviality_within_threshold}, \ref{imp:no_triviality}, \ref{imp:threshold_then_triviality}, \ref{imp:triviality_before_call}, \ref{imp:threshold_before_call}, \ref{imp:threshold_and_triviality_before_call}, \ref{imp:one_insertion}}
%				\expandafter\pgfplotsinvokeforeach\expandafter{\algos}{
%					\plotpernlogn{#1}{tableQuickRecNssUniEnd}
%				}
%				%
%				\nextgroupplot[title={Middle | Left First}, yticklabels={}]
%				\expandafter\pgfplotsinvokeforeach\expandafter{\algos}{
%					\plotpernlogn{#1}{tableQuickRecNssUniMiddle}
%				}
%				%
%				\nextgroupplot[title={Median | Left First}, yticklabels={}]
%				\expandafter\pgfplotsinvokeforeach\expandafter{\algos}{
%					\plotpernlogn{#1}{tableQuickRecNssUniMedian}
%				}
%				%
%				\nextgroupplot[title={Random | Left First}, yticklabel pos=right]
%				\expandafter\pgfplotsinvokeforeach\expandafter{\algos}{
%					\plotpernlogn{#1}{tableQuickRecNssUniRandom}
%				}
%				%
%				\nextgroupplot[title={Last | Right First}]
%				\expandafter\pgfplotsinvokeforeach\expandafter{\algos}{
%					\plotpernlogn{#1}{tableQuickRecSsUniEnd}
%				}
%				%
%				\nextgroupplot[title={Middle | Right First}, yticklabels={}]
%				\expandafter\pgfplotsinvokeforeach\expandafter{\algos}{
%					\plotpernlogn{#1}{tableQuickRecSsUniMiddle}
%				}
%				%
%				\nextgroupplot[title={Median | Right First}, yticklabels={}]
%				\expandafter\pgfplotsinvokeforeach\expandafter{\algos}{
%					\plotpernlogn{#1}{tableQuickRecSsUniMedian}
%				}
%				%
%				\nextgroupplot[title={Random | Right First}, yticklabel pos=right]
%				\expandafter\pgfplotsinvokeforeach\expandafter{\algos}{
%					\plotpernlogn{#1}{tableQuickRecSsUniRandom}
%				}
%			\end{groupplot}
%		\end{tikzpicture}
%		\caption{
%			Recursive Approach
%		}
%		\bigskip
%	\end{subfigure}
%	%
%	\begin{subfigure}{\textwidth}
%		\tikzsetnextfilename{quick_implementations_it}
%		\begin{tikzpicture}[plot]
%			\begin{groupplot}
%				\nextgroupplot[title={Last | Left First}]
%				\expandafter\pgfplotsinvokeforeach\expandafter{\algos}{
%					\plotpernlogn{#1}{tableQuickIterNssUniEnd}
%				}
%				%
%				\nextgroupplot[title={Middle | Left First}, yticklabels={}]
%				\expandafter\pgfplotsinvokeforeach\expandafter{\algos}{
%					\plotpernlogn{#1}{tableQuickIterNssUniMiddle}
%				}
%				%
%				\nextgroupplot[title={Median | Left First}, yticklabels={}]
%				\expandafter\pgfplotsinvokeforeach\expandafter{\algos}{
%					\plotpernlogn{#1}{tableQuickIterNssUniMedian}
%				}
%				%
%				\nextgroupplot[title={Random | Left First}, yticklabel pos=right]
%				\expandafter\pgfplotsinvokeforeach\expandafter{\algos}{
%					\plotpernlogn{#1}{tableQuickIterNssUniRandom}
%				}
%				%
%				\nextgroupplot[title={Last | Right First}]
%				\expandafter\pgfplotsinvokeforeach\expandafter{\algos}{
%					\plotpernlogn{#1}{tableQuickIterSsUniEnd}
%				}
%				%
%				\nextgroupplot[title={Middle | Right First}, yticklabels={}]
%				\expandafter\pgfplotsinvokeforeach\expandafter{\algos}{
%					\plotpernlogn{#1}{tableQuickIterSsUniMiddle}
%				}
%				%
%				\nextgroupplot[title={Median | Right First}, yticklabels={}]
%				\expandafter\pgfplotsinvokeforeach\expandafter{\algos}{
%					\plotpernlogn{#1}{tableQuickIterSsUniMedian}
%				}
%				%
%				\nextgroupplot[title={Random | Right First}, yticklabel pos=right]
%				\expandafter\pgfplotsinvokeforeach\expandafter{\algos}{
%					\plotpernlogn{#1}{tableQuickIterSsUniRandom}
%				}
%			\end{groupplot}
%		\end{tikzpicture}
%		\caption{
%			Iterative approach
%		}
%	\end{subfigure}
%
%	\bigskip
%	\hfil\pgfplotslegendfromname{leg:quick:implementations}\hfil
%	\caption{
%		Comparison of the different implementations (1--8) of \QS{} for all possible pivot choices.
%		In the first rows, the left-hand partitions are sorted before the right-hand ones, while it is the reverse in the second rows.
%	}
%	\label{fig:quick:implementations}
%\end{figure}
%
%\begin{figure}
%	\tikzsetnextfilename{quick_rec_vs_it}
%	\begin{tikzpicture}[plot]
%		\begin{groupplot}[
%			horizontal sep for labels,
%			adaptive group=1 by 2,
%			groupplot xlabel={Input Length \(n\)},
%			xmode=log,
%			xtick={20, 32, 64, 128, 256, 512, 1024},
%			xticklabels={\(20\), \(32\), \(64\), \(128\), \(256\), \(512\), \(1024\)},
%			legend columns=-1,
%		]
%			\nextgroupplot[ylabel=Cycles / \((n \lb n)\), ymin=55, ymax=65, extra y ticks={55, 65}, legend to name=leg:rec_vs_it]
%			\legend{Iterative, Recursive}
%			\plotpernlogn{It}{tableQuickRecVsIter}
%			\plotpernlogn{Rec}{tableQuickRecVsIter}
%			%
%			\nextgroupplot[ylabel=Speed-up, ymin=0.96, ymax=1.04, /pgf/number format/.cd, precision=2, fixed zerofill=true]
%			\plotspeedup{It}{Rec}{tableQuickRecVsIter}
%		\end{groupplot}
%	\end{tikzpicture}
%
%	\hfil\pgfplotslegendfromname{leg:rec_vs_it}\hfil
%	\caption{
%		Comparison of the fastest recursive and iterative \QS*{} (cf. \cref{subsubsec:tasklet:quick:compiler}).
%		The actual algorithm is compiled the very same in both cases, so that time differences are only due to the way \QS{} is applied to the partitions.
%		\todo{nicht mehr wegen des Pivots!}
%	}
%	\label{fig:rec_vs_it}
%\end{figure}

\clearpage

\def\quickpivots{LAST,MEDIAN,RANDOM,MEDIAN_OF_RANDOM}
\expandafter\pgfplotsinvokeforeach\expandafter{\quickpivots}{
	\pgfplotstablereadnamed{data/quick/matrix/iterative/#1/shorter/uniform.txt}{tableQuickMatrixIt#1Shorter}
	\pgfplotstablereadnamed{data/quick/matrix/iterative/#1/left/uniform.txt}{tableQuickMatrixIt#1Left}
	\pgfplotstablereadnamed{data/quick/matrix/iterative/#1/right/uniform.txt}{tableQuickMatrixIt#1Right}

	\pgfplotstablereadnamed{data/quick/matrix/recursive/#1/shorter/uniform.txt}{tableQuickMatrixRec#1Shorter}
	\pgfplotstablereadnamed{data/quick/matrix/recursive/#1/left/uniform.txt}{tableQuickMatrixRec#1Left}
	\pgfplotstablereadnamed{data/quick/matrix/recursive/#1/right/uniform.txt}{tableQuickMatrixRec#1Right}
}

\tikzexternaldisable

\begin{figure}[p]
	\pgfplotsset{
		height=2.6cm,
		horizontal sep for naught,
		vertical sep for naught,
		adaptive group=3 by 4,
		groupplot xlabel={Input Length \(n\)},
		groupplot ylabel={Cycles / \((n \lb n)\)},
		xmode=log,
		xtick={16, 64, 256, 1024},
		xticklabels={\(16\), \(64\), \(256\), \(1024\)},
		minor xtick={32, 128, 512},
		ymin=55,
		ymax=80,
		legend columns=-1,
	}
	\def\quickalgos{Normal,TrivInThresh,NoTrivial,ThreshThenTriv,TrivialBC,ThreshBC,ThreshTrivBC,OneInsertion}
	\captionsetup[subfigure]{aboveskip=0mm,belowskip=1mm}
	\begin{subfigure}{\textwidth}
		\tikzsetnextfilename{quick_implementations_rec}
		\begin{tikzpicture}[plot]
			\begin{groupplot}
				\nextgroupplot[title=Last, xticklabels={}, legend to name=leg:quick:implementations]
				\legend{\ref{imp:normal}, \ref{imp:triviality_within_threshold}, \ref{imp:no_triviality}, \ref{imp:threshold_then_triviality}, \ref{imp:triviality_before_call}, \ref{imp:threshold_before_call}, \ref{imp:threshold_and_triviality_before_call}, \ref{imp:one_insertion}}
				\expandafter\pgfplotsinvokeforeach\expandafter{\quickalgos}{
					\plotpernlogn{#1}{tableQuickMatrixRecLASTLeft}
				}
				\nextgroupplot[title=Median, xticklabels={}, yticklabels={}]
				\expandafter\pgfplotsinvokeforeach\expandafter{\quickalgos}{
					\plotpernlogn{#1}{tableQuickMatrixRecMEDIANLeft}
				}
				\nextgroupplot[title=Random, xticklabels={}, yticklabels={}]
				\expandafter\pgfplotsinvokeforeach\expandafter{\quickalgos}{
					\plotpernlogn{#1}{tableQuickMatrixRecRANDOMLeft}
				}
				\nextgroupplot[title=Median (Random), xticklabels={}, yticklabel pos=right]
				\expandafter\pgfplotsinvokeforeach\expandafter{\quickalgos}{
					\plotpernlogn{#1}{tableQuickMatrixRecMEDIAN_OF_RANDOMLeft}
				}
				%
				\nextgroupplot[xticklabels={}]
				\expandafter\pgfplotsinvokeforeach\expandafter{\quickalgos}{
					\plotpernlogn{#1}{tableQuickMatrixRecLASTRight}
				}
				\nextgroupplot[xticklabels={}, yticklabels={}]
				\expandafter\pgfplotsinvokeforeach\expandafter{\quickalgos}{
					\plotpernlogn{#1}{tableQuickMatrixRecMEDIANRight}
				}
				\nextgroupplot[xticklabels={}, yticklabels={}]
				\expandafter\pgfplotsinvokeforeach\expandafter{\quickalgos}{
					\plotpernlogn{#1}{tableQuickMatrixRecRANDOMRight}
				}
				\nextgroupplot[xticklabels={}, yticklabel pos=right]
				\expandafter\pgfplotsinvokeforeach\expandafter{\quickalgos}{
					\plotpernlogn{#1}{tableQuickMatrixRecMEDIAN_OF_RANDOMRight}
				}
				%
				\nextgroupplot
				\expandafter\pgfplotsinvokeforeach\expandafter{\quickalgos}{
					\plotpernlogn{#1}{tableQuickMatrixRecLASTShorter}
				}
				\nextgroupplot[yticklabels={}]
				\expandafter\pgfplotsinvokeforeach\expandafter{\quickalgos}{
					\plotpernlogn{#1}{tableQuickMatrixRecMEDIANShorter}
				}
				\nextgroupplot[yticklabels={}]
				\expandafter\pgfplotsinvokeforeach\expandafter{\quickalgos}{
					\plotpernlogn{#1}{tableQuickMatrixRecRANDOMShorter}
				}
				\nextgroupplot[yticklabel pos=right]
				\expandafter\pgfplotsinvokeforeach\expandafter{\quickalgos}{
					\plotpernlogn{#1}{tableQuickMatrixRecMEDIAN_OF_RANDOMShorter}
				}
			\end{groupplot}
		\end{tikzpicture}
		\caption{
			Recursive Approach
		}
	\end{subfigure}
	\begin{subfigure}{\textwidth}
		\tikzsetnextfilename{quick_implementations_it}
		\begin{tikzpicture}[plot]
			\begin{groupplot}
				\nextgroupplot[title=Last, xticklabels={}]
				\expandafter\pgfplotsinvokeforeach\expandafter{\quickalgos}{
					\plotpernlogn{#1}{tableQuickMatrixItLASTLeft}
				}
				\nextgroupplot[title=Median, xticklabels={}, yticklabels={}]
				\expandafter\pgfplotsinvokeforeach\expandafter{\quickalgos}{
					\plotpernlogn{#1}{tableQuickMatrixItMEDIANLeft}
				}
				\nextgroupplot[title=Random, xticklabels={}, yticklabels={}]
				\expandafter\pgfplotsinvokeforeach\expandafter{\quickalgos}{
					\plotpernlogn{#1}{tableQuickMatrixItRANDOMLeft}
				}
				\nextgroupplot[title=Median (Random), xticklabels={}, yticklabel pos=right]
				\expandafter\pgfplotsinvokeforeach\expandafter{\quickalgos}{
					\plotpernlogn{#1}{tableQuickMatrixItMEDIAN_OF_RANDOMLeft}
				}
				%
				\nextgroupplot[xticklabels={}]
				\expandafter\pgfplotsinvokeforeach\expandafter{\quickalgos}{
					\plotpernlogn{#1}{tableQuickMatrixItLASTRight}
				}
				\nextgroupplot[xticklabels={}, yticklabels={}]
				\expandafter\pgfplotsinvokeforeach\expandafter{\quickalgos}{
					\plotpernlogn{#1}{tableQuickMatrixItMEDIANRight}
				}
				\nextgroupplot[xticklabels={}, yticklabels={}]
				\expandafter\pgfplotsinvokeforeach\expandafter{\quickalgos}{
					\plotpernlogn{#1}{tableQuickMatrixItRANDOMRight}
				}
				\nextgroupplot[xticklabels={}, yticklabel pos=right]
				\expandafter\pgfplotsinvokeforeach\expandafter{\quickalgos}{
					\plotpernlogn{#1}{tableQuickMatrixItMEDIAN_OF_RANDOMRight}
				}
				%
				\nextgroupplot
				\expandafter\pgfplotsinvokeforeach\expandafter{\quickalgos}{
					\plotpernlogn{#1}{tableQuickMatrixItLASTShorter}
				}
				\nextgroupplot[yticklabels={}]
				\expandafter\pgfplotsinvokeforeach\expandafter{\quickalgos}{
					\plotpernlogn{#1}{tableQuickMatrixItMEDIANShorter}
				}
				\nextgroupplot[yticklabels={}]
				\expandafter\pgfplotsinvokeforeach\expandafter{\quickalgos}{
					\plotpernlogn{#1}{tableQuickMatrixItRANDOMShorter}
				}
				\nextgroupplot[yticklabel pos=right]
				\expandafter\pgfplotsinvokeforeach\expandafter{\quickalgos}{
					\plotpernlogn{#1}{tableQuickMatrixItMEDIAN_OF_RANDOMShorter}
				}
			\end{groupplot}
		\end{tikzpicture}
		\caption{
			Iterative Approach
		}
	\end{subfigure}

	\hfil\pgfplotslegendfromname{leg:quick:implementations}\hfil
	\caption{
		Comparison of \crefrange{imp:normal}{imp:one_insertion}.
		The left-hand partition is preferred in the first rows, the right-hand one in the second rows, and the shorter one in the third rows.
	}
	\label{fig:quick:implementations}
\end{figure}

\tikzexternalenable

\clearpage

