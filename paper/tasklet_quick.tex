\section{\texorpdfstring{\QS{}}{QuickSort}}
\label{sec:tasklet:quick}

\QS{} \cite{hoare1962quicksort} uses partitioning to sort in an expected average runtime of \(\bigoh{n \log n}\) and a worst-case runtime of \(\bigoh{n^2}\):
A pivot element is chosen from the input array, then the input array gets scanned and elements greater or lesser than the pivot are moved to the right or left side of the array, respectively.
Finally, \QS{} is used on the left and right side (the \enquote{partitions}).
\QS{} does not sort in-place, as additional space of size \(\bigoh{\log n}\) is needed for a call stack.
Furthermore, \QS{} is not stable.


\paragraph{Sentinel Values}
The partitioning is implemented using \citeauthor{hoare1962quicksort}'s original scheme \cite{hoare1962quicksort}:
At the start of each partitioning step, a pivot \lstinline|p| is chosen and swapped with the last element.
Then, two pointers are set to either end of the partition.
The left pointer \lstinline|i| moves rightwards until finding an element at least as great as the pivot (\lstinline|*i >= p|), while the right pointer \lstinline|j| moves leftwards until finding an element at most as great as the pivot (\lstinline|*l <= p|).
The two elements found are in the wrong order so they are swapped, and the pointers move onwards.
This process continues until the pointers meet.
Finally, the pivot is swapped with the first element of the right partition.

Only an explicit check for whether the pointers have met after stopping is needed.
Since the elements of the partitions to the left are at most as great as the elements of the current partition, they naturally act as bounds check for the pointer moving rightwards.
The pivot at the end acts as bounds check for the pointer moving leftwards.
Since the leftmost partitions have no neighbour to the left, one explicit sentinel values set to the minimum possible value must be placed at the start of the input.
The downside to this approach is that elements equal to the pivot are also swapped.

\paragraph{Base Cases}

\begin{figure}
	\pgfplotstableset{
		create on use/n/.style={create col/copy column from table={data/quick/fallback/uint32/16.txt}{n}},
	}
	\pgfplotsinvokeforeach{14,15,16,17,18,19,20}{
		\pgfplotstableset{create on use/µ_#1_32/.style={create col/copy column from table={data/quick/fallback/uint32/#1.txt}{µ_TrivialBC}}}
		\pgfplotstableset{create on use/µ_#1_64/.style={create col/copy column from table={data/quick/fallback/uint64/#1.txt}{µ_TrivialBC}}}
	}
	\pgfplotstablenew[columns={n,µ_14_32,µ_15_32,µ_16_32,µ_17_32,µ_18_32,µ_19_32,µ_20_32,µ_14_64,µ_15_64,µ_16_64,µ_17_64,µ_18_64,µ_19_64,µ_20_64}]{\pgfplotstablegetrowsof{data/quick/fallback/uint32/16.txt}}{\tableQuickFallback}

	\tikzsetnextfilename{quick_fallback}
	\begin{tikzpicture}[plot]
		\begin{groupplot}[
			horizontal sep for labels,
			adaptive group=1 by 2,
			groupplot ylabel={Speed-up},
			x from 16 to 1024,
			ymin=0.993,
			ymax=1.001,
			extra y ticks={0.993,1.001},
			yticklabel style={/pgf/number format/.cd, precision=3, fixed, zerofill},
		]
			\nextgroupplot[title=32-bit\strut]
			\pgfplotsset{legend to name=leg:quick:fallback, legend entries={15,...,20}}
			\pgfplotsset{update limits=false} \addplot coordinates {(15,0.99)}; \pgfplotsset{update limits=true}
			\pgfplotsinvokeforeach{16,17,18,19,20}{
				\ifnumequal{#1}{18}{
					\addplot coordinates {(18,0.99)};
				}{
					\plotspeedup{#1_32}{18_32}{tableQuickFallback}
				}
			}
			%
			\nextgroupplot[title=64-bit\strut]
			\pgfplotsinvokeforeach{15,16,17,18,19}{
				\ifnumequal{#1}{17}{
					\addplot coordinates {(17,0.99)};
				}{
					\plotspeedup{#1_64}{17_64}{tableQuickFallback}
				}
			}
		\end{groupplot}
	\end{tikzpicture}

	\hfil\pgfplotslegendfromname{leg:quick:fallback}\hfil
	\caption{
		Speed-ups of \QS*{} with different thresholds (15--20) for when to fall back to \IS{} over a threshold of 18 elements (32-bit) and 17 elements (64-bit).
		Using \ShS{} was not beneficial overall, likely because many partitions fall below the thresholds.
	}
	\label{fig:quick:fallback}
\end{figure}

When only a few elements remain in a partition, \QS{}'s overhead predominates such that \IS{} lends itself as fallback algorithm.
As seen in \cref{fig:quick:fallback}, the optimal threshold for switching the sorting algorithm is 18 elements for 32-bit integers on uniform inputs and likely similar on inputs following Zipf's or normal distributions.
For 64-bit integers, the optimal threshold is 17 elements, but we set 18 elements to be the default threshold for both data types to simplify matters since the impact is minuscule.
Up to 40\% of the runtime is saved compared to a \QS{} never falling back.
For sorted and almost sorted inputs, the threshold is higher since \IS{} is very fast on them so falling back earlier and, thus, ending the sorting process is better.
The same is true for reverse sorted inputs even though these are the worst-case inputs for \IS{} because \QS{}'s two pointers invert large swaths of the input.
However, these input distributions should be catered for by a pattern-defeating \QS{} as laid out in \cref{sec:tasklet:conclusion}, hence the 18 elements as default threshold.

To avoid unnecessary uses of \IS{}, another base case is imaginable, namely terminating when a partition contains at most 1 elements.
There are tremendous consequences for the runtime depending on the exact implementation of the base cases, as shown later in \enquote{\nameref{sec:tasklet:quick:compilation}}.


\paragraph{Recursion vs.\ Iteration}
In theory, the question of whether a DPU algorithm should be implemented recursively or iteratively comes down to convenience.
Due to the uniform costs of instructions, jumping to the start of a loop or to the start of a function essentially costs the same, as does managing arguments automatically through the regular call stack and manually through a simulated one.
Furthermore, in case of \QS{}, the compiler turns tail-recursive calls into jumps back to the function start, so that one partition is sorted recursively and the other iteratively.
All this would suggest a recursive implementation due to the reduced maintenance.

In practice, it comes down to the compilation.
Even parts of the algorithms which are independent from the choice between recursion and iteration can be compiled differently, such that there are implementations where iteration is faster than recursion and the other way around.
Overall though, iterative implementations \emph{tend} to be compiled better with superior register usage and less instructions used for the actual \QS{} algorithm.


\paragraph{Partition Prioritisation}
Whether the left-hand or the right-hand partition is sorted first should not make any difference for the runtime but actually does so because of different compilation, as shown later in \enquote{\nameref{sec:tasklet:quick:compilation}}.
Always sorting the shorter partition first and putting the longer partition on the call stack guarantees that the problem size is at least halved each step, so that the call stack stores \(\bigoh{\log n}\) elements at most.
This approach, however, is linked to huge speed penalties, which is why it is advisable to always prioritise the same side;
in this Thesis, the right-hand partitions are prioritised.
An overflow of the call stack becomes unlikely with the right pivot choice.


\paragraph{Pivot Choice}
Another parameter to tune is the way in which the pivot is chosen.
The following were implemented and tested:
\begin{itemize}
	\item
	Using the \emph{last element} is the fastest way, requiring zero additional instructions.

	\item
	Taking the \emph{deterministic median} of three elements, namely the first, middle, and last one, is far more computationally expensive since the position of the middle element must be calculated, the median be determined, and the pivot be swapped with the last element of the array, where it acts as sentinel.

	\item
	A \emph{random element} is most efficiently drawn using an xorshift random number generator and rejection sampling \cite{lukas_geis}.

	\item
	The \emph{random median} is a combination of the previous two methods, where the median of three random elements is taken.
	For simplicity, there is no check on whether an element is drawn twice or thrice.
	Since the partitions are rather long, this should happen seldom, anyhow.
\end{itemize}
A median increases the chances of choosing a pivot that is neither particularly high nor particularly low.
This leads to more balanced partitions such that the call stack is less likely to overflow and the base cases are reached faster.
But even then it is still possible to construct inputs where the runtime climbs up to \(\bigtheta{n^2}\) \cite{erkiö1984worstcase}, as everything is moved to the same partition so that the problem size is reduced by only one element (namely the pivot) after each partitioning step.

The random pivots circumvent this problem.
Whilst the pivots could, by ill luck, also lead to the same unbalanced partitions as the deterministic pivots, the worst-case expected runtime is \(\bigoh{n \log n}\) \cite{blum2011probabilistic}.
Using the median of medians \cite{blum1973median} could guarantee a runtime of \(\bigoh{n \log n}\) but was not implemented because a performant implementation would probably be quite complex and its benefit minuscule for this Thesis.

The general trend, as seen in \enquote{\nameref{sec:tasklet:quick:compilation}}, is the following:
A median gets more beneficial for the average runtime, the longer the input becomes, and leads to small pay-offs in the end.
Moreover, the standard deviations of the runtimes are cut roughly in half, although not shown in the figures of \enquote{\nameref{sec:tasklet:quick:compilation}} for reasons of clarity.
If the input is known to be fairly random, a deterministic choice yields a noticeably speed-up.
However, the gain remains in the single digits percentage-wise, supporting the findings by \citeauthor{lukas_geis}~\cite{lukas_geis} that drawing random numbers is quite cheap.
For this reason, the median of three random elements is used as default configuration throughout this Thesis.


\subsubsection*{Investigation of the Compilation}
\label{subsubsec:tasklet:quick:compilation}

%\pgfplotstablereadnamed{data/quick/rec_vs_it.txt}{tableQuickRecVsIter}
%\pgfplotstablereadnamed{data/quick/recursive/no switched sides/uniform/end.txt}{tableQuickRecNssUniEnd}
%\pgfplotstablereadnamed{data/quick/recursive/no switched sides/uniform/middle.txt}{tableQuickRecNssUniMiddle}
%\pgfplotstablereadnamed{data/quick/recursive/no switched sides/uniform/median_of_three.txt}{tableQuickRecNssUniMedian}
%\pgfplotstablereadnamed{data/quick/recursive/no switched sides/uniform/random.txt}{tableQuickRecNssUniRandom}
%\pgfplotstablereadnamed{data/quick/recursive/switched sides/uniform/end.txt}{tableQuickRecSsUniEnd}
%\pgfplotstablereadnamed{data/quick/recursive/switched sides/uniform/middle.txt}{tableQuickRecSsUniMiddle}
%\pgfplotstablereadnamed{data/quick/recursive/switched sides/uniform/median_of_three.txt}{tableQuickRecSsUniMedian}
%\pgfplotstablereadnamed{data/quick/recursive/switched sides/uniform/random.txt}{tableQuickRecSsUniRandom}
%\pgfplotstablereadnamed{data/quick/iterative/no switched sides/uniform/end.txt}{tableQuickIterNssUniEnd}
%\pgfplotstablereadnamed{data/quick/iterative/no switched sides/uniform/middle.txt}{tableQuickIterNssUniMiddle}
%\pgfplotstablereadnamed{data/quick/iterative/no switched sides/uniform/median_of_three.txt}{tableQuickIterNssUniMedian}
%\pgfplotstablereadnamed{data/quick/iterative/no switched sides/uniform/random.txt}{tableQuickIterNssUniRandom}
%\pgfplotstablereadnamed{data/quick/iterative/switched sides/uniform/end.txt}{tableQuickIterSsUniEnd}
%\pgfplotstablereadnamed{data/quick/iterative/switched sides/uniform/middle.txt}{tableQuickIterSsUniMiddle}
%\pgfplotstablereadnamed{data/quick/iterative/switched sides/uniform/median_of_three.txt}{tableQuickIterSsUniMedian}
%\pgfplotstablereadnamed{data/quick/iterative/switched sides/uniform/random.txt}{tableQuickIterSsUniRandom}

The quality of the compilation and thus the real performance of \QS{} is erratic to such an extent that one implementation variant may see a speed-up of 25\% over another one even with the same pivot choice although virtually none would be expected.
As hinted in the preceding paragraphs, this raises the need for a benchmark suite with the following parameters:
base case handling, recursion/iteration, pivot choice, and partition prioritisation.
Before the results are discussed, the first parameter shall be explained in more depth.

Besides falling back to \IS{} if 13 elements remain (\enquote{treshold undercut}), another base case is imaginable, namely a full termination if 1, 0, or --1 elements remain (\enquote{trivial length}).
Theoretically, it should not be needed to check for trivial lengths because even though it is doable with just one additional instruction, such short partitions are rare and the \IS{} would terminate after a few instructions anyway.
Nonetheless, its inclusion or exclusion can have significant impacts.
The following \nameCrefs{imp:normal} were tested:
\begin{enumerate}
	\item\label[implementation]{imp:normal}
	If the length is trivial, terminate.
	If not and if the threshold is undercut, sort with \IS{}.
	Otherwise, sort with \QS{} and use \QS{} on both partitions.
%	\textcolor{red}{[Normal]}

	\item\label[implementation]{imp:triviality_within_threshold}
	If the threshold is undercut, check if the length is trivial and terminate or sort with \IS{}, respectively.
	Otherwise, sort with \QS{} and use \QS{} on both partitions.
%	\textcolor{red}{[TrivInThresh]}
	\begin{itemize}
		\item
		This \nameCref{imp:triviality_within_threshold} significantly reduces the number of checks for trivial length.
	\end{itemize}

	\item\label[implementation]{imp:no_triviality}
	If the threshold is undercut, sort with \IS{}.
	Otherwise, sort with \QS{} and use \QS{} on both partitions.
%	\textcolor{red}{[NoTrivial]}
	\begin{itemize}
		\item
		This \nameCref{imp:no_triviality} forgoes the check for a trivial length completely, at the cost of unneeded \IS*{}.
	\end{itemize}

	\item\label[implementation]{imp:threshold_then_triviality}
	If the threshold is undercut, sort with \IS{}.
	If not and if the length is trivial, terminate.
	Otherwise, sort with \QS{} and use \QS{} on both partitions.
%	\textcolor{red}{[ThreshThenTriv]}
	\begin{itemize}
		\item
		This \nameCref{imp:threshold_then_triviality}, while nonsensical from a logical point of view, gives the compiler an explicit guarantee that the partitioning loop does not end immediately.
	\end{itemize}

	\item\label[implementation]{imp:triviality_before_call}
	If the threshold is undercut, sort with \IS{}.
	Otherwise, sort with \QS{}.
	Then check for either partition if its length is trivial and use \QS{} if not.
%	\textcolor{red}{[TrivialBC]}
	\begin{itemize}
		\item
		This \nameCref{imp:triviality_before_call}, as well as the next two, gets rid of some unneeded uses of \QS{}.
		In the recursive case, these \nameCrefs{imp:triviality_before_call} lose the property of being tail-recursive.
	\end{itemize}

	\item\label[implementation]{imp:threshold_before_call}
	Sort with \QS{}.
	Check for either partition if the threshold is undercut and use \IS{} or \QS{}, respectively.
%	\textcolor{red}{[ThreshBC]}

	\item\label[implementation]{imp:threshold_and_triviality_before_call}
	Sort with \QS{}.
	Check for either partition if its length is trivial or if the threshold is undercut and use \IS{}, \QS{}, or nothing, respectively.
%	\textcolor{red}{[ThreshTrivBC]}

	\item\label[implementation]{imp:one_insertion}
	If the threshold is undercut, terminate.
	Otherwise, sort with \QS{} and use \QS{} on both partitions.
	After all \QS*{} are done, sort the whole input array with \IS{}.
%	\textcolor{red}{[OneInsertion]}
	\begin{itemize}
		\item
		This \nameCref{imp:one_insertion} always does one pass of \IS{}.
		For example, the other \nameCrefs{imp:normal} do roughly 90 at 1024 elements.
	\end{itemize}
\end{enumerate}

All results are shown in \cref{fig:quick:implementations}.
When using recursion, \cref{imp:normal,imp:triviality_before_call} perform the best, especially for longer inputs.
Their compilations are fundamentally the same, including the conversion of the second recursive call into a jump back to the function start.
All other \nameCrefs{imp:normal} fare vastly worse.
Common occurrences are \dots{}
\begin{itemize}
	\item
	\dots{} one more instruction in the loop finding the next element to move to the right, \dots{}

	\item
	\dots{} one more instruction after such an element has been found, \dots{}

	\item
	\dots{} more stores and loads when entering and leaving the function.
\end{itemize}

\todo[inline]{%
	Ich kann mir leider nicht alles erklären.
	Als Beispiel habe ich die Kompilate von \cref{imp:normal} / Recursive / Last für Left First und Right First hochgeladen (die verkürzten Varianten besitzen nur noch Befehle und Sprungmarken).
	Ersteres ist ja die schnellste rekursive Variante, während letzteres deutlich schlechter abschneidet.
	Dennoch sehe ich bei der langsameren Variante keinen fundamental anderen Algorithmus.
	Je Funktionsaufruf kommen ≈3 Extra-Aufrufe hinzu (bei insgesamt ≈104 rekursiven und ≈104 \enquote{Endaufrufen} bei 1024 Elementen), was fast 70\,000 Takte Unterschied nicht erklären kann.
}

%\begin{figure}[p]
%	\def\algos{Normal,TrivInThresh,NoTrivial,ThreshThenTriv,TrivialBC,ThreshBC,ThreshTrivBC,OneInsertion}
%	\pgfplotsset{
%		height=3.5cm,
%		horizontal sep for naught,
%		adaptive group=2 by 4,
%		groupplot xlabel={Input Length \(n\)},
%		groupplot ylabel={Cycles / \((n \lb n)\)},
%		xmode=log,
%		xtick={20, 64, 256, 1024},
%		xticklabels={\(20\), \(64\), \(256\), \(1024\)},
%		minor xtick={32, 128, 512},
%		ymin=55,
%		ymax=80,
%		ytick distance=5,
%		legend columns=-1,
%	}
%	\begin{subfigure}{\textwidth}
%		\tikzsetnextfilename{quick_implementations_rec}
%		\begin{tikzpicture}[plot]
%			\begin{groupplot}
%				\nextgroupplot[title={Last | Left First}, legend to name=leg:quick:implementations]
%				\legend{\ref{imp:normal}, \ref{imp:triviality_within_threshold}, \ref{imp:no_triviality}, \ref{imp:threshold_then_triviality}, \ref{imp:triviality_before_call}, \ref{imp:threshold_before_call}, \ref{imp:threshold_and_triviality_before_call}, \ref{imp:one_insertion}}
%				\expandafter\pgfplotsinvokeforeach\expandafter{\algos}{
%					\plotpernlogn{#1}{tableQuickRecNssUniEnd}
%				}
%				%
%				\nextgroupplot[title={Middle | Left First}, yticklabels={}]
%				\expandafter\pgfplotsinvokeforeach\expandafter{\algos}{
%					\plotpernlogn{#1}{tableQuickRecNssUniMiddle}
%				}
%				%
%				\nextgroupplot[title={Median | Left First}, yticklabels={}]
%				\expandafter\pgfplotsinvokeforeach\expandafter{\algos}{
%					\plotpernlogn{#1}{tableQuickRecNssUniMedian}
%				}
%				%
%				\nextgroupplot[title={Random | Left First}, yticklabel pos=right]
%				\expandafter\pgfplotsinvokeforeach\expandafter{\algos}{
%					\plotpernlogn{#1}{tableQuickRecNssUniRandom}
%				}
%				%
%				\nextgroupplot[title={Last | Right First}]
%				\expandafter\pgfplotsinvokeforeach\expandafter{\algos}{
%					\plotpernlogn{#1}{tableQuickRecSsUniEnd}
%				}
%				%
%				\nextgroupplot[title={Middle | Right First}, yticklabels={}]
%				\expandafter\pgfplotsinvokeforeach\expandafter{\algos}{
%					\plotpernlogn{#1}{tableQuickRecSsUniMiddle}
%				}
%				%
%				\nextgroupplot[title={Median | Right First}, yticklabels={}]
%				\expandafter\pgfplotsinvokeforeach\expandafter{\algos}{
%					\plotpernlogn{#1}{tableQuickRecSsUniMedian}
%				}
%				%
%				\nextgroupplot[title={Random | Right First}, yticklabel pos=right]
%				\expandafter\pgfplotsinvokeforeach\expandafter{\algos}{
%					\plotpernlogn{#1}{tableQuickRecSsUniRandom}
%				}
%			\end{groupplot}
%		\end{tikzpicture}
%		\caption{
%			Recursive Approach
%		}
%		\bigskip
%	\end{subfigure}
%	%
%	\begin{subfigure}{\textwidth}
%		\tikzsetnextfilename{quick_implementations_it}
%		\begin{tikzpicture}[plot]
%			\begin{groupplot}
%				\nextgroupplot[title={Last | Left First}]
%				\expandafter\pgfplotsinvokeforeach\expandafter{\algos}{
%					\plotpernlogn{#1}{tableQuickIterNssUniEnd}
%				}
%				%
%				\nextgroupplot[title={Middle | Left First}, yticklabels={}]
%				\expandafter\pgfplotsinvokeforeach\expandafter{\algos}{
%					\plotpernlogn{#1}{tableQuickIterNssUniMiddle}
%				}
%				%
%				\nextgroupplot[title={Median | Left First}, yticklabels={}]
%				\expandafter\pgfplotsinvokeforeach\expandafter{\algos}{
%					\plotpernlogn{#1}{tableQuickIterNssUniMedian}
%				}
%				%
%				\nextgroupplot[title={Random | Left First}, yticklabel pos=right]
%				\expandafter\pgfplotsinvokeforeach\expandafter{\algos}{
%					\plotpernlogn{#1}{tableQuickIterNssUniRandom}
%				}
%				%
%				\nextgroupplot[title={Last | Right First}]
%				\expandafter\pgfplotsinvokeforeach\expandafter{\algos}{
%					\plotpernlogn{#1}{tableQuickIterSsUniEnd}
%				}
%				%
%				\nextgroupplot[title={Middle | Right First}, yticklabels={}]
%				\expandafter\pgfplotsinvokeforeach\expandafter{\algos}{
%					\plotpernlogn{#1}{tableQuickIterSsUniMiddle}
%				}
%				%
%				\nextgroupplot[title={Median | Right First}, yticklabels={}]
%				\expandafter\pgfplotsinvokeforeach\expandafter{\algos}{
%					\plotpernlogn{#1}{tableQuickIterSsUniMedian}
%				}
%				%
%				\nextgroupplot[title={Random | Right First}, yticklabel pos=right]
%				\expandafter\pgfplotsinvokeforeach\expandafter{\algos}{
%					\plotpernlogn{#1}{tableQuickIterSsUniRandom}
%				}
%			\end{groupplot}
%		\end{tikzpicture}
%		\caption{
%			Iterative approach
%		}
%	\end{subfigure}
%
%	\bigskip
%	\hfil\pgfplotslegendfromname{leg:quick:implementations}\hfil
%	\caption{
%		Comparison of the different implementations (1--8) of \QS{} for all possible pivot choices.
%		In the first rows, the left-hand partitions are sorted before the right-hand ones, while it is the reverse in the second rows.
%	}
%	\label{fig:quick:implementations}
%\end{figure}
%
%\begin{figure}
%	\tikzsetnextfilename{quick_rec_vs_it}
%	\begin{tikzpicture}[plot]
%		\begin{groupplot}[
%			horizontal sep for labels,
%			adaptive group=1 by 2,
%			groupplot xlabel={Input Length \(n\)},
%			xmode=log,
%			xtick={20, 32, 64, 128, 256, 512, 1024},
%			xticklabels={\(20\), \(32\), \(64\), \(128\), \(256\), \(512\), \(1024\)},
%			legend columns=-1,
%		]
%			\nextgroupplot[ylabel=Cycles / \((n \lb n)\), ymin=55, ymax=65, extra y ticks={55, 65}, legend to name=leg:rec_vs_it]
%			\legend{Iterative, Recursive}
%			\plotpernlogn{It}{tableQuickRecVsIter}
%			\plotpernlogn{Rec}{tableQuickRecVsIter}
%			%
%			\nextgroupplot[ylabel=Speed-up, ymin=0.96, ymax=1.04, /pgf/number format/.cd, precision=2, fixed zerofill=true]
%			\plotspeedup{It}{Rec}{tableQuickRecVsIter}
%		\end{groupplot}
%	\end{tikzpicture}
%
%	\hfil\pgfplotslegendfromname{leg:rec_vs_it}\hfil
%	\caption{
%		Comparison of the fastest recursive and iterative \QS*{} (cf. \cref{subsubsec:tasklet:quick:compiler}).
%		The actual algorithm is compiled the very same in both cases, so that time differences are only due to the way \QS{} is applied to the partitions.
%		\todo{nicht mehr wegen des Pivots!}
%	}
%	\label{fig:rec_vs_it}
%\end{figure}

\clearpage

\def\quickpivots{LAST,MEDIAN,RANDOM,MEDIAN_OF_RANDOM}
\expandafter\pgfplotsinvokeforeach\expandafter{\quickpivots}{
	\pgfplotstablereadnamed{data/quick/matrix/iterative/#1/shorter/uniform.txt}{tableQuickMatrixIt#1Shorter}
	\pgfplotstablereadnamed{data/quick/matrix/iterative/#1/left/uniform.txt}{tableQuickMatrixIt#1Left}
	\pgfplotstablereadnamed{data/quick/matrix/iterative/#1/right/uniform.txt}{tableQuickMatrixIt#1Right}

	\pgfplotstablereadnamed{data/quick/matrix/recursive/#1/shorter/uniform.txt}{tableQuickMatrixRec#1Shorter}
	\pgfplotstablereadnamed{data/quick/matrix/recursive/#1/left/uniform.txt}{tableQuickMatrixRec#1Left}
	\pgfplotstablereadnamed{data/quick/matrix/recursive/#1/right/uniform.txt}{tableQuickMatrixRec#1Right}
}

\tikzexternaldisable

\begin{figure}[p]
	\pgfplotsset{
		height=2.6cm,
		horizontal sep for naught,
		vertical sep for naught,
		adaptive group=3 by 4,
		groupplot xlabel={Input Length \(n\)},
		groupplot ylabel={Cycles / \((n \lb n)\)},
		xmode=log,
		xtick={16, 64, 256, 1024},
		xticklabels={\(16\), \(64\), \(256\), \(1024\)},
		minor xtick={32, 128, 512},
		ymin=55,
		ymax=80,
		legend columns=-1,
	}
	\def\quickalgos{Normal,TrivInThresh,NoTrivial,ThreshThenTriv,TrivialBC,ThreshBC,ThreshTrivBC,OneInsertion}
	\captionsetup[subfigure]{aboveskip=0mm,belowskip=1mm}
	\begin{subfigure}{\textwidth}
		\tikzsetnextfilename{quick_implementations_rec}
		\begin{tikzpicture}[plot]
			\begin{groupplot}
				\nextgroupplot[title=Last, xticklabels={}, legend to name=leg:quick:implementations]
				\legend{\ref{imp:normal}, \ref{imp:triviality_within_threshold}, \ref{imp:no_triviality}, \ref{imp:threshold_then_triviality}, \ref{imp:triviality_before_call}, \ref{imp:threshold_before_call}, \ref{imp:threshold_and_triviality_before_call}, \ref{imp:one_insertion}}
				\expandafter\pgfplotsinvokeforeach\expandafter{\quickalgos}{
					\plotpernlogn{#1}{tableQuickMatrixRecLASTLeft}
				}
				\nextgroupplot[title=Median, xticklabels={}, yticklabels={}]
				\expandafter\pgfplotsinvokeforeach\expandafter{\quickalgos}{
					\plotpernlogn{#1}{tableQuickMatrixRecMEDIANLeft}
				}
				\nextgroupplot[title=Random, xticklabels={}, yticklabels={}]
				\expandafter\pgfplotsinvokeforeach\expandafter{\quickalgos}{
					\plotpernlogn{#1}{tableQuickMatrixRecRANDOMLeft}
				}
				\nextgroupplot[title=Median (Random), xticklabels={}, yticklabel pos=right]
				\expandafter\pgfplotsinvokeforeach\expandafter{\quickalgos}{
					\plotpernlogn{#1}{tableQuickMatrixRecMEDIAN_OF_RANDOMLeft}
				}
				%
				\nextgroupplot[xticklabels={}]
				\expandafter\pgfplotsinvokeforeach\expandafter{\quickalgos}{
					\plotpernlogn{#1}{tableQuickMatrixRecLASTRight}
				}
				\nextgroupplot[xticklabels={}, yticklabels={}]
				\expandafter\pgfplotsinvokeforeach\expandafter{\quickalgos}{
					\plotpernlogn{#1}{tableQuickMatrixRecMEDIANRight}
				}
				\nextgroupplot[xticklabels={}, yticklabels={}]
				\expandafter\pgfplotsinvokeforeach\expandafter{\quickalgos}{
					\plotpernlogn{#1}{tableQuickMatrixRecRANDOMRight}
				}
				\nextgroupplot[xticklabels={}, yticklabel pos=right]
				\expandafter\pgfplotsinvokeforeach\expandafter{\quickalgos}{
					\plotpernlogn{#1}{tableQuickMatrixRecMEDIAN_OF_RANDOMRight}
				}
				%
				\nextgroupplot
				\expandafter\pgfplotsinvokeforeach\expandafter{\quickalgos}{
					\plotpernlogn{#1}{tableQuickMatrixRecLASTShorter}
				}
				\nextgroupplot[yticklabels={}]
				\expandafter\pgfplotsinvokeforeach\expandafter{\quickalgos}{
					\plotpernlogn{#1}{tableQuickMatrixRecMEDIANShorter}
				}
				\nextgroupplot[yticklabels={}]
				\expandafter\pgfplotsinvokeforeach\expandafter{\quickalgos}{
					\plotpernlogn{#1}{tableQuickMatrixRecRANDOMShorter}
				}
				\nextgroupplot[yticklabel pos=right]
				\expandafter\pgfplotsinvokeforeach\expandafter{\quickalgos}{
					\plotpernlogn{#1}{tableQuickMatrixRecMEDIAN_OF_RANDOMShorter}
				}
			\end{groupplot}
		\end{tikzpicture}
		\caption{
			Recursive Approach
		}
	\end{subfigure}
	\begin{subfigure}{\textwidth}
		\tikzsetnextfilename{quick_implementations_it}
		\begin{tikzpicture}[plot]
			\begin{groupplot}
				\nextgroupplot[title=Last, xticklabels={}]
				\expandafter\pgfplotsinvokeforeach\expandafter{\quickalgos}{
					\plotpernlogn{#1}{tableQuickMatrixItLASTLeft}
				}
				\nextgroupplot[title=Median, xticklabels={}, yticklabels={}]
				\expandafter\pgfplotsinvokeforeach\expandafter{\quickalgos}{
					\plotpernlogn{#1}{tableQuickMatrixItMEDIANLeft}
				}
				\nextgroupplot[title=Random, xticklabels={}, yticklabels={}]
				\expandafter\pgfplotsinvokeforeach\expandafter{\quickalgos}{
					\plotpernlogn{#1}{tableQuickMatrixItRANDOMLeft}
				}
				\nextgroupplot[title=Median (Random), xticklabels={}, yticklabel pos=right]
				\expandafter\pgfplotsinvokeforeach\expandafter{\quickalgos}{
					\plotpernlogn{#1}{tableQuickMatrixItMEDIAN_OF_RANDOMLeft}
				}
				%
				\nextgroupplot[xticklabels={}]
				\expandafter\pgfplotsinvokeforeach\expandafter{\quickalgos}{
					\plotpernlogn{#1}{tableQuickMatrixItLASTRight}
				}
				\nextgroupplot[xticklabels={}, yticklabels={}]
				\expandafter\pgfplotsinvokeforeach\expandafter{\quickalgos}{
					\plotpernlogn{#1}{tableQuickMatrixItMEDIANRight}
				}
				\nextgroupplot[xticklabels={}, yticklabels={}]
				\expandafter\pgfplotsinvokeforeach\expandafter{\quickalgos}{
					\plotpernlogn{#1}{tableQuickMatrixItRANDOMRight}
				}
				\nextgroupplot[xticklabels={}, yticklabel pos=right]
				\expandafter\pgfplotsinvokeforeach\expandafter{\quickalgos}{
					\plotpernlogn{#1}{tableQuickMatrixItMEDIAN_OF_RANDOMRight}
				}
				%
				\nextgroupplot
				\expandafter\pgfplotsinvokeforeach\expandafter{\quickalgos}{
					\plotpernlogn{#1}{tableQuickMatrixItLASTShorter}
				}
				\nextgroupplot[yticklabels={}]
				\expandafter\pgfplotsinvokeforeach\expandafter{\quickalgos}{
					\plotpernlogn{#1}{tableQuickMatrixItMEDIANShorter}
				}
				\nextgroupplot[yticklabels={}]
				\expandafter\pgfplotsinvokeforeach\expandafter{\quickalgos}{
					\plotpernlogn{#1}{tableQuickMatrixItRANDOMShorter}
				}
				\nextgroupplot[yticklabel pos=right]
				\expandafter\pgfplotsinvokeforeach\expandafter{\quickalgos}{
					\plotpernlogn{#1}{tableQuickMatrixItMEDIAN_OF_RANDOMShorter}
				}
			\end{groupplot}
		\end{tikzpicture}
		\caption{
			Iterative Approach
		}
	\end{subfigure}

	\hfil\pgfplotslegendfromname{leg:quick:implementations}\hfil
	\caption{
		Comparison of \crefrange{imp:normal}{imp:one_insertion}.
		The left-hand partition is preferred in the first rows, the right-hand one in the second rows, and the shorter one in the third rows.
	}
	\label{fig:quick:implementations}
\end{figure}

\tikzexternalenable

\clearpage

\subsection*{Evaluation of the Performance}
\label{sec:tasklet:quick:performance}
\addcontentsline{toc}{subsection}{\nameref{sec:tasklet:quick:performance}}

\pgfplotsinvokeforeach{sorted,reverse,almost,uniform,zipf,normal}{
	\pgfplotstablereadnamed{data/quick/matrix/iterative/Median_of_random/right/uint32/#1.txt}{tableQuickRand_32#1}
	\pgfplotstablereadnamed{data/quick/matrix/recursive/Median_of_random/right/uint64/#1.txt}{tableQuickRand_64#1}
}

Before turning to the performance of \QS{} on specific input distributions, the ratio between costs and benefits of the pivot choices shall be evaluated.
Looking again at \cref{fig:quick:implementations,fig:heap:runtime_uint64} shows that a median gets more beneficial, the longer the input becomes, achieving small pay-offs for the longest ones.
Moreover, the standard deviations of the runtimes, although not shown in the figures for reasons of clarity, are cut roughly in half.
Randomisation slows down noticeably, so random pivots are disadvantageous if the input is known to be fairly random.
However, the decrease remains in the single digits percentage-wise, supporting the findings by \citeauthor{lukas_geis}~\cite{lukas_geis} that drawing random numbers is quite cheap.
For this reason, the random median is used as default method throughout this thesis.

\begin{figure}
	\tikzsetnextfilename{quick_runtime}
	\begin{tikzpicture}[plot]
		\begin{groupplot}[
			adaptive group=1 by 2,
			groupplot ylabel={Cycles / \((n \lb n)\)},
			x from 16 to 1024,
			ytick distance=10,
		]
			\nextgroupplot[title/.add={}{32-bit}, ymin=30, ymax=80]
			\pgfplotsset{legend to name=leg:quick:runtime, legend entries={Sorted, Reverse S., Almost S., Uniform, Zipf's, Normal}}
			\pgfplotsinvokeforeach{sorted,reverse,almost,uniform,zipf,normal}{
				\plotpernlogn{TrivialBC}{tableQuickRand_32#1}
			}
			%
			\nextgroupplot[title/.add={}{64-bit}, ymin=40,ymax=90]
			\pgfplotsinvokeforeach{sorted,reverse,almost,uniform,zipf,normal}{
				\plotpernlogn{TrivialBC}{tableQuickRand_64#1}
			}
		\end{groupplot}
	\end{tikzpicture}

	\hfil\pgfplotslegendfromname{leg:quick:runtime}\hfil
	\caption{
		Mean runtime of \QS{} on all tested input distributions and data types.
	}
	\label{fig:quick:runtime}
\end{figure}

\Cref{fig:quick:runtime} shows the runtime of \QS{} in it default configuration, that is, with random medians.
\Cref{fig:quick:runtime_uint32,fig:quick:runtime_uint64} additionally contain the runtimes with deterministic medians as well as the standard deviations of the measurements.
The mean runtimes are rather close across all input distributions, a consequence of using random medians and of considering elements equal to the pivot as different.
In fact, it is \IS{} that primarily causes the discrepancies, as setting the threshold to one element proves.
This also explains why \QS{} performs so well on large inputs with Zipf's distribution:
This distributions generates many duplicates, which are put into the same partitions, so \IS{} performs many simple scans.

One might expect \QS{} to perform even better on sorted and reverse sorted input, since everything is either already in the correct position or because the two pointers quickly invert large swaths of the inputs.
However, a side effect of swapping the pivot twice can be that many elements are displaced by one position from where they should be in the sorted input.
Take reverse sorted inputs and the deterministic median as an example:
The element \(n/2\) is chosen as pivot out of the elements \(n\), \(n/2\), and \(0\) and then gets swapped with the last element, that is, with \(0\).
Thereupon, the pointers invert the rest of the input such that the start of the input looks something like \(1, 2, \dots, n/2-1, 0, n/2, \dots\) after the first partitioning step.
Indeed, this pattern makes \QS{} with deterministic medians degrade and eventually overflow the call stack, which is why the respective plots in \cref{fig:quick:runtime_uint32,fig:quick:runtime_uint64} leave the charts.
An implementation without swapping the pivot promises better performance for such cases, but in exploratory ones, the performance on more random input distributions suffered drastically.

