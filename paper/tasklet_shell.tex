\subsection{\texorpdfstring{\ShS{}}{ShellSort}}
\label{subsec:tasklet:shell}

\IS{} suffers from small elements at the end of the input, since those have to be brought to the front through \(\bigoh{n}\) comparisons and swaps.
\ShS{}, proposed by Donald L. Shell in 1959~\cite{Shell1959AHS}, gets around this by doing multiple passes of \IS{} with different step sizes:
In pass~\(p\) with step size \(\stepsizes_p\), the input array is divided into the subarrays of indices \((i, \stepsizes_p + i, 2 \stepsizes_p + i, \dots)\) for \(i = 0, \dots, \stepsizes_p - 1\) which then get sorted individually through \IS{}.
The step sizes get smaller each pass, with the final step size being \(1\) such that a regular \IS{} is performed.
Intuitively, the individual \IS*{} are fast since elements which need to travel long distances do big jumps.
Finding the right balance between the heightened overhead through multiple \IS{} passes and the shortened runtime of each \IS{} pass is subject to research to this day \cite{skean2023optimization,lee2021empirically} and depends on the cost of the operation types (comparing, swapping, looping).

\input{singletasklet_shell_performance}
