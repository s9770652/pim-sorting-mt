\chapter[Sorting in the \texorpdfstring{\abb{MRAM}}{MRAM}]{Sorting in the \acs*{MRAM}}
\label{sec:mram}

This \lcnamecref{sec:mram} is concerned with sorting data which resides in the \ac{MRAM}.
Both a sequential sorting algorithm for execution by a single tasklet as well as a parallel sorting algorithm for execution by a whole \ac{DPU} are presented.
We restrict ourselves to varieties of \MS{} since it allow for stable sorting and, also, the performance of \MS{} proved to be competitive in sorting \ac{WRAM} data in \cref{sec:tasklet}.
Next to the uniform cost of instructions, key characteristics of the \ac{DPU} architecture which are of especial importance in this \lcnamecref{sec:mram} are the two-tier memory hierarchy and inter-tasklet communication through shared memory.

A challenge in designing an algorithm for \ac{MRAM} data is the management of data transfers between the large \ac{MRAM} and the small \ac{WRAM}.
A key aspect of the transfer management is the use of the sequential reader, that is a software abstraction provided by \upmem{} which simplifies moving data from the \ac{MRAM} to the \ac{WRAM}.
\Cref{sec:mram:seq_reader} presents the advantages of the sequential reader and explains its usage and inner workings.
\Cref{sec:mram:triple} addresses the segmentation of the \ac{WRAM} into several buffers which are needed by both sequential readers and \MS{} itself.
\Cref{sec:mram:merge} deals with the sequential \ac{MRAM} \MS{}, shows how the sequential reader can be improved upon, and concludes with an analysis of the performance.
Finally, the parallel \ac{MRAM} \MS{} is built from the sequential one and analysed in \cref{sec:mram:par}.

In some instances, we employ \lstinline|T| to denote the data type of the input and \lstinline[keywords={}]|sizeof(T)| to denote the size of an element of type \lstinline|T| in bytes.
Every measurement presented in this \lcnamecref{sec:mram} was repeated ten times, which is a sufficient number given the large inputs and the low runtime variance of the \MS*{}.
\Cref{apx:mram} contains further measurements but the ones essential for following the content of this \lcnamecref{sec:mram} are also presented in figures herein.

\section{The Sequential Reader}
\label{sec:mram:seq_reader}

A \ac{DMA} takes half a cycle per transferred byte.
On top of that, each reading \ac{DMA} comes with an additional overhead of  \qty{77}{\cycles} and each writing \ac{DMA} with an overhead of \qty{61}{\cycles}.
To dilute this overhead, \ac{MRAM} data is preferably processed in blocks using the functions \lstinline|mram_read| and \lstinline|mram_write|.
The benefit of blockwise processing is so high that, according to \citeauthor{mutlu2022Benchmarking}~\cite[11]{mutlu2022Benchmarking}, using \lstinline|mram_read| and \lstinline|mram_write| remains beneficial compared to accessing single \ac{MRAM} elements even if only an eighth of the transferred data are actually of interest.
However, the functions \lstinline|mram_read| and \lstinline|mram_write| do come with some constraints.
Both the target and the source address must be aligned to 8 bytes.
Also, the number of transferred bytes must be at least 8, at most 2048, and a multiple of 8.
Furthermore, programmers are tasked with maintaining a buffer in the \ac{WRAM} and transferring data at appropriate times, which, given that the \ac{WRAM} is more than a thousand times smaller than the \ac{MRAM}, is likely frequent.

Since processing \ac{MRAM} data consecutively is a common occurrence, \upmem{} provides a data structure called \emph{sequential reader}.
Through a set of \langC{} functions, a sequential reader automates the read-in process and, thereby, removes any need to care for the alignment of addresses or the loading of new data.
On top of that, \upmem{} claims that \textquote{\bibellipsis{} this abstraction implementation has been optimized and will provide better performance than a standard \langC{} check of the cache boundaries.}~\cite[Memory management -- Sequential readers]{upmemSDK}

Of course, a sequential reader still requires a \ac{WRAM} buffer.
Its size in bytes is determined by the compile-time constant \seqreadcachesize{}, which can be set to either 32, 64, 128, 256, 512, or 1024.
The allocation of the buffer on the heap happens through the function \lstinline|seqread_alloc| and is worth a closer look.
Remember that the heap is actually implemented as a never-decreasing stack.
This means that new memory is only ever allocated behind the heap pointer, which stores the end of the heap.
With \(\seqreadcachesize{} = \twotoi\), the \(i\) least significant bits of the first byte address in the buffer are required to be zero, for reasons explained shortly.
Therefore, padding is introduced by having the heap pointer skip to the next higher multiple of \seqreadcachesize{} if not already on such a multiple.
Due to the nature of a stack, this has the drawback that the padding can never be allocated for something else.
After the skip of the heap pointer, a total of \(2 × \seqreadcachesize{}\) many bytes are allocated, also for reasons explained shortly.
All in all, the memory footprint of a sequential-read buffer is at least \(2 × \seqreadcachesize{}\) many bytes and less than \(3 × \seqreadcachesize{}\) many.

The function \lstinline|seqread_init| instructs a sequential reader to load data from a specified \ac{MRAM} address into its buffer.
Conceptually, the whole \ac{MRAM} is divided into \emph{pages} of size \seqreadcachesize{}.
To load data from the specified \ac{MRAM} address, the address is rounded down the the next multiple of \seqreadcachesize{}, which yields the beginning of the page containing the \ac{MRAM} address.
Then, 2 × \seqreadcachesize{} many bytes are loaded so that the buffer holds two pages.
This way, data of some long, compound type at the end of the first page are fully loaded even if extending into the other page.

The function \lstinline|seqread_init| also returns a pointer to the corresponding position of the specified \ac{MRAM} address within the sequential-read buffer, to which we will refer as pointer to the \emph{current element}.
Due to the page model, this pointer may not be set to the beginning of the buffer.
To access the current element, one simply dereferences the pointer.
Calling the function \lstinline|seqread_get| advances this pointer by a given number of bytes, which cannot be greater than \seqreadcachesize{};
this way of specifying bytes allows the sequential reader to support arbitrary data types.
Once the pointer to the current element ends up in the second half of the buffer, it is set \seqreadcachesize{} many bytes back so that it points to an address in the first half again.
In addition, the \ac{MRAM} address from which to read is increased by \seqreadcachesize{} many bytes, and the next two subsequent pages are loaded.
This means that the page stored in the second half of the buffer is loaded from the \ac{MRAM} again but stored in the first half this time.
\Cref{fig:merge:reader} visualises an intermediate state of a sequential reader, showcasing its characteristic read behaviour.

\begin{figure}
	\centering
	\tikzsetnextfilename{sequential_reader}
	\begin{tikzpicture}[sketch, curr/.style={accentcolor}]
		% MRAM.
		\def\mramlen{20}
		\def\mramstart{-2}
		\def\mramend{6}

		\fill[excess] (\mramstart, 0) rectangle (                0, 1);
		\fill[excess] (  \mramlen, 0) rectangle (\mramlen+\mramend, 1);
		\draw[excess] (\mramstart, 0)      grid (                0, 1);
		\draw[excess] (  \mramlen, 0)      grid (\mramlen+\mramend, 1);

		\draw (0, 0) grid (\mramlen,1);

		% Buffer.
		\def\buflen{8}
		\def\bufxoffset{10}
		\def\buffyoffset{-3}

		\fill[excess] (0.5*\buflen+\bufxoffset+\mramstart, \buffyoffset) rectangle (\buflen+\bufxoffset+\mramstart, 1+\buffyoffset);
		\draw[excess] (0.5*\buflen+\bufxoffset+\mramstart, \buffyoffset)      grid (\buflen+\bufxoffset+\mramstart, 1+\buffyoffset);

		\draw(\bufxoffset+\mramstart, \buffyoffset) grid (0.5*\buflen+\bufxoffset+\mramstart, 1+\buffyoffset);

		% Mapping lines.
		\draw[               name path=M2Bl] (1.5*\buflen+\mramstart, 0) to[out=-90, in=90] (            \bufxoffset+\mramstart, 1+\buffyoffset);
		\draw[line cap=butt, name path=M2Br] (2.5*\buflen+\mramstart, 0) to[out=-90, in=90] (    \buflen+\bufxoffset+\mramstart, 1+\buffyoffset);
		\draw[                       dashed] (2.0*\buflen+\mramstart, 0) to[out=-90, in=90] (0.5*\buflen+\bufxoffset+\mramstart, 1+\buffyoffset);
		\tikzfillbetween[of=M2Bl and M2Br, split=false] {gray, ultra nearly transparent}

		% Current item.
		\def\currindex{2}

		\fill[curr, fill opacity=.1]   (\currindex+\bufxoffset+\mramstart,      \buffyoffset) rectangle (1+\currindex+\bufxoffset+\mramstart, 1+\buffyoffset);
		\draw[curr, ultra thick, fill] (\currindex+\bufxoffset+\mramstart,      \buffyoffset)      grid (1+\currindex+\bufxoffset+\mramstart, 1+\buffyoffset);
		\draw[curr, flow, ->]          (\currindex+\bufxoffset+\mramstart, 1.25+\buffyoffset)       --+ (0:1);

		% Item identifiers.
		\def\mramfirst{10}

		\pgfmathsetmacro{\looplastindex}{int(\mramlen+\mramend-1)}
		\foreach \i in {\mramstart,...,\looplastindex}{
			\pgfmathsetmacro{\itemindex}{int( 4*(\mramfirst+\i) )}
			\node[address] at (\i+0.5, 0.5) { \hexa{\itemindex} };
		}

		\pgfmathsetmacro{\looplastindex}{int(\buflen-1)}
		\foreach \i in {0,...,\looplastindex}{
			\pgfmathsetmacro{\itemindex}{int( 4*(\mramfirst+1.5*\buflen+\mramstart+\i) )}
			\node[address] at (\i+\bufxoffset+\mramstart+0.5, \buffyoffset+0.5) { \hexa{\itemindex} };
		}

		% Pointers.
		\def\ptrdist{0.65}

		\draw[ptr, <-] (1.5*\buflen+\mramstart+0.5, 1) -- +(90:\ptrdist) node[above] {\lstinline|from|};

		\draw[ptr, <-] (\bufxoffset+\mramstart+0.5, \buffyoffset) -- +(-90:\ptrdist) node[below] {\lstinline|buffer|};

		\draw[ptr, <-] (\currindex+\bufxoffset+\mramstart+0.5, \buffyoffset) -- +(-90:\ptrdist) node[below] {\lstinline|curr|\vphantom{\lstinline|f|}};
	\end{tikzpicture}

	\caption{
		An exemplary sequential reader with \seqreadcachesize{} set to 16 being used to transfer 32-bit elements from the \ac{MRAM} (top row) into the sequential-read buffer (bottom row).
		The hexadecimal numbers denote the addresses of the respective elements within the \ac{MRAM}.
		Only the elements with addresses from \lstinline|0x28| to \lstinline|0x74| are sought to be read, however, the page model requires that the elements with addresses \lstinline|0x20|, \lstinline|0x24|, and \lstinline|0x78| to \lstinline|0x8c| are also loaded at some point.
		The pointer \lstinline|buffer| constantly points to the beginning of the sequential-read buffer.
		The pointer \lstinline|from| points to the \ac{MRAM} address of the first byte within the buffer.
		The pointer \lstinline|curr| moves from left to right, 4 bytes at a time.
		No byte of the second half of the buffer is ever read as the elements fit perfectly within the first half.
	}
	\label{fig:merge:reader}
\end{figure}

The acclaimed speedup through more performant bounds checks happens within the function \lstinline|seqread_get|, which in turn calls the function \lstinline|__builtin_dpu_seqread_get|.
An inspection of its compilation with \(\seqreadcachesize{} = \twotoi\) reveals the use of a combo operation.
The pointer to the current element is advanced by invoking an \lstinline|add| instruction to increase the stored address.
This \lstinline|add| instruction uses a condition to detect the generation of the \(i\)th carry bit.
To be more precise, this carry bit is set to \(\text{\lstinline|op1[i:0]|} + \text{\lstinline|op2[i:0]|}\), where \lstinline|op1[i:0]| and \lstinline|op2[i:0]| are the \(i + 1\) least significant bits of the involved operands, in this case the pointer and the number of bytes to advance.~\cite[DPU Handbook -- Specific Conditions Common To Addition and Subtraction]{upmemSDK}
Thanks to the carefully chosen size and alignment of the buffer, the generation of such a carry bit signifies that the pointer to the current element has left the first buffer half.
This means that it takes just one instruction to advance the pointer, check the buffer boundaries and jump over \Dash if needed \Dash the subsequent instructions responsible for updating the whole reader.


\section{The Triple Buffer}
\label{sec:mram:triple}

It is yet again beneficial to form starting runs.
This is done by loading a section of the MRAM into the WRAM, sorting it through one of the algorithms presented in \cref{sec:tasklet}, and writing the sorted section back to the MRAM.
As those sorting algorithm rely on the data being present entirely within the WRAM, the functions \lstinline|mram_read| and \lstinline|mram_write| are used directly.
Contrary to the WRAM \MS*{} with starting runs of length 14, the starting runs of the MRAM \MS{} are much longer, easily containing a thousand elements and more.
The reason is that longer starting runs reduce the number of rounds of MRAM merging and, thus, reduce DMAs, which are relatively costly.
However, again similar to the WRAM \MS*{}, it can be beneficial to slightly reduce the starting run length to achieve more balanced and faster rounds.
Nonetheless, the runtime difference between 1100 and 1200 elements per starting run is in the magnitude of one per mille, so for the sake of simplicity, the starting run length is set to the maximum a tasklet can store at once in the WRAM.

%Recall the rough calculation from the beginning of \cref{sec:tasklet} showing that tasklets can hold about \qty{5}{\kibi\byte} of data.
%This space must be used for both forming the starting runs (without any sequential-read buffers) and merging the runs (with two sequential-read buffers).
%One possibility would be to allocate one starting run buffer of size \qty{5}{\kibi\byte}, reset the entire heap after the starting runs have been formed and, then, allocate the sequential-read buffers.

This does raise the question how the starting run formation can allocate a large WRAM buffer while still leaving memory for two sequential-read buffers used during merging.
The answer is a \emph{triple buffer} which consists of a general-purpose buffer followed by two consecutive sequential-read buffers.
If sequential readers are not in use, the triple buffer can be regarded as one contiguous buffer.
To initialise the triple buffer, a tasklet first calls \lstinline|mem_alloc| to allocate \(\cachesize{} + \min\braces{ 8, \text{\lstinline[keywords={}]|sizeof(T)|} }\) many bytes on the heap, where \cachesize{} is a compile-time constant divisible by 8 and~\lstinline|T| is the data type of the input.
This memory, excluding the first \(\min\braces{ 8, \text{\lstinline[keywords={}]|sizeof(T)|} }\) many bytes, is referred to as \emph{cache} and will be used later to store merged runs.
The element right before the beginning of the cache is set to the least possible value of~\lstinline|T| so that it acts as sentinel value for \IS{}, hence the allocation of at least \lstinline[keywords={}]|sizeof(T)| additional bytes.
In case of integers smaller than 8 bytes, the beginning of the cache would not be aligned on 8 bytes anymore, hence the allocation of at least 8 additional bytes to simplify DMAs.
After the initialisation of the cache, the tasklet calls \lstinline|seqread_alloc| twice.
Due to the stack nature of the heap, the two sequential-read buffers are allocated directly behind the cache.
To ensure the contiguity of the triple buffer if more than one tasklet is present, a mutex is employed such that only one tasklet initialises its triple buffer at a time.
The entire triple buffer has the size \(\triplebuffersize \coloneqq \cachesize{} + 4 \times \seqreadcachesize{}\), which is, for simplicity, the minimum guaranteed number of allocated bytes and, therefore, the same for all tasklets even if some calls of \lstinline|seqread_alloc| skipped some bytes.
Note that only the first call of \lstinline|seqread_alloc| by each tasklet may skip some memory since the stack pointer is necessarily at a multiple of \seqreadcachesize{} after any call of \lstinline|seqread_alloc|.

\begin{figure}[b]
	\centering
	\tikzsetnextfilename{triple_buffer}
	\begin{tikzpicture}[reader figure, brace/.style={decorate, decoration={calligraphic brace, amplitude=5.5pt, raise=2pt}, line width=1.1pt}, brace down/.style={brace, decoration={mirror}}]
		\def\cache{6}
		\def\skipped{2}
		\def\buffer{8}

		% Buffer.
		\fill[excess] (0.5*\buffer+\skipped+\cache, 0) rectangle (1*\buffer+\skipped+\cache, 1);
		\fill[excess] (1.5*\buffer+\skipped+\cache, 0) rectangle (2*\buffer+\skipped+\cache, 1);

		\draw (-2, 0) grid (2*\buffer+\skipped+\cache, 1);

		% Braces.
		\draw[brace] (                      0, 1) --+ (  \cache, 0) node[pos=0.5, above=1ex] {cache\vphantom{p}};
		\draw[brace] (                 \cache, 1) --+ (\skipped, 0) node[pos=0.5, above=1ex] {skipped};
		\draw[brace] (        \skipped+\cache, 1) --+ ( \buffer, 0) node[pos=0.5, above=1ex] {first buffer\vphantom{p}};
		\draw[brace] (\buffer+\skipped+\cache, 1) --+ ( \buffer, 0) node[pos=0.5, above=1ex] {second buffer\vphantom{p}};

		\draw[brace down] (0, 0) --+ (2*\buffer+\cache, 0) node[pos=0.5, below=1ex] {\triplebuffersize\vphantom{f}};

		% Item identifiers.
		\pgfmathsetmacro{\looplastindex}{int(2*\buffer+\skipped+\cache+1)}
		\foreach \i in {0,...,\looplastindex}{
			\pgfmathsetmacro{\itemindex}{int( 4*(24+\i) )}
			\node[address] at (\i-1.5, 0.5) { \hexa{\itemindex}\vphantom{f} };
		}

		% Pointers.
		\draw[ptr] (-.5, 1) --+ (90:14.5pt) node[above] {sentinel};
	\end{tikzpicture}

	\caption{
		An exemplary triple buffer with \(\cachesize{} = 24\) and \(\seqreadcachesize{}= 16\) for 32-bit elements.
		The hexadecimal numbers denote WRAM addresses.
	}
	\label{fig:mram:triple}
\end{figure}


\section{A First \texorpdfstring{\MS{}}{MergeSort} on MRAM Data}
\label{sec:mram:merge}

The MRAM \MS{} is based on the half-space WRAM \MS{} as presented in \cref{sec:tasklet:merge} so only the adaptations to the two-tier memory hierarchy are discussed.

\paragraph{Starting Runs}
It is yet again beneficial to form starting runs.
This is done by loading a section of the MRAM into the WRAM, sorting it through one of the algorithms presented in \cref{sec:tasklet}, and writing the sorted section back to the MRAM.
As those sorting algorithm rely on the data being present entirely within the WRAM, sequential readers are not useful until later when merging runs.
Instead, the functions \lstinline|mram_read| and \lstinline|mram_write| are used directly.

Contrary to the WRAM \MS{} with starting runs of length 14, the starting runs of the MRAM \MS{} are much longer, easily containing a thousand elements and more.
The reason is that longer starting runs reduce the number of rounds of the MRAM \MS{} and, thus, reduce the DMAs to the MRAM, which are relatively costly compared to accesses to the WRAM.
According to \citeauthor{mutlu2022Benchmarking}~\cite[8\psq]{mutlu2022Benchmarking}, accessing 64-bit integers in the WRAM is about four and a half times faster than accessing 64-bit integers in the MRAM.
Since accessing 32-bit integers in the WRAM takes just as long as accessing 64-bit integers and since the performance of DMAs depends only on the total number of bytes, there is still a speedup of roughly 2 of WRAM accesses over MRAM accesses.
However, again similar to the WRAM \MS{}, it can be beneficial to slightly reduce the starting run length to achieve more balanced and faster rounds.
Nonetheless, the runtime difference between 1000, 1100, and 1200 elements per starting run is in the magnitude of one per mil.
For this reason, the starting run length is set simply to the maximum number of integers which the WRAM allotted to a tasklet can hold.

This does raise the question what said maximum is.
As calculated at the start of \cref{sec:tasklet}, each tasklet is allotted no more than \qty{5957}{\byte}.
Subtracting \qty{768}{\byte} for the stack of the tasklet and the call stack for the WRAM \QS{} leaves \qty{5189}{\byte} in the heap, that is a little less than 1300 32-bit integers.\todo{Noch einmal später aktualisieren}
This space must be used for both forming the starting runs (without any sequential-read buffers) and merging the runs (with two sequential-read buffers).
One possibility would be to allocate one starting run buffer of size \qty{5957}{\byte}, reset the entire heap after the starting runs have been formed and, then, allocate the sequential-read buffers.

Far more flexibility offers the introduction of a \emph{triple buffer} which consists of a general-purpose buffer followed by two consecutive sequential-read buffers.
If sequential readers are not in use, the triple buffer can be regarded as a single, big buffer.
To initialise the triple buffer, a tasklet first calls \lstinline|mem_alloc| to allocate \(\text{\lstinline|CACHE_SIZE|} + \min\braces{ 8, \text{\lstinline|sizeof(T)|} }\) many bytes on the heap, where \lstinline|CACHE_SIZE| is a user-defined compile-time constant and \lstinline|T| is the data type of the input.
This memory, excluding the first \(\min\braces{ 8, \text{\lstinline|sizeof(T)|} }\) many bytes, is referred to as \emph{cache} and will be used later to store merged runs.
The element right before the beginning of the cache is set to the lowest possible value of \lstinline|T| so that it acts as sentinel value for \IS{}, hence the allocation of at least \lstinline|sizeof(T)| additional bytes.
In case of integers smaller than 8 bytes, the address of the cache would not be aligned on 8 bytes anymore, hence the allocation of at least 8 additional bytes to simplify DMAs.
After the initialisation of the cache, the tasklet calls \lstinline|seqread_alloc| twice.
Due to the stack nature of the heap, the two sequential-read buffers are allocated directly behind the cache.
To ensure the contiguity of the triple buffer if more than one tasklet is present, a mutex is employed to ensure than only one tasklet initialises its triple buffer at a time.
The entire triple buffer has the size \(\text{\lstinline|TRIPLE_BUFFER_SIZE|} \coloneqq \text{\lstinline|CACHE_SIZE|} + 4 \times \text{\lstinline|SEQREAD_CACHE_SIZE|}\), which is, for simplicity, the minimum guaranteed number of allocated bytes and, therefore, the same for all tasklets even if some calls of \lstinline|seqread_alloc| skipped some bytes.
Note that only the first call of \lstinline|seqread_alloc| by each tasklet may skip some memory since the stack pointer is at a multiple of \lstinline|SEQREAD_CACHE_SIZE| after every call of \lstinline|seqread_alloc|.


\section{Parallel \texorpdfstring{\MS{}}{MergeSort}}
\label{sec:mram:par}

A simplistic way to parallelise \MS{} is the following:
Let the number of tasklets be a power of two and have the tasklet identifiers start from zero.
The whole input array is divided into as many shares of equal length as there are tasklets, and each tasklet sorts a share sequentially using the \ac{MRAM} \MS{} of \cref{sec:mram:merge} to form starting runs.
Once finished, Tasklet~\(t\) with~\(t \bmod 2 = 1\) informs Tasklet~\(t - 1\) that it is finished with sorting its share and suspends itself.
Tasklet \(t - 1\) merges its own share and that of Tasklet~\(t\) into a bigger run.
Once finished, Tasklet~\(t\) with~\(t \bmod 4 = 2\) informs Tasklet~\(t - 2\) that it is finished with sorting the run and suspends itself.
Then, Tasklet~\(t - 2\) merges its run with that of Tasklet~\(t\).
This scheme continues until the last round where the two remaining runs are sorted by Tasklet 0.

The bottleneck is the sequential execution of each merge which eventually leads to a single active tasklet.
Even with infinite many processors, this simplistic parallel \MS{} can achieve a theoretical parallel speedup\footnote{
	The \emph{parallel speedup} \(S = \operatorname{t}_1(A) / \operatorname{t}_p(A)\) of a parallel algorithm \(A\) is the ratio of its wall-clock time \(\operatorname{t}_1(A)\) when run with one processor and of its wall-clock time \(\operatorname{t}_p(A)\) when run with \(p\) processors.
} of at most \(\bigtheta{\log n}\).
\Citeauthor{cormen2013algorithmen}~\cite{cormen2013algorithmen} propose an alternative approach whose maximum theoretical parallel speedup is \(\bigtheta{n / \log^2 n}\).
Advantageously, \cref{alg:mram:two-tier merge} for merging \ac{MRAM} data can be reused without fundamental changes when adapting this parallel \MS{} to \acp{DPU}.
Also, the number of synchronisation points is logarithmic in the number of tasklets only, and the time which each synchronisation takes is insignificant compared to the total runtime.

\subsection{Presentation of Key Aspects}
\label{sec:mram:par:aspects}

The parallel \ac{MRAM} \MS{} is essentially a full-space \MS{}, meaning it needs an auxiliary array of the same size as the input array and the two arrays switch roles after each round.
The parallel merge procedure (\cref{fig:par:merge}) operates on arbitrary runs, that is, it is no longer demanded that two runs be neighboured when merging.
Likewise, the output location is arbitrary now, too, and not related to the indices of the two runs.
Please note that the algorithm presented here is not stable but will be stabilised later on in this \lcnamecref{sec:mram:par:aspects}.

Suppose that there are but two tasklets and both have formed a starting run using the sequential \ac{MRAM} \MS{}.
One of the tasklets is now temporarily suspended, whilst the other one determines which of the runs is longer.
Then, it determines the median of the longer run, which will act as \emph{pivot element} dividing the longer run into a front half and a back half.
The pivot is used to find an index~\(i\) which separates the shorter run into a front half and a back half such that any element with index~\(i' < i\) is not greater than the pivot and any element with index \(i' \ge i\) is at least as great as the pivot.
Finding such an index \(i\) can be implemented with a binary search.
The elements both front halves are at most as great as the pivot so they can go to the front of the output run.
This means that the position of the pivot can be calculated by taking the output location and offsetting it by the combined length of the two front halves.
Now, the front halves of both runs can be merged to the positions in front of the pivot using a sequential merge procedure, and the back halves can be merged to the positions behind the pivot.
Since these two merges affect distinct elements and addresses, they can be performed in parallel by the two tasklets.

The two tasklets do not merge the same number of elements necessarily, but the workloads cannot differ by a factor greater than three.
The longer run has a length of at least \(n/2\) elements and is divided into two halves with at least \(n/4\) elements each due to the pivot being the median.
Thus, both tasklets are guaranteed to merge at least \(n/4\) elements.
The shorter run has a length of at most \(n/2\) elements.
In the worst case, the pivot is either strictly less or strictly greater than all elements in the shorter run, meaning the shorter run is divided at its beginning or end, respectively, and is merged in its entirety by one of the tasklets.
This means that each tasklet merges at least \qty{25}{\percent} and at most \qty{75}{\percent} of the elements.

The parallel \MS{} can be generalised to work with more than two tasklets.
If there are, for example, four tasklets, then Tasklets~0 and 1 merge their runs in parallel, as do Tasklets~2 and~3.
Then, Tasklet~0 partitions the resulting two runs and assigns their back halves to Tasklet~2, so that both Tasklet~0 and~2 can partition their particular halves again and merge in parallel with Tasklets~1 and~3, respectively.
For simplicity, the number of tasklets is always a power of two in this \cref{sec:mram:par}.

\begin{figure}
	\centering
	\tikzsetnextfilename{par_merge}
	\begin{tikzpicture}[
		sketch,
		less/.style={very nearly transparent},
		greater/.style={nearly transparent},
		fade left/.style={dash pattern=on 1pt off 3pt on 1pt off 3pt on 2pt off 3pt on 3pt off 2pt on 12pt},
		fade right/.style={dash pattern=on 12pt off 2pt on 3pt off 3pt on 2pt off 3pt on 1pt off 3pt on 1pt},
		long fade left/.style={dash pattern=on 1pt off 3pt on 1pt off 3pt on 1pt off 3pt on 2pt off 3pt on 2pt off 2pt on 3pt off 2pt on 3pt off 2pt on 14pt},
		long fade right/.style={dash pattern=on 14pt off 2pt on 3pt off 2pt on 3pt off 3pt on 2pt off 3pt on 2pt off 2pt on 1pt off 3pt on 1pt off 3pt on 1pt},
	]
		\def\startA{\pad}
		\def\lenAA{5}
		\def\lenAB{5}

		\def\startB{\padMid+\lenAB+1+\lenAA+\startA}
		\def\lenBA{4}
		\def\lenBB{3}

		\def\startZ{\padMid/2+\pad}

		\def\pad{2}
		\def\padMid{2}
		\def\len{\pad+\lenBB+\lenBA+\padMid+\lenAB+1+\lenAA+\pad}
		\def\distance{3}

		% Arrows.
		\draw[ptr, <-]         (1+\startZ, 1.0) -- +(90:\distance);
		\draw[rounded corners] (1+\startZ, 1.5) -| ++(0, \distance/3-0.5) -| +(\padMid+\lenAB+\lenAA, \distance/3*2);

		\draw[white, line width=4pt] ({\len-(\startZ)-1}, \distance/3*2+1) -| +(-\lenBA-\padMid-\lenBB, \distance/3);
		\draw[rounded corners]       ({\len-(\startZ)-1},   \distance/3*2) -| ++(0, 1) -| +(-\lenBA-\padMid-\lenBB, \distance/3);
		\draw[ptr, <-]               ({\len-(\startZ)-1},               1) -- +(90:\distance);

		\draw[white, line width=4pt]    (\startZ+\lenAA+\lenBA+0.45, 1) -| +(0, \distance/4) -- (0.5+\lenAA+\startA, 1+\distance/16*13) -| ++(0, \distance/16*3);
		\draw[ptr, <-, rounded corners] (\startZ+\lenAA+\lenBA+0.50, 1) -| +(0, \distance/4) -- (0.5+\lenAA+\startA, 1+\distance/16*13) -| ++(0, \distance/16*3);

		% Filling.
		\fill[   less, accentcolor] (         \startA, 1+\distance) rectangle +(\lenAA, 1);
		\fill[greater, accentcolor] (1+\lenAA+\startA, 1+\distance) rectangle +(\lenAB, 1);

		\fill[   less, accentcolor] (       \startB, 1+\distance) rectangle +(\lenBA, 1);
		\fill[greater, accentcolor] (\lenBA+\startB, 1+\distance) rectangle +(\lenBB, 1);

		\fill[   less, accentcolor] (                \startZ, 0) rectangle +(\lenBA+\lenAA, 1);
		\fill[greater, accentcolor] (1+\lenBA+\lenAA+\startZ, 0) rectangle +(\lenBB+\lenAB, 1);

		% Horizontal lines.
		\draw[fade left]  (        0, 2+\distance) -- +(\pad, 0);
		\draw[fade left]  (        0, 1+\distance) -- +(\pad, 0);
		\draw[fade right] (\len-\pad, 2+\distance) -- +(\pad, 0);
		\draw[fade right] (\len-\pad, 1+\distance) -- +(\pad, 0);

		\draw (\startA, 2+\distance) -- (\len-\pad, 2+\distance);
		\draw (\startA, 1+\distance) -- (\len-\pad, 1+\distance);

		\draw[long fade left]  (               0, 1) -- +(\startZ, 0);
		\draw[long fade left]  (               0, 0) -- +(\startZ, 0);
		\draw[long fade right] ({\len-(\startZ)}, 1) -- +(\startZ, 0);
		\draw[long fade right] ({\len-(\startZ)}, 0) -- +(\startZ, 0);

		\draw (\startZ, 1) -- +({\len-(\startZ)*2}, 0);
		\draw (\startZ, 0) -- +({\len-(\startZ)*2}, 0);

		% Borders.
		\draw (                \startA, 1+\distance) -- +(90:1);
		\draw (         \lenAA+\startA, 1+\distance) -- +(90:1);
		\draw (       1+\lenAA+\startA, 1+\distance) -- +(90:1);
		\draw (\lenAB+1+\lenAA+\startA, 1+\distance) -- +(90:1);

		\draw (              \startB, 1+\distance) -- +(90:1);
		\draw (       \lenBA+\startB, 1+\distance) -- +(90:1);
		\draw (\lenBB+\lenBA+\startB, 1+\distance) -- +(90:1);

		\draw (                              \startZ, 0) -- +(90:1);
		\draw (                \lenBA+\lenAA+\startZ, 0) -- +(90:1);
		\draw (              1+\lenBA+\lenAA+\startZ, 0) -- +(90:1);
		\draw (\lenBB+\lenAB+1+\lenBA+\lenAA+\startZ, 0) -- +(90:1);

		% Braces.
		\draw[brace] (\startA, 2+\distance) -- +(\lenAB+1+\lenAA, 0) node[pos=0.5, above=1ex] {longer run\vphantom{p}};
		\draw[brace] (\startB, 2+\distance) -- +(  \lenBB+\lenBA, 0) node[pos=0.5, above=1ex] {shorter run\vphantom{p}};

		% Texts.
		\node[align=right, inner sep=0pt, text width=12mm, yshift=-0.4pt] at (-1.3, 1+\distance+0.5) {\lstinline|Input|};
		\node[align=right, inner sep=0pt, text width=12mm, yshift=-0.4pt] at (-1.3,             0.5) {\lstinline|Output|};

		\node at (\startB-\padMid/2, 1+\distance+0.45) {⋯};

		\node at (         \lenAA/2+\startA, 1+\distance+0.5) {\(\le p\)};
		\node at (       0.5+\lenAA+\startA, 1+\distance+0.5) {\(p\vphantom{>}\)};
		\node at (\lenAB/2+1+\lenAA+\startA, 1+\distance+0.5) {\(\ge p\)};

		\node at (       \lenBA/2+\startB, 1+\distance+0.5) {\(\le p\)};
		\node at (\lenBB/2+\lenBA+\startB, 1+\distance+0.5) {\(\ge p\)};

		\node at (                \lenBA/2+\lenAA/2+\startZ, 0.5) {\(\le p\)};
		\node at (                0.5+\lenBA+\lenAA+\startZ, 0.5) {\(p\vphantom{>}\)};
		\node at (\lenBB/2+\lenAB/2+1+\lenBA+\lenAA+\startZ, 0.5) {\(\ge p\)};
	\end{tikzpicture}

	\caption{
		Parallel merge of two runs.
		The first run is the longer one, so its median \(p\) is chosen as a pivot which is used to divide the shorter run.
		Afterwards, the pivot is moved to its output location.
		The two front halves of the runs are assigned to one tasklet and are merged to the positions in front of the pivot.
		At the same time, the two back halves of the runs are merged by another tasklet to the positions behind of the pivot.
		\cite[Figure~27.6]{cormen2013algorithmen}
	}
	\label{fig:par:merge}
\end{figure}


\paragraph{Communication \& Synchronisation}
The communication network can be visualised as forest of binomial trees.
During the first parallel merge, Tasklet~1 informs Tasklet~0 about being finished with sorting (\emph{bottom-up communication}), and Tasklet~0 informs Tasklet~1 about being finished with partitioning (\emph{top-down communication}).
Likewise, Tasklet~3 and~4 communicate, Tasklets~5 and~6, and so on.
At the beginning of the second parallel merge, Tasklets~1, 2, and~3 inform Tasklet~0 about being finished with sorting, and Tasklet~0 informs Tasklet~2 about being finished with partitioning.
Then, both tasklets partition again and inform Tasklet~1 and 3, respectively.
This bidirectional communication scheme is expanded for the third and fourth parallel merge if those exist.
We implemented the identification of communication partners through bitwise logic on tasklet identifiers.

To inform tasklets on \emph{which} elements they have to sort, there is a global \ac{WRAM} array whither the division points indices are written by partitioning tasklets.
To inform tasklets on \emph{when} they are finished with sorting or partitioning, tasklets employ \emph{handshakes}.
Handshakes allow for bilateral communication, which is enough for parallel \MS{}.
A tasklet can call \lstinline|handshake_wait_for(id)| and gets suspended until Tasklet~\lstinline|id| calls \lstinline|handshake_notify()|.
Likewise, if Tasklet \lstinline|id| calls \lstinline|handshake_notify()|, it gets suspended until some other tasklet calls \lstinline|handshake_wait_for(id)|.
If two or more tasklets wait for the same tasklet, an error is thrown and the execution halts.
Handshakes render bottom-up communication straightforward:
Each non-root of a communication tree calls \lstinline|handshake_notify| when done with sorting, whereas the root calls \lstinline|handshake_wait_for| for all of its successors.
Due to workload imbalances, some tasklets will try to shake hands earlier than others and will have to wait.
Because they are suspended while waiting, they free up the pipeline, thus accelerating the remaining tasklets.
Top-down communication is also straightforward:
After having shaken hands, each non-root immediately calls \lstinline|handshake_notify| again to get suspended once more.
The root repeatedly partitions the runs, writes the division points to the global \ac{WRAM} array mentioned earlier, and calls \lstinline|handshake_wait_for| to resume the next tasklet.


\paragraph{Binary Search}
The binary search is conventionally implemented, meaning elements are loaded individually from the \ac{MRAM} and there is no mechanism to load a block of data via \lstinline|mram_read| for a search within the \ac{WRAM} once the search interval has been narrowed down enough.
Whilst we did implement such a two-tier binary search, the speedup is below measurement uncertainty.
The reason is that the binary search is executed a few times in total only, so its impact on the total runtime is minuscule.
For the sake of code simplicity and kernel size, the \ac{WRAM} search tier has been removed.
Reducing the kernel size is a valid concern since the many unrolled loops bloat the kernel and only a handful of bytes in the \ac{IRAM} remain free.

Recall that the two halves assigned to a tasklet contain between \qty{25}{\percent} and \qty{75}{\percent} of the elements of the respective runs, which can lead to workload imbalances.
A more even workload of \qty{50}{\percent} would be achieved if the median of the merged run were chosen as pivot and the runs divided accordingly.
Finding the two positions where the runs should be divided according to this common median requires a modification to the binary search.
Two search intervals are set up, one for either run.
Then, the medians of both runs are determined.
If this does not produce valid division points, then one of the runs has more little elements in its front half than the other run.
This means that the front half of this run must become longer and the front half of the other run shorter.
Therefore, the search intervals can be narrowed down to the right side of the run with the less elements and to the left side of the run with the greater elements.
The process can now be repeated until two valid division points are found.

To our dissatisfaction, we did not manage to implement a bug-free version of this binary search within the timeframe of this thesis, so the effect of load imbalances manifests in our measurements.


\paragraph{Alignment}
In \cref{sec:mram:merge}, it is demanded that all sizes and positions be multiples of 8 even if the input consists of 32-bit elements.
Now, this can be ensured during the formation of starting runs by dividing the input accordingly and introducing dummy variables if need be.
Thereafter, however, there is no control over any alignment whatsoever because the sizes of run halves are arbitrary.
This raises the need for modifications to the sequential merge procedure.

Before beginning the first tier, the alignment of the output location must be checked.
If it is unaligned, the less of the first elements of both runs is written to the output location.
As a result, the updated output location is aligned to 8 bytes again, meaning the first tier can proceed as normal since emptying the cache through \lstinline|mram_write| is unproblematic.

At first, the second tier proceeds as normal, too.
Once the less run is depleted, the cache may contain an odd number of elements, so the current element of the greater run is written to the cache before emptying it.
However, if there are elements still remaining in the greater run, flushing the remainder becomes more complicated than in \cref{sec:mram:merge}.
There, it was sufficient to loop over the remainder in the \ac{MRAM}, write it to the cache, and move it to the respective output location.
Now, the current output location is still aligned to 8 bytes but it may very well be that the first element of the remainder has an unaligned address.
This indicates a mismatch, for all unaligned elements must be transferred to an aligned address and all aligned elements to an unaligned one.
When calling \lstinline|mram_read| and \lstinline|mram_write|, the alignment of elements within the \ac{MRAM} and within the \ac{WRAM} must be the same.
If such an instance is detected, the remainder is loaded blockwise from the \ac{MRAM} into the cache.
There, a loop shifts each element forward by one position, resolving the mismatch.
Afterwards, the shifted elements can be written to the output location.
Since only an even number of elements can be moved via \lstinline|mram_write|, it may be necessary to transfer a single item individually at the end in case that the remainder had an odd length.

Writing single 32-bit elements to the \ac{MRAM} is not threadsafe, since, internally, eight bytes are read to an oblique \ac{WRAM} cache, partially modified, and written back to the \ac{MRAM}.
Therefore, an atomic write, which utilises costly virtual mutexes, must be performed.


\paragraph{Stability}
The parallel merging algorithm as presented above makes \MS{} unstable.
The reason is that one or both runs may contain the pivot value multiple times and that a division point may separate the duplicates.
Therefore, it must be ensured that all duplicates of a run remain within the same half.
To do so, once the median is determined, it is checked if the left neighbour of the median has the same value .
If so, there may be even more duplicates so a binary search is employed to find the earliest occurrence of the pivot value within the longer run.
This earliest occurrence marks the division point for the longer run.
Similarly, the binary search in the shorter run now has to find the earliest possible division point, too.

To our great dissatisfaction, we did not manage to implement a bug-free version within the timeframe of this thesis.
With the zero-one input distribution, the output is corrupted, whence its exclusion in the respective figures.


\subsection{Evaluation of the Performance}
\label{sec:mram:par:performance}

\expandafter\pgfplotsinvokeforeach\expandafter{\alldists}{
	% 32-bit | Instabil
	\pgfplotstablenewnamed[create on use/n/.style={}, create on use/µ_MergePar/.style={}, create on use/σ_MergePar/.style={}, columns={n,µ_MergePar,σ_MergePar}]{0}{tableMergeParUnstable_32#1}
	\pgfplotstablevertcatnamed{tableMergeParUnstable_32#1}{data/merge_par/NR_TASKLETS=1/STABLE=false/uint32/#1.txt}
	\pgfplotstablevertcatnamed{tableMergeParUnstable_32#1}{data/merge_par/NR_TASKLETS=2/STABLE=false/uint32/#1.txt}
	\pgfplotstablevertcatnamed{tableMergeParUnstable_32#1}{data/merge_par/NR_TASKLETS=4/STABLE=false/uint32/#1.txt}
	\pgfplotstablevertcatnamed{tableMergeParUnstable_32#1}{data/merge_par/NR_TASKLETS=8/STABLE=false/uint32/#1.txt}
	\pgfplotstablevertcatnamed{tableMergeParUnstable_32#1}{data/merge_par/NR_TASKLETS=16/STABLE=false/uint32/#1.txt}

	% 64-bit | Instabil
	\pgfplotstablenewnamed[create on use/n/.style={}, create on use/µ_MergePar/.style={}, create on use/σ_MergePar/.style={}, columns={n,µ_MergePar,σ_MergePar}]{0}{tableMergeParUnstable_64#1}
	\pgfplotstablevertcatnamed{tableMergeParUnstable_64#1}{data/merge_par/NR_TASKLETS=1/STABLE=false/uint64/#1.txt}
	\pgfplotstablevertcatnamed{tableMergeParUnstable_64#1}{data/merge_par/NR_TASKLETS=2/STABLE=false/uint64/#1.txt}
	\pgfplotstablevertcatnamed{tableMergeParUnstable_64#1}{data/merge_par/NR_TASKLETS=4/STABLE=false/uint64/#1.txt}
	\pgfplotstablevertcatnamed{tableMergeParUnstable_64#1}{data/merge_par/NR_TASKLETS=8/STABLE=false/uint64/#1.txt}
	\pgfplotstablevertcatnamed{tableMergeParUnstable_64#1}{data/merge_par/NR_TASKLETS=16/STABLE=false/uint64/#1.txt}
}

\NewDocumentCommand{\speeduppar}{m}{
	\pgfplotstablegetelem{0}{µ_MergePar}\of#1
	\pgfmathsetmacro{\messlatte}{\pgfplotsretval}
	\addplot+ table[
		create on use/tasklets/.style={create col/set list={1,2,4,8,16}},
		create on use/factor/.style={create col/expr={\messlatte / \thisrow{µ_MergePar}}},
		x=tasklets, y=factor
	] {#1};
}

\begin{figure}
	\tikzsetnextfilename{par_speedup}
	\begin{tikzpicture}[plot]
		\begin{groupplot}[
			adaptive group=1 by 2,
			groupplot xlabel={Tasklets},
			groupplot ylabel={Parallel Speedup},
			xtick={1,2,4,8,16},
			xmode=log,
			ymin=0,
			ymax=12,
			ytick distance=2,
		]
			\nextgroupplot[title/.add={}{32-bit}]
			\pgfplotsset{legend to name=leg:par:speedup, legend entries={Sorted, Reverse S., Almost S., Zero-One, Uniform, Zipf's}}
			\addplot[forget plot, very nearly transparent] coordinates {(1,1)(2,2)(3,3)(4,4)(5,5)(6,6)(7,7)(8,8)(9,9)(10,10)(11,11)(16,11)};
			\expandafter\pgfplotsinvokeforeach\expandafter{\alldists}{
				\expandafter\speeduppar\expandafter{\csname tableMergeParUnstable_32#1\endcsname}
			}
			%
			\nextgroupplot[title/.add={}{64-bit}]
			\addplot[forget plot, very nearly transparent] coordinates {(1,1)(2,2)(3,3)(4,4)(5,5)(6,6)(7,7)(8,8)(9,9)(10,10)(11,11)(16,11)};
			\expandafter\pgfplotsinvokeforeach\expandafter{\alldists}{
				\expandafter\speeduppar\expandafter{\csname tableMergeParUnstable_64#1\endcsname}
			}
		\end{groupplot}
	\end{tikzpicture}

	\tikzexternaldisable\hfil\pgfplotslegendfromname{leg:par:speedup}\hfil\tikzexternalenable
	\caption{
		Mean parallel speedups of \MS{} on all benchmarked input distributions and data types with \qty{32}{\mebi\byte} of data.
		The grey line indicates an ideal, linear speedup which is capped at~11.
	}
	\label{fig:par:speedup}
\end{figure}

The parallel speedup of the unstable \MS{} on \qty{32}{\mebi\byte} of data is shown in \cref{fig:par:speedup}.
An ideal parallel speedup would be linear in the number of tasklets but capped at 11 or, rather, slightly above because of \ac{DMA} latency hiding.
For uniformly distributed inputs and inputs following Zipf's law, the measured speedup is very close to the optimum, reaching values above 10 for both 32-bit and 64-bit integers.
This is owed to workloads tending to be balanced naturally and tasklets being removed from the pipeline once they are finished.
For all other inputs, the parallel speedup is roughly between 7 and 9 with 32-bit integers and between~6 and~8 with 64-bit integers \Dash a consequence of workloads becoming more unbalanced.
This shall be illustrated by sorted inputs:
In the last round, two runs remain.
When Tasklet 0 performs the first partitioning step, the pivot divides the longer run into two equally long halves.
However, the pivot is strictly less or greater than any element in the shorter run, meaning the shorter run keeps its length of about \(n/2\) many elements as the division point is at one of its ends.
Such unbalanced divisions carry on to further partitioning steps.
The ratio between the least and the greatest number of assigned elements in the last round of parallel merging is about 2.3 for zero-one inputs and 4 for the three kinds of sorted inputs when the number of tasklets is 16.
A modified binary search as described in \cref{sec:mram:par:aspects} would presumably help bringing these little parallel speedups on par with those on the uniform and Zipf's input distribution.

\Cref{fig:par:phases} shows the wall-clock times of the unstable parallel \MS{}.
The measurements are subdivided into the three phases of the parallel \MS{}, that is the sequential \ac{WRAM} phase, the sequential \ac{MRAM} phase, and the parallel \ac{MRAM} phase.
On the one hand, most results are unsurprising.
The sorted, reverse sorted, and zero-one input distribution are sorted quickly as they lead to short first and second phases, whilst the uniform and Zipf's input distribution make these two phases last longer.
The third phase always takes roughly the same amount of time, implying that workload imbalances are cancelled out by earlier flushes, although the effect is weaker for 64-bit integers where computation is more costly.
The greater parallel speedup of the uniform and Zipf's input distribution shines through by the third phase being short compared to the relatively long first and second phase.

Almost sorted inputs, on the other hand, are clear outsiders because of the remarkably long third phase, making them the worst case amongst all benchmarked ones.
Despite the long runtime, the parallel speedup is about the same as for sorted and reverse sorted inputs.
Indeed, they all share equal or nearly equal workload imbalances.
The explanation for the third phase being so long is the same as the one given in \cref{sec:mram:merge:performance} with respect to the sequential \MS{}:
A few great elements placed in a run of mostly little elements delays flushes, and the longer the runs become, the likelier it is for a run to contain an overly great element.

\pgfplotstablenew[create on use/n/.style={}, create on use/µ_MergePar/.style={}, create on use/σ_MergePar/.style={}, columns={n,µ_MergePar,σ_MergePar}]{0}{\tableMergeParFirst}
\expandafter\pgfplotsinvokeforeach\expandafter{\alldists}{
	\pgfplotstablevertcat{\tableMergeParFirst}{data/merge_par/phase_1/STABLE=false/uint32/#1.txt}
}
\pgfplotstablenew[create on use/n/.style={}, create on use/µ_MergeFS/.style={}, create on use/σ_MergeFS/.style={}, columns={n,µ_MergeFS,σ_MergeFS}]{0}{\tableMergeParSec}
\expandafter\pgfplotsinvokeforeach\expandafter{\alldists}{
	\pgfplotstablevertcat{\tableMergeParSec}{data/merge_mram/NR_TASKLETS=16/CACHE_SIZE=1024/SEQREAD_CACHE_SIZE=512/uint32/#1.txt}
}
\pgfplotstablenew[create on use/n/.style={}, create on use/µ_MergePar/.style={}, create on use/σ_MergePar/.style={}, columns={n,µ_MergePar,σ_MergePar}]{0}{\tableMergeParThird}
\expandafter\pgfplotsinvokeforeach\expandafter{\alldists}{
	\pgfplotstablevertcat{\tableMergeParThird}{data/merge_par/NR_TASKLETS=16/STABLE=false/uint32/#1.txt}
}
\pgfplotstablenew[
	create on use/dist/.style={create col/set list={Sorted,Reverse S.,Almost S.,Zero-One,Uniform,Zipf's}},
	create on use/first/.style={create col/copy column from table={\tableMergeParFirst}{µ_MergePar}},
	create on use/second/.style={create col/copy column from table={\tableMergeParSec}{µ_MergeFS}},
	create on use/third/.style={create col/copy column from table={\tableMergeParThird}{µ_MergePar}},
	columns={dist,first,second,third},
]{6}{\tableMergeParXxxii}

\pgfplotstablenew[create on use/n/.style={}, create on use/µ_MergePar/.style={}, create on use/σ_MergePar/.style={}, columns={n,µ_MergePar,σ_MergePar}]{0}{\tableMergeParFirst}
\expandafter\pgfplotsinvokeforeach\expandafter{\alldists}{
	\pgfplotstablevertcat{\tableMergeParFirst}{data/merge_par/phase_1/STABLE=false/uint64/#1.txt}
}
\pgfplotstablenew[create on use/n/.style={}, create on use/µ_MergeFS/.style={}, create on use/σ_MergeFS/.style={}, columns={n,µ_MergeFS,σ_MergeFS}]{0}{\tableMergeParSec}
\expandafter\pgfplotsinvokeforeach\expandafter{\alldists}{
	\pgfplotstablevertcat{\tableMergeParSec}{data/merge_mram/NR_TASKLETS=16/CACHE_SIZE=1024/SEQREAD_CACHE_SIZE=512/uint64/#1.txt}
}
\pgfplotstablenew[create on use/n/.style={}, create on use/µ_MergePar/.style={}, create on use/σ_MergePar/.style={}, columns={n,µ_MergePar,σ_MergePar}]{0}{\tableMergeParThird}
\expandafter\pgfplotsinvokeforeach\expandafter{\alldists}{
	\pgfplotstablevertcat{\tableMergeParThird}{data/merge_par/NR_TASKLETS=16/STABLE=false/uint64/#1.txt}
}
\pgfplotstablenew[
	create on use/dist/.style={create col/set list={Sorted,Reverse S.,Almost S.,Zero-One,Uniform,Zipf's}},
	create on use/first/.style={create col/copy column from table={\tableMergeParFirst}{µ_MergePar}},
	create on use/second/.style={create col/copy column from table={\tableMergeParSec}{µ_MergeFS}},
	create on use/third/.style={create col/copy column from table={\tableMergeParThird}{µ_MergePar}},
	columns={dist,first,second,third},
]{6}{\tableMergeParLxiv}

\begin{figure}
	\tikzsetnextfilename{par_phases}
	\begin{tikzpicture}[plot]
		\begin{groupplot}[
			adaptive group=1 by 2,
			barplot,
			ybar=-\pgfkeysvalueof{/pgf/bar width},
%			yticklabel style={/pgf/number format/.cd, precision=1, fixed, zerofill},  % Cool, it is bugged out, so I have to place all ticks and labels manually…
		]
			\nextgroupplot[ymax=1.5e9, ytick distance=5e8, yticklabels={,\(0.0\), \(0.5\), \(1.0\), \(1.5\)}, title/.add={}{32-bit}]
			\pgfplotsset{legend to name=leg:par:phases, legend entries={Third Phase,Second Phase,First Phase}, legend reversed}
			\pgfplotsset{cycle list shift=2}
			\addplot+ table[x=dist, y=third] {\tableMergeParXxxii};
			\pgfplotsset{cycle list shift=0}
			\addplot+ table[x=dist, y=second] {\tableMergeParXxxii};
			\pgfplotsset{cycle list shift=-2}
			\addplot+ table[x=dist, y=first] {\tableMergeParXxxii};
			%
			\nextgroupplot[ymax=1e9, yticklabels={,\(0.0\), \(0.2\), \(0.4\), \(0.6\), \(0.8\), \(1.0\)}, title/.add={}{64-bit}]
			\pgfplotsset{cycle list shift=2}
			\addplot+ table[x=dist, y=third] {\tableMergeParLxiv};
			\pgfplotsset{cycle list shift=0}
			\addplot+ table[x=dist, y=second] {\tableMergeParLxiv};
			\pgfplotsset{cycle list shift=-2}
			\addplot+ table[x=dist, y=first] {\tableMergeParLxiv};
		\end{groupplot}
	\end{tikzpicture}

	\tikzexternaldisable\hfil\pgfplotslegendfromname{leg:par:phases}\hfil\tikzexternalenable
	\caption{
		Mean wall-clock times of the parallel \MS{}, broken down into its three phases, on all benchmarked input distributions and data types with \qty{32}{\mebi\byte} of data.
		The first phase comprises sequential sorting in the \ac{WRAM}, the second one sequential sorting in the \ac{MRAM}, and the third one parallel sorting in the \ac{MRAM}.
	}
	\label{fig:par:phases}
\end{figure}

\todo[inline]{stabiler \MS{}}


