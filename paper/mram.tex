\chapter{Sequential Sorting in the MRAM}
\label{sec:mram}

This chapter is concerned with the second phase of parallel sorting where tasklets sort MRAM data independently from each other, that is sequentially.
The sorting algorithm presented herein is a \MS{} since it allows for a stable parallel sorting algorithm and, also, the performance of \MS{} proved to be competitive in sorting WRAM data in \cref{sec:tasklet}.
A challenge in designing a \MS{} for MRAM data is the management of data transfers between the large MRAM and the small WRAM of a DPU.
A key aspect of the transfer management is the use of the sequential reader, that is a software abstraction provided by UPMEM which simplifies moving data from the MRAM to the WRAM.
\Cref{sec:mram:seq_reader} presents the advantages of the sequential reader and explains its usage and inner workings.
\Cref{sec:mram:triple} addresses the division of the WRAM into several buffers which are needed by both sequential readers and the MRAM \MS{} itself.
Finally, \cref{sec:mram:merge} deals with the sorting algorithm itself, shows how the sequential reader can be improved upon, and concludes with an analysis of the performance.
Every measurement shown in this \lcnamecref{sec:mram} was repeated a thousand times, which is a sufficient number given the large inputs and the low runtime variance of \MS{}.

%\begin{itemize}
%	\item
%	Sequential Reader
%	\begin{itemize}
%		\item
%		Verbesserungspotential
%		\begin{itemize}
%			\item
%			\lstinline|last_item| auch für den straighten Leser.
%			Das macht es billiger zu prüfen, wann die erste Stufe verlassen werden muss.
%			Außerdem macht sie das Entleeren eines Laufes billiger, da leicht das wahre Ende des Puffers festgestellt werden kann.
%		\end{itemize}
%	\end{itemize}
%
%	\item
%	Does reading less if the end is reached improve performance? No!
%
%	\item
%	Kann Flushed durch vollständiges Lesen des Puffers verschnellert werden? Ja! Siehe Erweiterung durch den reinen C-Leser
%
%	\item
%	Was sich alles lohnen könnte:
%	\begin{itemize}
%		\item
%		Aufweichung der triple buffer zum Erstellen längerer Startläufe
%	\end{itemize}
%\end{itemize}

\clearpage
\section{The Sequential Reader}
\label{sec:mram:seq_reader}

It is technically possible to access data in the MRAM in the same way as data in the WRAM.
For example, \lstinline|var = arr[i]| is valid code no matter whether the array \lstinline|arr|, the variable \lstinline|var|, or the index \lstinline|i| have been declared to reside in the WRAM or the MRAM.
Such \emph{direct memory accesses} (DMAs) to the MRAM are translated into the the assembler functions \lstinline|ldma| and \lstinline|sdma|, which load and store the respective data.
However, each DMA comes with an overhead, and accessing MRAM variables like WRAM variables means an execution of \lstinline|ldma| or \lstinline|sdma| on each accesses
For this reason, the preferred way to load contiguous data from the MRAM is through the function \lstinline|mram_read|.
This function takes the MRAM address of the first byte to load, the WRAM address of a buffer where the data are to be stored, and the number of consecutive bytes to load.
Likewise, there is the function \lstinline|mram_write| for writing contiguous WRAM data to the MRAM.
Internally, these two \langC{} functions make also use of the assembler functions \lstinline|ldma| and \lstinline|sdma| but pass them the number of bytes to load so that the overhead occurs only once and is spread amongst many bytes.
The effect is so high that, according to \citeauthor{mutlu2022Benchmarking}~\cite[11]{mutlu2022Benchmarking}, using \lstinline|mram_read| and \lstinline|mram_write| remains beneficial even if only an eighth of the transferred data is actually of interest (as is the case, for example, with strided access patterns).

However, the functions \lstinline|mram_read| and \lstinline|mram_write| do come with some constraints.
Both the target and the source address must be aligned on 8 bytes.
Also, the number of transferred bytes must be at least 8, at most 2048, and be a multiple of 8.
Failing to follow these constraints can result in missing or corrupt data.
Furthermore, programmers are tasked with maintaining the WRAM buffer and transferring data at appropriate times, which is likely frequent as the WRAM is more than a thousand times smaller than the MRAM.

Since reading data consecutively is a common occurrence, UPMEM provides a set of data structures and \langC{} functions which automate the process and, thereby, remove any need to care for the alignment of addresses, the maintenance of the WRAM buffer, or the loading of new data.
On top of that, UPMEM claims that \textquote{[\dots] this abstraction implementation has been optimized and will provide better performance than a standard \langC{} check of the cache boundaries.}~\cite[Memory management -- Sequential readers]{upmemSDK}

The compile-time constant \lstinline|SEQREAD_CACHE_SIZE| determines the size of the sequential-read buffer in bytes and can be set to either 32, 64, 128, 256, 512, or 1024.
Initialising the buffer must be done once through calling the function \lstinline|seqread_alloc|.
Several things are noteworthy:
\begin{itemize}
	\item
	The buffer is allocated on the heap, so the initialisation does take some time.
	Also, heap allocation is atomic, that is, it cannot happen concurrently, meaning that tasklets can be stalled by other tasklets.
	Last but not least, the addresses of the buffers are not known during compilation, so that they must be passed to assembler functions via registers.
	This could be a potential cause for a deterioration of the compilation if registers are too few for all required data and many loads and stores ensue.

	\item
	The allocated buffer has a size of 2 × \lstinline|SEQREAD_CACHE_SIZE| for reasons explained shortly.

	\item
	Remember that the heap is implemented as a never-decreasing stack.
	This means that new memory is only ever allocated after the \emph{stack pointer}, which stores the end of the stack.
	Before the buffers are actually allocated, the stack pointer skips to the next multiple of \lstinline|SEQREAD_CACHE_SIZE| if not already on such a multiple.
	This is done for address masking purposes, as \lstinline|SEQREAD_CACHE_SIZE| is a power of two.
	Due to the nature of the stack, this has the drawback that the skipped WRAM memory can never be allocated for something else.
\end{itemize}
All in all, the memory footprint of a sequential-read buffer is between 2 × \lstinline|SEQREAD_CACHE_SIZE| and 3 × \lstinline|SEQREAD_CACHE_SIZE| many bytes.

The function \lstinline|seqread_init| instructs a sequential reader and its buffer to load data from a specified MRAM address and returns a pointer to the corresponding WRAM address of said MRAM address.
Loading is done by, first, rounding the MRAM address down to the next multiple of \lstinline|SEQREAD_CACHE_SIZE| and, then, loading 2 × \lstinline|SEQREAD_CACHE_SIZE| many bytes.
In other words, the MRAM is divided into \emph{pages} of size \lstinline|SEQREAD_CACHE_SIZE| and the buffer holds two subsequent pages.
This way, data of some long, compound type at the end of the first page is fully loaded even if extending into the other page.

To access the data in the buffer, one can dereference the pointer returned by \lstinline|seqread_init|.
Calling the function \lstinline|seqread_get| advances this pointer to the current item by a given number of bytes;
it is permissible to advance by differing numbers of bytes on different calls.
This way of specifying bytes allows the sequential reader to support arbitrary data types.
Once the pointer to the current item ends up in the second half of the buffer, it is set \lstinline|SEQREAD_CACHE_SIZE| many bytes back so that it again points to an address in the first half.
Also, the stored MRAM address is increased by \lstinline|SEQREAD_CACHE_SIZE| many bytes, and the next two subsequent pages are loaded.
This means that the page which was stored in the second half of the buffer so far is loaded again from the MRAM but stored in the first half this time.

It is unclear how this implementation of sequential readers achieves the acclaimed speedup over regular \langC{} code.
The bounds checks happen when advancing the pointer to the current item, that is in the function \lstinline|seqread_get|.
This function, however, does nothing more than calling the obscure and undocumented function \lstinline|__builtin_dpu_seqread_get|.
It should be noted that the inner workings of sequential readers are poorly documented in the manual.
The descriptions given here are based on observations and careful studying of the \langC{}~source code of the functions other than \lstinline|seqread_get|.


\clearpage
\section{The Triple Buffer}
\label{sec:mram:triple}

It is yet again beneficial to form starting runs.
This is done by loading a section of the MRAM into the WRAM, sorting it through one of the algorithms presented in \cref{sec:tasklet}, and writing the sorted section back to the MRAM.
As those sorting algorithm rely on the data being present entirely within the WRAM, the functions \lstinline|mram_read| and \lstinline|mram_write| are used directly.
Contrary to the WRAM \MS*{} with starting runs of length 14, the lengths of the starting runs of the MRAM \MS{} go well into the hundreds.
The reason is that longer starting runs reduce the number of rounds of MRAM merging and, thus, reduce DMAs, which are relatively costly.
However, again similar to the WRAM \MS*{}, it can be beneficial to slightly reduce the starting run length to achieve more balanced and faster rounds.
Nonetheless, the runtime difference between 500, 600 and 700 elements per starting run is in the magnitude of one per mille, so for the sake of simplicity, the starting run length is set to the maximum a tasklet can store at once in the WRAM.

%Recall the rough calculation from the beginning of \cref{sec:tasklet} showing that tasklets can hold about \qty{5}{\kibi\byte} of data.
%This space must be used for both forming the starting runs (without any sequential-read buffers) and merging the runs (with two sequential-read buffers).
%One possibility would be to allocate one starting run buffer of size \qty{5}{\kibi\byte}, reset the entire heap after the starting runs have been formed and, then, allocate the sequential-read buffers.

This does raise the question how the starting run formation can allocate a large WRAM buffer while still leaving memory for two sequential-read buffers used later during merging.
The answer is a \emph{triple buffer} which consists of a general-purpose buffer followed by two consecutive sequential-read buffers.
If sequential readers are not in use, the triple buffer can be regarded as one contiguous buffer.
To initialise the triple buffer, a tasklet first calls \lstinline|mem_alloc| to allocate \(\cachesize{}\) many bytes on the heap, where \cachesize{} is a compile-time constant divisible by 8.
This memory is referred to as \emph{cache} and will be used later to store merged runs.
After the initialisation of the cache, the tasklet calls \lstinline|seqread_alloc| twice.
Due to the stack nature of the heap, the two sequential-read buffers are allocated directly behind the cache.
To ensure the contiguity of the triple buffer if more than one tasklet is present, a mutex is employed such that only one tasklet initialises its triple buffer at a time.
The entire triple buffer has the size \(\triplebuffersize \coloneqq \cachesize{} + 4 \times \seqreadcachesize{}\), which is, for simplicity, the minimum number of allocated bytes and, therefore, the same for all tasklets even if some calls of \lstinline|seqread_alloc| skipped some bytes.
Note that skipped memory may appear in front of only the first buffer, since the stack pointer is necessarily at a multiple of \seqreadcachesize{} after any call of \lstinline|seqread_alloc|.
Because of otherwise skipped memory, it makes little sense to set \cachesize{} to a value which is not a multiple of \seqreadcachesize{}.
The optimal values for both \cachesize{} and \seqreadcachesize{} are determined in \cref{sec:mram:merge:performance}.

\begin{figure}[b]
	\centering
	\tikzsetnextfilename{triple_buffer}
	\begin{tikzpicture}[reader figure, brace/.style={decorate, decoration={calligraphic brace, amplitude=5.5pt, raise=2pt}, line width=1.1pt}, brace down/.style={brace, decoration={mirror}}]
		\def\cache{8}
		\def\skipped{2}
		\def\buffer{8}

		% Buffer.
		\fill[excess] (0.5*\buffer+\skipped+\cache, 0) rectangle (1*\buffer+\skipped+\cache, 1);
		\fill[excess] (1.5*\buffer+\skipped+\cache, 0) rectangle (2*\buffer+\skipped+\cache, 1);

		\draw (0, 0) grid (2*\buffer+\skipped+\cache, 1);

		% Braces.
		\draw[brace] (                      0, 1) --+ (  \cache, 0) node[pos=0.5, above=1ex] {cache\vphantom{p}};
		\draw[brace] (                 \cache, 1) --+ (\skipped, 0) node[pos=0.5, above=1ex] {skipped};
		\draw[brace] (        \skipped+\cache, 1) --+ ( \buffer, 0) node[pos=0.5, above=1ex] {first buffer\vphantom{p}};
		\draw[brace] (\buffer+\skipped+\cache, 1) --+ ( \buffer, 0) node[pos=0.5, above=1ex] {second buffer\vphantom{p}};

		\draw[brace down] (0, 0) --+ (2*\buffer+\cache, 0) node[pos=0.5, below=1ex] {\triplebuffersize\vphantom{f}};

		% Item identifiers.
		\pgfmathsetmacro{\looplastindex}{int(2*\buffer+\skipped+\cache-1)}
		\foreach \i in {0,...,\looplastindex}{
			\pgfmathsetmacro{\itemindex}{int( 4*(22+\i) )}
			\node[address] at (\i+.5, 0.5) { \hexa{\itemindex}\vphantom{f} };
		}
	\end{tikzpicture}

	\caption{
		An exemplary triple buffer with \(\cachesize{} = 32\) and \(\seqreadcachesize{}= 16\) for 32-bit elements.
		The hexadecimal numbers denote WRAM addresses.
		Note that the skipped byte are still used when regarding the triple buffer as one contiguous buffer.
		Then, however, the last bytes of the second sequential-read buffer are unused.
	}
	\label{fig:merge:triple}
\end{figure}


\clearpage
\section{A First \texorpdfstring{\MS{}}{MergeSort} on MRAM Data}
\label{sec:mram:merge}

The MRAM \MS{} is based on the half-space WRAM \MS{} as presented in \cref{sec:tasklet:merge} so only the adaptations to the two-tier memory hierarchy are discussed.

\paragraph{Starting Runs}
It is yet again beneficial to form starting runs.
This is done by loading a section of the MRAM into the WRAM, sorting it through one of the algorithms presented in \cref{sec:tasklet}, and writing the sorted section back to the MRAM.
As those sorting algorithm rely on the data being present entirely within the WRAM, sequential readers are not useful until later when merging runs.
Instead, the functions \lstinline|mram_read| and \lstinline|mram_write| are used directly.

Contrary to the WRAM \MS{} with starting runs of length 14, the starting runs of the MRAM \MS{} are much longer, easily containing a thousand elements and more.
The reason is that longer starting runs reduce the number of rounds of the MRAM \MS{} and, thus, reduce the DMAs to the MRAM, which are relatively costly compared to accesses to the WRAM.
According to \citeauthor{mutlu2022Benchmarking}~\cite[8\psq]{mutlu2022Benchmarking}, accessing 64-bit integers in the WRAM is about four and a half times faster than accessing 64-bit integers in the MRAM.
Since accessing 32-bit integers in the WRAM takes just as long as accessing 64-bit integers and since the performance of DMAs depends only on the total number of bytes, there is still a speedup of roughly 2 of WRAM accesses over MRAM accesses.
However, again similar to the WRAM \MS{}, it can be beneficial to slightly reduce the starting run length to achieve more balanced and faster rounds.
Nonetheless, the runtime difference between 1000, 1100, and 1200 elements per starting run is in the magnitude of one per mil.
For this reason, the starting run length is set simply to the maximum number of integers which the WRAM allotted to a tasklet can hold.

This does raise the question what said maximum is.
As calculated at the start of \cref{sec:tasklet}, each tasklet is allotted no more than \qty{5957}{\byte}.
Subtracting \qty{768}{\byte} for the stack of the tasklet and the call stack for the WRAM \QS{} leaves \qty{5189}{\byte} in the heap, that is a little less than 1300 32-bit integers.\todo{Noch einmal später aktualisieren}
This space must be used for both forming the starting runs (without any sequential-read buffers) and merging the runs (with two sequential-read buffers).
One possibility would be to allocate one starting run buffer of size \qty{5957}{\byte}, reset the entire heap after the starting runs have been formed and, then, allocate the sequential-read buffers.

Far more flexibility offers the introduction of a \emph{triple buffer} which consists of a general-purpose buffer followed by two consecutive sequential-read buffers.
If sequential readers are not in use, the triple buffer can be regarded as a single, big buffer.
To initialise the triple buffer, a tasklet first calls \lstinline|mem_alloc| to allocate \(\text{\lstinline|CACHE_SIZE|} + \min\braces{ 8, \text{\lstinline|sizeof(T)|} }\) many bytes on the heap, where \lstinline|CACHE_SIZE| is a user-defined compile-time constant and \lstinline|T| is the data type of the input.
This memory, excluding the first \(\min\braces{ 8, \text{\lstinline|sizeof(T)|} }\) many bytes, is referred to as \emph{cache} and will be used later to store merged runs.
The element right before the beginning of the cache is set to the lowest possible value of \lstinline|T| so that it acts as sentinel value for \IS{}, hence the allocation of at least \lstinline|sizeof(T)| additional bytes.
In case of integers smaller than 8 bytes, the address of the cache would not be aligned on 8 bytes anymore, hence the allocation of at least 8 additional bytes to simplify DMAs.
After the initialisation of the cache, the tasklet calls \lstinline|seqread_alloc| twice.
Due to the stack nature of the heap, the two sequential-read buffers are allocated directly behind the cache.
To ensure the contiguity of the triple buffer if more than one tasklet is present, a mutex is employed to ensure than only one tasklet initialises its triple buffer at a time.
The entire triple buffer has the size \(\text{\lstinline|TRIPLE_BUFFER_SIZE|} \coloneqq \text{\lstinline|CACHE_SIZE|} + 4 \times \text{\lstinline|SEQREAD_CACHE_SIZE|}\), which is, for simplicity, the minimum guaranteed number of allocated bytes and, therefore, the same for all tasklets even if some calls of \lstinline|seqread_alloc| skipped some bytes.
Note that only the first call of \lstinline|seqread_alloc| by each tasklet may skip some memory since the stack pointer is at a multiple of \lstinline|SEQREAD_CACHE_SIZE| after every call of \lstinline|seqread_alloc|.

