\chapter{Sorting Sequentially in MRAM}
\label{sec:mram}

\begin{itemize}
	\item
	Sequential Reader
	\begin{itemize}
		\item
		nimmt viel Platz weg aufgrund der Maske

		\item
		Maske sorgt auch für viel unnötig geladenes beim Initialisieren, da half-space \MS{} erster Lauf alles verschieben kann.
		Insbesondere heißt das, das es sein kann, dass nur \lstinline|SEQREAD_CACHE_SIZE + 1| viel tatsächlich neues nach der Initialiserung im Speicher steht, da das vorherige dem vorherigen Laufe entspricht.

		\item
		stellt Ausrichtung bei jeder Initialisierung sicher

		\item
		geht nur vorwärts, nicht rückwärts

		\item
		sehr teuer zu wissen, wann Schluss
		\begin{itemize}
			\item
			umgehbar, indem ein Zähler verwendet wird
		\end{itemize}

		\item
		\lstinline|get| erzeugt einen Funktionsaufruf

		\item
		Empirisch festgestellt:
		Lädt nur neu, wenn \lstinline|ptr| noch im ersten Puffer liegt, \lstinline|ptr + 1| aber nicht mehr.
		Das bedeutet, wenn man \lstinline|ptr| händisch ohne \lstinline|get| ein paar Felder weiterlaufen lässt und dann \lstinline|get(ptr-1)| zur Synchronisation ausführt, funktioniert es nicht, wenn \lstinline|ptr| nicht an der letzten Stelle des ersten Puffers gelandet ist!
		Das macht unrollen schwieriger und der zweite Puffer ist für uns wirklich verloren.

		\item
		Verbesserungspotential
		\begin{itemize}
			\item
			\lstinline|last_item| auch für den straighten Leser

			\item
			Daraus folgt auch vielleicht, dass das zweistufige Verschmelzen verbessert werden kann.
			Bisher wird nach \unrolledcachelength{} Wiederholungen der Cache entleert und geprüft, ob in die zweite Stufe übergegangen werden muss.
			Unter Umständen kann man billig nach \unrollfactor{} Wiederholungen prüfen, ob der Cache geleert werden muss.
		\end{itemize}
	\end{itemize}

	\item
	eigener Reader
	\begin{itemize}
		\item
		Ausrichtung nicht sichergestellt. Daher:
		\begin{itemize}
			\item
			Eingabegröße muss durch 8 teilbar sein (Dummyvalue falls nötig)

			\item
			Aufteilung je Tasklet durch 8 teilbar

			\item
			starting run lengths durch 8 teilbar
		\end{itemize}

		\item
		Berechnung der Adresse des letzten Elementes
		\begin{itemize}
			\item
			\lstinline|(T *)buffers[me()].seq_2 + (2 * SEQREAD_CACHE_SIZE / sizeof(T)) - 1|
			\begin{itemize}
				\item
				wird als Offset des Startzeigers aufgefasst

				\item
				wird bei jeder Verwendung neu berechnet
			\end{itemize}

			\item
			\lstinline|(T *)(buffers[me()].seq_2 + 2 * SEQREAD_CACHE_SIZE) - 1|
			\begin{itemize}
				\item
				einmal berechnet

				\item
				\qty{-5,6}{\percent} Laufzeit
			\end{itemize}
		\end{itemize}

		\item
		Zugewinn nur durch größere Datenübertragung/weniger Wegschmiss bereits Bekanntem: etwa \qty{4}{\percent} bei uniform 32-bit

		\item
		Frühes Ende mitrechnen

		\item
		Does reading less if the end is reached improve performance? No!

		\item
		Statische Definition der Puffer möglich!
		Führt auch zu weniger Registernutzung.
	\end{itemize}

	\item
	Dreifachpuffer
	\begin{itemize}
		\item
		zusammenhängend dank Schranke

		\item
		\lstinline|TRIPLE_BUFFER_SIZE| = \lstinline|BLOCK_SIZE| + 4 × \lstinline|SEQREAD_CACHE_SIZE|

		\item
		\lstinline|MAX_TRANSFER_SIZE_TRIPLE| gg. \lstinline|MAX_TRANSFER_SIZE_CACHE|
	\end{itemize}

	\item
	Wie wenig einsparbar durch Startingrun MRAM → WRAM statt MRAM → MRAM → WRAM?

	\item
	Bemerkenswert:
	Aufgrund der Schleifenstreckung ist es besser, den Cache etwas kleiner zu machen.

	\item
	Kann Flushed durch vollständiges Lesen des Puffers verschnellert werden?
\end{itemize}

\clearpage
\section{The Sequential Reader}
\label{sec:mram:seq_reader}

A \ac{DMA} takes half a cycle per transferred byte.
On top of that, each reading \ac{DMA} comes with an additional overhead of  \qty{77}{\cycles} and each writing \ac{DMA} with an overhead of \qty{61}{\cycles}.
To dilute this overhead, \ac{MRAM} data is preferably processed in blocks using the functions \lstinline|mram_read| and \lstinline|mram_write|.
The benefit of blockwise processing is so high that, according to \citeauthor{mutlu2022Benchmarking}~\cite[11]{mutlu2022Benchmarking}, using \lstinline|mram_read| and \lstinline|mram_write| remains beneficial compared to accessing single \ac{MRAM} elements even if only an eighth of the transferred data are actually of interest.
However, the functions \lstinline|mram_read| and \lstinline|mram_write| do come with some constraints.
Both the target and the source address must be aligned to 8 bytes.
Also, the number of transferred bytes must be at least 8, at most 2048, and a multiple of 8.
Furthermore, programmers are tasked with maintaining a buffer in the \ac{WRAM} and transferring data at appropriate times, which, given that the \ac{WRAM} is more than a thousand times smaller than the \ac{MRAM}, is likely frequent.

Since processing \ac{MRAM} data consecutively is a common occurrence, \upmem{} provides a data structure called \emph{sequential reader}.
Through a set of \langC{} functions, a sequential reader automates the read-in process and, thereby, removes any need to care for the alignment of addresses or the loading of new data.
On top of that, \upmem{} claims that \textquote{\bibellipsis{} this abstraction implementation has been optimized and will provide better performance than a standard \langC{} check of the cache boundaries.}~\cite[Memory management -- Sequential readers]{upmemSDK}

Of course, a sequential reader still requires a \ac{WRAM} buffer.
Its size in bytes is determined by the compile-time constant \seqreadcachesize{}, which can be set to either 32, 64, 128, 256, 512, or 1024.
The allocation of the buffer on the heap happens through the function \lstinline|seqread_alloc| and is worth a closer look.
Remember that the heap is actually implemented as a never-decreasing stack.
This means that new memory is only ever allocated behind the heap pointer, which stores the end of the heap.
With \(\seqreadcachesize{} = \twotoi\), the \(i\) least significant bits of the first byte address in the buffer are required to be zero, for reasons explained shortly.
Therefore, padding is introduced by having the heap pointer skip to the next higher multiple of \seqreadcachesize{} if not already on such a multiple.
Due to the nature of a stack, this has the drawback that the padding can never be allocated for something else.
After the skip of the heap pointer, a total of \(2 × \seqreadcachesize{}\) many bytes are allocated, also for reasons explained shortly.
All in all, the memory footprint of a sequential-read buffer is at least \(2 × \seqreadcachesize{}\) many bytes and less than \(3 × \seqreadcachesize{}\) many.

The function \lstinline|seqread_init| instructs a sequential reader to load data from a specified \ac{MRAM} address into its buffer.
Conceptually, the whole \ac{MRAM} is divided into \emph{pages} of size \seqreadcachesize{}.
To load data from the specified \ac{MRAM} address, the address is rounded down the the next multiple of \seqreadcachesize{}, which yields the beginning of the page containing the \ac{MRAM} address.
Then, 2 × \seqreadcachesize{} many bytes are loaded so that the buffer holds two pages.
This way, data of some long, compound type at the end of the first page are fully loaded even if extending into the other page.

The function \lstinline|seqread_init| also returns a pointer to the corresponding position of the specified \ac{MRAM} address within the sequential-read buffer, to which we will refer as pointer to the \emph{current element}.
Due to the page model, this pointer may not be set to the beginning of the buffer.
To access the current element, one simply dereferences the pointer.
Calling the function \lstinline|seqread_get| advances this pointer by a given number of bytes, which cannot be greater than \seqreadcachesize{};
this way of specifying bytes allows the sequential reader to support arbitrary data types.
Once the pointer to the current element ends up in the second half of the buffer, it is set \seqreadcachesize{} many bytes back so that it points to an address in the first half again.
In addition, the \ac{MRAM} address from which to read is increased by \seqreadcachesize{} many bytes, and the next two subsequent pages are loaded.
This means that the page stored in the second half of the buffer is loaded from the \ac{MRAM} again but stored in the first half this time.
\Cref{fig:merge:reader} visualises an intermediate state of a sequential reader, showcasing its characteristic read behaviour.

\begin{figure}
	\centering
	\tikzsetnextfilename{sequential_reader}
	\begin{tikzpicture}[sketch, curr/.style={accentcolor}]
		% MRAM.
		\def\mramlen{20}
		\def\mramstart{-2}
		\def\mramend{6}

		\fill[excess] (\mramstart, 0) rectangle (                0, 1);
		\fill[excess] (  \mramlen, 0) rectangle (\mramlen+\mramend, 1);
		\draw[excess] (\mramstart, 0)      grid (                0, 1);
		\draw[excess] (  \mramlen, 0)      grid (\mramlen+\mramend, 1);

		\draw (0, 0) grid (\mramlen,1);

		% Buffer.
		\def\buflen{8}
		\def\bufxoffset{10}
		\def\buffyoffset{-3}

		\fill[excess] (0.5*\buflen+\bufxoffset+\mramstart, \buffyoffset) rectangle (\buflen+\bufxoffset+\mramstart, 1+\buffyoffset);
		\draw[excess] (0.5*\buflen+\bufxoffset+\mramstart, \buffyoffset)      grid (\buflen+\bufxoffset+\mramstart, 1+\buffyoffset);

		\draw(\bufxoffset+\mramstart, \buffyoffset) grid (0.5*\buflen+\bufxoffset+\mramstart, 1+\buffyoffset);

		% Mapping lines.
		\draw[               name path=M2Bl] (1.5*\buflen+\mramstart, 0) to[out=-90, in=90] (            \bufxoffset+\mramstart, 1+\buffyoffset);
		\draw[line cap=butt, name path=M2Br] (2.5*\buflen+\mramstart, 0) to[out=-90, in=90] (    \buflen+\bufxoffset+\mramstart, 1+\buffyoffset);
		\draw[                       dashed] (2.0*\buflen+\mramstart, 0) to[out=-90, in=90] (0.5*\buflen+\bufxoffset+\mramstart, 1+\buffyoffset);
		\tikzfillbetween[of=M2Bl and M2Br, split=false] {gray, ultra nearly transparent}

		% Current item.
		\def\currindex{2}

		\fill[curr, fill opacity=.1]   (\currindex+\bufxoffset+\mramstart,      \buffyoffset) rectangle (1+\currindex+\bufxoffset+\mramstart, 1+\buffyoffset);
		\draw[curr, ultra thick, fill] (\currindex+\bufxoffset+\mramstart,      \buffyoffset)      grid (1+\currindex+\bufxoffset+\mramstart, 1+\buffyoffset);
		\draw[curr, flow, ->]          (\currindex+\bufxoffset+\mramstart, 1.25+\buffyoffset)       --+ (0:1);

		% Item identifiers.
		\def\mramfirst{10}

		\pgfmathsetmacro{\looplastindex}{int(\mramlen+\mramend-1)}
		\foreach \i in {\mramstart,...,\looplastindex}{
			\pgfmathsetmacro{\itemindex}{int( 4*(\mramfirst+\i) )}
			\node[address] at (\i+0.5, 0.5) { \hexa{\itemindex} };
		}

		\pgfmathsetmacro{\looplastindex}{int(\buflen-1)}
		\foreach \i in {0,...,\looplastindex}{
			\pgfmathsetmacro{\itemindex}{int( 4*(\mramfirst+1.5*\buflen+\mramstart+\i) )}
			\node[address] at (\i+\bufxoffset+\mramstart+0.5, \buffyoffset+0.5) { \hexa{\itemindex} };
		}

		% Pointers.
		\def\ptrdist{0.65}

		\draw[ptr, <-] (1.5*\buflen+\mramstart+0.5, 1) -- +(90:\ptrdist) node[above] {\lstinline|from|};

		\draw[ptr, <-] (\bufxoffset+\mramstart+0.5, \buffyoffset) -- +(-90:\ptrdist) node[below] {\lstinline|buffer|};

		\draw[ptr, <-] (\currindex+\bufxoffset+\mramstart+0.5, \buffyoffset) -- +(-90:\ptrdist) node[below] {\lstinline|curr|\vphantom{\lstinline|f|}};
	\end{tikzpicture}

	\caption{
		An exemplary sequential reader with \seqreadcachesize{} set to 16 being used to transfer 32-bit elements from the \ac{MRAM} (top row) into the sequential-read buffer (bottom row).
		The hexadecimal numbers denote the addresses of the respective elements within the \ac{MRAM}.
		Only the elements with addresses from \lstinline|0x28| to \lstinline|0x74| are sought to be read, however, the page model requires that the elements with addresses \lstinline|0x20|, \lstinline|0x24|, and \lstinline|0x78| to \lstinline|0x8c| are also loaded at some point.
		The pointer \lstinline|buffer| constantly points to the beginning of the sequential-read buffer.
		The pointer \lstinline|from| points to the \ac{MRAM} address of the first byte within the buffer.
		The pointer \lstinline|curr| moves from left to right, 4 bytes at a time.
		No byte of the second half of the buffer is ever read as the elements fit perfectly within the first half.
	}
	\label{fig:merge:reader}
\end{figure}

The acclaimed speedup through more performant bounds checks happens within the function \lstinline|seqread_get|, which in turn calls the function \lstinline|__builtin_dpu_seqread_get|.
An inspection of its compilation with \(\seqreadcachesize{} = \twotoi\) reveals the use of a combo operation.
The pointer to the current element is advanced by invoking an \lstinline|add| instruction to increase the stored address.
This \lstinline|add| instruction uses a condition to detect the generation of the \(i\)th carry bit.
To be more precise, this carry bit is set to \(\text{\lstinline|op1[i:0]|} + \text{\lstinline|op2[i:0]|}\), where \lstinline|op1[i:0]| and \lstinline|op2[i:0]| are the \(i + 1\) least significant bits of the involved operands, in this case the pointer and the number of bytes to advance.~\cite[DPU Handbook -- Specific Conditions Common To Addition and Subtraction]{upmemSDK}
Thanks to the carefully chosen size and alignment of the buffer, the generation of such a carry bit signifies that the pointer to the current element has left the first buffer half.
This means that it takes just one instruction to advance the pointer, check the buffer boundaries and jump over \Dash if needed \Dash the subsequent instructions responsible for updating the whole reader.


\clearpage
\section{The Triple Buffer}
\label{sec:mram:triple}

It is yet again beneficial to form starting runs.
This is done by loading a section of the MRAM into the WRAM, sorting it through one of the algorithms presented in \cref{sec:tasklet}, and writing the sorted section back to the MRAM.
As those sorting algorithm rely on the data being present entirely within the WRAM, the functions \lstinline|mram_read| and \lstinline|mram_write| are used directly.
Contrary to the WRAM \MS*{} with starting runs of length 14, the starting runs of the MRAM \MS{} are much longer, easily containing a thousand elements and more.
The reason is that longer starting runs reduce the number of rounds of MRAM merging and, thus, reduce DMAs, which are relatively costly.
However, again similar to the WRAM \MS*{}, it can be beneficial to slightly reduce the starting run length to achieve more balanced and faster rounds.
Nonetheless, the runtime difference between 1100 and 1200 elements per starting run is in the magnitude of one per mille, so for the sake of simplicity, the starting run length is set to the maximum a tasklet can store at once in the WRAM.

%Recall the rough calculation from the beginning of \cref{sec:tasklet} showing that tasklets can hold about \qty{5}{\kibi\byte} of data.
%This space must be used for both forming the starting runs (without any sequential-read buffers) and merging the runs (with two sequential-read buffers).
%One possibility would be to allocate one starting run buffer of size \qty{5}{\kibi\byte}, reset the entire heap after the starting runs have been formed and, then, allocate the sequential-read buffers.

This does raise the question how the starting run formation can allocate a large WRAM buffer while still leaving memory for two sequential-read buffers used during merging.
The answer is a \emph{triple buffer} which consists of a general-purpose buffer followed by two consecutive sequential-read buffers.
If sequential readers are not in use, the triple buffer can be regarded as one contiguous buffer.
To initialise the triple buffer, a tasklet first calls \lstinline|mem_alloc| to allocate \(\cachesize{} + \min\braces{ 8, \text{\lstinline[keywords={}]|sizeof(T)|} }\) many bytes on the heap, where \cachesize{} is a compile-time constant divisible by 8 and~\lstinline|T| is the data type of the input.
This memory, excluding the first \(\min\braces{ 8, \text{\lstinline[keywords={}]|sizeof(T)|} }\) many bytes, is referred to as \emph{cache} and will be used later to store merged runs.
The element right before the beginning of the cache is set to the least possible value of~\lstinline|T| so that it acts as sentinel value for \IS{}, hence the allocation of at least \lstinline[keywords={}]|sizeof(T)| additional bytes.
In case of integers smaller than 8 bytes, the beginning of the cache would not be aligned on 8 bytes anymore, hence the allocation of at least 8 additional bytes to simplify DMAs.
After the initialisation of the cache, the tasklet calls \lstinline|seqread_alloc| twice.
Due to the stack nature of the heap, the two sequential-read buffers are allocated directly behind the cache.
To ensure the contiguity of the triple buffer if more than one tasklet is present, a mutex is employed such that only one tasklet initialises its triple buffer at a time.
The entire triple buffer has the size \(\triplebuffersize \coloneqq \cachesize{} + 4 \times \seqreadcachesize{}\), which is, for simplicity, the minimum guaranteed number of allocated bytes and, therefore, the same for all tasklets even if some calls of \lstinline|seqread_alloc| skipped some bytes.
Note that only the first call of \lstinline|seqread_alloc| by each tasklet may skip some memory since the stack pointer is necessarily at a multiple of \seqreadcachesize{} after any call of \lstinline|seqread_alloc|.

\begin{figure}[b]
	\centering
	\tikzsetnextfilename{triple_buffer}
	\begin{tikzpicture}[reader figure, brace/.style={decorate, decoration={calligraphic brace, amplitude=5.5pt, raise=2pt}, line width=1.1pt}, brace down/.style={brace, decoration={mirror}}]
		\def\cache{6}
		\def\skipped{2}
		\def\buffer{8}

		% Buffer.
		\fill[excess] (0.5*\buffer+\skipped+\cache, 0) rectangle (1*\buffer+\skipped+\cache, 1);
		\fill[excess] (1.5*\buffer+\skipped+\cache, 0) rectangle (2*\buffer+\skipped+\cache, 1);

		\draw (-2, 0) grid (2*\buffer+\skipped+\cache, 1);

		% Braces.
		\draw[brace] (                      0, 1) --+ (  \cache, 0) node[pos=0.5, above=1ex] {cache\vphantom{p}};
		\draw[brace] (                 \cache, 1) --+ (\skipped, 0) node[pos=0.5, above=1ex] {skipped};
		\draw[brace] (        \skipped+\cache, 1) --+ ( \buffer, 0) node[pos=0.5, above=1ex] {first buffer\vphantom{p}};
		\draw[brace] (\buffer+\skipped+\cache, 1) --+ ( \buffer, 0) node[pos=0.5, above=1ex] {second buffer\vphantom{p}};

		\draw[brace down] (0, 0) --+ (2*\buffer+\cache, 0) node[pos=0.5, below=1ex] {\triplebuffersize\vphantom{f}};

		% Item identifiers.
		\pgfmathsetmacro{\looplastindex}{int(2*\buffer+\skipped+\cache+1)}
		\foreach \i in {0,...,\looplastindex}{
			\pgfmathsetmacro{\itemindex}{int( 4*(24+\i) )}
			\node[address] at (\i-1.5, 0.5) { \hexa{\itemindex}\vphantom{f} };
		}

		% Pointers.
		\draw[ptr] (-.5, 1) --+ (90:14.5pt) node[above] {sentinel};
	\end{tikzpicture}

	\caption{
		An exemplary triple buffer with \(\cachesize{} = 24\) and \(\seqreadcachesize{}= 16\) for 32-bit elements.
		The hexadecimal numbers denote WRAM addresses.
	}
	\label{fig:mram:triple}
\end{figure}


\clearpage
\section{A First \texorpdfstring{\MS{}}{MergeSort} on MRAM Data}
\label{sec:mram:merge}

The MRAM \MS{} is based on the half-space WRAM \MS{} as presented in \cref{sec:tasklet:merge} so only the adaptations to the two-tier memory hierarchy are discussed.

\paragraph{Starting Runs}
It is yet again beneficial to form starting runs.
This is done by loading a section of the MRAM into the WRAM, sorting it through one of the algorithms presented in \cref{sec:tasklet}, and writing the sorted section back to the MRAM.
As those sorting algorithm rely on the data being present entirely within the WRAM, sequential readers are not useful until later when merging runs.
Instead, the functions \lstinline|mram_read| and \lstinline|mram_write| are used directly.

Contrary to the WRAM \MS{} with starting runs of length 14, the starting runs of the MRAM \MS{} are much longer, easily containing a thousand elements and more.
The reason is that longer starting runs reduce the number of rounds of the MRAM \MS{} and, thus, reduce the DMAs to the MRAM, which are relatively costly compared to accesses to the WRAM.
According to \citeauthor{mutlu2022Benchmarking}~\cite[8\psq]{mutlu2022Benchmarking}, accessing 64-bit integers in the WRAM is about four and a half times faster than accessing 64-bit integers in the MRAM.
Since accessing 32-bit integers in the WRAM takes just as long as accessing 64-bit integers and since the performance of DMAs depends only on the total number of bytes, there is still a speedup of roughly 2 of WRAM accesses over MRAM accesses.
However, again similar to the WRAM \MS{}, it can be beneficial to slightly reduce the starting run length to achieve more balanced and faster rounds.
Nonetheless, the runtime difference between 1000, 1100, and 1200 elements per starting run is in the magnitude of one per mil.
For this reason, the starting run length is set simply to the maximum number of integers which the WRAM allotted to a tasklet can hold.

This does raise the question what said maximum is.
As calculated at the start of \cref{sec:tasklet}, each tasklet is allotted no more than \qty{5957}{\byte}.
Subtracting \qty{768}{\byte} for the stack of the tasklet and the call stack for the WRAM \QS{} leaves \qty{5189}{\byte} in the heap, that is a little less than 1300 32-bit integers.\todo{Noch einmal später aktualisieren}
This space must be used for both forming the starting runs (without any sequential-read buffers) and merging the runs (with two sequential-read buffers).
One possibility would be to allocate one starting run buffer of size \qty{5957}{\byte}, reset the entire heap after the starting runs have been formed and, then, allocate the sequential-read buffers.

Far more flexibility offers the introduction of a \emph{triple buffer} which consists of a general-purpose buffer followed by two consecutive sequential-read buffers.
If sequential readers are not in use, the triple buffer can be regarded as a single, big buffer.
To initialise the triple buffer, a tasklet first calls \lstinline|mem_alloc| to allocate \(\text{\lstinline|CACHE_SIZE|} + \min\braces{ 8, \text{\lstinline|sizeof(T)|} }\) many bytes on the heap, where \lstinline|CACHE_SIZE| is a user-defined compile-time constant and \lstinline|T| is the data type of the input.
This memory, excluding the first \(\min\braces{ 8, \text{\lstinline|sizeof(T)|} }\) many bytes, is referred to as \emph{cache} and will be used later to store merged runs.
The element right before the beginning of the cache is set to the lowest possible value of \lstinline|T| so that it acts as sentinel value for \IS{}, hence the allocation of at least \lstinline|sizeof(T)| additional bytes.
In case of integers smaller than 8 bytes, the address of the cache would not be aligned on 8 bytes anymore, hence the allocation of at least 8 additional bytes to simplify DMAs.
After the initialisation of the cache, the tasklet calls \lstinline|seqread_alloc| twice.
Due to the stack nature of the heap, the two sequential-read buffers are allocated directly behind the cache.
To ensure the contiguity of the triple buffer if more than one tasklet is present, a mutex is employed to ensure than only one tasklet initialises its triple buffer at a time.
The entire triple buffer has the size \(\text{\lstinline|TRIPLE_BUFFER_SIZE|} \coloneqq \text{\lstinline|CACHE_SIZE|} + 4 \times \text{\lstinline|SEQREAD_CACHE_SIZE|}\), which is, for simplicity, the minimum guaranteed number of allocated bytes and, therefore, the same for all tasklets even if some calls of \lstinline|seqread_alloc| skipped some bytes.
Note that only the first call of \lstinline|seqread_alloc| by each tasklet may skip some memory since the stack pointer is at a multiple of \lstinline|SEQREAD_CACHE_SIZE| after every call of \lstinline|seqread_alloc|.

