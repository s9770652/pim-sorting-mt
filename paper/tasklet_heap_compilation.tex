\subsubsection*{Investigation of the Compilation}
\label{subsubsec:tasklet:heap:compilation}

The reason for the astoundingly poor performance is unclear.
Building the heap makes up about 10\% of the total runtime, so it can be excluded as reason.
One major difference to the other sorting algorithms is that array indices instead of pointers are used.
Nonetheless, this should not make a difference since the manipulation of indices and pointers take equally long, not least because of the \lstinline|lsl_add| instruction, and since the same instructions are used to load and store data.
Even so, this could factor in a suspected deterioration of the compiler optimisations.
Unfortunately, time restrains bar as us from a profound investigation.

While engineering, some strange observations were made.
For example, the runtime difference between stopping \HS{} when only one element remains in the heap and stopping \HS{} when only three elements remain (which then get sorted by \IS{}) can be in the tens of thousands of cycles in both directions.
Stopping \HS{} even earlier has comparatively little effect.
The measurements were conducted without employing \IS{}.

The undisputedly strangest observation was the following:
Before building a heap, a single sentinel leaf must be inserted if the input length is even.
Adding this leaf if the input length is odd makes no difference algorithmically, being a left leaf never to be accessed due to bounds checks.
However, adding an if-statement within which the sentinel leaf is placed has dramatic effects compared to placing the sentinel value unconditionally:
Since the parity of the input length is needed later anyway, the version with the if-statement is expected to gain one instruction.
Yet, when measuring the runtimes on 1024 elements, for example, one can observe anything between a reduction by 5000 cycles over changes within the margin of error to increases by 25\,000 cycles, depending on the implementation and the input distribution.
A comparison of the compilations reveals minute differences at the beginnings of functions, none of which affect loops.
Adding one or more sentinel leaves outside of the functions has no impact on this behaviour.
