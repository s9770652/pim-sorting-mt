\subsection*{Investigation of the Compilation}
\label{sec:tasklet:heap:compilation}

While engineering, some strange observations were made.
For example, the runtime difference between stopping \HS{} when only one element remains in the heap and stopping \HS{} when only three elements remain (which then get sorted by \IS{}) can be in the tens of thousands of cycles in both directions, depending on the sifting direction and the input distribution.
Stopping \HS{} even earlier has comparatively little effect.
\IS{} has ultimately been removed from all implementations.

The undisputedly strangest observation was the following:
Before building a heap, a single sentinel leaf must be inserted if the input length is even.
Adding this leaf if the input length is odd makes no difference algorithmically, as it would be a left leaf never to be accessed due to the bounds checks.
However, adding an if-statement determining whether the sentinel leaf has to be placed has dramatic effects compared to placing the sentinel value unconditionally.
Since the parity of the input length is needed later anyway, the conditional version is expected to gain one instruction.
Yet, when measuring the runtimes on 1024 elements, one can observe anything from a reduction by 5000 cycles over changes within the margin of error to increases by 25\,000 cycles, depending on the sifting direction and the input distribution.
Adding one or more sentinel leaves outside of the \HS{} functions has no impact on this behaviour.
A comparison of the compilations reveals minute differences at the beginnings of the \HS{} functions, none of which affect anything repeatedly executed.
The register usage does also not change in such a manner that the execution time of instructions is prolonged to 12 cycles, as described in the DPU SDK documentation \cite[Instruction Set Architecture -- Efficient scheduling]{upmemSDK}.
