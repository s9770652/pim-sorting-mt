\chapter[Sorting in the \texorpdfstring{\abb{WRAM}}{WRAM}]{Sorting in the \acs*{WRAM}}
\label{sec:tasklet}

This \lcnamecref{sec:tasklet} is concerned with sequentially sorting data which fits into the \ac{WRAM} entirely.
\Cref{sec:tasklet:insertion} discusses \IS{} which is a component of all algorithms presented thereafter as it is lightweight and performant on short inputs.
\Cref{sec:tasklet:shell} covers \ShS{} which is a generalisation of \IS{}.
\hyperref[sec:tasklet:heap]{\nameCrefs{sec:tasklet:heap} }\labelcref{sec:tasklet:heap}, \labelcref{sec:tasklet:quick}, and \labelcref{sec:tasklet:merge} deal with \HS{}, \QS{}, and \MS{}, respectively, which are more elaborate algorithms suitable for long inputs.
These \lcnamecrefs{sec:tasklet:insertion} give a short presentation of their respective algorithm at the beginning, ensued by a discussion of key parameters in their designs.
This is usually followed by an insight into non-algorithmic challenges faced during development caused by the compiler whose optimisations are often of suboptimal quality.
Finally, an evaluation of the performance of the respective algorithm completes each \lcnamecref{sec:tasklet:insertion}.
\Cref{sec:tasklet:conclusion} summarises the findings on the presented algorithms and gives a brief outlook on future improvements.

\Cref{apx:tasklet} contains a comprehensive collection of measurements but the ones essential for following the content of this \lcnamecref{sec:tasklet} are also presented in figures herein.
Every measurement was repeated a thousand times with the sorting algorithms in their default configuration unless explicitly noted otherwise.
The meaning behind and reasoning for the individual parameters in the configurations are subject in \crefrange{sec:tasklet:insertion}{sec:tasklet:merge} but shall be mentioned already for ease of reference:
\begin{description}
	\item[\IS{}]
	explicit sentinel value

	\item[\ShS{}]
	explicit sentinel values;
	step sizes \(\stepsizes = \paren{1, 6}\) for inputs with at most 64 elements and \(\stepsizes = \paren{1, 4, 17}\) for longer ones

	\item[\HS{}]
	top-down for 32-bit integers;
	bottom-up with swap disparity for 64-bit integers

	\item[\QS{}]
	fallback threshold of 18 elements;
	random medians as pivots;
	prioritisation of right-hand partitions over left-hand partitions;
	iterative for 32-bit integers;
	recursive for 64-bit integers;
	\Cref*{imp:triviality_before_call}

	\item[\MS{}]
	half-space;
	starting run length of 14 elements
\end{description}

The measurements were confined to at most \num{1024} elements.
The reason is that \qty{64}{\kibi\byte} of \ac{WRAM} are available and that most kernels will run with at least 11 tasklets to saturate the instruction pipeline.
In consequence, at most \qty{5957}{\byte} are allotted to each tasklet.
To account for the tasklet stack, this number is reduced by \qty{600}{\byte}, leaving space for about \num{1339} 32-bit elements.
On that score, 1024 elements is a reasonable cutoff.

\section{\texorpdfstring{\IS{}}{InsertionSort}}
\label{sec:tasklet:insertion}

This stable sorting algorithm works by moving the \(i\)th element leftwards as long as its left neighbour is greater, assuming that the elements at the indices \(0\) to \(i - 1\) are already sorted.
Its asymptotic runtime is bad, reaching \(\bigoh*{n^2}\) not only in the worst case but also in the average case, where each of the \(\binom{n}{2}\) pairs of input elements is in wrong order and will be swapped at some point in the execution with probability 50\%.
Nonetheless, \IS{} does have some saving graces:
\begin{enumerate*}
	\item
	If the input array is mostly or even fully sorted, the runtime drops down to \(\bigoh{n}\).

	\item
	It works in-place, needing only \(\bigoh{1}\) additional space.

	\item
	Its program code is short, lending itself to inlining.

	\item
	The overhead is small.
\end{enumerate*}
Especially the last two points make \IS{} a good fall-back algorithm for asymptotically better sorting algorithms to use on short subarrays.

\paragraph{Sentinel Values}
When moving an element to the left, two checks are needed:
Does the left neighbour exist and is it smaller than the element to move?
The first check can be omitted through the use of \emph{sentinel values}:
If the element at index \(-1\) is the smallest possible value of the chosen data type, it is at least as small as any value in the input array, and the leftwards motion stops there at the latest.
Since a DPU has no branch predictor, the slowdown from performing twice as many checks as needed is quite high and lies at about 30\% for short inputs (\cref{fig:insertion:against_others}).

Setting such an \emph{explicit} sentinel value can be omitted by using \emph{implicit} sentinel values.
At the start of each round, one can check if the element at index \(0\) is at least as small as the element at index~\(i\).
If yes, the former is a sufficient sentinel value, and \IS{} can proceed as normal.
If not, the latter must be the minimum of the first \(i + 1\) elements and, therefore, can be moved immediately to the front.

\subsection{Evaluation of the Performance}
\label{sec:tasklet:insertion:performance}

\def\insertionalgos{1NoSentinel,1,1Implicit,BubbleNonAdapt,BubbleAdapt,Selection}

\expandafter\pgfplotsinvokeforeach\expandafter{\alldists}{
	\pgfplotstablereadnamed{data/small sorts/uint32/#1.txt}{tableSmallSorts_32#1}
	\pgfplotstablereadnamed{data/small sorts/uint64/#1.txt}{tableSmallSorts_64#1}
}

The runtimes of the three \IS*{} can be compared in the \cref{fig:insertion:against_others,fig:insertion:against_others_uint32,fig:insertion:against_others_uint64}.
The sentinel-less \IS{} is consistently worse than the explicit one.
For most input distributions, the implicit \IS{} is also a bit slower, as it effectively performs one check more for each element.
Of course, the gap becomes less significant with increasing input length as the other operations of the loops dominate the runtime.

An outlier, however, are the reverse sorted inputs.
For 32-bit numbers (\cref{fig:insertion:against_others}), the speedup\footnote{
	The \emph{speedup} \(S\) of an algorithm~\(A\) over an algorithm~\(B\) is defined as the ratio \(\operatorname{t}\mkern1mu(B\mkern1mu) / \operatorname{t}\mkern1mu(A)\) of their runtimes \(\operatorname{t}\mkern1mu(A)\) and \(\operatorname{t}\mkern1mu(B\mkern1mu)\).
	Values below 1 indicate that algorithm~\(A\) runs slower than algorithm~\(B\).
} of the implicit \IS{} over the explicit one drops down to as little as \num{0.68}.
This comes as a surprise since both versions effectively execute the same loop body while shifting everything one position backwards, with only the loop condition being different.
Due to \acp{DPU} being unit-cost machines, a value check on whether the preceding element is less (explicit \IS{}) and an address check on whether the preceding position is the start of the array (implicit \IS{}) should take the same amount of time.
Yet, even the sentinel-less \IS{} surpasses the implicit \IS{}, despite doing both value checks and address checks.
For 64-bit numbers (\cref{fig:insertion:against_others_uint64}), the implicit \IS{} would be expected to perform better than the explicit one, considering that a value check now takes two instructions and an address check still only one.
Nonetheless, the two \IS*{} tie.
This constitutes another case of bad compilation.
We did not troubleshoot as the explicit \IS{} would still be expected to offer superior performance in most cases.
The explicit \IS{} is, therefore, used in the rest of this thesis and referred to plainly as \enquote{\IS{}} henceforth.

\begin{figure}
	\tikzsetnextfilename{insertion_against_others}
	\begin{tikzpicture}[plot]
		\begin{groupplot}[
			adaptive group=1 by 2,
			groupplot ylabel={Cycles / \(n^2\)},
			xtick distance=3,
			minor xtick=data,
			ymin=0,
			ymax=60,
			legend columns=3,
		]
			\nextgroupplot[title/.add={}{Reverse Sorted}]
			\pgfplotsset{legend to name=leg:insertion:against_others, legend entries={\IS{} (sentinel-less), \IS{} (explicit), \IS{} (implicit), \BS{} (not adaptive), \BS{} (adaptive), \SelS{}}}
			\expandafter\pgfplotsinvokeforeach\expandafter{\insertionalgos}{
				\plotpernn{#1}{tableSmallSorts_32reverse}
			}
			%
			\nextgroupplot[title/.add={}{Uniform}]
			\expandafter\pgfplotsinvokeforeach\expandafter{\insertionalgos}{
				\plotpernn{#1}{tableSmallSorts_32uniform}
			}
		\end{groupplot}
	\end{tikzpicture}

	\tikzexternaldisable\hfil\pgfplotslegendfromname{leg:insertion:against_others}\hfil\tikzexternalenable
	\caption{
		Mean runtimes of sorting algorithms with a runtime in \(\bigoh*{n^2}\) on 32-bit integers.
	}
	\label{fig:insertion:against_others}
\end{figure}

\begin{note}
	Other known simple sorting algorithm are \SelS{} and \BS{}.
	\emph{\SelS{}}~\cites[83]{maurer1974datenstrukturen}[Section~2.2.2]{wirth1975algorithmen} assumes, like \IS{}, that the elements with indices \(0\) to \(i - 1\) are already sorted in round~\(i\).
	It scans the elements with indices \(i\) to \(n\) and finds their minimum.
	Then, it swaps a minimum element with the element with index \(i\).
	\emph{\BS{}}~\cite[Section~2.2.3]{wirth1975algorithmen} scans the elements with indices \(0\) to \(n - i + 1\) and swaps each pair of neighbouring elements if they are in the wrong order.
	An easy extension is adaptive \BS{} which sorts only up to the position of the last swap.

	The average-runtime complexity of \SelS{} and \BS{} is the same as that of \IS{}.
	The asymptoticity, however, hides much higher constant factors such that \IS{} should always be preferred, as seen in \cref{fig:insertion:against_others,fig:insertion:against_others_uint32,fig:insertion:against_others_uint64}.
	Consequently, they will not be expanded on further in this thesis.
\end{note}


\subsection*{Investigation of the Compilation}
\label{sec:tasklet:insertion:compilation}
\addcontentsline{toc}{subsection}{\nameref{sec:tasklet:insertion:compilation}}

A common theme when developing for DPUs is a nosediving quality of the compilation.
This is no different for \IS{} upon which shall be shed some light in this \lcnamecref{sec:tasklet:insertion:compilation}.

A naïve implementation of \IS{} begins sorting at the very start of the input and is shown in \cref{fig:insertion:impl:pred_first}.
Obviously, the first element alone is already sorted, so it is algorithmically sound to let \IS{} begin at the second element.
This optimisation is accomplished in \cref{fig:insertion:impl:pred_sec}.
Surprisingly, it leads to a nine instructions longer runtime at 16 integers!
The same happens if, in \cref{fig:insertion:impl:pred_sec}, one keeps \lstinline|*i = start| and instead uses \lstinline|curr = ++i|.

Looking at the compilation reveals the reason:
In the naïve version, the pointer \lstinline|pred| is optimised away and, in its stead, the pointer \lstinline|curr| is passed to all load operations together with a constant offset as second argument.
In the optimised version, the pointer \lstinline|pred| is used with an offset to fetch the values of \lstinline|to_sort| and \lstinline|*pred| at the beginning of each iteration of the outer loop.
Then, the pointer \lstinline|curr| is initialised using the pointer \lstinline|pred| before being used in the inner loop as in the naïve version.
This initialisation is done through one additional \lstinline|move| instruction.
% This is a consequence of reusing the register of the \lstinline|start| pointer for \lstinline|pred| instead of for \lstinline|i|, whose incremented value is put into another register.

These changes fully explain the prolongation of the runtime by nine instructions:
The optimised version loops 15 times in total, each time laboriously initialising the pointer \lstinline|curr|, and executes one \lstinline|add| instruction at the beginning of the function to advance the starting position.
The naïve version loops 16 times, the first time executing seven instructions for naught.

\begin{figure}
	\lstset{basicstyle=\ttfamily\small}
	\def\iscodewidth{0.47\linewidth}
	\begin{subfigure}{\iscodewidth}
		\begin{lstlisting}
void InsertionSort(int *start, int *end) {
	int *curr, *i = start;
	while ((curr = i++) <= end) {
		int to_sort = *curr;
		int *pred = curr - 1;
		while (*pred > to_sort) {
			*curr = *pred;
			curr = pred--;
		}
		*curr = to_sort;
	}
}
		\end{lstlisting}
		\caption{
			Start at the first element and with predecessor pointer.
		}
		\label{fig:insertion:impl:pred_first}
	\end{subfigure}
	\hfill
	\begin{subfigure}{\iscodewidth}
		\begin{lstlisting}
void InsertionSort(int *start, int *end) {
	int *curr, *i = start + 1;
	while ((curr = i++) <= end) {
		int to_sort = *curr;
		int *pred = curr - 1;
		while (*pred > to_sort) {
			*curr = *pred;
			curr = pred--;
		}
		*curr = to_sort;
	}
}
		\end{lstlisting}
		\caption{
			Start at the second element and with predecessor pointer.
		}
		\label{fig:insertion:impl:pred_sec}
	\end{subfigure}

	\begin{subfigure}{\iscodewidth}
		\begin{lstlisting}
void InsertionSort(int *start, int *end) {
	int *curr, *i = start;
	while ((curr = i++) <= end) {
		int to_sort = *curr;
		while (*(curr - 1) > to_sort) {
			*curr = *(curr - 1);
			curr--;
		}
		*curr = to_sort;
	}
}
		\end{lstlisting}
		\caption{
			Start at the first element and without predecessor pointer.
		}
		\label{fig:insertion:impl:offset_first}
	\end{subfigure}
	\hfill
	\begin{subfigure}{\iscodewidth}
		\begin{lstlisting}
void InsertionSort(int *start, int *end) {
	int *curr, *i = start + 1;
	while ((curr = i++) <= end) {
		int to_sort = *curr;
		while (*(curr - 1) > to_sort) {
			*curr = *(curr - 1);
			curr--;
		}
		*curr = to_sort;
	}
}
		\end{lstlisting}
		\caption{
			Start at the second element and without predecessor pointer.
		}
		\label{fig:insertion:impl:offset_sec}
	\end{subfigure}
	\caption{
		Four different implementations of \IS{} in C.
		\Cref{fig:insertion:impl:pred_first,fig:insertion:impl:offset_first} are compiled the same.
		\Cref{fig:insertion:impl:pred_sec,fig:insertion:impl:offset_sec} are compiled differently.
	}
\end{figure}

Multiple workarounds exist to sidestep this problem.
One workaround is to take the unoptimised code and change the starting position via inline assembler.
This is trivial for the explicit \IS{} since one can simply inject an \lstinline|add| instruction at the beginning of the function to increment the pointer \lstinline|start|.
The implicit and the sentinel-less \IS*{} need to know the original starting address \lstinline|start| later on, though, and initialise the actual starting point rather late;
injecting inline assembler proves more difficult as a consequence.
Moreover, as \IS{} is to be used as fallback algorithm by other functions which might also need to keep the original value of \lstinline|start|, inline assembler is a bad choice even for the explicit \IS{}.

Another workaround is the usage of a wrapper function calling \IS{} with the arguments \lstinline|start + 1| and \lstinline|end|.
This works quite well:
First, the register holding \lstinline|start| is incremented, and, then, the inlined code from the actual \IS{} follows.
Doing so makes the runtime drop as expected.

Recall how in the faster version (\cref{fig:insertion:impl:pred_first}), the pointer \lstinline|pred| is always deduced from the pointer \lstinline|curr| using an offset.
This gives the cue for yet another workaround:
In \cref{fig:insertion:impl:offset_first,fig:insertion:impl:offset_sec}, every occurrence of \lstinline|pred| is replaced with \lstinline|curr - 1|.
As a consequence, the code of \cref{fig:insertion:impl:offset_first} compiles the very same as the one of \cref{fig:insertion:impl:pred_first}, while \cref{fig:insertion:impl:offset_sec} yields the same compilation as the versions with the wrapper function or the inline assembly.
This workaround is clearly the best of the three and, hence, the one used in the rest of this thesis.

Alas, the eternal struggle with the compiler endeth not herewith.
A deeper look into the compilation reveals the following sequence:
\begin{center}
	\begin{tabular}{ll}
		\lstinline|move r3, r0| & \makebox[0pt][l]{\textit{// copy content of register \lstinline|r0| to \lstinline|r3|}}
		\\ \lstinline|jleu r4, r2, .LABEL| & \makebox[0pt][l]{\textit{// jump to \lstinline|.LABEL| if \lstinline|r4| \(\le\) \lstinline|r2|}}
		\\ \lstinline|move r3, r0| &
	\end{tabular}
\end{center}
Without delving further into its significance \Dash the second \lstinline|move r3, r0| is unneeded as it is impossible to jump directly to it nor to return via \lstinline|jleu|.
Also, \lstinline|move| does not set any flags like the zero flag or carry flag, as some other instructions do, so such a side effect can be excluded as justification.
Copying the whole assembler code and injecting it as inline assembler but with this second \lstinline|move r3, r0| removed pushes the runtime even further down whilst still sorting correctly.
New issues, especially for inlining, are introduced, though, and we deem a proper assembly implementation as out of scope for this thesis.



\section{\texorpdfstring{\ShS{}}{ShellSort}}
\label{sec:tasklet:shell}

\IS{} suffers from little elements in the back of the input, since those have to be brought to the front through \(\bigtheta{n}\) comparisons and swaps.
\ShS{}~\cites{Shell1959AHS}[Chapter~2.2.4]{wirth1975algorithmen} circumvents this by doing \(k\) passes of \IS{} with decreasing step sizes:
In pass~\(p = 1, \dots, k\) with step size \(\stepsizes_{k - p}\), the input array is divided into \(\stepsizes_{k - p}\) subarrays so that the \(i\)th subarray contains the elements with indices \(\paren{i, \: i + \stepsizes_{k - p}, \: i + 2 \stepsizes_{k - p}, \: \dots}\), for \(0 \le i < \stepsizes_{k - p}\).
These subarrays then get sorted individually through \IS{}.
The final step size is \(\stepsizes_0 = 1\) such that a regular \IS{} is performed.
Intuitively, early \IS*{} are fast as they touch only few elements and little elements in the back are brought forward in large strides.
Later \IS*{} are also fast as elements are close to being sorted.
Like regular \IS{}, \ShS{} also works in place but loses the stability property.

Finding the right balance between the heightened overhead through multiple \IS{} passes and the shortened runtime of each \IS{} pass is subject to research to this day \cite{skean2023optimization,lee2021empirically} and depends on the cost of the operation types (comparing, swapping, looping).
Traditionally, step sizes were constructed mathematically, allowing to determine \ShS{}'s runtime to be, for example, \(\bigoh[\big]{n^{1.2}}\)~\cite[106]{wirth1975algorithmen} or \(\bigoh[\big]{n \log^2 n}\)~\cite[Section 2]{skean2023optimization}, that is better than \IS{}.
Nowadays, well-performing step sizes are identified empirically~\cite{10.1007/3-540-44669-9_12,skean2023optimization,lee2021empirically}, making a generalisation and, thus, asymptotic analysis more difficult.

\subsection*{Evaluation of the Performance}
\label{sec:tasklet:shell:performance}
\addcontentsline{toc}{subsection}{\nameref{sec:tasklet:shell:performance}}

\pgfplotstablereadnamed{data/shell/two-tier/uint32/reverse.txt}{tableShellTwo_32reverse}
\pgfplotstablereadnamed{data/shell/two-tier/uint32/uniform.txt}{tableShellTwo_32uniform}
\pgfplotsinvokeforeach{7,...,17}{
	\pgfplotstablereadnamed{data/shell/h1=#1/uint32/reverse.txt}{tableShell#1_32reverse}
	\pgfplotstablereadnamed{data/shell/h1=#1/uint32/uniform.txt}{tableShell#1_32uniform}
}

Let us first focus on short inputs where only two passes with step sizes~\(\stepsizes = \paren{1, \stepsizes_1}\) suffice.
The previous results on \IS{} suggest that a fast \ShS{} should make use of~\(\stepsizes_1\) sentinel values.
\Cref{fig:shell:two-tier,fig:shell:two-tier_uint32,fig:shell:two-tier_uint64} show that, with the exception of the \ShS{} with step size \(\stepsizes_1 = 2\), the additional pass start to pay off at around 16 elements for both 32-bit and 64-bit values with the fully random input distributions, reaching a speedup of around 115\%--120\% at 24 elements.
In case of the reverse sorted input, the speedup is practically always positive even for very short inputs, reaching around 125\%--210\% at 24 elements.
On sorted and almost sorted inputs, \ShS{} exhibits a slowdown in the tested range of input lengths.

\begin{figure}
	\tikzsetnextfilename{shell_two-tier}
	\begin{tikzpicture}[plot]
		\begin{groupplot}[
			horizontal sep for labels,
			adaptive group=1 by 2,
			groupplot xlabel={Input Length \(n\)},
			xtick distance=3,
			minor xtick=data,
			title={},
		]
			\nextgroupplot[ylabel=Cycles / \(n^2\), ymin=0, ymax=50, ytick distance=10]
			\pgfplotsset{legend to name=leg:shell:two-tier, legend entries={\(1\), \(...\), \(9\)}}
			\pgfplotsinvokeforeach{1,...,9}{
				\plotpernn[x filter/.expression={x > #1 ? x : nan}]{#1}{tableSmallSorts_32uniform}
			}
			%
			\nextgroupplot[ylabel=Speedup, ymin=0.7, ymax=1.2, ytick distance=0.1, yticklabel style={/pgf/number format/.cd, precision=1, fixed, zerofill}]
			\pgfplotsset{cycle list shift=1}
			\pgfplotsinvokeforeach{2,...,9}{
				\plotspeedup[x filter/.expression={x > #1 ? x : nan}]{#1}{1}{tableSmallSorts_32uniform}
			}
		\end{groupplot}
	\end{tikzpicture}

	\hfil\pgfplotslegendfromname{leg:shell:two-tier}\hfil
	\caption{
		Comparison of \IS{} (\(1\)) and various two-tier \ShS*{} (\(2\)--\(9\)), whose step sizes \(\stepsizes_1\) are indicated by their labels, on uniformly distributed 32-bit integers.
		The speedups are with respect to the \IS{}.
	}
	\label{fig:shell:two-tier}
\end{figure}

\NewDocumentCommand{\shellscatter}{m m m}{
	\addplotnamedtable[select row={#1}, forget plot][x=µ_#2, y expr={6}]{tableShellTwo_#3};
	\ifnumless{#2}{7}{
		\addplotnamedtable[select row={#1}, forget plot][x=µ_#2, y expr={7}]{tableShell7_#3};
	}{}
	\ifnumless{#2}{8}{
		\addplotnamedtable[select row={#1}, forget plot][x=µ_#2, y expr={8}]{tableShell8_#3};
	}{}
	\ifnumless{#2}{9}{
		\addplotnamedtable[select row={#1}, forget plot][x=µ_#2, y expr={9}]{tableShell9_#3};
	}{}
	\addplotnamedtable[select row={#1}][x=µ_#2, y expr={10}, forget plot]{tableShell10_#3};
	\addplotnamedtable[select row={#1}][x=µ_#2, y expr={11}, forget plot]{tableShell11_#3};
	\addplotnamedtable[select row={#1}][x=µ_#2, y expr={12}, forget plot]{tableShell12_#3};
	\addplotnamedtable[select row={#1}][x=µ_#2, y expr={13}, forget plot]{tableShell13_#3};
	\addplotnamedtable[select row={#1}][x=µ_#2, y expr={14}, forget plot]{tableShell14_#3};
	\addplotnamedtable[select row={#1}][x=µ_#2, y expr={15}, forget plot]{tableShell15_#3};
	\ifnumgreater{#1}{16}{
		\addplotnamedtable[select row={#1}][x=µ_#2, y expr={16}, forget plot]{tableShell16_#3};
		\addplotnamedtable[select row={#1}][x=µ_#2, y expr={17}]{tableShell17_#3};
	}{
		\addplotnamedtable[select row={#1}, opacity=0][x=µ_#2, y expr={17}]{tableShell15_#3};
	}
}

\pgfplotsset{
	shell scatter plot/.style={
		adaptive group=3 by 2,
		groupplot xlabel={Runtime [\(10^4\) Cycles]},
		groupplot ylabel={\(\stepsizes_2\)},
		scaled x ticks=base 10:-4,
		xtick scale label code/.code={},  % removes exponent underneath the axis
		ytick={6, 7, 9, ..., 15},
		yticklabels={/, \(7\), \(9\), \(...\), \(17\)},
		minor ytick={8, 10, ..., 13},
		/tikz/only marks,
		cycle list shift=2,  % for sharing colours with the previous figure
		xmajorgrids=false,
		title/.add={}{Input Length \textit{n} =},
		legend columns=-1,
	},
}

\begin{figure}[p]
	\tikzsetnextfilename{shell_three-tier}
	\begin{tikzpicture}[plot]
		\newcommand{\type}{32uniform}
		\begin{groupplot}[shell scatter plot]
			\pgfplotsinvokeforeach{16,32,48,64,96,128}{
				\nextgroupplot[title/.add={}{ #1}]
				\pgfplotsforeachungrouped\h in {3,...,9}{
					\shellscatter{#1}{\h}{\type}
				}
				\ifnumgreater{#1}{17}{ \pgfplotsset{extra y ticks={17}, minor ytick={8, 10, ..., 16}} }{}
			}
			\pgfplotsset{legend to name=leg:shell:three-tier, legend entries={\(\stepsizes_1 = 3\), \(\stepsizes_1 = 4\), \(\stepsizes_1 = 5\), \(\stepsizes_1 = 6\), \(\stepsizes_1 = 7\), \(\stepsizes_1 = 8\), \(\stepsizes_1 = 9\)}}
		\end{groupplot}
	\end{tikzpicture}

	\hfil\pgfplotslegendfromname{leg:shell:three-tier}\hfil
	\caption{
		Mean runtimes of two-tier and three-tier \ShS*{} on uniformly distributed 32-bit integers.
		The two-tier \ShS*{} are situated on the lowest file, which is labelled \enquote{/}.
		The three-tier \ShS*{} are situated on the files above, whose \(y\) values indicate the step size \(\stepsizes_2\).
		The coloured symbols encode the step sizes \(\stepsizes_1\).
	}
	\label{fig:shell:three-tier}
\end{figure}

When moving to greater input lengths (\cref{fig:shell:three-tier,fig:shell:three-tier_uint32reverse,fig:shell:three-tier_uint32uniform,fig:shell:three-tier_uint64reverse,fig:shell:three-tier_uint64uniform}), the differences in performance between the two-tier \ShS*{} become more pronounced;
%especially the ones with \(\stepsizes_1 = 3\) and \(\stepsizes_1 = 4\) fall off whereas the one with \(\stepsizes_1 = 6\) holds its ground quite well.
Between 48 and 64 elements, three passes get worthwhile to consider.
%Interestingly, many \ShS*{} with \(\stepsizes_2 = 4\) take the lead whilst the ones with \(\stepsizes_2 = 6\) are mid-table.
On the one hand, the findings are in accordance with the well-known ones by \citeauthor{10.1007/3-540-44669-9_12}~\cites{10.1007/3-540-44669-9_12}[cf.][]{skean2023optimization} who, for 128 elements, determined \(\stepsizes = \paren{1, 9}\) to be the optimal pair and \(\stepsizes = \paren{1, 4, 17}\) to be the optimal triplet, which is also the case in \cref{fig:shell:three-tier}.
On the other hand, the gain from doing three passes is far smaller:
\Citeauthor{10.1007/3-540-44669-9_12} calculated an average speedup of 133\% over doing two passes, while it is only 116\% on a DPU.
In opposition to his findings, this also makes it unlikely that doing four passes would already net any gain at this input length.
This shows that past findings on non-uniform cost models cannot be applied one-to-one to DPUs.

But would pushing the limits of \ShS{} even be rewarding?
Greater input lengths require greater steps \Dash probably well into the three digits for \(n \approx 1000\) \cite{skean2023optimization,10.1007/3-540-44669-9_12} \Dash and those in turn require more sentinel values.
Implicit sentinel values could provide relief since the slowdown from implicitness should trend to zero for longer inputs, as was the case for \IS{}.
Still, finding the best step sizes for longer inputs requires a lot more work because the length and, thus, the number of reasonable combinations of step sizes become larger.
Unfortunately, longer optimal tuples cannot be constructed straightforwardly from shorter optimal ones, as seen in this \lcnamecref{sec:tasklet:shell:performance}.

Its application, on the other hand, would likely be niche.
\ShS{} is outperformed by other algorithms presented hereafter, and those have no use for a \ShS{} adjusted to longer inputs.
Its only silver lining could be its in-place property (especially when relying solely on implicit sentinel values) combined with its medium speed, as discussed in \cref{sec:tasklet:conclusion}.



\section{\texorpdfstring{\HS{}}{HeapSort}}
\label{sec:tasklet:heap}

\HS{}~\cites{floyd1964treesort}{williams1964heapsort}[Chapter~2.2.5]{wirth1975algorithmen} makes use of a so-called \emph{heap}, which is a priority queue allowing to retrieve and remove the maximum element stored in time \(\bigoh{\log n}\).
Repeated retrieval and removal of the maximum allows to sort in place in time \(\bigoh{n \log n}\), although the sorting is not stable.

A heap (or, more specifically, a \emph{binary max-heap}) is a binary tree of logarithmic depth whose layers are fully filled, that is, the layer of depth \(i\) contains \(\twotoi\) vertices.
The only exception is the last layer, which may contain less vertices but must be filled from left to right.
In the context of \HS{}, the vertices are identified with the elements to sort.
The \emph{heap order} dictates that each parent must be at least as great as its child.
Consequently, the root is a greatest element.
A heap with \(n\) vertices can be represented as an array of length \(n\) using a bijective mapping between the vertices and the array indices:
If the root is stored at position \(1\), the children of the vertex with index \(i\) have the indices \(\twoi\) and \(\twoi + 1\), whilst its parent has index \(i \div 2\), where the obelus (\(\div\)) denotes an integer division.

After the heap has been built in place from the input array in time \(\bigoh{n}\), the sorting works as follows:
At the start of round~\(r = 1, \dots, n\), the first \(n - (r - 1)\) elements of the array represent the heap and the last~\(r - 1\) elements the end of the sorted output.
The root, which is the \(r\itordinal\)th greatest element of the input, gets removed and, since the heap cannot contain holes, a reparation procedure is performed.
Since the heap has shrunken by one vertex, the removed root can be stored at index \(n - (r - 1)\), that is the freed-up position directly behind the end of the heap.

\subsection{Presentation of Key Aspects}
\label{sec:tasklet:heap:aspects}

\paragraph{Sifting Direction}
Once the heap is built, the \emph{top-down} \HS{} proceeds as follows:
At the start of each round, the root and the rightmost leaf in the bottom layer (\enquote{last leaf}) swap places.
The root is now in the correct output position, but the former last leaf may violate the heap order, that is, the root may be less than one or both of its children.
The greater of the two children is determined, and the root and the greater child swap places.
This downwards-sifting of the former last leaf continues iteratively until the heap order is restored.

In contrast, the \emph{bottom-up} \HS{}~\cite{wegener1993heapsort} works as follows:
At the start of each round, the root is removed so that a hole sits now at the top of the heap.
Then, the hole and the greater of its two children swap places.
This downwards-sifting of the hole continues iteratively until it becomes a leaf.
Now, the last leaf is moved to the position of the hole.
Should this former last leaf be greater than its new parent, then the heap order is now violated.
It needs to be sifted upwards by iteratively swapping positions with its respective parent until the heap order is restored.
At last, the original root element can be put where the former last leaf used to be.

The motivation behind these variants is at follows:
In each step where the top-down \HS{} sifts a former last leaf downwards, two value checks (Which child is greater? Is the parent less than the greater child?) need to be done.
The leaves of a heap tend to be little so the downwards-sifting normally lasts awhile.
As opposed to this, each step of the bottom-up \HS{} needs only one value check (Which child is greater?).
Both \HS*{} sift downwards similarly long so many checks can be saved.
Since the last leaf effectively takes the place of another leaf and since both are likely little, the upwards-sifting should be short-lived and not eat the gain up.

The upwards-sifting reverts some of the changes done by the downwards-sifting.
The bottom-up \HS{} can be brought to swap parity with the top-down \HS{} by the following change:
The downwards-sifting is traced but the vertices are not actually moved.
Once the leaf where the hole would end up is reached, the sifting is backtracked until the bottommost vertex which is greater than the last leaf.
The position found is where the last leaf would end up after the upwards-sifting, so all vertices below can stay put and all vertices above move to their parents' positions, that is, thither the swaps from the downwards-sifting would have put them.
This implementation variant makes the downwards-sifting even cheaper, but the upwards-sifting must now go all the way up to the root.


\paragraph{Sentinel Values}
When \HS{} sifts a vertex downwards, it needs to determine the greater of its two children before deciding whether and whither to move.
If and only if the heap has an even number of vertices, there is a left child without a right sibling:
the rightmost leaf in the bottom layer.
Instead of adding some check on whether the right sibling exists, one can rather add the missing leaf and set it to the least possible value each time the heap reaches an even size.
Thus, if it has been confirmed that a left child exists, a right one does also exist.
%It is necessary now that if two siblings are equal, the left one should be considered greater, lest the sentinel leaf be touched.
Bounds checks on whether a left child exists are still required lest \HS{} loses its in-place property, since there are about \(n/2\) leaves of which all would need sentinel children.

Likewise, whenever \HS{} sifts upwards and considers the parent \(i \div 2\) of a vertex \(i\), it will only proceed if the parent is less.
Since the parentless root has index \(1\), the formula \(i \div 2\) yields~\(0\), so it makes sense to set the element with index \(0\) to the greatest possible value to stop any upwards-sifting.
The speedup from using sentinel values is about \num{1.15}.


\paragraph{Code Duplication}
A strategy particularly useful for \HS{}, although also employed in other sorting algorithms, is \emph{code duplication}.
Downwards-sifting can be broken down into two steps:
\begin{enumerate*}
	\item
	Find the greater child.

	\item
	Perform some operations on said child.
\end{enumerate*}
A natural, low-maintenance implementation would determine the greater child and, then, store it in a variable on which the operations are performed afterwards.
However, the quality of the compilation improves if the operations are implemented twice, once for either child, and executed conditionally.
This approach led to a speedup of about \num{1.07}.


\paragraph{Base Cases}
When 15 elements or fewer remain in the heap, \IS{} is used to sort them.
Admittedly, the impact of this one-time use is rather negligible, and \ShS{} would not make much of a difference.
%However, it serves as reminder to the fragility of the quality of the compilation, as shown in the following part.


\subsection*{Investigation of the Compilation}
\label{sec:tasklet:heap:compilation}
\addcontentsline{toc}{subsection}{\nameref{sec:tasklet:heap:compilation}}

Under zero-based indexing, the indices of the sons of a vertex with index \(i\) are \(2 i + 1\) and \(2 i + 2\), whilst the one of its father is \(\floor{(i - 1)/ 2}\).
Under one-based indexing, the indices of the sons of a vertex with index \(i\) are \(2 i\) and \(2 i + 1\), whilst the one of its father is \(\floor{i / 2}\).
The formula \(\floor{i / 2}\) is computable through a bitwise shift one place to the right, whereas \(\floor{(i - 1)/ 2}\) requires a subtraction before the bitwise shift.
Since the bottom-up \HS*{} rely heavily on finding fathers during backtracking, one-based indexing is clearly superior.

Consistency alone would suggest one-based indexing for all types of \HS{}.
However, the first \HS{} implemented was the top-down \HS{}, which only ever sifts down.
The picture is not so clear if focussing only on that version of \HS{}.
The compiler automatically turns multiplications by \(2\) into a bitwise shift by one place to the left.
Next to a regular \lstinline|lsl| instruction for such bitwise shifts to the left, DPUs also possess an instruction called \lstinline|lsl_add| which first shifts to the left and then adds a number.
This way, the formulas \(2i + 1\) and \(2i\) take the same amount of time to compute.

Notwithstanding \lstinline|lsl_add| being indeed employed in the compilation, the zero-based indexing is about 7\% slower than one-based indexing.
The reason is that only the number of places to shift can be passed as immediate value, that is, as plain number, but not the addend, which must be passed via a register.
Whilst DPUs have a read-only register permanently storing the number~\(1\) at disposal, read-only registers can only ever be the first register argument, not the second one, which is the addend in case of \lstinline|lsl_add|.
As a consequence, the compiler moves the number \(1\) to a register whenever \(2i + 1\) is to be computed, only to immediately overwrite it with the result from \lstinline|lsl_add|.
Hence, the calculation of \(2i + 1\) does take twice as long as \(2 i\) after all.

There are plenty of other curious observations.
For example, the runtime difference between stopping \HS{} when one element remains in the heap and stopping \HS{} when only three elements remain (which then get sorted by \IS{}) reduces the runtime by tens of thousand of cycles.
Stopping \HS{} even earlier has comparatively little effect.
For comparison, sorting just three elements solely with \HS{} barely takes one thousand cycles at worst.

The undisputedly strangest observation was the following:
Before building a heap, a single sentinel leaf must be inserted if the input length is even.
Adding this leaf if the input length is odd makes no difference algorithmically, as it would be a left leaf never to be accessed due to the bounds checks.
However, adding an if-statement determining whether the sentinel leaf has to be placed has dramatic effects compared to placing the sentinel leaf unconditionally.
Since the parity of the input length is needed later in the function anyway, the conditional version is expected to gain one instruction.
Yet, when measuring the runtimes on 1024 elements, one can observe anything from a reduction by 5000 cycles over changes within the margin of error to increases by 25\,000 cycles, depending on the sifting direction and the input distribution.
Adding one or more sentinel leaves outside of the \HS{} functions has no impact on this behaviour.
A comparison of the compilations reveals minute differences at the beginnings of the \HS{} functions, none of which affect anything repeatedly executed.
The register usage does also not change in such a manner that the execution time of instructions is prolonged to 12 cycles, as described in the DPU SDK documentation \cite[Instruction Set Architecture -- Efficient scheduling]{upmemSDK}.


\subsection*{Evaluation of the Performance}
\label{sec:tasklet:heap:performance}

\pgfplotsinvokeforeach{sorted,reverse,almost,uniform,zipf,normal}{
	\pgfplotstablereadnamed{data/heap/uint32/#1.txt}{tableHeap_32#1}
	\pgfplotstablereadnamed{data/heap/uint64/#1.txt}{tableHeap_64#1}
}

\pgfplotsset{
	heap/.style={
		adaptive group=1 by 2,
		groupplot ylabel={Cycles / \((n \lb n)\)},
		x from 16 to 1024,
	},
}

\def\heapalgos{HeapOnlyDown,HeapUpDown,HeapSwapParity}

\begin{figure}
	\tikzsetnextfilename{heap_runtime}
	\begin{tikzpicture}[plot]
		\begin{groupplot}[heap, ytick distance=5]
			\nextgroupplot[title/.add={}{32-bit}, ymin=130, ymax=155, legend to name=leg:heap:runtime]
			\legend{\HS{} (top-down), \HS{} (bottom-up), \HS{} (swap parity)}
			\expandafter\pgfplotsinvokeforeach\expandafter{\heapalgos}{
				\plotpernlogn{#1}{tableHeap_32uniform}
			}
			%
			\nextgroupplot[title/.add={}{64-bit}, ymin=155, ymax=180]
			\expandafter\pgfplotsinvokeforeach\expandafter{\heapalgos}{
				\plotpernlogn{#1}{tableHeap_64uniform}
			}
		\end{groupplot}
	\end{tikzpicture}

	\hfil\pgfplotslegendfromname{leg:heap:runtime}\hfil
	\caption{
		Comparison of the runtimes of three different \HS{} implementations on uniformly distributed 32-bit integers and 64-bit integers, respectively.
	}
	\label{fig:heap:runtime}
\end{figure}

The measurements are visualised in \cref{fig:heap:runtime,fig:heap:runtime_uint32,fig:heap:runtime_uint64}.
In general, the performance of \HS{} is even less volatile than that of \MS{} with respect to the input distribution.
Reverse sorted inputs are sorted the fastest since they are already max-heaps so the heap-building phase is short, while sorted inputs are sorted the slowest since they are min-heaps so the heap-building phase is long.
Nonetheless, the reverse sorted inputs get sorted just about 10\% faster and less on average.
This low volatility cannot hide the fact that the total runtime is abysmal across the board:
Compared to \QS{} and \MS{}, the runtimes are between 50\% to 250\% higher, depending on the input distribution and the data type!

The normalised runtimes of the top-down \HS{} and the bottom-up \HS{} with swap parity show a slight upwards trends, whereas those of the bottom-up \HS{} with swap disparity mostly shows a slight downwards trends with the exception of reverse sorted inputs, where it is also an upwards trend, albeit even slighter.
Interesting are their rankings relative to each other:
Value checks on 64-bit integers take two instructions, so that the savings of the bottom-up \HS{} with swap disparity allow it to outperform the top-down \HS{} even for short inputs, and the advantage grows with the input length.
This makes sense as roughly 50\% of the vertices are leaves and 25\% are fathers of leaves, no matter the total size.
Therefore, the percentage of formerly last leaves sifting down to the bottom remains steady but the travelled distance increases.
Value checks on 32-bit integers, on the other hand, take only one instruction, so that their reduction is overshadowed by the increased overhead from the longer downwards-sifting and the added upwards-sifting, whence the lead of the top-down \HS{}.
Indeed, at around 2000 elements, the bottom-up \HS{} with swap disparity overtakes the top-down \HS{} because of their inverse trends, but the lead is narrow even at 6000 elements.
However, these long inputs are less interesting anyway due to the limited WRAM and the multiple tasklets used in most applications.

The bottom-up \HS{} with swap parity consistently trails behind.
This comes as no surprise since the overhead of its considerably prolonged upwards-sifting bears no proportion to the few swaps saved.
This holds true even for 64-bit integers as moves still cost only one instruction.
Unrolling the upwards-sifting proved to be unhelpful.



\subsection{\texorpdfstring{\QS{}}{QuickSort}}
\label{subsec:tasklet:quick}

\pgfplotstablereadnamed{data/quick/fallback/uint32/composite.txt}{tableQuickFallback_32}
\pgfplotstablereadnamed{data/quick/fallback/uint64/composite.txt}{tableQuickFallback_64}

\QS{} \cite{hoare1962quicksort} uses partitioning to sort in an expected average runtime of \(\bigoh{n \log n}\) and a worst-case runtime of \(\bigoh{n^2}\):
A pivot element is chosen from the input array, then the input array gets scanned and elements greater or lesser than the pivot are moved to the right or left side of the array, respectively.
Finally, \QS{} is used on the left and right side (the \enquote{partitions}).
The \QS*{} implementations presented here are neither stable nor in-place.


\paragraph{Sentinel Values}
The partitioning is implemented using \citeauthor{hoare1962quicksort}'s original scheme \cite{hoare1962quicksort}:
At the start of each partitioning step, a pivot \lstinline|p| is chosen and swapped with the last element.
Then, two pointers are set to either end of the partition.
The left pointer \lstinline|i| moves rightwards until finding an element at least as great as the pivot (\lstinline|*i >= p|), while the right pointer \lstinline|j| moves leftwards until finding an element at most as great as the pivot (\lstinline|*l <= p|).
The two elements found are in the wrong order so they are swapped, and the pointers move onwards.
This process continues until the pointers meet.
Finally, the pivot is swapped with the first element of the right partition.

Only an explicit check for whether the pointers have met after stopping is needed.
Since the elements of the partitions to the left are at most as great as the elements of the current partition, they naturally act as bounds check for the pointer moving rightwards.
The pivot at the end acts as bounds check for the pointer moving leftwards.
Since the leftmost partitions have no neighbour to the left, one explicit sentinel values set to the minimum possible value must be placed at the start of the input.
The downside to this approach is that elements equal to the pivot are also swapped.

\paragraph{Base Cases}

\begin{figure}
	\tikzsetnextfilename{quick_fallback}
	\begin{tikzpicture}[plot]
		\begin{groupplot}[
			horizontal sep for labels,
			adaptive group=1 by 2,
			groupplot xlabel={Input Length \(n\)},
			groupplot ylabel={Speed-up},
			xmode=log,
			xtick={16, 64, 256, 1024},
			xticklabels={\(16\), \(64\), \(256\), \(1024\)},
			minor xtick={32, 128, 512},
			ymin=0.994,
			ymax=1.001,
			extra y ticks={1.001},
			yticklabel style={/pgf/number format/.cd, precision=3, fixed, zerofill},
			legend columns=-1,
		]
			\nextgroupplot[title=32-bit\strut]
			\pgfplotsset{legend to name=leg:quick:fallback, legend entries={16,17,19,20}}
			\pgfplotsinvokeforeach{16,17,19,20}{
				\plotspeedup{#1}{18}{tableQuickFallback_32}
			}
			%
			\nextgroupplot[title=64-bit\strut]
			\pgfplotsinvokeforeach{16,17,19,20}{
				\plotspeedup{#1}{18}{tableQuickFallback_64}
			}
		\end{groupplot}
	\end{tikzpicture}

	\hfil\pgfplotslegendfromname{leg:quick:fallback}\hfil
	\caption{
		Speed-ups of \QS*{} with different thresholds for when to fall back to \IS{} over a threshold of 18 elements.
		Using \ShS{} was not beneficial overall, likely because many partitions fall below the thresholds.
	}
	\label{fig:quick:fallback}
\end{figure}

When only a few elements remain in a partition, \QS{}'s overhead predominates such that \IS{} lends itself as fallback algorithm.
As seen in \cref{fig:quick:fallback}, the optimal threshold for switching the sorting algorithm is 18 elements for uniform inputs and likely similar for inputs following Zipf's or normal distributions;
about 15\% of the runtime is saved compared to a \QS{} never falling back.
For sorted and almost sorted inputs, the threshold is higher since \IS{} is very fast on them so falling back earlier and, thus, ending the sorting process is better.
The same is true for reverse sorted inputs even though these are the worst-case inputs for \IS{} because \QS{}'s two pointers invert large swaths of the input.
However, these input distributions should be catered for by a pattern-defeating \QS{} as laid out in \todo{Verweis auf später}, hence the 18 elements as default threshold.

To avoid unnecessary uses of \IS{}, another base case is imaginable, namely terminating when a partition contains at most 1 elements.
There are tremendous consequences for the runtime depending on the exact implementation of the base cases, as shown later in \enquote{\nameref{subsubsec:tasklet:quick:compilation}}.


\paragraph{Recursion vs.\ Iteration}
In theory, the question of whether a DPU algorithm should be implemented recursively or iteratively comes down to convenience.
Due to the uniform costs of instructions, jumping to the start of a loop or to the start of a function essentially costs the same, as does managing arguments automatically through the regular call stack and manually through a simulated one.
Furthermore, in case of \QS{}, the compiler turns tail-recursive calls into jumps back to the function start, so that one partition is sorted recursively and the other iteratively.
All this would suggest a recursive implementation due to the reduced maintenance.

In practice, it comes down to the compilation.
Even parts of the algorithms which are independent from the choice between recursion and iteration can be compiled differently, such that there are implementations where iteration is faster than recursion and the other way around.
Overall though, iterative implementations \emph{tend} to be compiled better with superior register usage and less instructions used for the actual \QS{} algorithm.


\paragraph{Partition Prioritisation}
Whether the left-hand or the right-hand partition is sorted first should not make any difference for the runtime but actually does so because of different compilation, as shown later in \enquote{\nameref{subsubsec:tasklet:quick:compilation}}.
Always sorting the shorter partition first and putting the longer partition on the call stack guarantees that the problem size is at least halved each step, so that the call stack stores \(\bigoh{log n}\) elements at most.
This approach, however, is linked to huge speed penalties, which is why it is advisable to always prioritise the same side;
in this Thesis, the right-hand partitions are prioritised.
An overflow of the call stack becomes unlikely with the right pivot choice.


\paragraph{Pivot Choice}
Another parameter to tune is the way in which the pivot is chosen.
The following were implemented and tested:
\begin{itemize}
	\item
	Using the \emph{last element} is the fastest way, requiring zero additional instructions.

	\item
	Taking the \emph{median of three elements}, namely the first, middle, and last one, is far more computationally expensive since the position of the middle element must be calculated, the median be determined, and the pivot be swapped with the last element of the array, where it acts as sentinel.

	\item
	A \emph{random element} is most efficiently drawn using an xorshift random number generator and rejection sampling \cite{lukas_geis}.
	This takes some instructions but impedes deterministically chosen worst-case inputs.

	\item
	Taking the \emph{median of three random elements} is a combination of the previous two methods.
	For simplicity, there is no check on whether an element is drawn twice or thrice.
	Since the partitions are rather long, this should happen seldom, anyhow.
\end{itemize}
A median increases the chances of choosing a pivot that is neither particularly high nor particularly low.
This leads to more balanced partitions such that the call stack is less likely to overflow and the base cases are reached faster.
But even then it is still possible to construct inputs where the runtime climbs up to \(\bigtheta{n^2}\) \cite{erkiö1984worstcase}, as everything is moved to the same partition so that the problem size is reduced by only one element (namely the pivot) after each partitioning step.

The random pivots circumvent this problem.
Whilst the pivots could, by ill luck, also lead to the same unbalanced partitions as the deterministic pivots, the worst-case expected runtime is \(\bigoh{n \log n}\) \cite{blum2011probabilistic}.
Using the median of medians \cite{blum1973median} could guarantee a runtime of \(\bigoh{n \log n}\) but was not implemented because a performant implementation would probably be quite complex and its benefit minuscule for this Thesis.

The general trend, as seen in \enquote{\nameref{subsubsec:tasklet:quick:compilation}}, is the following:
A median gets more beneficial for the average runtime, the longer the input becomes, and leads to small pay-offs in the end.
Moreover, the standard deviations of the runtimes are cut roughly in half, although not shown in the figures of \enquote{\nameref{subsubsec:tasklet:quick:compilation}} for reasons of clarity.
If the input is known to be fairly random, a deterministic choice yields a noticeably speed-up.
However, the gain remains in the single digits percentage-wise, supporting the findings by \citeauthor{lukas_geis}~\cite{lukas_geis} that drawing random numbers is quite cheap.
For this reason, the median of three random elements is used as default configuration throughout this Thesis.


\subsubsection*{Investigation of the Compilation}
\label{subsubsec:tasklet:quick:compilation}

\def\quickpivots{LAST,MEDIAN,RANDOM,MEDIAN_OF_RANDOM}
\expandafter\pgfplotsinvokeforeach\expandafter{\quickpivots}{
	\pgfplotstablereadnamed{data/quick/matrix/iterative/#1/shorter/uint32/uniform.txt}{tableQuickMatrixIt#1Shorter_32}
	\pgfplotstablereadnamed{data/quick/matrix/iterative/#1/left/uint32/uniform.txt}{tableQuickMatrixIt#1Left_32}
	\pgfplotstablereadnamed{data/quick/matrix/iterative/#1/right/uint32/uniform.txt}{tableQuickMatrixIt#1Right_32}

	\pgfplotstablereadnamed{data/quick/matrix/recursive/#1/shorter/uint32/uniform.txt}{tableQuickMatrixRec#1Shorter_32}
	\pgfplotstablereadnamed{data/quick/matrix/recursive/#1/left/uint32/uniform.txt}{tableQuickMatrixRec#1Left_32}
	\pgfplotstablereadnamed{data/quick/matrix/recursive/#1/right/uint32/uniform.txt}{tableQuickMatrixRec#1Right_32}
}
\def\quickalgos{Normal,TrivInThresh,NoTrivial,ThreshThenTriv,TrivialBC,ThreshBC,ThreshTrivBC,OneInsertion}

The quality of the compilation of \QS{} is highly erratic to such an extent that \Dash even with the same pivots! \Dash one implementation variant may see a reduction of 25\% from another one where none should be.
There are small details influencing the runtime, like storing the value of the pivot in a dedicated variable instead of accessing it through a pointer changing the runtime by a few percentage points in both directions, depending on the rest of implementation.
But as hinted at in the preceding part of this section, there are four major parameters to examine:
handling of the base cases, recursion/iteration, pivot choice, and partition prioritisation.
Before the findings are discussed, the first parameter shall be explained in more depth.

Besides falling back to \IS{} if 18 elements remain (\enquote{treshold undercut}), another base case is imaginable, namely a termination if at most one element remains (\enquote{trivial length}).
Realistically speaking, it should not be needed to check for trivial lengths because even though it is doable with just one additional instruction, such short partitions are rare and \IS{} would terminate after a few instructions anyway.
Nonetheless, its inclusion or exclusion can have significant impacts.
The following handlings were tested:
\begin{enumerate}[label=(\liningnums{\arabic*})]
	\item\label[implementation]{imp:normal}
	If the length is trivial, terminate immediately.
	If if the threshold is undercut, sort with \IS{} and terminate.
	Otherwise, sort with \QS{} and use \QS{} on both partitions.
%	\textcolor{red}{[Normal]}

	\item\label[implementation]{imp:triviality_within_threshold}
	If the threshold is undercut, check if the length is trivial and terminate immediately or sort with \IS{} and then terminate, respectively.
	Otherwise, sort with \QS{} and use \QS{} on both partitions.
%	\textcolor{red}{[TrivInThresh]}
	\begin{itemize}
		\item
		This handling significantly reduces the number of checks for trivial length.
	\end{itemize}

	\item\label[implementation]{imp:no_triviality}
	If the threshold is undercut, sort with \IS{} and terminate.
	Otherwise, sort with \QS{} and use \QS{} on both partitions.
%	\textcolor{red}{[NoTrivial]}
	\begin{itemize}
		\item
		This handling forgoes the check for a trivial length completely, at the cost of some unneeded \IS*{}.
	\end{itemize}

	\item\label[implementation]{imp:threshold_then_triviality}
	If the threshold is undercut, sort with \IS{} and terminate.
	If the length is trivial, terminate immediately.
	Otherwise, sort with \QS{} and use \QS{} on both partitions.
%	\textcolor{red}{[ThreshThenTriv]}
	\begin{itemize}
		\item
		This handling, while nonsensical from a logical point of view, gives the compiler an explicit guarantee that the partitioning loop does not end immediately.
	\end{itemize}

	\item\label[implementation]{imp:triviality_before_call}
	If the threshold is undercut, sort with \IS{} and terminate.
	Otherwise, sort with \QS{}.
	Then check for either partition if its length is trivial and use \QS{} if not.
%	\textcolor{red}{[TrivialBC]}
	\begin{itemize}
		\item
		This handling, as well as the next two, gets rid of some unneeded uses of \QS{}.
		In the recursive case, these handlings lose the property of being tail-recursive.
	\end{itemize}

	\item\label[implementation]{imp:threshold_before_call}
	Sort with \QS{}.
	Check for either partition if the threshold is undercut and use \IS{} or \QS{} on them, respectively.
%	\textcolor{red}{[ThreshBC]}

	\item\label[implementation]{imp:threshold_and_triviality_before_call}
	Sort with \QS{}.
	Check for either partition if its length is trivial or if the threshold is undercut and use \IS{}, \QS{}, or nothing on them, respectively.
%	\textcolor{red}{[ThreshTrivBC]}

	\item\label[implementation]{imp:one_insertion}
	If the threshold is undercut, terminate immediately.
	Otherwise, sort with \QS{} and use \QS{} on both partitions.
	After all \QS*{} are done, sort the whole input array with \IS{}.
%	\textcolor{red}{[OneInsertion]}
	\begin{itemize}
		\item
		This handling always does one pass of \IS{}.
		For example, the other handlings use \IS{} roughly 91 times on 1024 uniformly distributed elements.
	\end{itemize}
\end{enumerate}
The performances of all tested implementation for 32-bit integers are shown in \cref{fig:quick:implementations}.
The measurements were done on uniform input distributions so the deterministic pivots are, in expectation, of the same quality as the random ones.

\pgfplotsset{
	quick matrix/.style={
		height=2.567cm,
		horizontal sep for naught,
		vertical sep for naught,
		adaptive group=3 by 4,
		groupplot xlabel={Input Length \(n\)},
		groupplot ylabel={Cycles / \((n \lb n)\)},
		xmode=log,
		xtick={16, 64, 256, 1024},
		xticklabels={\(16\), \(64\), \(256\), \(1024\)},
		minor xtick={32, 128, 512},
		ymin=55,
		ymax=80,
		/tikz/mark repeat=2,
		legend columns=-1,
	}
}

\begin{figure}[p]
	\captionsetup[subfigure]{aboveskip=0mm,belowskip=1mm}
	\begin{subfigure}{\textwidth}
		\tikzsetnextfilename{quick_implementations_rec}
		\begin{tikzpicture}[plot]
			\begin{groupplot}[quick matrix]
				\nextgroupplot[title=Last, xticklabels={}]
				\pgfplotsset{legend to name=leg:quick:implementations, legend entries={\ref{imp:normal}, \ref{imp:triviality_within_threshold}, \ref{imp:no_triviality}, \ref{imp:threshold_then_triviality}, \ref{imp:triviality_before_call}, \ref{imp:threshold_before_call}, \ref{imp:threshold_and_triviality_before_call}, \ref{imp:one_insertion}}}
				\expandafter\pgfplotsinvokeforeach\expandafter{\quickalgos}{
					\plotpernlogn{#1}{tableQuickMatrixRecLASTLeft_32}
				}
				\nextgroupplot[title=Median, xticklabels={}, yticklabels={}]
				\expandafter\pgfplotsinvokeforeach\expandafter{\quickalgos}{
					\plotpernlogn{#1}{tableQuickMatrixRecMEDIANLeft_32}
				}
				\nextgroupplot[title=Random, xticklabels={}, yticklabels={}]
				\expandafter\pgfplotsinvokeforeach\expandafter{\quickalgos}{
					\plotpernlogn{#1}{tableQuickMatrixRecRANDOMLeft_32}
				}
				\nextgroupplot[title=Median (Random), xticklabels={}, yticklabel pos=right]
				\expandafter\pgfplotsinvokeforeach\expandafter{\quickalgos}{
					\plotpernlogn{#1}{tableQuickMatrixRecMEDIAN_OF_RANDOMLeft_32}
				}
				%
				\nextgroupplot[xticklabels={}]
				\expandafter\pgfplotsinvokeforeach\expandafter{\quickalgos}{
					\plotpernlogn{#1}{tableQuickMatrixRecLASTRight_32}
				}
				\nextgroupplot[xticklabels={}, yticklabels={}]
				\expandafter\pgfplotsinvokeforeach\expandafter{\quickalgos}{
					\plotpernlogn{#1}{tableQuickMatrixRecMEDIANRight_32}
				}
				\nextgroupplot[xticklabels={}, yticklabels={}]
				\expandafter\pgfplotsinvokeforeach\expandafter{\quickalgos}{
					\plotpernlogn{#1}{tableQuickMatrixRecRANDOMRight_32}
				}
				\nextgroupplot[xticklabels={}, yticklabel pos=right]
				\expandafter\pgfplotsinvokeforeach\expandafter{\quickalgos}{
					\plotpernlogn{#1}{tableQuickMatrixRecMEDIAN_OF_RANDOMRight_32}
				}
				%
				\nextgroupplot
				\expandafter\pgfplotsinvokeforeach\expandafter{\quickalgos}{
					\plotpernlogn{#1}{tableQuickMatrixRecLASTShorter_32}
				}
				\nextgroupplot[yticklabels={}]
				\expandafter\pgfplotsinvokeforeach\expandafter{\quickalgos}{
					\plotpernlogn{#1}{tableQuickMatrixRecMEDIANShorter_32}
				}
				\nextgroupplot[yticklabels={}]
				\expandafter\pgfplotsinvokeforeach\expandafter{\quickalgos}{
					\plotpernlogn{#1}{tableQuickMatrixRecRANDOMShorter_32}
				}
				\nextgroupplot[yticklabel pos=right]
				\expandafter\pgfplotsinvokeforeach\expandafter{\quickalgos}{
					\plotpernlogn{#1}{tableQuickMatrixRecMEDIAN_OF_RANDOMShorter_32}
				}
			\end{groupplot}
		\end{tikzpicture}
		\caption{
			Recursive Approach
		}
	\end{subfigure}
	\begin{subfigure}{\textwidth}
		\tikzsetnextfilename{quick_implementations_it}
		\begin{tikzpicture}[plot]
			\begin{groupplot}[quick matrix]
				\nextgroupplot[title=Last, xticklabels={}]
				\expandafter\pgfplotsinvokeforeach\expandafter{\quickalgos}{
					\plotpernlogn{#1}{tableQuickMatrixItLASTLeft_32}
				}
				\nextgroupplot[title=Median, xticklabels={}, yticklabels={}]
				\expandafter\pgfplotsinvokeforeach\expandafter{\quickalgos}{
					\plotpernlogn{#1}{tableQuickMatrixItMEDIANLeft_32}
				}
				\nextgroupplot[title=Random, xticklabels={}, yticklabels={}]
				\expandafter\pgfplotsinvokeforeach\expandafter{\quickalgos}{
					\plotpernlogn{#1}{tableQuickMatrixItRANDOMLeft_32}
				}
				\nextgroupplot[title=Median (Random), xticklabels={}, yticklabel pos=right]
				\expandafter\pgfplotsinvokeforeach\expandafter{\quickalgos}{
					\plotpernlogn{#1}{tableQuickMatrixItMEDIAN_OF_RANDOMLeft_32}
				}
				%
				\nextgroupplot[xticklabels={}]
				\expandafter\pgfplotsinvokeforeach\expandafter{\quickalgos}{
					\plotpernlogn{#1}{tableQuickMatrixItLASTRight_32}
				}
				\nextgroupplot[xticklabels={}, yticklabels={}]
				\expandafter\pgfplotsinvokeforeach\expandafter{\quickalgos}{
					\plotpernlogn{#1}{tableQuickMatrixItMEDIANRight_32}
				}
				\nextgroupplot[xticklabels={}, yticklabels={}]
				\expandafter\pgfplotsinvokeforeach\expandafter{\quickalgos}{
					\plotpernlogn{#1}{tableQuickMatrixItRANDOMRight_32}
				}
				\nextgroupplot[xticklabels={}, yticklabel pos=right]
				\expandafter\pgfplotsinvokeforeach\expandafter{\quickalgos}{
					\plotpernlogn{#1}{tableQuickMatrixItMEDIAN_OF_RANDOMRight_32}
				}
				%
				\nextgroupplot
				\expandafter\pgfplotsinvokeforeach\expandafter{\quickalgos}{
					\plotpernlogn{#1}{tableQuickMatrixItLASTShorter_32}
				}
				\nextgroupplot[yticklabels={}]
				\expandafter\pgfplotsinvokeforeach\expandafter{\quickalgos}{
					\plotpernlogn{#1}{tableQuickMatrixItMEDIANShorter_32}
				}
				\nextgroupplot[yticklabels={}]
				\expandafter\pgfplotsinvokeforeach\expandafter{\quickalgos}{
					\plotpernlogn{#1}{tableQuickMatrixItRANDOMShorter_32}
				}
				\nextgroupplot[yticklabel pos=right]
				\expandafter\pgfplotsinvokeforeach\expandafter{\quickalgos}{
					\plotpernlogn{#1}{tableQuickMatrixItMEDIAN_OF_RANDOMShorter_32}
				}
			\end{groupplot}
		\end{tikzpicture}
		\caption{
			Iterative Approach
		}
	\end{subfigure}

	\tikzexternaldisable
	\hfil\pgfplotslegendfromname{leg:quick:implementations}\hfil
	\tikzexternalenable
	\caption{
		Comparison of \crefrange{imp:normal}{imp:one_insertion} and different pivots.
		Left-hand partitions are prio\-ri\-tised in the first rows, right-hand ones in the second rows, and shorter ones in the third~rows.
	}
	\label{fig:quick:implementations}
\end{figure}

Even when ignoring the differences between specific handlings, the high fluctuations between the plots leap to the eye.
Plots within the same column share the same method to chose pivots, plots within the same row share the same prioritisation of partitions.
In general, it would be expected that plots within the same column are fairly similar, yet prioritising the shorter partition is almost universally associated with an increase in runtime.
When focussing on the top-performing implementations, the increase can reach more than 15\%.
There is no clear trend between the consistent prioritisations of either side, although the difference can be huge in individual cases as well.
However, recursion is more susceptible to the partition prioritisation than iteration.

The correlation of recursive and iterative performance is weak.
On one hand, there is, for example, \cref{imp:triviality_before_call} with deterministic medians and prioritisation of shorter partitions where the runtimes are essentially the same.
On the other hand, there is \cref{imp:one_insertion} with deterministic medians and prioritisation of right-hand partitions where recursion is slower by more than a third.
All in all, iterative implementations usually perform better, though, especially when focussing on the top-performing implementations of each pivot choice.

The ranking of the different handlings is rather incoherent.
\Cref{imp:triviality_before_call}, which does not call \QS{} on trivial partitions, decidedly fares the best out of all handlings, being amongst the top performers across all tested implementations.
\Cref{imp:threshold_before_call,imp:threshold_and_triviality_before_call}, which call \QS{} even less often than \cref{imp:triviality_before_call}, are the polar opposite and bring up the rear of the ranking every single time.
Recursive implementations of \cref{imp:triviality_within_threshold}, where triviality is checked for only if the threshold is undercut, are often worse than recursive implementations of \cref{imp:normal}, where triviality is always checked for, whilst it is the other way around for iterative implementations.
Interestingly, for all investigated implementations, the compiler is capable of eliminating the last possible recursive calls by turning them into jumps back to the function start, regardless of whether these were properly tail recursive or not.

These observations, however, only apply to 32-bit integers.
\Cref{fig:quick:implementations_64} shows the same measurements for 64-bit integers.
Whilst the general trend for pivots, partition prioritisation and recursion/iteration hold true, the rankings are vastly different.
\Cref{imp:triviality_before_call} is not undisputedly the best anymore.
\Cref{imp:threshold_before_call,imp:threshold_and_triviality_before_call} switch back and forth between being the worst and the best handlings.
Most notably, the two top-performing implementations using deterministic and random medians as pivots, respectively, are actually recursive.
Luckily, both use \cref{imp:triviality_before_call} so the only difference between the default configurations of 32-bit and 64-bit \QS{} is the usage of recursion/iteration.

What is causing these huge disparities?
There is a great variety in the compilations but some of the common occurrences are \dots{}
\begin{itemize}
	\item
	\dots{} one instruction more before (re-)starting to move the pointers, \dots{}

	\item
	\dots{} one instruction more while moving the left pointer by one element, \dots{}

	\item
	\dots{} one instruction more after the left pointer has stopped, \dots{}

	\item
	\dots{} more stores and loads when entering and leaving the function.
\end{itemize}
This focus on the left pointer \lstinline|i| is partially explainable by it being used to calculate the final position of the pivot \lstinline|p| and, thus, the inner boundaries of both new partitions:
When the left pointer \lstinline|i| stops, it holds \lstinline|*i >= p|.
Also, the left pointer either passed over the preceding element if \lstinline|*(i - 1) < p|, or it stopped there if \lstinline|*(i - 1) >= p|.
However, the right pointer \lstinline|j| stopped on some value fulfilling \lstinline|*j <= p|, so it holds \lstinline|*(i - 1) <= p| after swapping.
Either way, it holds \lstinline|*(i - 1) <= p|.
If the right pointer \lstinline|j| meets the left pointer \lstinline|i|, it either stops there immediately if \lstinline|*i = p|, or it stops at address \lstinline|i - 1| because of \lstinline|*(i - 1) <= p|.
In all cases, the pivot, which was moved to the right of the partition at the start, can now swap with pointer \lstinline|i|, and the addresses \lstinline|i - 1| and \lstinline|i + 1| form the end of the left-hand partition and the start of the right-hand partition, respectively.
We spot-checked implementations to see whether using the right pointer alone or both of them to calculate the boundaries could alleviate the problems but the results were mixed:
from betterment over indifference to worsening, everything was observable.

\subsection*{Evaluation of the Performance}
\label{sec:tasklet:quick:performance}
\addcontentsline{toc}{subsection}{\nameref{sec:tasklet:quick:performance}}

\pgfplotsinvokeforeach{sorted,reverse,almost,uniform,zipf,normal}{
	\pgfplotstablereadnamed{data/quick/matrix/iterative/Median_of_random/right/uint32/#1.txt}{tableQuickRand_32#1}
	\pgfplotstablereadnamed{data/quick/matrix/recursive/Median_of_random/right/uint64/#1.txt}{tableQuickRand_64#1}
}

Before turning to the performance of \QS{} on specific input distributions, the ratio between costs and benefits of the pivot selection methods shall be evaluated.
Looking again at \cref{fig:quick:implementations,fig:heap:runtime_uint64} shows that a median gets more beneficial, the longer the input becomes, achieving small pay-offs for the longest ones.
Moreover, the standard deviations of the runtimes, although not shown in the figures for reasons of clarity, are cut roughly in half.
Randomisation slows down noticeably, so random pivots are disadvantageous if the input is known to be fairly random.
However, the decrease remains in the single digits percentage-wise, supporting the findings by \citeauthor{lukas_geis}~\cite{lukas_geis} that drawing random numbers is quite cheap.
For this reason, the random median is used as default method throughout this thesis.

\begin{figure}
	\tikzsetnextfilename{quick_runtime}
	\begin{tikzpicture}[plot]
		\begin{groupplot}[
			adaptive group=1 by 2,
			groupplot ylabel={Cycles / \((n \lb n)\)},
			x from 16 to 1024,
			ytick distance=10,
		]
			\nextgroupplot[title/.add={}{32-bit}, ymin=30, ymax=80]
			\pgfplotsset{legend to name=leg:quick:runtime, legend entries={Sorted, Reverse S., Almost S., Uniform, Zipf's, Normal}}
			\pgfplotsinvokeforeach{sorted,reverse,almost,uniform,zipf,normal}{
				\plotpernlogn{TrivialBC}{tableQuickRand_32#1}
			}
			%
			\nextgroupplot[title/.add={}{64-bit}, ymin=40,ymax=90]
			\pgfplotsinvokeforeach{sorted,reverse,almost,uniform,zipf,normal}{
				\plotpernlogn{TrivialBC}{tableQuickRand_64#1}
			}
		\end{groupplot}
	\end{tikzpicture}

	\hfil\pgfplotslegendfromname{leg:quick:runtime}\hfil
	\caption{
		Mean runtime of \QS{} on all tested input distributions and data types.
	}
	\label{fig:quick:runtime}
\end{figure}

\Cref{fig:quick:runtime} shows the runtime of \QS{} in it default configuration, that is, with random medians.
\Cref{fig:quick:runtime_uint32,fig:quick:runtime_uint64} additionally contain the runtimes with deterministic medians as well as the standard deviations of the measurements.
The mean runtimes are rather close across all input distributions, a consequence of using random medians and of considering elements equal to the pivot as different.
In fact, it is \IS{} that primarily causes the discrepancies, as setting the threshold to one element proves.
This also explains why \QS{} performs so well on large inputs with Zipf's distribution:
This distributions generates many duplicates, which are put into the same partitions, so \IS{} performs many simple scans.

One might expect \QS{} to perform even better on sorted and reverse sorted input, since everything is either already in the correct position or because the two pointers quickly invert large swaths of the inputs.
However, a side effect of swapping the pivot twice can be that many elements are displaced by one position from where they should be in the sorted input.
Take reverse sorted inputs and the deterministic median as an example:
The element \(n/2\) is selected as pivot out of the elements \(n\), \(n/2\), and \(0\) and then gets swapped with the last element, that is, with \(0\).
Thereupon, the pointers invert the rest of the input such that the start of the input looks something like \(1, 2, \dots, n/2-1, 0, n/2, \dots\) after the first partitioning step.
Indeed, this pattern makes \QS{} with deterministic medians degrade and eventually overflow the call stack, which is why the respective plots in \cref{fig:quick:runtime_uint32,fig:quick:runtime_uint64} leave the charts.
An implementation without swapping the pivot promises better performance for such cases, but in exploratory ones, the performance on more random input distributions suffered drastically.



\subsection{\texorpdfstring{\MS{}}{MergeSort}}
\label{subsec:tasklet:merge}

\MS{} repeatedly compares two sorted subarrays and merges them into a longer sorted array in time \(\bigtheta{n \log n}\).
Unlike \QS{}, this runtime is guaranteed.
Furthermore, the sorting is naturally stable but at the cost of not happening in-place.

\paragraph{Starting Runs}
Instead of starting by merging runs of length 1, it is beneficial to first create longer starting runs using \ShS{}.
Unlike \QS{}, where each partition naturally acted as sentinel for the subsequent one, it is necessary to temporarily place sentinels values in front of each starting run and later restore the original values of the preceding run.
The step sizes used for \ShS{} \Dash namely \(\stepsizes = (1)\) for lengths up to 16, \(\stepsizes = (6, 1)\) for lengths up to 48, and \(\stepsizes = (12, 5, 1)\) for everything above \Dash have been chosen based on the results in \cref{subsec:tasklet:shell}, according to which these step sizes offer top performance for uniformly distributed inputs and medial performance for the reverse sorted inputs.
Spot-check inspection suggest no deterioration of \ShS{}'s compilation due to inlining.

\paragraph{Memory Footprint}
A simple but fast implementation of \MS{} writes all merged runs to an auxiliary array, raising the need for space for \(n\) additional elements (\enquote{full space}).
After a round is finished and all pairs of runs have been merged, the input array and the auxiliary array switch roles, and the merging starts anew.
Are the final sorted elements supposed to be saved in the original input array, a final round with a write-back from the auxiliary array to the input array is needed for some input lengths.

A slightly more sophisticated implementation needs space for only \(n/2\) additional elements (\enquote{half space}):
When two adjacent runs are to be merged, the first one can be copied to an auxiliary array.
Then, the copy and the second run are merged to the start of the first run.
As a side effect, no write-back is ever needed and, additionally, the merging of two runs can be terminated prematurely once the last element of the copied run is merged, since the last elements of the other run are already in place.
%As a consequence, flushes will only be performed on at most half of the runs.
Further optimised, \MS{} would not need to copy the first runs immediately.
It suffices to search for the foremost element of the first run which is greater than the first element of the second element.
All previous elements are already in the correct position so only the following elements need to be copied to the auxiliary array.
This optimisation, although examined during development, was not in use when measuring runtimes since it unfortunately complicates another optimisation, namely unrolling.

\paragraph{Unrolling}
There are four common reasons for \emph{flushing}, that is, writing \Dash many oftwhiles \Dash consecutive elements:
\begin{enumerate}
	\item
	When two runs are merged and the end of one of them is reached, the remaining elements of the other one can be moved safely to the output location.
	Especially with the sorted, reverse sorted, and almost sorted input distributions, the number of remaining elements will be high.

	\item
	The number of runs is odd, so the full-space \MS{} moves the last run to the output location immediately.

	\item
	The full-space \MS{} may write all elements from the auxiliary array back to the input array if the former contains the final sorted sequence.

	\item
	The half-space \MS{} copies runs, whose length are always a multiple of the the starting run length, before each merger of pairs.
\end{enumerate}
Therefore, flushing account for a considerable part of the runtime, and reducing the loop overhead (variable incrementation and bounds checking) is helpful.
This can be done via \emph{unrolling}:
As long as at least, let us say, \(x\) elements still need to be flushed, the \(x\) foremost elements are moved first and then all necessary variables are incremented by \(x\).
Is \(x\) a compile-time constant, the compiler implements the moving of the elements through \(x\) instruction which use constant, pre-calculated offsets.
Once less than \(x\) elements remain, an ordinary loop which moves elements individually is used.
In good cases, this approach reduces the loop overhead to an \(x\)th, whilst in bad cases, where less than \(x\) are to be flushed, the overhead is increased by one additional check.

Due to time reasons, we refrained from doing automatic and extensive tests and relied on manual and exploratory tests to come up with the following strategy:
When the full-space \MS{} performs a write-back or when the half-space \MS{} copies the first run, \(x\) is set to the starting run length.
In all other cases, \(x\) is set to 24.
This strategy, albeit not optimal, makes the \MS*{} significantly faster:
Sorting sorted, reverse sorted, and almost sorted inputs sees speed-ups up to 30\%, whereas sorting more random inputs still sees speed-ups for the most part and slow-downs into low single-digits at worst, depending on the starting run length.

\subsection*{Investigation of the Compilation}
\label{sec:tasklet:merge:compilation}

Yet again, the compiler shows unforeseen behaviour.
The following is a collocation of some of the issues found while engineering \MS{}.
They will not be discussed in detail here but still provide a point of reference for future work:
\begin{itemize}
	\item
	As already mentioned, sorting the starting runs via \ShS{} requires the placement and later removal of temporary sentinel values.
	For the very first starting run, one can omit storing and restoring the overwritten elements by using permanent sentinel values;
	this optimisation was in use when measuring runtimes.
	On the downside, this leads to a bigger compilation as \ShS{} is inlined twice.
	If the size of the whole compilation is already close to the maximum, one might be inclined to handle the first starting run just like the others.
	Realistically, this should slow down the total runtime by just a few hundreds of cycles, yet the real slow-down is in the thousands.

	\item
	If the input is so short that it fits entirely within the first starting run, one can immediately end the execution after \ShS{} is done.
	Several implementations were tested, with unsatisfactory results:
	Some increased the runtime for longer inputs, others decreased it but also increased it for shorter inputs.
	The settled-on implementation is of the former variety since the increases hit shorter inputs harder relatively and a more thorough solution would not further the purpose of this section.

	\item
	Concerning the half-space \MS{}:
	Treating the copied run logically as the second run and the uncopied run as the first one nets a noticeable decrease in runtime compared to an implementation with flipped logic.
	Even worse, only with the former does unrolling improve the speed, being an impairment with the latter!
	This behaviour occurs with both immediate and deferred copying of the first runs.
	An inspection of the issue unearthed marvels like code of the form
	\begin{center}
		\vspace{-\baselineskip}
		\texttt{*i++ = *j; a = b - i; c = i; i = d;}
	\end{center}
	leading to 5\% longer runtimes compared to
	\begin{center}
		\texttt{*i = *j; a = b - (i + 1); c = i + 1; i = d;}
	\end{center}
	even though executed at most once per merger, but we could sadly not pinpoint the fundamental cause for the behaviour.
\end{itemize}


\subsection*{Evaluation of the Performance}
\label{sec:tasklet:merge:performance}
\addcontentsline{toc}{subsection}{\nameref{sec:tasklet:merge:performance}}

\pgfplotsinvokeforeach{14,15,16,17,18,24,32,48,64,96}{
	\pgfplotstablereadnamed{data/merge/threshold=#1/uint32/sorted.txt}{tableMergeStart#1_32sorted}
	\pgfplotstablereadnamed{data/merge/threshold=#1/uint32/reverse.txt}{tableMergeStart#1_32reverse}
	\pgfplotstablereadnamed{data/merge/threshold=#1/uint32/almost.txt}{tableMergeStart#1_32almost}
	\pgfplotstablereadnamed{data/merge/threshold=#1/uint32/uniform.txt}{tableMergeStart#1_32uniform}
	\pgfplotstablereadnamed{data/merge/threshold=#1/uint32/zipf.txt}{tableMergeStart#1_32zipf}
	\pgfplotstablereadnamed{data/merge/threshold=#1/uint32/normal.txt}{tableMergeStart#1_32normal}

	\pgfplotstablereadnamed{data/merge/threshold=#1/uint64/sorted.txt}{tableMergeStart#1_64sorted}
	\pgfplotstablereadnamed{data/merge/threshold=#1/uint64/reverse.txt}{tableMergeStart#1_64reverse}
	\pgfplotstablereadnamed{data/merge/threshold=#1/uint64/almost.txt}{tableMergeStart#1_64almost}
	\pgfplotstablereadnamed{data/merge/threshold=#1/uint64/uniform.txt}{tableMergeStart#1_64uniform}
	\pgfplotstablereadnamed{data/merge/threshold=#1/uint64/zipf.txt}{tableMergeStart#1_64zipf}
	\pgfplotstablereadnamed{data/merge/threshold=#1/uint64/normal.txt}{tableMergeStart#1_64normal}
}
\pgfplotsinvokeforeach{sorted,reverse,almost,uniform,zipf,normal}{
	\pgfplotstablereadnamed{data/merge/uint32/#1.txt}{tableMerge_32#1}
	\pgfplotstablereadnamed{data/merge/uint64/#1.txt}{tableMerge_64#1}
}

Three implementations using \IS{} on starting runs of length 16 have been tested:
full-space \MS{} without write-backs, full-space \MS{} with write-backs, and half-space \MS{}.
The results are shown in \cref{fig:merge:runtime,fig:merge:runtime_uint32,fig:merge:runtime_uint64}.
Besides the mean runtimes on all tested input distributions, \cref{fig:merge:runtime_uint32,fig:merge:runtime_uint64} additionally contain the standard error of the measurements.
Note that the tested input lengths have been chosen in such a way that the plots exhibit \MS{}'s characteristic zigzagging to the full extent:
The merging process can be visualised as binary tree, with the starting runs as leaves and two vertices being brothers if the corresponding runs get merged together;
the root contains the final sorted run.
This way, the height of the tree is equal to the number of rounds of merging.
For \(n = 2^i\) input elements, the tree is complete, and the normalised runtime is locally minimal.
For \(n = 2^i + 1\) input elements, the root has a leaf with one element as son, and the normlised runtime is locally maximal, as the number of rounds increased to accommodate for just one element.

\pgfplotsset{
	merge/.style={
		horizontal sep for ticks,
		adaptive group=1 by 3,
		groupplot ylabel={Cycles / \((n \lb n)\)},
		x from 16 to 1024 minor,
	},
	merge filter/.style={x filter/.expression={(\thisrow{n} > #1) ? \pgfmathresult : nan}},
}

\begin{figure}
	\tikzsetnextfilename{merge_runtime}
	\begin{tikzpicture}[plot]
		\begin{groupplot}[
			merge,
			ymin=0,
			ymax=150,
		]
			\nextgroupplot[title/.add={}{Without Write-Backs}]
			\pgfplotsinvokeforeach{sorted,reverse,uniform}{
				\plotpernlogn{Merge}{tableMerge_32#1}
			}
			%
			\nextgroupplot[title/.add={}{With Write-Backs}]
			\pgfplotsinvokeforeach{sorted,reverse,uniform}{
				\plotpernlogn{MergeWriteBack}{tableMerge_32#1}
			}
			%
			\nextgroupplot[title/.add={}{Half-Space}]
			\pgfplotsinvokeforeach{sorted,reverse,uniform}{
				\plotpernlogn{MergeHalfSpace}{tableMerge_32#1}
			}
			\pgfplotsset{legend to name=leg:merge:runtime, legend entries={Sorted, Reverse Sorted, Uniform}}
		\end{groupplot}
	\end{tikzpicture}

	\hfil\pgfplotslegendfromname{leg:merge:runtime}\hfil
	\caption{
		Mean runtimes of the full-space \MS*{} with and without write-backs and the half-space \MS{} on 32-bit integers, for \(n = 2^i\) and \(n = 2^i + 1\) with \(i = 1, \dots, 10\).
	}
	\label{fig:merge:runtime}
\end{figure}

The measurements show that \MS{} guarantees a runtime of \(\bigtheta{n \log n}\) for the tested input distributions as expected.
The differences in runtime between the different input distributions get smaller with increasing input length and are ascribable to \IS{} and to the differing suitability of the unrolling.
Especially because of \IS{}, sorted inputs take the shortest and reverse sorted ones the longest;
cases where the usage of \IS{} worsened the runtime are unbeknown.
The differences across the input distributions become smaller with increasing input length but remain large even for \(n \approx 1000\) elements.
For the full-space \MS*{}, reverse sorted inputs get sorted 60\%--80\% slower than sorted inputs with 32-bit integers and 80\%--100\% slower with 64-bit integers.
For the half-space \MS{}, these ranges climb to 80\%--110\% and 110\%--150\%, respectively.

The half-space \MS{} delivers a strong performance despite its vastly lower memory footprint.
With sorted inputs, it takes the lead over the full-space \MS*{}, since the second runs need not be flushed and the additional copying of the first runs is unrolled.
For other inputs, the half-space \MS{} takes just about 10\% longer than the full-space \MS{} without write-backs for both 32-bit and 64-bit integers.
The leeway to the full-space \MS{} with write-backs is even smaller.
The slow-down is likely because of most elements of the second runs being moved forwards anyway and the elements of the first runs being both copied and moved.

\Cref{apx:tasklet:merge} contains further measurements on \MS{}, showing why a starting run length of 16 is a good choice.
\Cref{fig:merge:starting_runs_is_uint32,fig:merge:starting_runs_is_uint64} show the average runtimes of starting runs of lengths 14 to 18, all sorted with \IS{}.
Overall, the differences are small, yet 16 elements lead to top performance for all input distributions but the reverse sorted one.
The disadvantage there is not grave, though.
\Cref{fig:merge:starting_runs_shs_uint32,fig:merge:starting_runs_shs_uint64} include longer starting runs of lengths between 24 and 96 elements in order to see whether giving up stability by using \ShS{} can yield substantial gains.
The disparities between the runtimes of the different starting run lengths are strikingly small despite the wide range tested.
By and large, however, the savings are not big enough to warrant consideration of two different \MS{} configurations \Dash stable and unstable \Dash in this thesis.


\begin{figure}
	\tikzsetnextfilename{wram_sorts}
	\begin{tikzpicture}[plot]
		\begin{groupplot}[
			horizontal sep for labels,
			adaptive group=1 by 2,
			groupplot xlabel={Input Length \(n\)},
			xmode=log,
			xtick={20, 32, 64, 128, 256, 512, 1024},
			xticklabels={\(20\), \(32\), \(64\), \(128\), \(256\), \(512\), \(1024\)},
			legend columns=-1,
		]
			\nextgroupplot[ylabel=Cycles / \((n \lb n)\), ymin=0, ymax=150, legend to name=leg:wram_sorts]
			\legend{\QS{}, \ShS, \HS{}, \MS{}}
%			\plotpernlogn{Merge}{tableWramSorts}
			\plotpernlogn{Quick}{tableWramSorts}
			\plotpernlogn{Shell}{tableWramSorts}
%			\plotpernlogn{MergeWriteBack}{tableWramSorts}
			\plotpernlogn{Heap}{tableWramSorts}
			\plotpernlogn{MergeHalfSpace}{tableWramSorts}
			%
			\nextgroupplot[ylabel=Speed-up, ymin=0.3, ymax=1, extra y ticks={0.3}]
%			\plotspeedup{Merge}{Quick}{tableWramSorts}
			\pgfplotsset{cycle list shift=1}
			\plotspeedup{Shell}{Quick}{tableWramSorts}
%			\plotspeedup{MergeWriteBack}{Quick}{tableWramSorts}
			\plotspeedup{Heap}{Quick}{tableWramSorts}
			\plotspeedup{MergeHalfSpace}{Quick}{tableWramSorts}
		\end{groupplot}
	\end{tikzpicture}

	\hfil\pgfplotslegendfromname{leg:wram_sorts}\hfil
	\caption{
		Comparison of \MS{}, \HS{}, \ShS{}, and \QS{}.
		Due to \MS{}'s increased space requirements, its runtime was measured only for up to 768 elements.
		The \ShS{} uses the step sizes from \cref{fig:shell:against_others}, which are unoptimised for long inputs.
		The speed-ups are with respect to the \QS{}.
	}
	\label{fig:wram_sorts}
\end{figure}


\section{Interim Conclusion}
\label{sec:tasklet:conclusion}

\pgfplotsinvokeforeach{sorted,reverse,almost,uniform,zipf,normal}{
	\pgfplotstablereadnamed{data/shell/ciura/uint32/#1.txt}{tableShellCiura_32#1}
	\pgfplotstablereadnamed{data/shell/ciura/uint64/#1.txt}{tableShellCiura_64#1}
}

This subsection offers a summary the algorithms presented so far and the findings on them.
\Cref{fig:tasklet:summary} serves as succinct overview of their runtimes on all tested input distributions and data types.

For small inputs with up to 16 elements, \IS{} is arguably the best sorting algorithm as it offers the best performance on the tested input distributions and, additionally, sorts stably and in-place.
However, there is still room for improvement as its compilation is suboptimal, especially in case of the \IS{} with implicit sentinel values.

\begin{figure}[p]
	\pgfplotsset{
		height=2.925cm,
		adaptive group=2 by 3,
		groupplot ylabel={Cycles / \((n \lb n)\)},
		x from 16 to 1024 minor,
		/tikz/mark repeat=2,
		legend columns=3,
	}

	\NewDocumentCommand{\allalgos}{m}{
		\plotpernlognwitherrors{TrivialBC}{tableQuickRand_32#1}
		\pgfplotsset{update limits=false}
		\plotpernlognwitherrors[x filter/.expression={\thisrow{n} <= 1024 ? \pgfmathresult : nan}]{MergeHalfSpace}{tableMergeStart16_32#1}
		\plotpernlognwitherrors[x filter/.expression={\thisrow{n} <= 1024 ? \pgfmathresult : nan}]{MergeHalfSpace}{tableMergeStart32_32#1}
		\pgfplotsset{update limits=true}
		\plotpernlognwitherrors{HeapOnlyDown}{tableHeap_32#1}
		\pgfplotsset{update limits=false}
		\plotpernlognwitherrors{Ciura}{tableShellCiura_32#1}
	}
	\begin{subfigure}{\textwidth}
		\tikzsetnextfilename{tasklet_summary_32}
		\begin{tikzpicture}[plot]
			\begin{groupplot}[
				ymin=0,
				ymax=150,
			]
				\nextgroupplot[title=Sorted\strut]
				\pgfplotsset{legend to name=leg:tasklet:summary, legend entries={\QS{}, \MS{} (stable), \MS{} (unstable), \HS{}, \ShS{}}}
				\allalgos{sorted}
				%
				\nextgroupplot[title=Reverse Sorted\strut]
				\allalgos{reverse}
				%
				\nextgroupplot[title=Almost Sorted\strut]
				\allalgos{almost}
				%
				\nextgroupplot[title=Uniform\strut]
				\allalgos{uniform}
				%
				\nextgroupplot[title=Zipf's\strut]
				\allalgos{zipf}
				%
				\nextgroupplot[title=Normal\strut]
				\allalgos{normal}
			\end{groupplot}
		\end{tikzpicture}
		\caption{
			32-bit
		}
		\medskip
	\end{subfigure}

	\RenewDocumentCommand{\allalgos}{m}{
		\plotpernlognwitherrors{TrivialBC}{tableQuickRand_64#1}
		\pgfplotsset{update limits=false}
		\plotpernlognwitherrors[x filter/.expression={\thisrow{n} <= 1024 ? \pgfmathresult : nan}]{MergeHalfSpace}{tableMergeStart16_64#1}
		\plotpernlognwitherrors[x filter/.expression={\thisrow{n} <= 1024 ? \pgfmathresult : nan}]{MergeHalfSpace}{tableMergeStart32_64#1}
		\pgfplotsset{update limits=true}
		\plotpernlognwitherrors{HeapOnlyDown}{tableHeap_64#1}
		\pgfplotsset{update limits=false}
		\plotpernlognwitherrors{Ciura}{tableShellCiura_64#1}
	}
	\begin{subfigure}{\textwidth}
		\tikzsetnextfilename{tasklet_summary_64}
		\begin{tikzpicture}[plot]
			\begin{groupplot}[
				ymin=0,
				ymax=200,
			]
				\nextgroupplot[title=Sorted\strut]
				\allalgos{sorted}
				%
				\nextgroupplot[title=Reverse Sorted\strut]
				\allalgos{reverse}
				%
				\nextgroupplot[title=Almost Sorted\strut]
				\allalgos{almost}
				%
				\nextgroupplot[title=Uniform\strut]
				\allalgos{uniform}
				%
				\nextgroupplot[title=Zipf's\strut]
				\allalgos{zipf}
				%
				\nextgroupplot[title=Normal\strut]
				\allalgos{normal}
			\end{groupplot}
		\end{tikzpicture}
		\caption{
			64-bit
		}
		\bigskip
	\end{subfigure}

	\hfil\pgfplotslegendfromname{leg:tasklet:summary}\hfil
	\caption{
		Comparison of the average runtimes of the main algorithms presented in this section.
		The coloured areas denote the three-sigma intervals, that is, the 99.7\% confidence intervals.
	}
	\label{fig:tasklet:summary}
\end{figure}

