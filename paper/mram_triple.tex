\section{The Triple Buffer}
\label{sec:mram:triple}

Before beginning to merge, it is yet again beneficial to form starting runs.
The formation works by loading a block of \ac{MRAM} data into the \ac{WRAM}, sorting it with one of the algorithms presented in \cref{sec:tasklet}, and writing the sorted block back to the \ac{MRAM}.
As those sorting algorithm rely on the data being present entirely within the \ac{WRAM}, the functions \lstinline|mram_read| and \lstinline|mram_write| are used directly.
Contrary to the \ac{WRAM} \MS*{} with starting runs of length 14, the lengths of the starting runs of the \ac{MRAM} \MS{} go well into the hundreds.
The reason is that longer starting runs reduce the number of rounds of \ac{MRAM} merging and, thus, reduce \acp{DMA}, which are relatively costly.
However, again similar to the \ac{WRAM} \MS*{}, it can be beneficial to slightly reduce the starting run length to achieve more balanced and faster rounds.
Nonetheless, the impact on the total runtime between 500, 600 and 700 elements per starting run is in the magnitude of one per mille, so for the sake of simplicity, the starting run length is set to the maximum amount of data which a tasklet can store in the \ac{WRAM}.

This does raise the question how a large \ac{WRAM} buffer for the starting run formation can be allocated while still leaving memory for the two sequential-read buffers used later during merging.
The answer is a \emph{triple buffer} which consists of a general-purpose buffer followed by two consecutive sequential-read buffers.
If no sequential reader is in use, the triple buffer can be regarded as one contiguous buffer.
To initialise the triple buffer, a tasklet first calls \lstinline|mem_alloc| to allocate \(\cachesize{}\) many bytes on the heap, with \cachesize{} being a compile-time constant divisible by 8.
This memory is referred to as \emph{cache} and will be used later to store merged runs.
Subsequently, the tasklet calls \lstinline|seqread_alloc| twice.
Due to the stack nature of the heap, the two sequential-read buffers are allocated directly behind the cache.
To ensure the contiguity of the triple buffer if more than one tasklet is present, a mutex is employed such that only one tasklet initialises its triple buffer at a time.
The entire triple buffer has the size \(\triplebuffersize ≔ \cachesize{} + 4 × \seqreadcachesize{}\), which is, for simplicity, the minimum number of allocated bytes and, therefore, the same for all tasklets even if some calls of \lstinline|seqread_alloc| introduced padding.
Note that padding may appear in front of each first sequential-read buffer only, since the heap pointer is necessarily at a multiple of \seqreadcachesize{} after any call of \lstinline|seqread_alloc|.
Because of otherwise wasted memory, it makes little sense to set \cachesize{} to a value which is not a multiple of \seqreadcachesize{}.
The optimal values for both \cachesize{} and \seqreadcachesize{} are determined in \cref{sec:mram:merge:performance}.

\begin{figure}[b]
	\centering
	\tikzsetnextfilename{triple_buffer}
	\begin{tikzpicture}[sketch]
		\def\cache{8}
		\def\padding{2}
		\def\buffer{8}
		\def\shortenbraceby{0.1}

		% Buffer.
		\fill[excess] (0.5*\buffer+\padding+\cache, 0) rectangle (1*\buffer+\padding+\cache, 1);
		\fill[excess] (1.5*\buffer+\padding+\cache, 0) rectangle (2*\buffer+\padding+\cache, 1);

		\draw (0, 0) grid (2*\buffer+\padding+\cache, 1);

		% Braces.
		\draw[brace] (                      0+\shortenbraceby, 1) -- +(  \cache-2*\shortenbraceby, 0) node[pos=0.5, above=1ex] {cache\vphantom{p}};
		\draw[brace] (                 \cache+\shortenbraceby, 1) -- +(\padding-2*\shortenbraceby, 0) node[pos=0.5, above=1ex] {padding};
		\draw[brace] (        \padding+\cache+\shortenbraceby, 1) -- +( \buffer-2*\shortenbraceby, 0) node[pos=0.5, above=1ex] {first buffer\vphantom{p}};
		\draw[brace] (\buffer+\padding+\cache+\shortenbraceby, 1) -- +( \buffer-2*\shortenbraceby, 0) node[pos=0.5, above=1ex] {second buffer\vphantom{p}};

		\draw[brace down] (\shortenbraceby, 0) -- +(2*\buffer+\cache-2*\shortenbraceby, 0) node[pos=0.5, below=1ex] {\triplebuffersize\vphantom{f}};

		% Item identifiers.
		\pgfmathsetmacro{\looplastindex}{int(2*\buffer+\padding+\cache-1)}
		\foreach \i in {0,...,\looplastindex}{
			\pgfmathsetmacro{\itemindex}{int( 4*(22+\i) )}
			\node[address] at (\i+.5, 0.5) { \hexa{\itemindex}\vphantom{f} };
		}
	\end{tikzpicture}

	\caption{
		An exemplary triple buffer for 32-bit elements with \cachesize{} set to 32 and \seqreadcachesize{} set to 16.
		The hexadecimal numbers denote \ac{WRAM} addresses.
		Note that the padding is still used when regarding the triple buffer as one contiguous buffer.
		Then, however, the last bytes of the second sequential-read buffer are unused.
	}
	\label{fig:merge:triple}
\end{figure}
