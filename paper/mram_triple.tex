\section{Triple Buffers}
\label{sec:mram:triple}

It is yet again beneficial to form starting runs.
This is done by loading a section of the MRAM into the WRAM, sorting it through one of the algorithms presented in \cref{sec:tasklet}, and writing the sorted section back to the MRAM.
As those sorting algorithm rely on the data being present entirely within the WRAM, sequential readers are not useful until later when merging runs.
Instead, the functions \lstinline|mram_read| and \lstinline|mram_write| are used directly.

Contrary to the WRAM \MS{} with starting runs of length 14, the starting runs of the MRAM \MS{} are much longer, easily containing a thousand elements and more.
The reason is that longer starting runs reduce the number of rounds of the MRAM \MS{} and, thus, reduce the DMAs to the MRAM, which are relatively costly compared to accesses to the WRAM.
According to \citeauthor{mutlu2022Benchmarking}~\cite[8\psq]{mutlu2022Benchmarking}, accessing 64-bit integers in the WRAM is about four and a half times faster than accessing 64-bit integers in the MRAM.
Since accessing 32-bit integers in the WRAM takes just as long as accessing 64-bit integers and since the performance of DMAs depends only on the total number of bytes, there is still a speedup of roughly 2 of WRAM accesses over MRAM accesses.
However, again similar to the WRAM \MS{}, it can be beneficial to slightly reduce the starting run length to achieve more balanced and faster rounds.
Nonetheless, the runtime difference between 1000, 1100, and 1200 elements per starting run is in the magnitude of one per mil.
For this reason, the starting run length is set simply to the maximum number of integers which the WRAM allotted to a tasklet can hold.

This does raise the question what said maximum is.
As calculated at the start of \cref{sec:tasklet}, each tasklet is allotted no more than \qty{5957}{\byte}.
Subtracting \qty{768}{\byte} for the stack of the tasklet and the call stack for the WRAM \QS{} leaves \qty{5189}{\byte} in the heap, that is a little less than 1300 32-bit integers.\todo{Noch einmal später aktualisieren}
This space must be used for both forming the starting runs (without any sequential-read buffers) and merging the runs (with two sequential-read buffers).
One possibility would be to allocate one starting run buffer of size \qty{5957}{\byte}, reset the entire heap after the starting runs have been formed and, then, allocate the sequential-read buffers.

Far more flexibility offers the introduction of a \emph{triple buffer} which consists of a general-purpose buffer followed by two consecutive sequential-read buffers.
If sequential readers are not in use, the triple buffer can be regarded as a single, big buffer.
To initialise the triple buffer, a tasklet first calls \lstinline|mem_alloc| to allocate \(\text{\lstinline|CACHE_SIZE|} + \min\braces{ 8, \text{\lstinline|sizeof(T)|} }\) many bytes on the heap, where \lstinline|CACHE_SIZE| is a user-defined compile-time constant and \lstinline|T| is the data type of the input.
This memory, excluding the first \(\min\braces{ 8, \text{\lstinline|sizeof(T)|} }\) many bytes, is referred to as \emph{cache} and will be used later to store merged runs.
The element right before the beginning of the cache is set to the lowest possible value of \lstinline|T| so that it acts as sentinel value for \IS{}, hence the allocation of at least \lstinline|sizeof(T)| additional bytes.
In case of integers smaller than 8 bytes, the address of the cache would not be aligned on 8 bytes anymore, hence the allocation of at least 8 additional bytes to simplify DMAs.
After the initialisation of the cache, the tasklet calls \lstinline|seqread_alloc| twice.
Due to the stack nature of the heap, the two sequential-read buffers are allocated directly behind the cache.
To ensure the contiguity of the triple buffer if more than one tasklet is present, a mutex is employed to ensure than only one tasklet initialises its triple buffer at a time.
The entire triple buffer has the size \(\text{\lstinline|TRIPLE_BUFFER_SIZE|} \coloneqq \text{\lstinline|CACHE_SIZE|} + 4 \times \text{\lstinline|SEQREAD_CACHE_SIZE|}\), which is, for simplicity, the minimum guaranteed number of allocated bytes and, therefore, the same for all tasklets even if some calls of \lstinline|seqread_alloc| skipped some bytes.
Note that only the first call of \lstinline|seqread_alloc| by each tasklet may skip some memory since the stack pointer is at a multiple of \lstinline|SEQREAD_CACHE_SIZE| after every call of \lstinline|seqread_alloc|.
