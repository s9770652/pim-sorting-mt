\subsection{\texorpdfstring{\ShS{}}{ShellSort}}

\pgfplotstableread{data/shell/no_third_step.txt}{\tableShellNoThirdStep}
\pgfplotstableread{data/shell/h1=7.txt}{\tableShellVII}
\pgfplotstableread{data/shell/h1=8.txt}{\tableShellVIII}
\pgfplotstableread{data/shell/h1=9.txt}{\tableShellIX}
\pgfplotstableread{data/shell/h1=10.txt}{\tableShellX}
\pgfplotstableread{data/shell/h1=11.txt}{\tableShellXI}
\pgfplotstableread{data/shell/h1=12.txt}{\tableShellXII}
\pgfplotstableread{data/shell/h1=13.txt}{\tableShellXIII}
\pgfplotstableread{data/shell/h1=14.txt}{\tableShellXIV}
\pgfplotstableread{data/shell/h1=15.txt}{\tableShellXV}
\pgfplotstableread{data/shell/h1=16.txt}{\tableShellXVI}
\pgfplotstableread{data/shell/h1=17.txt}{\tableShellXVII}
\pgfplotstableread{data/shell/custom.txt}{\tableShellCustom}

\begin{figure}
	\begin{tikzpicture}[plot]
		\begin{groupplot}[
			width=0.4358\linewidth,
			group/group size=2 by 1,
			groupplot xlabel={Input Length \(n\)},
			xtick distance=3,
			minor xtick=data,
			legend columns=-1,
		]
			\nextgroupplot[ylabel=Cycles / \(n^2\), ymin=0, ymax=70, extra y ticks={70}, legend to name=leg:shell_sort]
			\legend{\(1\), \(...\), \(9\)}
			\pgfplotsinvokeforeach{1,...,9}{
				\plotpernn{#1}{\tableSmallSorts}
			}
			%
			\nextgroupplot[ylabel=Speed-up, ymin=0.6, ymax=1.2]
			\pgfplotsset{cycle list shift=1}
			\pgfplotsinvokeforeach{2,...,9}{
				\plotspeedup{#1}{1}{\tableSmallSorts}
			}
		\end{groupplot}
	\end{tikzpicture}

	\hfil\pgfplotslegendfromname{leg:shell_sort}\hfil
	\caption{
		Comparison of \IS{} (1) and various \ShS*{} (2--9).
		Each \ShS{} does one \IS{} pass with a step size between 2 and 9 before doing a pass of regular \IS{}.
		The speed-ups are with respect to the \IS{}.
	}
	\label{fig:shell_sort}
\end{figure}

\IS{} suffers from small elements at the end of the input, since those have to be brought to the front through \(\bigoh{n}\) comparisons and swaps.
\ShS{}, proposed by Donald L. Shell in 1959~\cite{Shell1959AHS}, remedies this by doing multiple passes of \IS{} with different step sizes:
In round~\(r\) with step size \(h_r\), the input array is divided into the subarrays of indices \((i, h_r + i, 2 h_r + i, \dots)\) for \(i = 0, \dots, h_r - 1\) which then get sorted individually through \IS{}.
The step size get smaller each round, with the final step size being \(1\) such that a regular \IS{} is performed.
Intuitively, the individual \IS*{} are fast since elements which need to travel long distances already did big jumps.
Finding the right balance between the heightened overhead through multiple \IS{} passes and the shortened runtime of each \IS{} pass is subject to research to this day \cite{skean2023optimization, lee2021empirically} and depends on the cost of the operations (comparing, swapping, looping).

Let us first focus on small input arrays where only two rounds with step sizes~\(h_1\) and \(1\) suffice.
The previous results on \IS{} suggest that \ShS{} should make use of~\(h_1\) sentinel values lest bounds checking eats any gain up.
\Cref{fig:shell_sort} shows that the additional rounds starts to pay off at around 20 elements for \(h_1 \ge 3\).
Bear in mind that these measurements were conducted on a uniform input distribution;
if \ShS{} is used by another algorithm on a subarray, these thresholds may be higher or even non-existent due to some degree of presorting.

\NewDocumentCommand{\shellscatter}{m m}{
	\addplot+ plot [select coords between index={#1}{#1}, forget plot] table [x=µ_#2, y expr={6}] {\tableShellNoThirdStep};
	\ifnumless{#2}{7}{
		\addplot+ plot [select coords between index={#1}{#1}, forget plot] table [x=µ_#2, y expr={7}] {\tableShellVII};
	}{}
	\ifnumless{#2}{8}{
		\addplot+ plot [select coords between index={#1}{#1}, forget plot] table [x=µ_#2, y expr={8}] {\tableShellVIII};
	}{}
	\ifnumless{#2}{9}{
		\addplot+ plot [select coords between index={#1}{#1}, forget plot] table [x=µ_#2, y expr={9}] {\tableShellIX};
	}{}
	\addplot+ plot [select coords between index={#1}{#1}] table [x=µ_#2, y expr={10}, forget plot] {\tableShellX};
	\addplot+ plot [select coords between index={#1}{#1}] table [x=µ_#2, y expr={11}, forget plot] {\tableShellXI};
	\addplot+ plot [select coords between index={#1}{#1}] table [x=µ_#2, y expr={12}, forget plot] {\tableShellXII};
	\addplot+ plot [select coords between index={#1}{#1}] table [x=µ_#2, y expr={13}, forget plot] {\tableShellXIII};
	\addplot+ plot [select coords between index={#1}{#1}] table [x=µ_#2, y expr={14}, forget plot] {\tableShellXIV};
	\addplot+ plot [select coords between index={#1}{#1}] table [x=µ_#2, y expr={15}, forget plot] {\tableShellXV};
	\addplot+ plot [select coords between index={#1}{#1}] table [x=µ_#2, y expr={16}, forget plot] {\tableShellXVI};
	\addplot+ plot [select coords between index={#1}{#1}] table [x=µ_#2, y expr={17}] {\tableShellXVII};
}

\begin{figure}[p]
	\begin{tikzpicture}[plot]
		\newcommand{\setn}[1]{\textit{n} = #1}
		\begin{groupplot}[
			width=0.4617\linewidth,
			group/group size=2 by 3,
			group style={horizontal sep=3em, vertical sep=3.5em},
			groupplot xlabel={Mean [\(10^4\) Cycles]},
			groupplot ylabel={\(h_1\)},
			scaled x ticks=base 10:-4,
			xtick scale label code/.code={},  % removes exponent underneath the axis
			ytick={6, 7, 9, ..., 17},
			yticklabels={/, \(7\), \(9\), \(...\), \(17\)},
			/tikz/only marks,
			cycle list shift=2,  % for sharing colours with the previous figure
			legend columns=-1,
		]
			\nextgroupplot[title={Input Length \setn{24}}]
			\pgfplotsinvokeforeach{3,...,9}{
				\shellscatter{0}{#1}
			}
			%
			\nextgroupplot[title={Input Length \setn{32}}]
			\pgfplotsinvokeforeach{3,...,9}{
				\shellscatter{1}{#1}
			}
			%
			\nextgroupplot[title={Input Length \setn{48}}]
			\pgfplotsinvokeforeach{3,...,9}{
				\shellscatter{2}{#1}
			}
			%
			\nextgroupplot[title={Input Length \setn{64}}]
			\pgfplotsinvokeforeach{3,...,9}{
				\shellscatter{3}{#1}
			}
			%
			\nextgroupplot[title={Input Length \setn{96}}]
			\pgfplotsinvokeforeach{3,...,9}{
				\shellscatter{4}{#1}
			}
			%
			\nextgroupplot[title={Input Length \setn{128}}, legend to name=leg:shell_three_rounds]
			\legend{\(3\), \(4\), \(5\), \(6\), \(7\), \(8\), \(9\)}
			\pgfplotsinvokeforeach{3,...,9}{
				\shellscatter{5}{#1}
			}
		\end{groupplot}
	\end{tikzpicture}

	\hfil\pgfplotslegendfromname{leg:shell_three_rounds}\hfil
	\caption{
		Runtimes of \ShS*{} with two rounds (/) and three rounds (7--17).
		The coloured symbols encode the step size \(h_1\) for two-round \ShS*{} and the step size~\(h_2\) for three-round \ShS*{}.
		For the latter, the step size \(h_1\) is noted on the y-axes.
	}
	\label{fig:shell_three_rounds}
\end{figure}

\begin{figure}
	\begin{tikzpicture}[plot]
		\begin{groupplot}[
			width=0.4358\linewidth,
			group/group size=2 by 1,
			groupplot xlabel={Input Length \(n\)},
			xtick distance=16,
			minor xtick=data,
			legend columns=-1,
		]
			\nextgroupplot[ylabel=Cycles / \((n \lb n)\), ymin=0, ymax=225, extra y ticks={50, 150, 225}, legend to name=leg:quick_vs_insert_vs_shell]
			\legend{Insertion, Shell, Quick}
			\plotpernlogn{InsertSent}{\tableShellCustom}
			\plotpernlogn{ShellCiura}{\tableShellCustom}
			\plotpernlogn{Quick}{\tableShellCustom}
			%
			\nextgroupplot[ylabel=Speed-up, ymin=0.2, ymax=1.2]
			\plotspeedup{InsertSent}{Quick}{\tableShellCustom}
			\plotspeedup{ShellCiura}{Quick}{\tableShellCustom}
		\end{groupplot}
	\end{tikzpicture}

	\hfil\pgfplotslegendfromname{leg:quick_vs_insert_vs_shell}\hfil
	\caption{
		Comparison of \IS{}, \ShS{} and \QS{}.
		The \ShS{} uses the steps sizes \(h = (6, 1)\) for \(n \le 64\) and \(h = (17, 4, 1)\) elsewise.
		The speed-ups are with respect to the \QS{}.
		\todo[inline]{\QS{} mit Zufallspivot auch noch einbauen?}
		\todo[inline]{verschiedene Verteilungen?}
	}
	\label{fig:quick_vs_insert_vs_shell}
\end{figure}

When moving to greater input lengths (\cref{fig:shell_three_rounds}), the differences in performance between the two-round \ShS*{} become more pronounced;
especially the ones with \(h_1 = 3\) and \(h_1 = 4\) fall off whereas the one with \(h_1 = 6\) holds its ground quite well.
With more than 64 elements, three round get worthwhile to consider.
Interestingly, many \ShS*{} with \(h_2 = 4\) take the lead whilst the ones with \(h_2 = 6\) are mid-table.
This is in accordance with \citeauthor{10.1007/3-540-44669-9_12} \cite{10.1007/3-540-44669-9_12} who noted~\(h = (17, 4, 1)\) to be the optimal triplet for 128 elements.
It is noteworthy, though, that he measured the quadruplet~\(h = (38, 9, 4, 1)\) to be about 5\% faster in the MIX machine model.
On a DPU, this sequence leads to a runtime of nearly exactly 74.000 cycles, placing it only mid-table.
Without access to Ciura's original code, giving a satisfactory explanation for the discrepancy is hard, however.

But would pushing the limits of \ShS{} even be rewarding?
Two issues come up.
Firstly, greater input lengths require greater steps \Dash well into the three digits for \(n \approx 1000\) \cite{skean2023optimization, 10.1007/3-540-44669-9_12} \Dash and those in turn require more sentinel values.
But the more sentinel values are stored, the less space is available for the actual input array, leading to smaller runs and thus hurting the overall sorting algorithm.
Explorative testing suggests that falling back to bounds checking for big steps is too punishing.
Secondly, there simply are better alternatives, namely \QS{} (which will be discussed in more detail in the next section).
\Cref{fig:quick_vs_insert_vs_shell} shows that even though \ShS{} takes just a fraction of the time \IS{} takes \Dash apparently achieving a runtime between \(\bigomega{n \lb n}\) and \(\bigoh{n \lb^2 n}\) \Dash\negthinspace, \QS{} beats both from 20 elements onwards.
Even \QS{}'s standard deviation of 1429 cycles at 128 elements is superior to \ShS{}'s 2670 cycles.
Together with \cref{fig:shell_sort}, this means that \ShS{} is not worth using at all and will, consequently, not be improved upon in this thesis.
