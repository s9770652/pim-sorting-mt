\subsubsection*{Evaluation of the Performance}
\label{subsubsec:tasklet:heap:performance}

\pgfplotsinvokeforeach{sorted,reverse,almost,uniform,zipf,normal}{
	\pgfplotstablereadnamed{data/heap/uint32/#1.txt}{tableHeap_32#1}
	\pgfplotstablereadnamed{data/heap/uint64/#1.txt}{tableHeap_64#1}
}

\pgfplotsset{
	heap/.style={
		adaptive group=1 by 2,
		groupplot xlabel={Input Length \(n\)},
		groupplot ylabel={Cycles / \((n \lb n)\)},
		xmode=log,
		xtick={16, 32, 64, 128, 256, 512, 1024},
		xticklabels={\(16\), \(32\), \(64\), \(128\), \(256\), \(512\), \(1024\)},
		legend columns=-1,
	},
}

\def\heapalgos{HeapOnlyDown,HeapUpDown,HeapSwapParity}

\begin{figure}
	\tikzsetnextfilename{heap_runtime}
	\begin{tikzpicture}[plot]
		\begin{groupplot}[heap, ytick distance=5]
			\nextgroupplot[title={32-bit\strut}, ymin=130, ymax=155, legend to name=leg:heap:runtime]
			\legend{\HS{} (top-down), \HS{} (bottom-up), \HS{} (swap parity)}
			\expandafter\pgfplotsinvokeforeach\expandafter{\heapalgos}{
				\plotpernlogn{#1}{tableHeap_32uniform}
			}
			%
			\nextgroupplot[title={64-bit\strut}, ymin=155, ymax=180]
			\expandafter\pgfplotsinvokeforeach\expandafter{\heapalgos}{
				\plotpernlogn{#1}{tableHeap_64uniform}
			}
		\end{groupplot}
	\end{tikzpicture}

	\hfil\pgfplotslegendfromname{leg:heap:runtime}\hfil
	\caption{
		Comparison of the runtimes of three different \HS{} implementations on uniformly distributed 32-bit integers and 64-bit integers, respectively.
	}
	\label{fig:heap:runtime}
\end{figure}

The measurements are visualised in \cref{fig:heap:runtime,fig:heap:runtime_uint32,fig:heap:runtime_uint64}.
In general, the performance of \HS{} is even less volatile than that of \MS{} with respect to the input distribution.
Reverse sorted inputs are sorted the fastest since they are already max-heaps so the heap-building phase is short, while sorted inputs are sorted the slowest since they are min-heaps so the heap-building phase is long.
Nonetheless, the reverse sorted inputs get sorted just about 10\% faster and less on average.
This low volatility cannot hide the fact that the total runtime is abysmal across the board:
Compared to \MS{}, the runtimes are between 50\% to 100\% higher, depending on the input distribution!

The normalised runtimes of the top-down \HS{} and the bottom-up \HS{} with swap parity show a slight upwards trends, whereas those of the bottom-up \HS{} with swap disparity mostly shows a slight downwards trends with the exception of reverse sorted inputs, where it is also an upwards trend, albeit even slighter.
Interesting are their rankings relative to each other:
Value checks on 64-bit integers take two instructions, so that the savings of the bottom-up \HS{} with swap disparity allow it to outperform the top-down \HS{} even for short inputs, and the advantage grows with the input length.
This makes sense as roughly 50\% of the vertices are leaves and 25\% are fathers of leaves, no matter the total size.
Therefore, the percentage of formerly last leaves sifting down to the bottom remains steady but the travelled distance increases.
Value checks on 32-bit integers, on the other hand, take only one instruction, so that their reduction is overshadowed by the increased overhead from the longer downwards-sifting and the added upwards-sifting, whence the lead of the top-down \HS{}.
Indeed, at around 2000 elements, the bottom-up \HS{} with swap disparity overtakes the top-down \HS{} because of their inverse trends, but the lead is narrow even at 6000 elements.
However, these long inputs are less interesting anyway due to the limited WRAM and the multiple tasklets used in most applications.

The bottom-up \HS{} with swap parity consistently trails behind.
This comes as no surprise since the overhead of its considerably prolonged upwards-sifting bears no proportion to the few swaps saved.
This holds true even for 64-bit integers as moves still cost only one instruction.
Unrolling the upwards-sifting proved to be unhelpful.
