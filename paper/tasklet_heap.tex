\section{\texorpdfstring{\HS{}}{HeapSort}}
\label{sec:tasklet:heap}

This algorithm guarantees a runtime of \(\bigoh{n \log n}\), too, and also sorts in-place at the cost of being unstable.
A \emph{max-heap} (or, simpler, \emph{heap}) is a binary tree of logarithmic depth whose layers are fully filled, possibly with the exception of the last layer, which must be filled from left to right.
Each vertex contains a key, and the key of each father must be at least as great as those of his sons.
As a consequence of this \emph{heap order}, the root contains the greatest key.

A heap with \(n\) keys can be represented as an array of length \(n\) using a bijective mapping between the vertex positions and the array indices (see later).
After the heap has been built in-place from the input array in time \(\bigoh{n}\), the sorting works as follows:
At the start of round~\(r = 1, \dots, n\), the first \(n - (r - 1)\) elements of the array represent the heap and the last \(r - 1\) elements the end of the sorted output.
Upon removal of the root, which contains the \(r\)th greatest element of the input, the heap structure must be restored in time \(\bigoh{\log n}\).
Since the heap has shrunken by one key, the key of the removed root can be stored at the freed-up position directly behind the end of the heap.

\paragraph{Indexing}
With a zero-based indexing, the sons of a vertex \(i\) can be calculated with the well-known formulas \(2i + 1\) and \(2n + 2\).
With a one-based indexing, the formulas turn into \(2i\) and \(2i + 1\).
The compiler automatically turns the multiplication by two into a left-shift by one.
Since DPUs possess an instruction called \lstinline|lsl_add| which first shifts leftwards and then adds an offset (useful \eg{} for array indexing), the formulas \(2i + 1\) and \(2i\) take the same amount of time to compute.

Nevertheless, the zero-based indexing is about 7\% slower despite \lstinline|lsl_add| being indeed in use.
The reason is that only the number of bits to shift can be passed as immediate value, that is as plain number, but not the offset, which must be passed via a register.
While DPUs have a read-only register storing the number \(1\) at disposal, read-only registers can only ever be the first register argument, not the second one, which, for \lstinline|lsl_add|, would be the offset.
As a consequence, the compiler moves the number \(1\) to a register whenever \(2i + 1\) is to be computed, only to immediately overwrite the \(1\) with the result from \lstinline|lsl_add|.
Hence, the calculation of \(2i + 1\) does take one instruction more than \(2n\) after all.

\paragraph{Sifting Direction}
Once the heap is built, the \emph{top-down} \HS{} proceeds as follows:
At the start of each round, the root and the rightmost leaf (\enquote{last leaf}) in the bottom layer swap places.
The root is now in the right position, but the formerly last leaf may violate the heap order, that is, the root may have a less key than one or both of its sons.
The greater of the two sons is determined, and the root and the greater son swap places.
This downwards-sifting of the former leaf continues iteratively until the heap order is restored.

In contrast, the \emph{bottom-up} \HS{} \cite{wegener1993heapsort} works as follows:
At the start of each round, the key of the root is removed so that a hole is now at the top of the heap.
Then, the greater of the two sons of the hole is determined, and they swap places.
This downwards-sifting of the hole continues iteratively until it becomes a leaf.
Now, the last leaf is moved to the position of the hole.
Should the moved leaf be greater than its father, then the heap order is now violated.
If so, it needs to be sifted upwards by iteratively swapping positions with its respective father until the heap order is restored.
At last, the original root key can be put where the formerly last leaf used to be.

The motivation behind these variants is at follows:
In each step where the top-down \HS{} sifts the formerly last leaf downwards, two value checks (Which son is greater? Is the father less than the greater son?) need to be done.
The leaves of a heap tend to be small so the downwards-sifting lasts awhile.
As opposed to this, each step of the bottom-up \HS{} needs only one value check (Is the father less than the greater son?).
Both \HS*{} sift downwards similarly long so many checks can be saved.
Since the last leaf effectively takes the place of another leaf and since both are likely small, the upwards-sifting should be short-lived and, hopefully, not eat the gain up.

The upwards-sifting reverts some of the changes done by the downwards-sifting.
The bottom-up \HS{} can be brought to swap parity with the top-down \HS{} by the following change:
The downwards-sifting is traced but the keys are not actually moved.
Once the leaf where the hole would end up is reached, the sifting is backtracked until the bottommost key which is at least as great as the last leaf.
The backtracking is easily implemented as the father \(\floor{i/2}\) of a vertex \(i\) can be computed through one bitwise right-shift.
The position found is where the last leaf would end up after the upwards-sifting, so all keys below can stay put and all keys above move to their fathers' positions, that is, thither the swaps from the downwards-sifting would have put them.
This implementation variant makes the downwards-sifting even cheaper but the upwards-sifting must now go all the way up to the root.

\paragraph{Sentinel Values}
When \HS{} sifts a vertex downwards, it needs to determine the greater of its two sons before deciding whether and whither to move.
If and only if the heap has an even number of vertices, there is a left son without a right brother:
the rightmost leaf in the bottom layer.
Instead of adding some check on whether the right brother exists, one can rather add the missing leaf and give it the smallest possible key each time the heap reaches an even size.
Thus, if it has been confirmed that a left son exists, a right one does also exist, and if two brothers contain the same key, the left one should be considered greater.
Bounds checks on whether a left son exists are still required lest \HS{} loses its in-place property, since there are about \(n/2\) leaves which all would need sentinel sons.

Likewise, whenever \HS{} sifts upwards and considers the father \(\floor{i/2}\) of a vertex \(i\), it will only proceed if the father is less.
Since the fatherless root has index \(1\) and the result of a right-shift is truncated towards \(0\), the formula \(\floor{i/2}\) yields \(0\), so it makes sense to set the element at index \(0\) to the greatest possible key to stop any upwards-sifting.
The savings from these approaches were around the 13\% mark.

\paragraph{Code Duplication}
A strategy particularly useful for \HS{}, although also employed in \MS{}, is code duplication.
Handling the greater of two sons is the fastest if the logic is written twice, once for either son, and then executed conditionally;
logic written once for a generalised variable holding the greater son is compiled considerably worse.
The savings from this approach were around the 7\% mark.

\subsection*{Investigation of the Compilation}
\label{sec:tasklet:heap:compilation}
\addcontentsline{toc}{subsection}{\nameref{sec:tasklet:heap:compilation}}

Under zero-based indexing, the indices of the sons of a vertex with index \(i\) are \(2 i + 1\) and \(2 i + 2\), whilst the one of its father is \(\floor{(i - 1)/ 2}\).
Under one-based indexing, the indices of the sons of a vertex with index \(i\) are \(2 i\) and \(2 i + 1\), whilst the one of its father is \(\floor{i / 2}\).
The formula \(\floor{i / 2}\) is computable through a bitwise shift one place to the right, whereas \(\floor{(i - 1)/ 2}\) requires a subtraction before the bitwise shift.
Since the bottom-up \HS*{} rely heavily on finding fathers during backtracking, one-based indexing is clearly superior.

Consistency alone would suggest one-based indexing for all types of \HS{}.
However, the first \HS{} implemented was the top-down \HS{}, which only ever sifts down.
The picture is not so clear if focussing only on that version of \HS{}.
The compiler automatically turns multiplications by \(2\) into a bitwise shift by one place to the left.
Next to a regular \lstinline|lsl| instruction for such bitwise shifts to the left, DPUs also possess an instruction called \lstinline|lsl_add| which first shifts to the left and then adds a number.
This way, the formulas \(2i + 1\) and \(2i\) take the same amount of time to compute.

Notwithstanding \lstinline|lsl_add| being indeed employed in the compilation, the zero-based indexing is about 7\% slower than one-based indexing.
The reason is that only the number of places to shift can be passed as immediate value, that is, as plain number, but not the addend, which must be passed via a register.
Whilst DPUs have a read-only register permanently storing the number~\(1\) at disposal, read-only registers can only ever be the first register argument, not the second one, which is the addend in case of \lstinline|lsl_add|.
As a consequence, the compiler moves the number \(1\) to a register whenever \(2i + 1\) is to be computed, only to immediately overwrite it with the result from \lstinline|lsl_add|.
Hence, the calculation of \(2i + 1\) does take twice as long as \(2 i\) after all.

There are plenty of other curious observations.
For example, the runtime difference between stopping \HS{} when one element remains in the heap and stopping \HS{} when only three elements remain (which then get sorted by \IS{}) reduces the runtime by tens of thousand of cycles.
Stopping \HS{} even earlier has comparatively little effect.
For comparison, sorting just three elements solely with \HS{} barely takes one thousand cycles at worst.

The undisputedly strangest observation was the following:
Before building a heap, a single sentinel leaf must be inserted if the input length is even.
Adding this leaf if the input length is odd makes no difference algorithmically, as it would be a left leaf never to be accessed due to the bounds checks.
However, adding an if-statement determining whether the sentinel leaf has to be placed has dramatic effects compared to placing the sentinel leaf unconditionally.
Since the parity of the input length is needed later in the function anyway, the conditional version is expected to gain one instruction.
Yet, when measuring the runtimes on 1024 elements, one can observe anything from a reduction by 5000 cycles over changes within the margin of error to increases by 25\,000 cycles, depending on the sifting direction and the input distribution.
Adding one or more sentinel leaves outside of the \HS{} functions has no impact on this behaviour.
A comparison of the compilations reveals minute differences at the beginnings of the \HS{} functions, none of which affect anything repeatedly executed.
The register usage does also not change in such a manner that the execution time of instructions is prolonged to 12 cycles, as described in the DPU SDK documentation \cite[Instruction Set Architecture -- Efficient scheduling]{upmemSDK}.


\subsection*{Evaluation of the Performance}
\label{sec:tasklet:heap:performance}

\pgfplotsinvokeforeach{sorted,reverse,almost,uniform,zipf,normal}{
	\pgfplotstablereadnamed{data/heap/uint32/#1.txt}{tableHeap_32#1}
	\pgfplotstablereadnamed{data/heap/uint64/#1.txt}{tableHeap_64#1}
}

\pgfplotsset{
	heap/.style={
		adaptive group=1 by 2,
		groupplot ylabel={Cycles / \((n \lb n)\)},
		x from 16 to 1024,
	},
}

\def\heapalgos{HeapOnlyDown,HeapUpDown,HeapSwapParity}

\begin{figure}
	\tikzsetnextfilename{heap_runtime}
	\begin{tikzpicture}[plot]
		\begin{groupplot}[heap, ytick distance=5]
			\nextgroupplot[title/.add={}{32-bit}, ymin=130, ymax=155, legend to name=leg:heap:runtime]
			\legend{\HS{} (top-down), \HS{} (bottom-up), \HS{} (swap parity)}
			\expandafter\pgfplotsinvokeforeach\expandafter{\heapalgos}{
				\plotpernlogn{#1}{tableHeap_32uniform}
			}
			%
			\nextgroupplot[title/.add={}{64-bit}, ymin=155, ymax=180]
			\expandafter\pgfplotsinvokeforeach\expandafter{\heapalgos}{
				\plotpernlogn{#1}{tableHeap_64uniform}
			}
		\end{groupplot}
	\end{tikzpicture}

	\hfil\pgfplotslegendfromname{leg:heap:runtime}\hfil
	\caption{
		Comparison of the runtimes of three different \HS{} implementations on uniformly distributed 32-bit integers and 64-bit integers, respectively.
	}
	\label{fig:heap:runtime}
\end{figure}

The measurements are visualised in \cref{fig:heap:runtime,fig:heap:runtime_uint32,fig:heap:runtime_uint64}.
In general, the performance of \HS{} is even less volatile than that of \MS{} with respect to the input distribution.
Reverse sorted inputs are sorted the fastest since they are already max-heaps so the heap-building phase is short, while sorted inputs are sorted the slowest since they are min-heaps so the heap-building phase is long.
Nonetheless, the reverse sorted inputs get sorted just about 10\% faster and less on average.
This low volatility cannot hide the fact that the total runtime is abysmal across the board:
Compared to \QS{} and \MS{}, the runtimes are between 50\% to 250\% higher, depending on the input distribution and the data type!

The normalised runtimes of the top-down \HS{} and the bottom-up \HS{} with swap parity show a slight upwards trends, whereas those of the bottom-up \HS{} with swap disparity mostly shows a slight downwards trends with the exception of reverse sorted inputs, where it is also an upwards trend, albeit even slighter.
Interesting are their rankings relative to each other:
Value checks on 64-bit integers take two instructions, so that the savings of the bottom-up \HS{} with swap disparity allow it to outperform the top-down \HS{} even for short inputs, and the advantage grows with the input length.
This makes sense as roughly 50\% of the vertices are leaves and 25\% are fathers of leaves, no matter the total size.
Therefore, the percentage of formerly last leaves sifting down to the bottom remains steady but the travelled distance increases.
Value checks on 32-bit integers, on the other hand, take only one instruction, so that their reduction is overshadowed by the increased overhead from the longer downwards-sifting and the added upwards-sifting, whence the lead of the top-down \HS{}.
Indeed, at around 2000 elements, the bottom-up \HS{} with swap disparity overtakes the top-down \HS{} because of their inverse trends, but the lead is narrow even at 6000 elements.
However, these long inputs are less interesting anyway due to the limited WRAM and the multiple tasklets used in most applications.

The bottom-up \HS{} with swap parity consistently trails behind.
This comes as no surprise since the overhead of its considerably prolonged upwards-sifting bears no proportion to the few swaps saved.
This holds true even for 64-bit integers as moves still cost only one instruction.
Unrolling the upwards-sifting proved to be unhelpful.

