\pgfplotstableread{data/quick/fallback.txt}{\tablequickfallback}

\begin{figure}
	\begin{tikzpicture}[plot]
		\begin{groupplot}[
			width=0.4358\linewidth,
			group/group size=2 by 2,
			group style={horizontal sep=5em, vertical sep=3em},  % For some reason, `group/vertical sep` does not work.
			groupplot xlabel={Input Length \(n\)},
			xmode=log,
			xtick={20, 32, 64, 128, 256, 512, 1024},
			xticklabels={\(20\), \(32\), \(64\), \(128\), \(256\), \(512\), \(1024\)},
			minor xtick={24, 48, 96, 192, 384, 768, 1536},  % For some reason, `data` does not work.
			legend columns=-1,
		]
			\nextgroupplot[ylabel=Cycles / \((n \lb n)\), ymin=0, legend to name=leg:quick_fallback]
			\legend{None, \(10\), \(11\), \(...\), \(16\)}
			\plotpernlogn{None}{\tablequickfallback}
			\pgfplotsinvokeforeach{10,...,16}{
				\plotpernlogn{#1}{\tablequickfallback}
			}
			%
			\nextgroupplot[ylabel=Speed-up]
			\plotspeedup{None}{13}{\tablequickfallback}
			\pgfplotsinvokeforeach{10,...,16}{
				\ifnumequal{#1}{13}{
					\pgfplotsset{cycle list shift=1}
				}{
					\plotspeedup{#1}{13}{\tablequickfallback}
				}
			}
			%
			\nextgroupplot[ylabel=Cycles / \((n \lb n)\)]
			\pgfplotsset{cycle list shift=1}
			\pgfplotsinvokeforeach{10,...,16}{
				\plotpernlogn{#1}{\tablequickfallback}
			}
			%
			\nextgroupplot[ylabel=Speed-up, /pgf/number format/.cd, precision=3, fixed zerofill=true]
			\pgfplotsset{cycle list shift=1}
			\pgfplotsinvokeforeach{10,...,16}{
				\ifnumequal{#1}{13}{
					\pgfplotsset{cycle list shift=2}
				}{
					\plotspeedup{#1}{13}{\tablequickfallback}
				}
			}
		\end{groupplot}
	\end{tikzpicture}

	\hfil\pgfplotslegendfromname{leg:quick_fallback}\hfil
	\caption{
		Comparison of \QS{} without fallback algorithm (None) and \QS*{} using \IS{} if only a few elements remain in a partition (10--16).
		The bottom row is zoomed in on the latter.
		The speed-ups are with respect to the \QS{} without fallback algorithm.
	}
	\label{fig:quick_fallback}
\end{figure}

\begin{description}
	\item[r0] start
	\item[r1] end
	\item[r23] return address
\end{description}
\begin{verbatim}
insertion_sort_nosentinel:
    jleu r0, r1, .LBB2_1  // Continue if array of positive length …
.LBB2_8:
    jump r23  // … else leave the function.
.LBB2_1:
    move r2, r0, true, .LBB2_2  // i ← start; Jump to beginning of outer loop.
.LBB2_5:
    move r4, r5  // ?
.LBB2_7:
    add r2, r2, 4  // i++
    sw r4, 0, r3  // *curr ← to_sort
    jgtu r2, r1, .LBB2_8  // If i > end, terminate.
.LBB2_2:  // Beginning of outer loop.
    lw r3, r2, 0  // to_sort ← *i;
    add r5, r2, -4  // prev ← i - 1
    move r4, r2  // curr ← i
    jltu r5, r0, .LBB2_7  // If prev < start, skip to the next iteration of the outer loop.
    move r5, r2  // (prev + 1) ← i
.LBB2_4:
    lw r6, r5, -4  // *prev
    jleu r6, r3, .LBB2_5  // If *prev > to_sort, terminate inner loop.
    add r4, r5, -4  // Store prev.
    add r7, r5, -8  // Store prev--.
    sw r5, 0, r6  // *curr ← *prev
    move r5, r4  // curr ← prev
    jgeu r7, r0, .LBB2_4  // If prev >= start, continue with the next iteration of the inner loop.
    jump .LBB2_7  // Continue with the next iteration of the outer loop.
\end{verbatim}

\begin{verbatim}
insertion_sort_sentinel:
    jleu r0, r1, .LBB3_1  // Continue if array of positive length …
.LBB3_5:
    jump r23  // … else leave the function.
.LBB3_4:
    add r0, r0, 4  // i++
    sw r3, 0, r2  // *curr ← to_sort
    jgtu r0, r1, .LBB3_5  // If i > end, leave the function.
.LBB3_1:
    lw r2, r0, 0  // to_sort ← *i
    lw r4, r0, -4  // *prev
    move r3, r0  // curr ← i
    jleu r4, r2, .LBB3_4  // If *prev > to_sort, terminate inner loop.
    move r3, r0  // ???
.LBB3_3:
    sw r3, 0, r4  // *curr ← *prev
    lw r4, r3, -8  // *(prev - 1)
    add r3, r3, -4  // curr ← prev
    jgtu r4, r2, .LBB3_3  // If *(prev - 1) > to_sort, continue with the next iteration of the inner loop.
    jump .LBB3_4  // Leave inner loop.
\end{verbatim}