\subsection{\texorpdfstring{\ShS{}}{ShellSort}}
\label{subapx:single:shell}

\pgfplotsinvokeforeach{reverse,uniform}{
	\pgfplotstablereadnamed{data/shell/two-tier/uint64/#1.txt}{tableShellTwo_64#1}
	\pgfplotstablereadnamed{data/shell/h1=7/uint64/#1.txt}{tableShell7_64#1}
	\pgfplotstablereadnamed{data/shell/h1=8/uint64/#1.txt}{tableShell8_64#1}
	\pgfplotstablereadnamed{data/shell/h1=9/uint64/#1.txt}{tableShell9_64#1}
	\pgfplotstablereadnamed{data/shell/h1=10/uint64/#1.txt}{tableShell10_64#1}
	\pgfplotstablereadnamed{data/shell/h1=11/uint64/#1.txt}{tableShell11_64#1}
	\pgfplotstablereadnamed{data/shell/h1=12/uint64/#1.txt}{tableShell12_64#1}
	\pgfplotstablereadnamed{data/shell/h1=13/uint64/#1.txt}{tableShell13_64#1}
	\pgfplotstablereadnamed{data/shell/h1=14/uint64/#1.txt}{tableShell14_64#1}
	\pgfplotstablereadnamed{data/shell/h1=15/uint64/#1.txt}{tableShell15_64#1}
	\pgfplotstablereadnamed{data/shell/h1=16/uint64/#1.txt}{tableShell16_64#1}
	\pgfplotstablereadnamed{data/shell/h1=17/uint64/#1.txt}{tableShell17_64#1}
}

\begin{figure}
	\tikzsetnextfilename{shell_two-tier_uint32}
	\begin{tikzpicture}[plot]
		\begin{groupplot}[
			adaptive group=3 by 2,
			groupplot xlabel={Input Length \(n\)},
			groupplot ylabel={Speed-up},
			xtick distance=3,
			minor xtick=data,
			legend columns=-1,
		]
			\nextgroupplot[title={Sorted}, legend to name=leg:shell:two-tier_uint32]
			\legend{\(2\), \(...\), \(9\)}
			\pgfplotsset{cycle list shift=1}
			\pgfplotsinvokeforeach{2,...,9}{
				\plotspeedup[x filter/.expression={x > #1 ? x : nan}]{#1}{1}{\tableSmallSortsXxxiiSorted}
			}
			%
			\nextgroupplot[title={Reverse Sorted}]
			\pgfplotsset{cycle list shift=1}
			\pgfplotsinvokeforeach{2,...,9}{
				\plotspeedup[x filter/.expression={x > #1 ? x : nan}]{#1}{1}{\tableSmallSortsXxxiiReverse}
			}
			%
			\nextgroupplot[title={Almost Sorted}]
			\pgfplotsset{cycle list shift=1}
			\pgfplotsinvokeforeach{2,...,9}{
				\plotspeedup[x filter/.expression={x > #1 ? x : nan}]{#1}{1}{\tableSmallSortsXxxiiAlmost}
			}
			%
			\nextgroupplot[title={Uniform}]
			\pgfplotsset{cycle list shift=1}
			\pgfplotsinvokeforeach{2,...,9}{
				\plotspeedup[x filter/.expression={x > #1 ? x : nan}]{#1}{1}{\tableSmallSortsXxxiiUniform}
			}
			%
			\nextgroupplot[title={Zipf's}]
			\pgfplotsset{cycle list shift=1}
			\pgfplotsinvokeforeach{2,...,9}{
				\plotspeedup[x filter/.expression={x > #1 ? x : nan}]{#1}{1}{\tableSmallSortsXxxiiZipf}
			}
			%
			\nextgroupplot[title={Normal}]
			\pgfplotsset{cycle list shift=1}
			\pgfplotsinvokeforeach{2,...,9}{
				\plotspeedup[x filter/.expression={x > #1 ? x : nan}]{#1}{1}{\tableSmallSortsXxxiiNormal}
			}
		\end{groupplot}
	\end{tikzpicture}

	\hfil\pgfplotslegendfromname{leg:shell:two-tier_uint32}\hfil
	\caption{
		A continuation of \cref{fig:shell:two-tier} with more input distributions.
		Instead of total runtimes, the speed-ups with respect to \IS{} are given for better clarity.
		The data type is 32-bit unsigned integers.
	}
	\label{fig:shell:two-tier_uint32}
\end{figure}

\begin{figure}
	\tikzsetnextfilename{shell_two-tier_uint64}
	\begin{tikzpicture}[plot]
		\begin{groupplot}[
			adaptive group=3 by 2,
			groupplot xlabel={Input Length \(n\)},
			groupplot ylabel={Speed-up},
			xtick distance=3,
			minor xtick=data,
			legend columns=-1,
		]
			\nextgroupplot[title={Sorted}, legend to name=leg:shell:two-tier_uint64]
			\legend{\(2\), \(...\), \(9\)}
			\pgfplotsset{cycle list shift=1}
			\pgfplotsinvokeforeach{2,...,9}{
				\plotspeedup[x filter/.expression={x > #1 ? x : nan}]{#1}{1}{\tableSmallSortsLxivSorted}
			}
			%
			\nextgroupplot[title={Reverse Sorted}]
			\pgfplotsset{cycle list shift=1}
			\pgfplotsinvokeforeach{2,...,9}{
				\plotspeedup[x filter/.expression={x > #1 ? x : nan}]{#1}{1}{\tableSmallSortsLxivReverse}
			}
			%
			\nextgroupplot[title={Almost Sorted}]
			\pgfplotsset{cycle list shift=1}
			\pgfplotsinvokeforeach{2,...,9}{
				\plotspeedup[x filter/.expression={x > #1 ? x : nan}]{#1}{1}{\tableSmallSortsLxivAlmost}
			}
			%
			\nextgroupplot[title={Uniform}]
			\pgfplotsset{cycle list shift=1}
			\pgfplotsinvokeforeach{2,...,9}{
				\plotspeedup[x filter/.expression={x > #1 ? x : nan}]{#1}{1}{\tableSmallSortsLxivUniform}
			}
			%
			\nextgroupplot[title={Zipf's}]
			\pgfplotsset{cycle list shift=1}
			\pgfplotsinvokeforeach{2,...,9}{
				\plotspeedup[x filter/.expression={x > #1 ? x : nan}]{#1}{1}{\tableSmallSortsLxivZipf}
			}
			%
			\nextgroupplot[title={Normal}]
			\pgfplotsset{cycle list shift=1}
			\pgfplotsinvokeforeach{2,...,9}{
				\plotspeedup[x filter/.expression={x > #1 ? x : nan}]{#1}{1}{\tableSmallSortsLxivNormal}
			}
		\end{groupplot}
	\end{tikzpicture}

	\hfil\pgfplotslegendfromname{leg:shell:two-tier_uint64}\hfil
	\caption{
		A continuation of \cref{fig:shell:two-tier} with more input distributions.
		Instead of total runtimes, the speed-ups with respect to \IS{} are given for better clarity.
		The data type is 64-bit unsigned integers.
	}
	\label{fig:shell:two-tier_uint64}
\end{figure}

\begin{figure}
	\tikzsetnextfilename{shell_two-tier_uint32reverse}
	\begin{tikzpicture}[plot]
		\newcommand{\setn}[1]{\textit{n} = #1}
		\newcommand{\type}{32reverse}
		\begin{groupplot}[shell scatter plot]
			\pgfplotsinvokeforeach{16,32,48,64,96,128}{
				\nextgroupplot[title={Input Length \setn{#1}}]
				\pgfplotsforeachungrouped\h in {3,...,9}{
					\shellscatter{#1}{\h}{\type}
				}
			}
			\pgfplotsset{legend to name=leg:shell:three-tier_uint32reverse}
			\legend{\(3\), \(4\), \(5\), \(6\), \(7\), \(8\), \(9\)}
		\end{groupplot}
	\end{tikzpicture}

	\hfil\pgfplotslegendfromname{leg:shell:three-tier_uint32reverse}\hfil
	\caption[]{
		Runtimes of \ShS*{} with two passes (/) and three passes (7--17).
		The coloured symbols encode the step size \(\stepsizes_1\) for two-tier \ShS*{} and the step size~\(\stepsizes_2\) for three-tier \ShS*{}.
		For the latter, the step size \(\stepsizes_1\) is noted on the y-axes.

		A variation of \cref{fig:shell:three-tier}.
		The data type is 32-bit unsigned integers.
		The input distribution is the reverse sorted one.
	}
	\label{fig:shell:three-tier_uint32reverse}
\end{figure}

\begin{figure}
	\tikzsetnextfilename{shell_two-tier_uint32uniform}
	\begin{tikzpicture}[plot]
		\newcommand{\setn}[1]{\textit{n} = #1}
		\newcommand{\type}{32uniform}
		\begin{groupplot}[shell scatter plot]
			\pgfplotsinvokeforeach{16,32,48,64,96,128}{
				\nextgroupplot[title={Input Length \setn{#1}}]
				\pgfplotsforeachungrouped\h in {3,...,9}{
					\shellscatter{#1}{\h}{\type}
				}
			}
			\pgfplotsset{legend to name=leg:shell:three-tier_uint32uniform}
			\legend{\(3\), \(4\), \(5\), \(6\), \(7\), \(8\), \(9\)}
		\end{groupplot}
	\end{tikzpicture}

	\hfil\pgfplotslegendfromname{leg:shell:three-tier_uint32uniform}\hfil
	\caption[]{
		Runtimes of \ShS*{} with two passes (/) and three passes (7--17).
		The coloured symbols encode the step size \(\stepsizes_1\) for two-tier \ShS*{} and the step size~\(\stepsizes_2\) for three-tier \ShS*{}.
		For the latter, the step size \(\stepsizes_1\) is noted on the y-axes.

		A repetition of \cref{fig:shell:three-tier}.
		The data type is 32-bit unsigned integers.
		The input distribution is the uniform one.
	}
	\label{fig:shell:three-tier_uint32uniform}
\end{figure}

\begin{figure}
	\tikzsetnextfilename{shell_three-tier_uint64reverse}
	\begin{tikzpicture}[plot]
		\newcommand{\setn}[1]{\textit{n} = #1}
		\newcommand{\type}{64reverse}
		\begin{groupplot}[shell scatter plot]
			\pgfplotsinvokeforeach{16,32,48,64,96,128}{
				\nextgroupplot[title={Input Length \setn{#1}}]
				\pgfplotsforeachungrouped\h in {3,...,9}{
					\shellscatter{#1}{\h}{\type}
				}
			}
			\pgfplotsset{legend to name=leg:shell:three-tier_uint64reverse}
			\legend{\(3\), \(4\), \(5\), \(6\), \(7\), \(8\), \(9\)}
		\end{groupplot}
	\end{tikzpicture}

	\hfil\pgfplotslegendfromname{leg:shell:three-tier_uint64reverse}\hfil
	\caption[]{
		Runtimes of \ShS*{} with two passes (/) and three passes (7--17).
		The coloured symbols encode the step size \(\stepsizes_1\) for two-tier \ShS*{} and the step size~\(\stepsizes_2\) for three-tier \ShS*{}.
		For the latter, the step size \(\stepsizes_1\) is noted on the y-axes.

		A variation of \cref{fig:shell:three-tier}.
		The data type is 64-bit unsigned integers.
		The input distribution is the reverse sorted one.
	}
	\label{fig:shell:three-tier_uint64reverse}
\end{figure}

\begin{figure}
	\tikzsetnextfilename{shell_three-tier_uint64uniform}
	\begin{tikzpicture}[plot]
		\newcommand{\setn}[1]{\textit{n} = #1}
		\newcommand{\type}{64uniform}
		\begin{groupplot}[shell scatter plot]
			\pgfplotsinvokeforeach{16,32,48,64,96,128}{
				\nextgroupplot[title={Input Length \setn{#1}}]
				\pgfplotsforeachungrouped\h in {3,...,9}{
					\shellscatter{#1}{\h}{\type}
				}
			}
			\pgfplotsset{legend to name=leg:shell:three-tier_uint64uniform}
			\legend{\(3\), \(4\), \(5\), \(6\), \(7\), \(8\), \(9\)}
		\end{groupplot}
	\end{tikzpicture}

	\hfil\pgfplotslegendfromname{leg:shell:three-tier_uint64uniform}\hfil
	\caption[]{
		Runtimes of \ShS*{} with two passes (/) and three passes (7--17).
		The coloured symbols encode the step size \(\stepsizes_1\) for two-tier \ShS*{} and the step size~\(\stepsizes_2\) for three-tier \ShS*{}.
		For the latter, the step size \(\stepsizes_1\) is noted on the y-axes.

		A variation of \cref{fig:shell:three-tier}.
		The data type is 64-bit unsigned integers.
		The input distribution is the uniform one.
	}
	\label{fig:shell:three-tier_uint64uniform}
\end{figure}
