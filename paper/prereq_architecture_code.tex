\subsection{Kernel Programming}
\label{sec:prereq:arch:code}

Executing tasks on \acp{DPU} requires both a program executed on the host \ac{CPU} and a kernel executed on the \acp{DPU}.
\Acp{DPU} are handled in groups, called \emph{ranks}, of up to 64 \acp{DPU} from the same side of a \ac{DIMM}.
The ranks, in turn, are aggregated in a \emph{\ac{DPU} set}.
A typical course of action is the following:
\begin{enumerate}
	\item
	Start the host program.

	\item
	Write the input to the \ac{MRAM} and/or \ac{WRAM} of all involved \acp{DPU}.

	\item
	Boot the \acp{DPU} and execute the kernel synchronously or asynchronously.

	\item
	Read the output from the \ac{MRAM} and/or \ac{WRAM}.

	\item
	Go back to the third step if needed.
\end{enumerate}
When kernels are executed synchronously, the host cannot access the memory of a \ac{DPU} until all \acp{DPU} in the whole set have finished.
But even with an asynchronous execution, the host cannot access the memory until all \acp{DPU} in the respective rank have finished.
Note that data is generally not deleted when a kernel finishes, so subsequent executions can hark back to previous results.
Also, any communication between the host and the \acp{DPU} must be initiated by the host.
The host program can be written in \langC{}, \langCpp{}, \langJava{}, or \langPython{}.
Apart from a few additional functionalities provided by the \upmem{} \ac{API} for communicating with the \acp{DPU}, the host program is a regular executable.

The software development kit includes a simulator which allows to run kernels on machines without \upmem{} \acp{DIMM}.
The kernel has to be written in either \langC{} or assembler, but we will focus on the former.
Its entry point is the \lstinline|main| function, thence one can proceed as in any \langC{} program.
All tasklets execute the same kernel but their control flow can be changed by simply including conditionals on the tasklet identifiers.
Synchronisation between tasklets can be achieved, amongst others, via barriers, mutual exclusion, and semaphores.
Communication between tasklets is achievable by defining global variables in the \ac{WRAM}.
The \langC{} standard library is only partially available as some compute-intensive functionalities have been not been implemented, for example the entire \lstinline|math| library.

The biggest changes to a regular \langC{} program are in relation to the memory.
Any variable resides in the \ac{WRAM} by default, but creating an \ac{MRAM} variable is as easy as adding the qualifier \lstinline|__mram| to the variable declaration.
By default, too, any pointer is assumed to point to data in the \ac{WRAM}, which can be changed by adding the qualifier \lstinline|__mram_ptr|.
The compiler correctly identifies confusion between pointers of different address spaces.

Local \ac{WRAM} variables are created on the stack of the respective tasklet, and the stack sizes can be set for each tasklet individually at compile time.
Nonetheless, it is possible for tasklets to dynamically allocate more space on the \ac{WRAM} via an allocator similar to the standard \langC{} function \lstinline|malloc|.
Although this is called \emph{heap allocation}, the name is misleading.
The compiler organises the \ac{WRAM} such that all tasklet stacks and anything else statically allocated on the \ac{WRAM} is in the front, so that the free memory comprises a contiguous block in the back of the \ac{WRAM}.
Then, the so-called \emph{heap pointer} is set to the beginning of the free block.
When memory is allocated on the heap, the heap pointer is sufficiently incremented to mark the space as reserved.
Afterwards, the original position of the heap pointer is returned to the allocating tasklet.
In other words:
the heap memory is simply a stack memory shared by all tasklets.
Indeed, a \ac{DPU} lacks an equivalent to the standard \langC{} function \lstinline|free| to deallocate heap memory.
The only possibility is to reset the entire heap by setting the heap pointer back to its initial position.
There is also no model of ownership, so tasklets can write to any memory address, including the stack and heap memory of other tasklets.
Heed must be paid when structuring the scarce \ac{WRAM}, which is subject again in \cref{sec:mram:triple}.

As hinted before, transferring data between the \ac{MRAM} and the \ac{WRAM} is in the responsibility of the programmer.
When only single elements are to be accessed, \ac{MRAM} data can be treated like normal variables.
For example, \lstinline|var = arr[i]| is valid code no matter whether the array \lstinline|arr|, the variable \lstinline|var|, or the index \lstinline|i| have been declared to reside in the WRAM or the MRAM.
However, this still constitutes one or more \acp{DMA} on each use, and each \ac{DMA} comes at a cost.
According to measurements~\cite{mutlu2022Benchmarking}, reading from the \ac{MRAM} has an overhead of \qty{77}{\cycles}, whilst writing to the \ac{MRAM} has an overhead of \qty{61}{\cycles}.
The transfer of each byte costs a further \qty{0.5}{\cycles}.
This means that \acp{DMA} for about \qty{128}{\byte} of data or less are dominated by the overhead.
Therefore, it recommended to move large blocks of \ac{MRAM} data into the \ac{WRAM}, perform calculations there, and move the modified data block back to the \ac{MRAM}.
This way, the overhead is mitigated.
Please note that, like for \ac{WRAM} data, the time to access \ac{MRAM} data is independent of the exact location \Dash only the memory type matters.

To perform such blockwise moves, one calls the \langC{} function \lstinline|mram_read| and \lstinline|mram_write|, which take a source address, a target address, and the number of bytes to use transfer.
There are, however, several limitations.
\begin{itemize}
	\item
	The \ac{WRAM} address must be aligned on 8 bytes.
	This can be ensured automatically by adding appropriate qualifiers to stack variables or by using heap memory which gets properly aligned automatically.

	\item
	The \ac{MRAM} address must be aligned on 8 bytes.
	No special functionality exists to this end;
	it is up to the programmer to organise the \ac{MRAM} with this limitation in mind and to resort to \acp{DMA} to single elements if such an alignment is not given.

	\item
	The number of transferred bytes must be at least 8, at most 2048, and a multiple of 8.
\end{itemize}
Failing to fulfil these constraints can result in missing or corrupt data.
The \ac{DMA} engine works sequentially, meaning data for only tasklet can be transferred at a time.
If multiple tasklets call \lstinline|mram_read| or \lstinline|mram_write|, some of them will be suspended for longer as they wait for the other \acp{DMA} to finish.
If \acp{DMA} are very frequent, having many active tasklets is especially important to keep the pipeline full.

For a performant kernel execution, it is generally recommended to restrict oneself to simple 32-bit logic as much as possible.
Some 64-bit functionalities are executed in eleven cycles, like loads and stores, but most take twice or even thrice as long.
Multiplication, division, and floating point arithmetic are emulated in software, so they should be avoided if necessary.
In their stead, addition, subtraction, and bitwise logic should be used.
Also, due to the uniform-cost model, a decrease in the count of instructions translates into a performance gain.
Unfortunately, a common issue is a nosediving quality of the compilation, perhaps resulting from a wrong configuration of the \abb{LLVM}-based compiler.
Investigating the compilation and trying different approaches \Dash including even mundane alternatives like reordering independent if statements \Dash is paramount when aiming for top performance and, thus, a recurring theme in this thesis.
General experience shows that explicitly saving the result of a computation if the value is reused later on prevents the compiler from issuing a recalculation which would hurt elsewise due to the compute-boundedness of the architecture.
Also, it seems that pointer logic tends to be compiled better than index logic.
Lastly, inlining leads to a performance gain oftentimes as the overhead for function calls is quite heavy.
Even though the call itself is a mere jump taking elven cycles, several registers must be saved and reloaded on entering and exiting a function;
for an empty function with two arguments, we determined a call overhead of \qty{144}{\cycles}, which can easily rise with heavier register usage.
This may also explain why turning arguments into global variables nets a performance gain in some cases as well.
