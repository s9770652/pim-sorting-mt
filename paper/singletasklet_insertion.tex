\subsection{\texorpdfstring{\IS{}}{InsertionSort}}
\label{subsec:single:insertion}

\pgfplotstableread{data/small sorts/uint32/uniform.txt}{\tableSmallSortsXxxiiUniform}

This stable sorting algorithm works by moving the \(i\)th element leftwards as long as its left neighbour is bigger, assuming that the elements at the indices \(0\) to \(i - 1\) are already sorted.
Its asymptotic runtime is bad, reaching \(\bigoh*{n^2}\) not only in the worst case but also in the average case, where each of the \(\binom{n}{2}\) pairs of input elements is in wrong order and will be swapped at some point in the execution with probability 50\%.
Nonetheless, \IS{} does have some saving graces:
\begin{enumerate*}
	\item
	If the input array is mostly or even fully sorted, the runtime drops down to \(\bigoh{n}\).

	\item
	It works in-place, needing only \(\bigoh{1}\) additional space.

	\item
	Its program code is short, lending itself to inlining.

	\item
	The overhead is small.
\end{enumerate*}
Especially the last two points make \IS{} a good fall-back algorithm for asymptotically better sorting algorithms to use on small subarrays.

When moving an element to the left, two checks are needed:
Does the left neighbour exist and is it smaller than the element to move?
The first check can be omitted through the use of \emph{sentinel values}:
If the element at index \(-1\) is at least as small as any value in the input array, the leftwards motion stops there at the latest.
Since a DPU has no branch predictor, the slowdown from performing twice as many checks as needed is quite high and lies at about 30\% for short inputs (\cref{fig:insertion:against_others}).

Setting an \emph{explicit} sentinel value can be omitted by using \emph{implicit} sentinel values.
At the start of each round, one can check if the element at index \(0\) is at least as small as the element at index~\(i\).
If yes, the former is a sufficient sentinel value, and \IS{} can proceed as normal.
If not, the latter must be the minimum of the first \(i + 1\) elements and, therefore, can be moved immediately to the front.

The runtimes of the explicit and implicit \IS*{} can be compared in the \cref{fig:insertion:against_others,fig:insertion:against_others_uint32,fig:insertion:against_others_uint64}.
For most input distributions, the implicit \IS{} is logically a bit slower, effectively adding one check instruction to each round.
An outlier, however, are the reverse sorted inputs.
For 32-bit numbers (\cref{fig:insertion:against_others_uint32}), the implicit version is up to 45\% slower than the explicit one.
This comes at a surprise since both versions effectively execute the same loop body while shifting everything one index backwards, with only the loop condition being different.
Due to the uni-cost model, a value check on whether the preceding element is smaller (explicit) and an address check on whether the preceding position is the start of the array (implicit) should take the same amount of time.
Yet, even the \IS{} not relying on sentinel values surpasses the implicit \IS{}, although doing both value checks and address checks!
For 64-bit numbers (\cref{fig:insertion:against_others_uint64}), the implicit \IS{} would be expected to perform better than the explicit one, considering that a value check now takes two instructions and an address check still only one.
Nonetheless, the two \IS*{} tie.

\begin{figure}
	\tikzsetnextfilename{insertion_against_others}
	\begin{tikzpicture}[plot]
		\begin{groupplot}[
			horizontal sep for labels,
			adaptive group=1 by 2,
			groupplot xlabel={Input Length \(n\)},
			xtick distance=3,
			minor xtick=data,
			legend columns=3,
		]
			\nextgroupplot[ylabel=Cycles / \(n^2\), ymin=0, ymax=60, legend to name=leg:insertion:against_others]
			\legend{\IS{} (no sentinel), \IS{} (explicit), \IS{} (implicit), \BS{} (not adaptive), \BS{} (adaptive), \SelS{}}
			\plotpernn{1NoSentinel}{\tableSmallSortsXxxiiUniform}
			\plotpernn{1}{\tableSmallSortsXxxiiUniform}
			\plotpernn{1Implicit}{\tableSmallSortsXxxiiUniform}
			\plotpernn{BubbleNonAdapt}{\tableSmallSortsXxxiiUniform}
			\plotpernn{BubbleAdapt}{\tableSmallSortsXxxiiUniform}
			\plotpernn{Selection}{\tableSmallSortsXxxiiUniform}
			%
			\nextgroupplot[ylabel=Speed-up, ymin=0.2, ymax=1]
			\plotspeedup{1NoSentinel}{1}{\tableSmallSortsXxxiiUniform}
			\pgfplotsset{cycle list shift=1}
			\plotspeedup{1Implicit}{1}{\tableSmallSortsXxxiiUniform}
			\plotspeedup{BubbleNonAdapt}{1}{\tableSmallSortsXxxiiUniform}
			\plotspeedup{BubbleAdapt}{1}{\tableSmallSortsXxxiiUniform}
			\plotspeedup{Selection}{1}{\tableSmallSortsXxxiiUniform}
		\end{groupplot}
	\end{tikzpicture}

	\hfil\pgfplotslegendfromname{leg:insertion:against_others}\hfil
	\caption{
		Comparison of sorting algorithms with a runtime in \(\bigoh{n^2}\).
		The adaptive \BS{} terminates prematurely if no changes were made to the input array during an iteration.
		The speed-ups are with respect to the \IS{} relying on explicit sentinel values.
	}
	\label{fig:insertion:against_others}
\end{figure}

\subsubsection*{Investigating the Compilation}
\label{subsubsec:tasklet:insertion:compiler}

At times, the quality of the compilation nosedives, and \IS{} serves as great example.
But before we focus on the poor performance of the implicit \IS{}, let us first look at something far more basic.

When sorting an element, all previous elements must already be in sorted order;
an exemplary implementation of this is shown in \cref{fig:insertion:impl:pred_first}.
Obviously, the first element alone is already sorted, so \IS{} will not perform any changes and proceed to the second element.
In order to get rid of some loop overhead, it would make sense to let \IS{} start at the second element, as in \cref{fig:insertion:impl:pred_sec}.
Surprisingly, starting at the first element yields a runtime of 4166 cycles at 16 elements, whereas starting at the second yields a runtime of 4266 cycles.
The same happens if, in \cref{fig:insertion:impl:pred_sec}, one keeps \lstinline|*i = start| and instead uses \lstinline|curr = ++i|.

Looking at the compilation reveals the reason:
In the faster case, the pointer \lstinline|pred| is optimised away and, in its stead, the pointer \lstinline|curr| and an offset of \lstinline|-4| is used.
In the slower case, the pointer \lstinline|pred| is \emph{sometimes} used with an offset to get the true value of \lstinline|curr|.
Two registers are used to store the addresses of \lstinline|curr| and \lstinline|pred| at different points in time, resulting in one additional instruction at the start of each round, nullifying any gain from starting with the second element.
This is a consequence of reusing the register of the \lstinline|start| pointer for \lstinline|pred| instead of for \lstinline|i|, whose incremented value is put into the additionally used register.

\begin{figure}
	\def\iscodewidth{0.47\linewidth}
	\begin{subfigure}{\iscodewidth}
		\begin{lstlisting}
void InsertionSort(int *start, int *end) {
	int *curr, *i = start;
	while ((curr = i++) <= end) {
		int to_sort = *curr;
		int *pred = curr - 1;
		while (*pred > to_sort) {
			*curr = *pred;
			curr = pred--;
		}
		*curr = to_sort;
	}
}
		\end{lstlisting}
		\caption{
			Start at the first element and with predecessor pointer.
		}
		\label{fig:insertion:impl:pred_first}
	\end{subfigure}
	\hfill
	\begin{subfigure}{\iscodewidth}
		\begin{lstlisting}
void InsertionSort(int *start, int *end) {
	int *curr, *i = start + 1;
	while ((curr = i++) <= end) {
		int to_sort = *curr;
		int *pred = curr - 1;
		while (*pred > to_sort) {
			*curr = *pred;
			curr = pred--;
		}
		*curr = to_sort;
	}
}
		\end{lstlisting}
		\caption{
			Start at the second element and with predecessor pointer.
		}
		\label{fig:insertion:impl:pred_sec}
	\end{subfigure}

	\begin{subfigure}{\iscodewidth}
		\begin{lstlisting}
void InsertionSort(int *start, int *end) {
	int *curr, *i = start;
	while ((curr = i++) <= end) {
		int to_sort = *curr;
		while (*(curr - 1) > to_sort) {
			*curr = *(curr - 1);
			curr--;
		}
		*curr = to_sort;
	}
}
		\end{lstlisting}
		\caption{
			Start at the first element and without predecessor pointer.
		}
		\label{fig:insertion:impl:offset_first}
	\end{subfigure}
	\hfill
	\begin{subfigure}{\iscodewidth}
		\begin{lstlisting}
void InsertionSort(int *start, int *end) {
	int *curr, *i = start + 1;
	while ((curr = i++) <= end) {
		int to_sort = *curr;
		while (*(curr - 1) > to_sort) {
			*curr = *(curr - 1);
			curr--;
		}
		*curr = to_sort;
	}
}
		\end{lstlisting}
		\caption{
			Start at the second element and without predecessor pointer.
		}
		\label{fig:insertion:impl:offset_sec}
	\end{subfigure}
	\caption{
		Four different implementations of \IS{}.
		\Cref{fig:insertion:impl:pred_first,fig:insertion:impl:offset_first} are compiled the same.
		\Cref{fig:insertion:impl:pred_sec,fig:insertion:impl:offset_sec} are compiled differently.
	}
\end{figure}

Multiple workarounds exist to sidestep this problem.
One workaround is to take the faster code of \cref{fig:insertion:impl:pred_first} and add inline assembler to the start which increments the register holding the \lstinline|start| pointer.
While this does work for the explicit \IS{}, the other two versions of \IS{} need to know the original starting address later on and, thus, actually set the address of \lstinline|i| rather late;
adding inline assembler proves fare more difficult as a consequence.
And as \IS{} is to be used as fallback algorithms by other functions, which might also need to keep the original value of \lstinline|start|, inline assembler is a bad choice even for the explicit \IS{}.

Another workaround is the usage of a wrapper function calling \IS{} with the arguments \lstinline|start + 1| and \lstinline|end|.
This works quite well:
First, the register holding \lstinline|start| is incremented and, then, follows the inlined code from the actual \IS{}.
Doing so makes the runtime drop from 4166 cycles down to 4101 cycles.

Recall how in the faster version (\cref{fig:insertion:impl:pred_first}), the pointer \lstinline|pred| is always deduced from the pointer \lstinline|curr|.
This is manually enforced in \cref{fig:insertion:impl:offset_first,fig:insertion:impl:offset_sec}, where \lstinline|pred| is replaced with \lstinline|curr - 1|.
As a consequence, the code of \cref{fig:insertion:impl:offset_first} compiles the very same as the one of \cref{fig:insertion:impl:pred_first}, while \cref{fig:insertion:impl:offset_sec} yields the same compilation as the wrapper functions.
This workaround is clearly the best of the three and, hence, the one used in the whole code base.

Alas, the struggle with the compiler endeth not herewith.
A deeper look into the compilation reveals the following three instructions:
\lstinline|move r3, r0;| \lstinline|jleu r4, r2,| \lstinline|.LBB1_4; move r3, r0|.
Without delving further into their meaning \Dash the second \lstinline|move r3, r0| is unneeded as it is impossible to jump directly to it and, apparently, it does also not set any flags.
Copying the whole assembler code and injecting it as inline assembler but with this second \lstinline|move r3, r0| removed pushes the runtime down to just 3961 cycles while still sorting correctly;
the gain is even greater for input distributions other than the uniform one.
New issues, especially for inlining, are introduced, though, and we deem a proper assembly implementation as out of scope for this thesis.

At this point, it should have become apparent that the poor performance of the implicit \IS{} is also due to bad compilation.
We refrain from going into details because we failed to find an easy fix and because the implicit \IS{} is not used in the rest of the code base.
Thence, a sole \enquote{\IS{}} refers to the version relying on explicit sentinel values henceforth.
Still, the idea of implicit sentinel values will be of use for \ShS{} in \cref{subsec:tasklet:shell}.


\begin{note}
	Other known simple sorting algorithm with a runtime complexity similar to \IS{} are \SelS{} and \BS{}.
	The asymptoticity, however, hides much higher constant factors such that \IS{} should always be preferred (cf.\ \cref{fig:insertion:against_others,fig:insertion:against_others_uint32,fig:insertion:against_others_uint64}).
\end{note}
