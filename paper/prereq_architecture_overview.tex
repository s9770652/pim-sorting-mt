\subsection{Overview}
\label{sec:prereq:architecture:overview}

The \ac{PIM} capabilities are realised on, at its base, sticks of regular \ac{RAM} or, to be more precise, on \acp{DIMM} of \aclu{DDR}\label{acro:DDR} \aclu{SDRAM}\label{acro:SDRAM} with a transfer rate of \qty{2400}{\mega\transfer\per\second} (\acs{DDR}-2400 \acs{SDRAM}).
Therefore, \ac{PIM} \acp{DIMM} can act as replacement for \acp{DIMM} already present in existing systems without repercussion for tasks which do not rely on in-memory processing.
With 8 on either side, a \acp{PIM} \acp{DIMM} contains 16 modified \acsu{DRAM} packages called \emph{PIM chips}.
Each \ac{PIM} chip, in turn, contains 8 \acfip{DPU}, so there are 128 \acp{DPU} per \ac{DIMM}.
Each \ac{DPU} is closely situated to one of the memory banks of size \qty{64}{\mebi\byte} which constitute the memory of a regular \ac{DDR} \Ac{DIMM}.
Due to the spatial proximity to its memory bank, a \ac{DPU} is capable of rapidly accessing data stored on a \ac{DIMM}.

A \ac{DPU} possesses 24 hardware threads, whose software abstraction are called \emph{tasklets}.
Taklets work independently from each other, meaning programs can use different control flows to process different data.
The orchestration of \acp{DPU} and their tasklets pose a major challenge during programming.
To put things into perspective:
UPMEM sells systems with up to 28 \ac{PIM} \acp{DIMM}, setting the total count of \acp{DPU} to \num{3584} and of tasklets to \num{86016}.
Hence, for a task to run well on a \ac{PIM} system, it not only needs to frequently access the \ac{RAM}, it also needs to be highly parallelisable.
If such a highly parallelisable task is indeed on hand, speedups well in the double digits for memory-bound tasks, compared to an execution on a \ac{CPU} or \ac{GPU}, are possible (compare \citeauthor{mutlu2022Benchmarking}~\cite{mutlu2022Benchmarking}).
Next to a faster execution, a gain in power efficiency is also to be expected, since data transfers between the \ac{RAM} and a host \ac{CPU} drive the energy consumption in regular systems significantly;
UPMEM claims a tenfold increase of the power efficiency.

The retention of the general \ac{DDR} architecture comes at a price.
A \ac{DPU} is implemented using only three layers of silicon, resulting in three times slower transistors compared to other transistors of the same process node.
Also, their density is considerably reduced.
In consequence, \acp{DPU} are not suitable for computing-intensive tasks (compare also \citeauthor{mutlu2022Benchmarking}~\cite{mutlu2022Benchmarking}).
