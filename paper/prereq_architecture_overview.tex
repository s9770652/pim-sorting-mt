\subsubsection{Overview}
\label{sec:prereq:architecture:overview}

The in-memory processing (PIM) capabilities are realised on, at its base, sticks of regular \emph{random access memory} (RAM) or, to be more precise, on \emph{Dual In-Line Memory Modules} (DIMMs) of \emph{Double Data Rate 4 Synchronous Dynamic Random-Access Memory} with a transfer rate of \qty{2400}{\mega\transfer\per\second} (DDR4-2400 SDRAM).
Therefore, PIM DIMMs can act as replacement for DIMMs already present in existing systems without repercussion for tasks which do not rely on in-memory processing.
With 8 on either side, a PIM DIMM contains 16 modified DRAM packages called \emph{PIM chips}.
Each PIM chip, in turn, contains 8 \emph{DRAM processing units} (DPUs), so there are 128 DPUs per DIMM.
Each DPU is closely situated to one of the memory banks of size \qty{64}{\mebi\byte} which constitute the memory of a regular DDR4 DIMM.
Due to the spatial proximity to its memory bank, a DPU is capable of rapidly accessing data stored on a DIMM.

A DPU possesses 24 hardware threads, whose software abstraction are called \emph{tasklets}.
Taklets work independently from each other, meaning programs can use different control flows to process different data.
The orchestration of DPUs and their tasklets pose a major challenge during programming.
To put things into perspective:
UPMEM sells systems with up to 28 PIM DIMMS, setting the total count of DPUs to \num{3584} and of tasklets to \num{86016}.
Hence, for a task to run well on a PIM system, it not only needs to frequently access the RAM, it also needs to be highly parallelisable.
If such a highly parallelisable task is indeed on hand, speedups well in the double digits for memory-bound tasks, compared to an execution on a CPU or GPU, are possible (compare \citeauthor{mutlu2022Benchmarking}~\cite{mutlu2022Benchmarking}).
Next to a faster execution, a gain in power efficiency is also to be expected, since data transfers between the RAM and a host CPU drive the energy consumption in regular systems significantly;
UPMEM claims a tenfold increase of the power efficiency.

The retention of the general DDR4 architecture comes at a price.
A DPU is implemented using only three layers of silicon, resulting in three times slower transistors compared to other transistors of the same process node.
Also, their density is considerably reduced.
In consequence, DPUs are not suitable for computing-intensive tasks (compare also \citeauthor{mutlu2022Benchmarking}~\cite{mutlu2022Benchmarking}).
