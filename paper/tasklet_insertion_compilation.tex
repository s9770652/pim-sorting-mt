\subsection*{Investigation of the Compilation}
\label{sec:tasklet:insertion:compilation}

At times, the quality of the compilation nosedives, and \IS{} serves as great example.
But before we focus on the poor performance of the implicit \IS{}, let us first look at something far more basic.

When sorting an element, all previous elements must already be in sorted order;
an exemplary implementation of this is shown in \cref{fig:insertion:impl:pred_first}.
Obviously, the first element alone is already sorted, so \IS{} will not perform any changes and proceed to the second element.
In order to get rid of some loop overhead, it would make sense to let \IS{} start at the second element, as in \cref{fig:insertion:impl:pred_sec}.
Surprisingly, starting at the first element yields a runtime of 4166 cycles at 16 elements, whereas starting at the second yields a runtime of 4266 cycles.
The same happens if, in \cref{fig:insertion:impl:pred_sec}, one keeps \lstinline|*i = start| and instead uses \lstinline|curr = ++i|.

Looking at the compilation reveals the reason:
In the faster case, the pointer \lstinline|pred| is optimised away and, in its stead, the pointer \lstinline|curr| and an offset of \lstinline|-4| is used.
In the slower case, the pointer \lstinline|pred| is used with an offset to get the true value of \lstinline|curr| in some places.
Two registers are used to store the addresses of \lstinline|curr| and \lstinline|pred| at different points in time, resulting in one additional instruction at the start of each round, nullifying any gain from starting with the second element.
This is a consequence of reusing the register of the \lstinline|start| pointer for \lstinline|pred| instead of for \lstinline|i|, whose incremented value is put into the additionally used register.

\begin{figure}
	\lstset{basicstyle=\ttfamily\small}
	\def\iscodewidth{0.47\linewidth}
	\begin{subfigure}{\iscodewidth}
		\begin{lstlisting}
void InsertionSort(int *start, int *end) {
	int *curr, *i = start;
	while ((curr = i++) <= end) {
		int to_sort = *curr;
		int *pred = curr - 1;
		while (*pred > to_sort) {
			*curr = *pred;
			curr = pred--;
		}
		*curr = to_sort;
	}
}
		\end{lstlisting}
		\caption{
			Start at the first element and with predecessor pointer.
		}
		\label{fig:insertion:impl:pred_first}
	\end{subfigure}
	\hfill
	\begin{subfigure}{\iscodewidth}
		\begin{lstlisting}
void InsertionSort(int *start, int *end) {
	int *curr, *i = start + 1;
	while ((curr = i++) <= end) {
		int to_sort = *curr;
		int *pred = curr - 1;
		while (*pred > to_sort) {
			*curr = *pred;
			curr = pred--;
		}
		*curr = to_sort;
	}
}
		\end{lstlisting}
		\caption{
			Start at the second element and with predecessor pointer.
		}
		\label{fig:insertion:impl:pred_sec}
	\end{subfigure}

	\begin{subfigure}{\iscodewidth}
		\begin{lstlisting}
void InsertionSort(int *start, int *end) {
	int *curr, *i = start;
	while ((curr = i++) <= end) {
		int to_sort = *curr;
		while (*(curr - 1) > to_sort) {
			*curr = *(curr - 1);
			curr--;
		}
		*curr = to_sort;
	}
}
		\end{lstlisting}
		\caption{
			Start at the first element and without predecessor pointer.
		}
		\label{fig:insertion:impl:offset_first}
	\end{subfigure}
	\hfill
	\begin{subfigure}{\iscodewidth}
		\begin{lstlisting}
void InsertionSort(int *start, int *end) {
	int *curr, *i = start + 1;
	while ((curr = i++) <= end) {
		int to_sort = *curr;
		while (*(curr - 1) > to_sort) {
			*curr = *(curr - 1);
			curr--;
		}
		*curr = to_sort;
	}
}
		\end{lstlisting}
		\caption{
			Start at the second element and without predecessor pointer.
		}
		\label{fig:insertion:impl:offset_sec}
	\end{subfigure}
	\caption{
		Four different implementations of \IS{}.
		\Cref{fig:insertion:impl:pred_first,fig:insertion:impl:offset_first} are compiled the same.
		\Cref{fig:insertion:impl:pred_sec,fig:insertion:impl:offset_sec} are compiled differently.
	}
\end{figure}

Multiple workarounds exist to sidestep this problem.
One workaround is to take the faster code of \cref{fig:insertion:impl:pred_first} and add inline assembler to the start which increments the register holding the \lstinline|start| pointer.
While this does work for the explicit \IS{}, the other two versions of \IS{} need to know the original starting address later on and, thus, actually set the address of \lstinline|i| rather late;
adding inline assembler proves fare more difficult as a consequence.
And as \IS{} is to be used as fallback algorithms by other functions, which might also need to keep the original value of \lstinline|start|, inline assembler is a bad choice even for the explicit \IS{}.

Another workaround is the usage of a wrapper function calling \IS{} with the arguments \lstinline|start + 1| and \lstinline|end|.
This works quite well:
First, the register holding \lstinline|start| is incremented and, then, follows the inlined code from the actual \IS{}.
Doing so makes the runtime drop from 4166 cycles down to 4101 cycles.

Recall how in the faster version (\cref{fig:insertion:impl:pred_first}), the pointer \lstinline|pred| is always deduced from the pointer \lstinline|curr|.
This is manually enforced in \cref{fig:insertion:impl:offset_first,fig:insertion:impl:offset_sec}, where \lstinline|pred| is replaced with \lstinline|curr - 1|.
As a consequence, the code of \cref{fig:insertion:impl:offset_first} compiles the very same as the one of \cref{fig:insertion:impl:pred_first}, while \cref{fig:insertion:impl:offset_sec} yields the same compilation as the wrapper functions.
This workaround is clearly the best of the three and, hence, the one used in the rest of this thesis.

Alas, the struggle with the compiler endeth not herewith.
A deeper look into the compilation reveals the following three instructions:
\begin{center}
	\begin{tabular}{ll}
		\lstinline|move r3, r0| & // copy content of register \lstinline|r0| to \lstinline|r3|
		\\ \lstinline|jleu r4, r2, .LBL| & // jump to \lstinline|.LBL| if \lstinline|r4| \(\le\) \lstinline|r2|
		\\ \lstinline|move r3, r0| &
	\end{tabular}
\end{center}
Without delving further into their significance \Dash the second \lstinline|move r3, r0| is unneeded as it is impossible to jump directly to it nor to return via \lstinline|jleu| and, apparently, \lstinline|move| does not set any flags.
Copying the whole assembler code and injecting it as inline assembler but with this second \lstinline|move r3, r0| removed pushes the runtime down to just 3961 cycles while still sorting correctly;
the gain is even greater for input distributions other than the uniform one.
New issues, especially for inlining, are introduced, though, and we deem a proper assembly implementation as out of scope for this thesis.

At this point, it should have become apparent that the poor performance of the implicit \IS{} is also due to bad compilation.
We refrain from going into details because we failed to find an easy fix and because the implicit \IS{} is not used in the rest of the thesis.
Thence, a sole \enquote{\IS{}} refers to the version relying on explicit sentinel values henceforth.
Still, the idea of implicit sentinel values will be of use in the following section on \ShS{}.

\raggedbottom
