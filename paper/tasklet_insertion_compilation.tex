\subsection{Investigation of the Compilation}
\label{sec:tasklet:insertion:compilation}

\Cref{fig:insertion:impl:pred_first} shows a naïve implementation of \IS{} which begins at the very start of the input.
Obviously, nothing happens in the first round as a lone element is already sorted, so it is algorithmically sound to let \IS{} begin at the second element.
This optimisation is accomplished in \cref{fig:insertion:impl:pred_sec}.
Surprisingly, it leads to a longer runtime!
For instance with sixteen 32-bit integers, the runtime is increased by nine instruction.
The same increase happens if one keeps \lstinline|*i = start| and uses \lstinline|curr = ++i| instead to begin at the second element.

Looking at the compilation reveals the reason:
In the naïve version, the pointer \lstinline|pred| is optimised away and, in its stead, the pointer \lstinline|curr| is passed to all load instructions together with a constant offset of \lstinline|-4|.
In the optimised version, the pointer \lstinline|pred| is used together with offsets of \lstinline|+4| and \lstinline|0| to fetch the values of \lstinline|to_sort| and \lstinline|*pred| at the beginning of each round.
Only then, the pointer \lstinline|curr| is initialised before being used in the inner loop as in the naïve version.
Effectively, this initialisation introduces one more instruction.
% This is a consequence of reusing the register of the \lstinline|start| pointer for \lstinline|pred| instead of for \lstinline|i|, whose incremented value is put into another register.

These changes fully explain the extension of the runtime by nine instructions:
The optimised version invokes one \lstinline|add| instruction at the beginning of the function to advance the starting position and loops 15 times in total, each time initialising the pointer \lstinline|curr|.
The naïve version loops 16 times, invoking seven instructions for naught in the first iteration.

\begin{figure}
	\lstset{basicstyle=\ttfamily\small}
	\def\codewidth{0.47\linewidth}
	\begin{subfigure}{\codewidth}
		\begin{lstlisting}
void InsertionSort(int *start, int *end) {
	int *curr, *i = start;
	while ((curr = i++) <= end) {
		int to_sort = *curr;
		int *pred = curr - 1;
		while (*pred > to_sort) {
			*curr = *pred;
			curr = pred--;
		}
		*curr = to_sort;
	}
}
		\end{lstlisting}
		\caption{
			Start at the first element and with predecessor pointer.
		}
		\label{fig:insertion:impl:pred_first}
	\end{subfigure}
	\hfill
	\begin{subfigure}{\codewidth}
		\begin{lstlisting}
void InsertionSort(int *start, int *end) {
	int *curr, *i = start + 1;
	while ((curr = i++) <= end) {
		int to_sort = *curr;
		int *pred = curr - 1;
		while (*pred > to_sort) {
			*curr = *pred;
			curr = pred--;
		}
		*curr = to_sort;
	}
}
		\end{lstlisting}
		\caption{
			Start at the second element and with predecessor pointer.
		}
		\label{fig:insertion:impl:pred_sec}
	\end{subfigure}

	\smallskip

	\begin{subfigure}{\codewidth}
		\begin{lstlisting}
void InsertionSort(int *start, int *end) {
	int *curr, *i = start;
	while ((curr = i++) <= end) {
		int to_sort = *curr;
		while (*(curr - 1) > to_sort) {
			*curr = *(curr - 1);
			curr--;
		}
		*curr = to_sort;
	}
}
		\end{lstlisting}
		\caption{
			Start at the first element and without predecessor pointer.
		}
		\label{fig:insertion:impl:offset_first}
	\end{subfigure}
	\hfil
	\begin{subfigure}{\codewidth}
		\begin{lstlisting}
void InsertionSort(int *start, int *end) {
	int *curr, *i = start + 1;
	while ((curr = i++) <= end) {
		int to_sort = *curr;
		while (*(curr - 1) > to_sort) {
			*curr = *(curr - 1);
			curr--;
		}
		*curr = to_sort;
	}
}
		\end{lstlisting}
		\caption{
			Start at the second element and without predecessor pointer.
		}
		\label{fig:insertion:impl:offset_sec}
	\end{subfigure}
	\caption{
		Four different implementations of \IS{} in C.
		\Cref{fig:insertion:impl:pred_first,fig:insertion:impl:offset_first} are compiled the same.
		\Cref{fig:insertion:impl:pred_sec,fig:insertion:impl:offset_sec} are compiled differently.
	}
	\label{fig:insertion:impl}
\end{figure}

Multiple workarounds exist to sidestep this problem.
One workaround is to take the unoptimised code and change the starting position via inline assembler.
This is trivial for the explicit \IS{} since one can simply inject an \lstinline|add| instruction at the beginning of the function to increment the pointer \lstinline|start|.
The implicit and the sentinel-less \IS*{} need to know the original starting address \lstinline|start| later on, though, and initialise the actual starting point rather late;
injecting inline assembler proves more difficult as a consequence.
Moreover, as \IS{} is to be used as fallback algorithm by other sorting algorithms which might also need to keep the original value of \lstinline|start|, inline assembler is a bad choice even for the explicit \IS{}.

Another workaround is the usage of a wrapper function calling \IS{} with the arguments \lstinline|start + 1| and \lstinline|end|.
This works quite well:
First, the register holding \lstinline|start| is incremented, and, then, the inlined code from the actual \IS{} follows.
Doing so makes the runtime drop as expected.

Recall how in the faster version (\cref{fig:insertion:impl:pred_first}), the pointer \lstinline|pred| is always deduced from the pointer \lstinline|curr| using an offset.
This gives the cue for yet another workaround:
In \cref{fig:insertion:impl:offset_first,fig:insertion:impl:offset_sec}, every occurrence of \lstinline|pred| is replaced with \lstinline|curr - 1|.
As a consequence, the code of \cref{fig:insertion:impl:offset_first} compiles the very same as the one of \cref{fig:insertion:impl:pred_first}, while \cref{fig:insertion:impl:offset_sec} yields the same compilation as the versions with the wrapper function or the inline assembly.
This workaround is clearly the best of the three and, hence, the one used in the rest of this thesis.

Alas, the eternal struggle with the compiler endeth not herewith.
A deeper look into the compilation reveals the following sequence:
\begin{center}
	\lstset{language={[DPU]Assembler}}
	\begin{tabular}{ll}
		\lstinline|move r3, r0| & \makebox[0pt][l]{\textit{// copy content of register \lstinline|r0| to \lstinline|r3|}} \\
		\lstinline|jleu r4, r2, .LABEL| & \makebox[0pt][l]{\textit{// jump to \lstinline|.LABEL| if \lstinline|r4| ≤ \lstinline|r2|}} \\
		\lstinline|move r3, r0| &
	\end{tabular}
\end{center}
Without delving further into its significance \Dash the second \lstinline|move r3, r0| is unneeded as it is impossible to jump directly to it nor to return via \lstinline|jleu|.
Also, \lstinline|move| does not set the zero or carry flag, so such a side effect as justification can be ruled out.
Copying the whole assembler code and injecting it as inline assembler but without this second \lstinline|move r3, r0| pushes the runtime even further down whilst maintaining the correctness of the output.
New issues, especially for inlining, are introduced, though, and we deem a proper assembly implementation as out of scope for this thesis.
