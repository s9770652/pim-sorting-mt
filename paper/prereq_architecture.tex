\section{Architecture}
\label{sec:prereq:architecture}

\begin{itemize}
	\item
	unmodified DRAM process (DDR4 2400 DIMM)
	\begin{itemize}
		\item
		replacement for standard DIMMS
		\begin{itemize}
			\item
			two (?) regular DIMMS must still be present

			\item
			conventional memory controllers
		\end{itemize}

		\item
		2 × 8 PIM chips

		\item
		chips on top

		\item
		64 MiB RAM per chip

		\item
		8 DPUs per PIM chip → 128 DPUs and 8 GiB per DIMM

		\item
		up to 28 DIMMs → 3584 DPUs

		\item
		software tasklets = hardware threads (24)
		\begin{itemize}
			\item
			independent (Single Program Multiple Data (SPMD))
		\end{itemize}

		\item
		high parallelisation is needed!

		\item
		reminiscent of GPU programming

		\item
		fast memory access due to spatial proximity → data movement bottleneck bypassed

		\item
		bad:
		less dense and 3 times slower than ASICs

		\item
		compute-bound architecture, so memory-bound problems better

		\item
		energiesparsam:
		etwa 1,2 W je Die
		(+ ein paar Zahlen aus HotChips heraussuchen)
	\end{itemize}

	\item
	memories
	\begin{itemize}
		\item
		MRAM:
		64 MiB
		\begin{itemize}
			\item
			slow

			\item
			readily available to both DPU and host, but not at the same time

			\item
			structure up to the programmer
		\end{itemize}

		\item
		WRAM:
		64 KiB (less with v1B DPUs)
		\begin{itemize}
			\item
			fast

			\item
			only available to host if specified

			\item
			allocated memory for stacks for each tasklet

			\item
			free memory allocatable later
		\end{itemize}

		\item
		IRAM:
		24 KiB IRAM (less with v1B DPUs)
		\begin{itemize}
			\item
			contains instructions (up to 4096/3968)

			\item
			readable and writeable by the host, readable by the DPU
		\end{itemize}

		\item
		atomic memory:
		256 bits
		\begin{itemize}
			\item
			hardware support for atomic accesses
		\end{itemize}

		\item
		run memory:
		64 bits
		\begin{itemize}
			\item
			booting, suspending, and resuming threads/tasklets
		\end{itemize}

		\item
		Grafik:
		HotChips 31, Folie 14
	\end{itemize}

	\item
	pipeline
	\begin{itemize}
		\item
		Grafik:
		HotChips 31, Folie 12

		\item
		350 MHz (Handbuch), 400 MHz (Handbuch), 450 MHz (\enquote{Reference Platfrom}), 500 MHz (Hot chip), 600 MHz (White paper)

		\item
		14 pipe stages; 3 overlap → 11 cycles effectively per instruction (uniform cost model → counting cycles/instructions useful)
		\begin{itemize}
			\item
			Ausnahmen:
			DMAs;
			12 Takte
		\end{itemize}

		\item
		interleaved → nominal performance of 1 instruction per cycle with 11+ tasklets (more not harmful)

		\item
		autonomous DMA engine with little to no effect on pipeline performance
	\end{itemize}

	\item
	instruction set architecture (ISA)
	\begin{itemize}
		\item
		RISC

		\item
		mostly no

		\item
		32-bit mostly with few 64-bit instructions

		\item
		many instructions for 64 bit emulated

		\item
		no native multiplication or division;
		function calls if not emulatable through bitwise shifts

		\item
		no native floating point arithmetic

		\item
		registers
		\begin{itemize}
			\item
			32 in total
			\begin{itemize}
				\item
				r0 -- r7:
				private,
				general purpose,
				caller saved,
				argument 1 -- 7,
				return register(s)

				\item
				r8 -- 13:
				private,
				general purpose,
				caller saved

				\item
				r14 -- 21:
				private,
				general purpose,
				callee saved

				\item
				r22:
				private,
				stack pointer

				\item
				r23:
				private,
				return address

				\item
				zero, one, lneg (--1), mneg (--2\textsuperscript{31}):
				common,
				read-only

				\item
				id, id2, id4, id8:
				private,
				read-only

				\item
				d0 -- d22:
				64-bit integers

				\item
				still more:
				program counter (12--16 bit);
				time counter (36 bit);
				carry bit for 64-bit instructions (persistent 1-bit flag);
				zero flag (persistent 1-bit flag)
			\end{itemize}

			\item
			64-bit loads, stores, moves
		\end{itemize}

		\item
		3-operands design, but labels and immediate values also possible

		\item
		registers written in reverse order

		\item
		result register always last

		\item
		plethora of conditions (51!)
		\begin{itemize}
			\item
			statuses used in conjunction with conditions

			\item
			jump, but also do something and jump

			\item
			cheaper loops
		\end{itemize}

		\item
		no branch prediction

		\item
		costs for memory accesses independent from address
	\end{itemize}

	\item
	programming model
	\begin{itemize}
		\item
		C for DPU, C, C++, Java, or Python for host
		\begin{itemize}
			\item
			theoretically no need for inline assembler
		\end{itemize}

		\item
		software development kit includes simulator

		\item
		main orchestration by CPU

		\item
		typical approach:
		boot host program → write to MRAM and/or WRAM → boot DPUs → compute on DPUs → read from MRAM and/or WRAM once finished → repeat if needed (no memory deletion)

		\item
		wichtig:
		\lstinline|mram_read| und \lstinline|mram_write| statt normaler Zugriffe
		\begin{itemize}
			\item
			DMAs:
			MRAM → WRAM;
			WRAM → MRAM;
			MRAM → IRAM (not interesting)

			\item
			alignment on 8 bytes of targets and source addresses

			\item
			transfer size between 8 and 2048 bytes + multiple of 8

			\item
			serial transfer

			\item
			the greater the transfer, the less significant the overhead

			\item
			Angabe der Geschwindigkeiten der Zürcher
		\end{itemize}

		\item
		WRAM heap allocation
		\begin{itemize}
			\item
			several option:
			incremental allocator (≈ \lstinline|malloc|), buddy allocator, block allocator

			\item
			free WRAM after the stacks, global variables, internal and oblique caches

			\item
			heap misnomer;
			actually a stack

			\item
			no possibility to partially free;
			full resets only

			\item
			no true ownership;
			paying heed duty by programmer
		\end{itemize}

		\item
		communication \& synchronisation with other tasklets mainly via global variables in WRAM and MRAM, barriers, semaphores and so on

		\item
		stick to 32-bit if possible
		\begin{itemize}
			\item
			64 bit costs the same, twice or thrice
		\end{itemize}

		\item
		stick to addition, subtraction, bitwise logic

		\item
		fewer instructions → better
		\begin{itemize}
			\item
			store reused results as compiler may discard them

			\item
			global variables better than arguments if function called oftentimes

			\item
			inlining häufig besser, da kein Funktionsaufruf (empirisch 144 Takte bei zwei Argumenten), aber nicht immer

			\item
			pointer arithmetic tends to be compiler better
		\end{itemize}

		\item
		aber:
		Compiler macht eh immer einen Strich durch die Rechnung

		\item
		recall:
		no imminent locality of time and space
		\begin{itemize}
			\item
			of course, DMAs somewhat reintroduce it
		\end{itemize}

		\item
		use as little synchronisation as possible, especially between DPUs

		\item
		utilise the pipeline as much as possible
	\end{itemize}
\end{itemize}
