\section{Architecture}
\label{sec:prereq:arch}

\begin{itemize}
	\item
	unmodified DRAM process (DDR4 2400 DIMM)
	\begin{itemize}
		\item
		replacement for standard DIMMS
		\begin{itemize}
			\item
			two (?) regular DIMMS must still be present

			\item
			conventional memory controllers
		\end{itemize}

		\item
		2 × 8 PIM chips

		\item
		chips on top

		\item
		64 MiB RAM per chip

		\item
		8 DPUs per PIM chip → 128 DPUs and 8 GiB per DIMM

		\item
		up to 28 DIMMs → 3584 DPUs

		\item
		software tasklets = hardware threads (24)
		\begin{itemize}
			\item
			independent (Single Program Multiple Data (SPMD))
		\end{itemize}

		\item
		high parallelisation is needed!

		\item
		reminiscent of GPU programming

		\item
		fast memory access due to spatial proximity → data movement bottleneck bypassed

		\item
		bad:
		less dense and 3 times slower than ASICs

		\item
		compute-bound architecture, so memory-bound problems better

		\item
		energiesparsam:
		etwa 1,2 W je Die
		(+ ein paar Zahlen aus HotChips heraussuchen)
	\end{itemize}

	\item
	memories
	\begin{itemize}
		\item
		MRAM:
		64 MiB
		\begin{itemize}
			\item
			slow

			\item
			readily available to both DPU and host, but not at the same time

			\item
			structure up to the programmer
		\end{itemize}

		\item
		WRAM:
		64 KiB (less with v1B DPUs)
		\begin{itemize}
			\item
			fast

			\item
			only available to host if specified

			\item
			allocated memory for stacks for each tasklet

			\item
			free memory allocatable later
		\end{itemize}

		\item
		IRAM:
		24 KiB IRAM (less with v1B DPUs)
		\begin{itemize}
			\item
			contains instructions (up to 4096/3968)

			\item
			readable and writeable by the host, readable by the DPU
		\end{itemize}

		\item
		atomic memory:
		256 bits
		\begin{itemize}
			\item
			hardware support for atomic accesses
		\end{itemize}

		\item
		run memory:
		64 bits
		\begin{itemize}
			\item
			booting, suspending, and resuming threads/tasklets
		\end{itemize}

		\item
		Grafik:
		HotChips 31, Folie 14
	\end{itemize}

	\item
	pipeline
	\begin{itemize}
		\item
		Grafik:
		HotChips 31, Folie 12

		\item
		267 MHz (\enquote{Benchmarking …}), 350 MHz (Handbuch), 400 MHz (Handbuch), 450 MHz (\enquote{Reference Platfrom}), 500 MHz (Hot chip), 600 MHz (White paper)

		\item
		14 pipe stages; 3 overlap → 11 cycles effectively per instruction (uniform cost model → counting cycles/instructions useful)
		\begin{itemize}
			\item
			Ausnahmen:
			DMAs;
			12 Takte
		\end{itemize}

		\item
		interleaved → nominal performance of 1 instruction per cycle with 11+ tasklets (more not harmful)

		\item
		autonomous DMA engine with little to no effect on pipeline performance
	\end{itemize}

	\item
	instruction set architecture (ISA)
	\begin{itemize}
		\item
		RISC

		\item
		32-bit mostly with few 64-bit instructions

		\item
		many instructions for 64 bit emulated

		\item
		no native multiplication or division;
		function calls if not emulatable through bitwise shifts

		\item
		no native floating point arithmetic

		\item
		registers
		\begin{itemize}
			\item
			32 in total
			\begin{itemize}
				\item
				r0 -- r7:
				private,
				general purpose,
				caller saved,
				argument 1 -- 7,
				return register(s)

				\item
				r8 -- 13:
				private,
				general purpose,
				caller saved

				\item
				r14 -- 21:
				private,
				general purpose,
				callee saved

				\item
				r22:
				private,
				stack pointer

				\item
				r23:
				private,
				return address

				\item
				zero, one, lneg (--1), mneg (--2\textsuperscript{31}):
				common,
				read-only

				\item
				id, id2, id4, id8:
				private,
				read-only

				\item
				d0 -- d22:
				64-bit integers

				\item
				still more:
				program counter (12--16 bit);
				time counter (36 bit);
				carry bit for 64-bit instructions (persistent 1-bit flag);
				zero flag (persistent 1-bit flag)
			\end{itemize}

			\item
			64-bit loads, stores, moves
		\end{itemize}

		\item
		3-operands design, but labels and immediate values also possible

		\item
		registers written in reverse order

		\item
		result register always last

		\item
		plethora of conditions (51!)
		\begin{itemize}
			\item
			statuses used in conjunction with conditions

			\item
			jump, but also do something and jump

			\item
			cheaper loops
		\end{itemize}

		\item
		no branch prediction

		\item
		costs for memory accesses independent from address
	\end{itemize}

	\item
	programming model
	\begin{itemize}
		\item
		C for DPU, C, C++, Java, or Python for host
		\begin{itemize}
			\item
			theoretically no need for inline assembler
		\end{itemize}

		\item
		software development kit includes simulator

		\item
		main orchestration by CPU

		\item
		typical approach:
		boot host program → write to MRAM and/or WRAM → boot DPUs → compute on DPUs → read from MRAM and/or WRAM once finished → repeat if needed (no memory deletion)

		\item
		wichtig:
		\lstinline|mram_read| und \lstinline|mram_write| statt normaler Zugriffe
		\begin{itemize}
			\item
			DMAs:
			MRAM → WRAM;
			WRAM → MRAM;
			MRAM → IRAM (not interesting)

			\item
			alignment on 8 bytes of targets and source addresses

			\item
			transfer size between 8 and 2048 bytes + multiple of 8

			\item
			serial transfer

			\item
			the greater the transfer, the less significant the overhead

			\item
			Angabe der Geschwindigkeiten der Zürcher
		\end{itemize}

		\item
		WRAM heap allocation
		\begin{itemize}
			\item
			several option:
			incremental allocator (≈ \lstinline|malloc|), buddy allocator, block allocator

			\item
			free WRAM after the stacks, global variables, internal and oblique caches

			\item
			heap misnomer;
			actually a stack

			\item
			no possibility to partially free;
			full resets only

			\item
			no true ownership;
			paying heed duty by programmer
		\end{itemize}

		\item
		communication \& synchronisation with other tasklets mainly via global variables in WRAM and MRAM, barriers, semaphores and so on

		\item
		stick to 32-bit if possible
		\begin{itemize}
			\item
			64 bit costs the same, twice or thrice
		\end{itemize}

		\item
		stick to addition, subtraction, bitwise logic

		\item
		fewer instructions → better
		\begin{itemize}
			\item
			store reused results as compiler may discard them

			\item
			global variables better than arguments if function called oftentimes

			\item
			inlining häufig besser, da kein Funktionsaufruf (empirisch 144 Takte bei zwei Argumenten), aber nicht immer

			\item
			pointer arithmetic tends to be compiler better
		\end{itemize}

		\item
		aber:
		Compiler macht eh immer einen Strich durch die Rechnung

		\item
		recall:
		no imminent locality of time and space
		\begin{itemize}
			\item
			of course, DMAs somewhat reintroduce it
		\end{itemize}

		\item
		use as little synchronisation as possible, especially between DPUs

		\item
		utilise the pipeline as much as possible
	\end{itemize}
\end{itemize}

\subsection{Overview}
\label{sec:prereq:arch:overview}

The \ac{PIM} capabilities are realised on modules of regular \ac{RAM} or, to be more precise, on \acp{DIMM} of \acrolabel{DDR}\aclu{DDR} \acrolabel{SDRAM}\aclu{SDRAM} with a transfer rate of \qty{2400}{\mega\transfer\per\second} (\acs{DDR}-2400 \acs{SDRAM}).
Therefore, \ac{PIM} \acp{DIMM} can act as replacement for \acp{DIMM} already present in existing systems without repercussion for tasks which do not rely on in-memory processing.
A \ac{PIM} \ac{DIMM} contains sixteen \emph{PIM chips}, that is modified \acrolabel{DRAM}\acsu{DRAM} packages, which contain the memory storage cells.
Each \ac{PIM} chip, in turn, contains eight \acrolabel{DPU}\emph{\aclp{DPU}}\acused{DPU} (\acsp{DPU}), so there are 128 \acp{DPU} per \ac{DIMM}.
Each \ac{DPU} is closely situated to one of the memory banks of size \qty{64}{\mebi\byte}.
Due to the spatial proximity to its memory bank, a \ac{DPU} is capable of rapidly accessing data stored on a \ac{DIMM}.

Depending on the model, a \ac{DPU} possesses either 16 or 24 hardware threads\footnote{
	There are two \ac{DPU} models, v1A and v1B.
	The former runs at \qty{350}{\mega\hertz} and is equipped with 24 threads, whereas the latter runs at \qty{400}{\mega\hertz} and is equipped with 16 threads.
	Measurements for this thesis were conducted on v1A.
}, whose software abstraction are called \emph{tasklets}.
Taklets work independently from each other, meaning programs can use different control flows to process different data.
An \upmem{} system can boast up to 20 \ac{PIM} \acp{DIMM}, setting the total count of \acp{DPU} to \num{2560} and of tasklets to \num{40960} or \num{61440}, respectively.
Whereas tasklets of the same \ac{DPU} communicate using shared memory, \acp{DPU} have no direct way to communicate or even share data with each other.
Instead, inter-\ac{DPU} communication is implemented by the host \ac{CPU} fetching data from one \ac{DPU} and sending it to another one.
Hence, for a task to run well on a \ac{PIM} system, it not only needs to frequently access the \ac{RAM}, it also needs to consist of many, fairly independent subtasks.
If such a task is indeed on hand, speedups well in the double digits for memory-bound tasks, compared to an execution on a \ac{CPU} or \ac{GPU}, are possible (compare \citeauthor{mutlu2022Benchmarking}~\cite{mutlu2022Benchmarking}).
Next to a faster execution, a gain in power efficiency is also to be expected, since data transfers between the \ac{RAM} and a host \ac{CPU} drive the energy consumption in regular systems significantly;
\upmem{} claims a tenfold increase of the power efficiency.

The retention of the general \ac{DDR} architecture comes at a price.
A \ac{DPU} is implemented using only three layers of silicon, resulting in transistors three times slower than other transistors of the same process node.
Also, their density is considerably reduced.
In consequence, \acp{DPU} are not suitable for computing-intensive tasks (compare also \citeauthor{mutlu2022Benchmarking}~\cite{mutlu2022Benchmarking}).


\subsection{Structure of an \texorpdfstring{\upmem{}}{UPMEM} Chip}
\label{sec:prereq:arch:structure}

\begin{figure}
	\centering
	\tikzsetnextfilename{arch_chip}
	\begin{tikzpicture}[
		sketch,
		dpu/.style={ fill=black!5 },
		mem d/.style={ fill=accentcolor!10!white },
		mem u/.style={ fill=accentcolor!25!white },
		flow/.style={ {Straight Barb[width=1.5mm]}-{Straight Barb[width=1.5mm]} },
		flow left/.style={ {Straight Barb[width=1.5mm]}- },
		flow right/.style={ -{Straight Barb[width=1.5mm]} },
	]
		\def\lenx{24}
		\def\pad{0.5}
		\def\numdpu{4}
		\def\paddpu{0.35}
		\def\paddpuflow{(0.75)}
		\def\intermem{1}
		\def\lendpux{(\lenx-2*\pad-\numdpu*\paddpu)}
		\def\lendpuy{(5)}
		\def\lencontrolx{((\lenx-2*\pad-\intermem)/2)}
		\def\lencontroly{1}
		\def\leny{(\lendpuy+\lencontroly+2*\pad+\numdpu*\paddpu+1)}

		% DPUs.
		%			\draw (0, 0) rectangle +(\lenx, \leny);

		\coordinate (dpu) at (\pad, \pad);
		\draw[dpu] (dpu) rectangle +({\lendpux}, {\lendpuy});
		\foreach \i in {\numdpu,...,0}{
			\draw[dpu, opacity=(\numdpu+1-\i)/(\numdpu+1)] ($(dpu) + (\i*\paddpu, \i*\paddpu)$) coordinate (dpu\i) rectangle +({\lendpux}, {\lendpuy});
		}
		\draw[latex-latex] ($(dpu) +  ({\lendpux+\pad/2}, 0)$) -- +(\numdpu*\paddpu, \numdpu*\paddpu) node[midway, below right=-1mm] {×8};
		\node[above, opacity=0] at (dpu) {×8};

		% Memories.
		\def\padmem{\paddpu}
		\def\lenmemtotalx{(\lendpux-2*\padmem-3*\intermem)}
		\def\lenmramx{(\lenmemtotalx/2)}
		\def\lenmemx{((\lenmemtotalx-\lenmramx)/3)}
		\def\lenmemy{(\lendpuy-2*\padmem)}
		\def\lenmemhalfy{((\lendpuy-2*\padmem-\intermem)/2)}

		\coordinate (mem) at ($(dpu) + (\padmem, \padmem)$);
		\draw[mem u] (mem)                                                        coordinate (pipe) rectangle +( {\lenmemx},     {\lenmemy}) node[midway] {Pipeline};
		\draw[mem u] ($(mem) + ({\lenmemx+\intermem}, 0)$)                        coordinate (wram) rectangle +( {\lenmemx}, {\lenmemhalfy}) node[midway] {WRAM};
		\draw[mem u] ($(mem) + ({\lenmemx+\intermem}, {\lenmemhalfy+\intermem})$) coordinate (iram) rectangle +( {\lenmemx}, {\lenmemhalfy}) node[midway] {IRAM};
		\draw[mem u] ($(mem) + ({2*(\lenmemx+\intermem)}, 0)$)                    coordinate (dma)  rectangle +( {\lenmemx},     {\lenmemy}) node[midway, align=center] {DMA\\Engine};
		\draw[mem d] ($(mem) + ({3*(\lenmemx+\intermem)}, 0)$)                    coordinate (mram) rectangle +({\lenmramx},     {\lenmemy}) node[midway] {MRAM};

		% Data flows.
		\draw[flow]       ($(iram) + (0, {\lenmemhalfy/2})$)          -- +(-\intermem, 0);  % Pipeline ↔ IRAM
		\draw[flow]       ($(wram) + (0, {\lenmemhalfy/2})$)          -- +(-\intermem, 0);  % Pipeline ↔ WRAM
		\draw[flow right] ($(pipe) + ({\lenmemx}, {\lenmemy/2})$)     -- +({2*\intermem+\lenmemx}, 0);  % Pipeline ↔ WRAM

		\draw[flow]       ($(wram) + ({\lenmemx}, {\lenmemhalfy/2})$) -- +(\intermem, 0);  % WRAM ↔ DMA
		\draw[flow left]  ($(iram) + ({\lenmemx}, {\lenmemhalfy/2})$) -- +(\intermem, 0);  % IRAM → DMA

		\draw[flow]       ($(dma)  + ({\lenmemx}, {\lenmemy/2})$)     -- +(\intermem, 0);  % DMA ↔ MRAM

		\draw[flow]       ($({\pad+\lencontrolx}, {\leny-\pad-\lencontroly/2})$)     -- +(\intermem, 0);  % CSI ↔ DDR
		\foreach \i in {\numdpu,...,0}{
			\pgfmathsetmacro{\fade}{int( (\numdpu+1-\i)/(\numdpu+1) * 100 )}
			\def\flowlen{\leny-2*\pad-\lencontroly-\i*\paddpu-\lendpuy+\padmem}
			\draw[flow, draw=black!\fade] ($(dpu\i) + ({\paddpu+\lenmemx/2+\i*\paddpuflow}, {\lendpuy-\padmem})$) -- +($(0, {\flowlen})$);
			\draw[flow, draw=black!\fade] ($(dpu\i) + ({\lendpux-\padmem-\lenmramx+\lenmemx*1.5+\i*\paddpuflow}, {\lendpuy-\padmem})$) -- +($(0, {\flowlen})$);
		}

		% Controllers.
		\draw[mem u] (\pad, {\leny-\pad})                      coordinate (csi) rectangle +({\lencontrolx}, -\lencontroly) node[midway] {Control/Status Interface};
		\draw[mem d] ($(csi) + ({\intermem+\lencontrolx}, 0)$) coordinate (ddr) rectangle +({\lencontrolx}, -\lencontroly) node[midway] {DDR\liningnums{4} Interface};
	\end{tikzpicture}

	\caption{
		The structure of a \ac{PIM} chip.
		The bright components are part of a standard \ac{DDR} package, the dark components are exclusive to \ac{PIM} chips.
	}
	\label{fig:arch:chip}
\end{figure}

A \ac{PIM} chip (\cref{fig:arch:chip}) contains eight \ac{DRAM} banks of \qty{64}{\mebi\byte} each.
These are connected with a regular memory controller through which a host \ac{CPU} can access the memory.
Next to each \ac{DRAM} bank is a \ac{DPU}.
The eight \acp{DPU} are connected with a special control interface which, in turn, is connected with the memory controller.
This allows the host to communicate with the \acp{DPU} but it does not allow \acp{DPU} to access \ac{DRAM} banks other than their own.
Instead, a \ac{DPU} possesses a direct connection to its \ac{DRAM} bank, thus bypassing the memory controller.
Such an access is also called \acfi{DMA} and is handled by the so-called \emph{\ac{DMA} engine}.
It is not possible for a \ac{DPU} and the host to access a \ac{DRAM} bank concurrently.

\Acp{DPU} contain several major and minor memories.
The memory of the \ac{DRAM} bank is also referred to as \acfi{MRAM}.
It is by far the largest memory of a \ac{DPU} and typically holds both the input provided by the host and the output calculated by the \ac{DPU}.
However, the \ac{MRAM} is also the slowest memory, for each access comes with non-negligible latency.

The \acfi{WRAM} is far smaller, comprising only \qty{64}{\kibi\byte}, yet the latency is practically zero.
A typical workflow is, hence, to load input data from the \ac{MRAM} into the \ac{WRAM}, process it, and write the output data back into the \ac{MRAM}.
The \ac{WRAM} also contains the stacks of the tasklets, where their local variables are stored, and global variables which are visible to all tasklets.
Moreover, tasklets can dynamically allocate further memory, which is also located in the \ac{WRAM}.
A \ac{DPU} does not dispose of a multilevel cache hierarchy moving data automatically like a \ac{CPU} does, and it is in the responsibility of the programmer to ensure that critical data are stored in the \ac{WRAM}.
Anyway, there is still a small number of automatically managed registers (see also \cref{sec:prereq:arch:isa}).
The host may access a specific section of the \ac{WRAM} only if the data has been specifically designated for this purpose, and such transfers are slower than transfers with the \ac{MRAM}.

Whilst the \ac{WRAM} holds the data which is processed, the \acfi{IRAM} contains the program (also called \emph{kernel}) which a \ac{DPU} executes.
The \ac{IRAM} has a size of \qty{24}{\kibi\byte} which translates to a maximum of \num{4096} instructions out of which a kernel has to be built.
This memory can be modified only by the host, as the \ac{DPU} can only read it, which is an automated process usually.

Next to these major memories, there is also a \qty{256}{\bit} large \emph{atomic memory} whose bits are accessible in a thread-safe way, allowing for mutual exclusion, barriers, and the like.
Furthermore, there is a \emph{run memory} through which threads can be booted, suspended, and resumed.

\begin{note}
	There are two \ac{DPU} models, v1A and v1B.
	The former runs at \qty{350}{\mega\hertz} and is equipped with 24 threads, whereas the latter runs at \qty{400}{\mega\hertz} and is equipped with 16 threads.
	Additionally, the model v1B can hold \qty{2}{\kibi\byte} of data less in its \ac{MRAM} and 128 instructions less in its \ac{IRAM}, since parts of those are \textquote{reserved for production and quality control purposes.}~\cite[Introduction~-- DPU chip characteristics]{upmemSDK}
	The model used for the measurements of this thesis is v1A.
\end{note}


\subsection{The Pipeline}
\label{sec:prereq:arch:pipeline}

Instructions are executed using \emph{pipelining}, that is, instructions are divided into several steps which are performed one after another, with each step taking exactly one cycle.
Once a step has been completed, the respective transistors are free to process the next instruction even if the previous instruction has not reached the end of the pipeline yet.
This allows all threads to use the same pipeline such that a nominal throughput of one instruction per cycle is achieved if enough threads are active and, thus, all steps of the pipeline are continuously performed.
\Cref{fig:arch:pipeline} shows the steps of the pipeline.
First, the instruction itself has to be loaded from the \ac{IRAM}.
Then, its argument are loaded from the registers before the actual computation is performed while accessing the \ac{WRAM} if necessary.
Although the step count is 14, the last three steps can be performed in parallel with the first three steps.
This means that the pipeline length is effectively reduced to 11, meaning only eleven active threads are needed to fully exploit the computing capabilities of a DPU.

Nevertheless, having more than eleven threads active is not detrimental to the throughput, which remains at one instruction per cycle, it only means that individual threads have to wait for some cycles.
This not only may make a parallel task easier to program, it can result in a performance gain when \acp{DMA} are involved.
\Acp{DMA} are mainly executed by the autonomous \ac{DMA} engine.
Whilst a thread is performing a \ac{DMA}, it is essentially suspended and removed from the pipeline, freeing a slot up.
Therefrom, the employment of more than eleven threads allows to hide absence by keeping the pipeline full.

As concluding remark, it shall be mentioned that there are circumstances under which the execution of an instruction takes twelve instead of eleven cycles.
This is related to the identifiers of the used registers, however the compiler usually manages to avoid these situations.
Hence, one can regard a \ac{DPU} as a \emph{uniform-cost machine} where each instruction takes eleven cycles to complete with the seldom exception of some taking twelve cycles and with the exception of \acp{DMA}.
Counting instructions is, therefore, a valid technique to assess the performance of some piece of code.

\begin{figure}
	\centering
	\includegraphics[page=65]{example-image-duck}

	\caption{
		The pipeline of a \ac{DPU}.
	}
	\label{fig:arch:pipeline}
\end{figure}


\subsection{Instruction Set Architecture}
\label{sec:prereq:arch:isa}

Each thread owns several private \emph{general-purpose registers} labelled \lstinline|r0| to \lstinline|r23| which can hold arbitrary 32-bit values and are freely readable and writeable by the respective thread.
Any even Register \lstinline|r|\(\mathtt{(2i)}\) and subsequent odd Register \lstinline|r|\(\mathtt{(2i+1)}\) form the 64-bit Register \lstinline|d|\(\mathtt{(2i)}\).
Furthermore, there are the read-only Registers \lstinline|id|, \lstinline|id2|, \lstinline|id4|, and \lstinline|id8|, which hold the identifier of the respective thread, multiplied by 1, 2, 4, and 8.
Also, there are special registers for the program counter holding the \ac{IRAM} index of the next instruction to execute, a performance counter used for measuring the time, a carry bit, and the zero flag.
Last but not least, there are four read-only registers which are shared by all threads:
the Registers~\lstinline|zero| and \lstinline|one| hold, as their names suggest, the constants \(0\) and \(1\), whereas the Registers~\lstinline|lneg| and \lstinline|mneg| hold the least negative and most negative 32-bit values, that is \(-1\) and \(-2^{31}\).

A \ac{DPU} is a \ac{RISC} with mainly 32-bit instructions \Dash most 64-bit instructions are pieced together from several 32-bit ones, thereby taking more than eleven cycles.
There is no hardware support for multiplication or division, so these are emulated by bitwise instructions, thereby taking even longer.
On top of that, there is no hardware support for floating point arithmetic, requiring costly emulation as well.
Instructions follow a 3-operands design, meaning there can be up to three register arguments to an instruction, with the target register coming first.
Next to registers, it is also possible to pass \emph{immediate values}, that is constant values passed directly without a register, and \emph{labels}, which are effectively \ac{IRAM} indices of instructions.
%The usage of instructions is intuitive and similar to ordinary assembler code.
Some examples:
\begin{itemize}
	\item
	\lstinline|move r6, 4| stores the immediate value \lstinline|4| in Register~\lstinline|r6|.

	\item
	\lstinline|lw r13, r12, -4| loads the 32-bit word which is four bytes away from the \ac{WRAM} address stored in Register~\lstinline|r12| into Register~\lstinline|r13|.
	Note that all addresses are physical.

	\item
	\lstinline|add r1, r5, r11| takes the 32-bit integers in Registers~\lstinline|r5| and \lstinline|r11|, adds them, stores the result in Register~\lstinline|r1|, and sets the carry bit accordingly.

	\item
	\lstinline|addc r0, r4, r10| performs an addition taking the carry bit into account, allowing to perform one 64-bit addition by invoking two 32-bit instructions.

	\item
	\lstinline|jump .LABEL| sets the program counter to the index of the labelled instruction.
\end{itemize}
Despite their name, some of the general-purpose registers have conventional uses.
The Registers~\lstinline|r0| to \lstinline|r7| are filled with the up to eight arguments of a function before it is called.
The return value of a function is written to the Registers~\lstinline|r0| or \lstinline|d0|, depending on whether it is \qty{32}{\bit} or \qty{64}{\bit} large.
Register~\lstinline|r22| contains the stack pointer, that is the address of the currently last element in the stack of the respective tasklet.
When a function is called and it need store data on the stack, it saves the original value of the stack pointer on the stack itself before incrementing the stack pointer, therethrough allocating new memory.
When the function terminates, it loads the original stack pointer value back into Register~\lstinline|r22|, therethrough deallocating memory.
Register~\lstinline|r23| contains the return address, that is the \ac{IRAM} index of the instruction whither to jump after the termination of a function.
Here, the instruction to load and store 64-bit large double words are of particular use.
By invoking \lstinline|sd r22, <offset>, d22|, the content of both Registers~\lstinline|r22| and \lstinline|r23| is stored to some position relative to the current stack pointer, whence it can be recovered by invoking \lstinline|ld d22, -<offset>, r22| later on.
Thereby, the bandwidth of the \ac{WRAM} is effectively doubled and the instruction count is reduced.

The capabilities of \ac{DPU} instructions is substantially enhanced by the plethora of \emph{conditions}, of which there are a total of 51.
Conditions are binary flags which are passed as additional arguments to instructions so that they act as either test operation or combo operations.
A \emph{test operation} performs its usual purpose but stores the evaluation of the condition in the target register.
For example, the instruction \lstinline|add r0, r0, -1, pl| takes the content of Register~\lstinline|r0|, decrements it, and checks the condition \lstinline|pl|.
This condition evaluates to true if the result is greater than or equal to zero.
Therefore, Register~\lstinline|r0| will contain the value \(1\) if and only if Register~\lstinline|r0| used to store the number \(1\) or greater, and will contain \(0\) otherwise.
A \emph{combo operation} takes a label as yet another argument.
The instruction performs its usual purpose, checks whether the result fulfils the condition, and, if yes, performs a jump to the label.
An example is the instruction \lstinline|add r0, r0, -1, pl, .LABEL_LOOP|, where Register~\lstinline|r0| holds a loop index which get decremented.
Should Register~\lstinline|r0| now hold a value greater or equal to zero, a jump back to the beginning of the loop body marked by the label \lstinline|.LABEL_LOOP| is performed.
Otherwise, the next line of the compilation is executed.
This way, it takes just eleven cycles to update the loop index, check the loop condition, and perform the appropriate action.
Such techniques of saving instructions are especially valuable because \acp{DPU} are incapable of instruction level parallelism.
Although conditions are employed automatically by the compiler for the most part, \cref{sec:mram} includes a manual use of conditions.


\subsection{Programming a Kernel}
\label{sec:prereq:arch:code}

Executing tasks on an \upmem{} system requires both a program executed on the host \ac{CPU} and a kernel executed on the \acp{DPU}.
\Acp{DPU} are handled in groups of up to 64 \acp{DPU} from the same rank of a \ac{DIMM}.
The groups, in turn, are aggregated in a \emph{\ac{DPU} set}.
A typical course of action is the following:
\begin{enumerate}
	\item
	Start the host program.

	\item
	Write the input to the \ac{MRAM} and/or \ac{WRAM} of all involved \acp{DPU}.

	\item
	Boot the \acp{DPU} and execute the kernel synchronously or asynchronously.

	\item
	Read the output from the \ac{MRAM} and/or \ac{WRAM}.

	\item
	Go back to the second or third step if needed.
\end{enumerate}
When kernels are executed synchronously, the host cannot access the memory of a \ac{DPU} until all \acp{DPU} in the whole set have finished.
But even with an asynchronous execution, the host cannot access the memory until all \acp{DPU} in the same rank have finished.
Note that data is generally not deleted when a kernel finishes, so subsequent executions can hark back to previous results.
Also, any communication between the host and the \acp{DPU} must be initiated by the host.
The host program can be written in \langC{}, \langCpp{}, \langJava{}, or \langPython{}.
Apart from a few additional functionalities provided by the \upmem{} \ac{API} for communicating with the \acp{DPU}, the host program is a regular executable.

The software development kit includes a simulator which allows to run kernels on machines without \upmem{} \acp{DIMM}.
The kernel has to be written in either \langC{} or assembler, but we will focus on the former.
Its entry point is the \lstinline|main| function, thence one can proceed as in any \langC{} program.
All tasklets execute the same kernel, but their control flow can be changed by simply including conditionals on the tasklet identifiers.
Synchronisation between tasklets can be achieved, amongst others, via barriers, mutual exclusion, and semaphores.
Communication between tasklets is achievable by defining global variables.
The \langC{} standard library is only partially available as some compute-intensive functionalities have been not been implemented, for example the entire \lstinline|math| library.

The biggest changes to a regular \langC{} program are in relation to the memory.
Any variable resides in the \ac{WRAM} by default, but creating an \ac{MRAM} variable is as easy as adding the qualifier \lstinline|__mram| to the variable declaration.
By default, too, any pointer is assumed to point to data in the \ac{WRAM}, which can be changed by adding the qualifier \lstinline|__mram_ptr|.
The compiler correctly identifies confusion between pointers of different address spaces.

Local \ac{WRAM} variables are created on the stack of the respective tasklet, and the stack sizes can be set for each tasklet individually at compile time.
Nonetheless, it is possible for tasklets to dynamically allocate more space on the \ac{WRAM} via an allocator similar to the standard \langC{} function \lstinline|malloc|.
Although this is called \emph{heap allocation}, the name is misleading.
The compiler organises the \ac{WRAM} such that all tasklet stacks and anything else statically allocated on the \ac{WRAM} is in the front, so that the free memory comprises a contiguous block in the back of the \ac{WRAM}.
Then, the so-called \emph{heap pointer} is set to the beginning of the free block.
When memory is allocated on the heap, the heap pointer is sufficiently incremented to mark the space as reserved.
Afterwards, the original position of the heap pointer is returned to the allocating tasklet.
In other words:
the heap memory is simply a stack memory shared by all tasklets.
Indeed, a \ac{DPU} lacks an equivalent to the standard \langC{} function \lstinline|free| to deallocate heap memory.
The only possibility is to reset the entire heap by setting the heap pointer back to its initial position.
There is also no model of ownership, so tasklets can write to any memory address, including the stack and heap memory of other tasklets.
Heed must be paid when structuring the scarce \ac{WRAM}, which is subject again in \cref{sec:mram:triple}.

As hinted before, transferring data between the \ac{MRAM} and the \ac{WRAM} is in the responsibility of the programmer.
When only single elements are to be accessed, variables in the \ac{MRAM} can be treated like normal variables.
For example, \lstinline|var = arr[i]| is valid code no matter whether the array \lstinline|arr|, the variable \lstinline|var|, or the index \lstinline|i| have been declared to reside in the WRAM or the MRAM.
However, this still constitutes one or more \acp{DMA} on each use, and each \ac{DMA} comes at a cost.
According to measurements~\cite{mutlu2022Benchmarking}, reading from the \ac{MRAM} has an overhead of \qty{77}{\cycles}, whilst writing to the \ac{MRAM} has an overhead of \qty{61}{\cycles}.
The transfer of each byte costs a further \qty{0.5}{\cycles}.
This means that \acp{DMA} of about \qty{128}{\byte} or less are dominated by the overhead.
Therefore, it is recommended to move large blocks of \ac{MRAM} data into the \ac{WRAM}, perform calculations there, and move the modified data block back to the \ac{MRAM}.
This way, the overhead is mitigated.
Note that, like for \ac{WRAM} data, the time to access \ac{MRAM} data is independent of the exact location \Dash only the memory type matters.

To perform such blockwise moves, one calls the \langC{} function \lstinline|mram_read| and \lstinline|mram_write|, which take a source address, a target address, and the number of bytes to use transfer.
There are, however, several limitations.
\begin{itemize}
	\item
	The \ac{WRAM} address must be aligned to 8 bytes.
	This can be ensured automatically by adding appropriate qualifiers to stack variables or by using heap memory which gets properly aligned automatically.

	\item
	The \ac{MRAM} address must be aligned to 8 bytes.
	No special functionality exists to this end;
	it is up to the programmer to organise the \ac{MRAM} with this limitation in mind and to resort to \acp{DMA} to single elements if such an alignment is not given.

	\item
	The number of transferred bytes must be at least 8, at most 2048, and a multiple of 8.
\end{itemize}
Failing to fulfil these constraints can result in missing or corrupt data.
The \ac{DMA} engine works sequentially, meaning data for only tasklet can be transferred at a time.
If multiple tasklets call \lstinline|mram_read| or \lstinline|mram_write|, some of them will be suspended for longer as they wait for the other \acp{DMA} to finish.
If \acp{DMA} are very frequent, having many active tasklets is especially important to keep the pipeline full.

For a performant kernel execution, it is generally recommended to restrict oneself to simple 32-bit logic as much as possible.
Some 64-bit functionalities are executed in eleven cycles, like loads and stores, but most take twice or even thrice as long.
Multiplication, division, and floating point arithmetic are emulated in software, so they should be avoided if necessary.
In their stead, addition, subtraction, and bitwise logic should be used.
Also, due to the unit-cost model, a decrease in the count of instructions translates into a performance gain.
Unfortunately, a common issue is a nosediving quality of the compilation, perhaps resulting from a wrong configuration of the \abb{LLVM}-based compiler.
Investigating the compilation and trying different approaches \Dash including even mundane alternatives like reordering independent \lstinline[keywords={}]|if| statements \Dash is paramount when aiming for top performance and, therefore, a recurring theme in this thesis.
In our experience, explicitly saving the result of a computation if the value is reused at a later time prevents the compiler from issuing a recalculation which would elsewise hurt due to the compute-boundedness of the architecture.
Also, it seems that pointer logic tends to be compiled better than index logic.
Lastly, inlining leads to a performance gain oftentimes as the overhead for function calls is quite heavy.
Even though the call itself is a mere jump taking elven cycles, several registers must be saved and reloaded on entering and exiting a function;
for an empty function with two arguments, we determined a call overhead of \qty{144}{\cycles}.
This number can easily rise with heavier register usage.
This may also explain why turning arguments into global variables nets a performance gain in some cases as well.

