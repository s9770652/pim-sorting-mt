\section{Architecture}
\label{sec:prereq:arch}

\begin{itemize}
	\item
	unmodified DRAM process (DDR4 2400 DIMM)
	\begin{itemize}
		\item
		replacement for standard DIMMS
		\begin{itemize}
			\item
			two (?) regular DIMMS must still be present

			\item
			conventional memory controllers
		\end{itemize}

		\item
		2 × 8 PIM chips

		\item
		chips on top

		\item
		64 MiB RAM per chip

		\item
		8 DPUs per PIM chip → 128 DPUs and 8 GiB per DIMM

		\item
		up to 28 DIMMs → 3584 DPUs

		\item
		software tasklets = hardware threads (24)
		\begin{itemize}
			\item
			independent (Single Program Multiple Data (SPMD))
		\end{itemize}

		\item
		high parallelisation is needed!

		\item
		reminiscent of GPU programming

		\item
		fast memory access due to spatial proximity → data movement bottleneck bypassed

		\item
		bad:
		less dense and 3 times slower than ASICs

		\item
		compute-bound architecture, so memory-bound problems better

		\item
		energiesparsam:
		etwa 1,2 W je Die
		(+ ein paar Zahlen aus HotChips heraussuchen)
	\end{itemize}

	\item
	memories
	\begin{itemize}
		\item
		MRAM:
		64 MiB
		\begin{itemize}
			\item
			slow

			\item
			readily available to both DPU and host, but not at the same time

			\item
			structure up to the programmer
		\end{itemize}

		\item
		WRAM:
		64 KiB (less with v1B DPUs)
		\begin{itemize}
			\item
			fast

			\item
			only available to host if specified

			\item
			allocated memory for stacks for each tasklet

			\item
			free memory allocatable later
		\end{itemize}

		\item
		IRAM:
		24 KiB IRAM (less with v1B DPUs)
		\begin{itemize}
			\item
			contains instructions (up to 4096/3968)

			\item
			readable and writeable by the host, readable by the DPU
		\end{itemize}

		\item
		atomic memory:
		256 bits
		\begin{itemize}
			\item
			hardware support for atomic accesses
		\end{itemize}

		\item
		run memory:
		64 bits
		\begin{itemize}
			\item
			booting, suspending, and resuming threads/tasklets
		\end{itemize}

		\item
		Grafik:
		HotChips 31, Folie 14
	\end{itemize}

	\item
	pipeline
	\begin{itemize}
		\item
		Grafik:
		HotChips 31, Folie 12

		\item
		267 MHz (\enquote{Benchmarking …}), 350 MHz (Handbuch), 400 MHz (Handbuch), 450 MHz (\enquote{Reference Platfrom}), 500 MHz (Hot chip), 600 MHz (White paper)

		\item
		14 pipe stages; 3 overlap → 11 cycles effectively per instruction (uniform cost model → counting cycles/instructions useful)
		\begin{itemize}
			\item
			Ausnahmen:
			DMAs;
			12 Takte
		\end{itemize}

		\item
		interleaved → nominal performance of 1 instruction per cycle with 11+ tasklets (more not harmful)

		\item
		autonomous DMA engine with little to no effect on pipeline performance
	\end{itemize}

	\item
	instruction set architecture (ISA)
	\begin{itemize}
		\item
		RISC

		\item
		mostly no

		\item
		32-bit mostly with few 64-bit instructions

		\item
		many instructions for 64 bit emulated

		\item
		no native multiplication or division;
		function calls if not emulatable through bitwise shifts

		\item
		no native floating point arithmetic

		\item
		registers
		\begin{itemize}
			\item
			32 in total
			\begin{itemize}
				\item
				r0 -- r7:
				private,
				general purpose,
				caller saved,
				argument 1 -- 7,
				return register(s)

				\item
				r8 -- 13:
				private,
				general purpose,
				caller saved

				\item
				r14 -- 21:
				private,
				general purpose,
				callee saved

				\item
				r22:
				private,
				stack pointer

				\item
				r23:
				private,
				return address

				\item
				zero, one, lneg (--1), mneg (--2\textsuperscript{31}):
				common,
				read-only

				\item
				id, id2, id4, id8:
				private,
				read-only

				\item
				d0 -- d22:
				64-bit integers

				\item
				still more:
				program counter (12--16 bit);
				time counter (36 bit);
				carry bit for 64-bit instructions (persistent 1-bit flag);
				zero flag (persistent 1-bit flag)
			\end{itemize}

			\item
			64-bit loads, stores, moves
		\end{itemize}

		\item
		3-operands design, but labels and immediate values also possible

		\item
		registers written in reverse order

		\item
		result register always last

		\item
		plethora of conditions (51!)
		\begin{itemize}
			\item
			statuses used in conjunction with conditions

			\item
			jump, but also do something and jump

			\item
			cheaper loops
		\end{itemize}

		\item
		no branch prediction

		\item
		costs for memory accesses independent from address
	\end{itemize}

	\item
	programming model
	\begin{itemize}
		\item
		C for DPU, C, C++, Java, or Python for host
		\begin{itemize}
			\item
			theoretically no need for inline assembler
		\end{itemize}

		\item
		software development kit includes simulator

		\item
		main orchestration by CPU

		\item
		typical approach:
		boot host program → write to MRAM and/or WRAM → boot DPUs → compute on DPUs → read from MRAM and/or WRAM once finished → repeat if needed (no memory deletion)

		\item
		wichtig:
		\lstinline|mram_read| und \lstinline|mram_write| statt normaler Zugriffe
		\begin{itemize}
			\item
			DMAs:
			MRAM → WRAM;
			WRAM → MRAM;
			MRAM → IRAM (not interesting)

			\item
			alignment on 8 bytes of targets and source addresses

			\item
			transfer size between 8 and 2048 bytes + multiple of 8

			\item
			serial transfer

			\item
			the greater the transfer, the less significant the overhead

			\item
			Angabe der Geschwindigkeiten der Zürcher
		\end{itemize}

		\item
		WRAM heap allocation
		\begin{itemize}
			\item
			several option:
			incremental allocator (≈ \lstinline|malloc|), buddy allocator, block allocator

			\item
			free WRAM after the stacks, global variables, internal and oblique caches

			\item
			heap misnomer;
			actually a stack

			\item
			no possibility to partially free;
			full resets only

			\item
			no true ownership;
			paying heed duty by programmer
		\end{itemize}

		\item
		communication \& synchronisation with other tasklets mainly via global variables in WRAM and MRAM, barriers, semaphores and so on

		\item
		stick to 32-bit if possible
		\begin{itemize}
			\item
			64 bit costs the same, twice or thrice
		\end{itemize}

		\item
		stick to addition, subtraction, bitwise logic

		\item
		fewer instructions → better
		\begin{itemize}
			\item
			store reused results as compiler may discard them

			\item
			global variables better than arguments if function called oftentimes

			\item
			inlining häufig besser, da kein Funktionsaufruf (empirisch 144 Takte bei zwei Argumenten), aber nicht immer

			\item
			pointer arithmetic tends to be compiler better
		\end{itemize}

		\item
		aber:
		Compiler macht eh immer einen Strich durch die Rechnung

		\item
		recall:
		no imminent locality of time and space
		\begin{itemize}
			\item
			of course, DMAs somewhat reintroduce it
		\end{itemize}

		\item
		use as little synchronisation as possible, especially between DPUs

		\item
		utilise the pipeline as much as possible
	\end{itemize}
\end{itemize}

\subsection{Overview}
\label{sec:prereq:arch:overview}

The \ac{PIM} capabilities are realised on, at its base, sticks of regular \ac{RAM} or, to be more precise, on \acp{DIMM} of \aclu{DDR} \aclu{SDRAM} with a transfer rate of \qty{2400}{\mega\transfer\per\second} (\acs{DDR}-2400 \acs{SDRAM}).
Therefore, \ac{PIM} \acp{DIMM} can act as replacement for \acp{DIMM} already present in existing systems without repercussion for tasks which do not rely on in-memory processing.
With eight on each side, a \ac{PIM} \ac{DIMM} contains sixteen \emph{PIM chips}, that is modified \acsu{DRAM} packages, which contain the memory storage cells.
Each \ac{PIM} chip, in turn, contains eight \emph{\aclp{DPU}}\acused{DPU} (\acsp{DPU}), so there are 128 \acp{DPU} per \ac{DIMM}.
Each \ac{DPU} is closely situated to one of the memory banks of size \qty{64}{\mebi\byte}.
Due to the spatial proximity to its memory bank, a \ac{DPU} is capable of rapidly accessing data stored on a \ac{DIMM}.

A \ac{DPU} possesses either 16 or 24 hardware threads, whose software abstraction are called \emph{tasklets}, depending on the model.
Taklets work independently from each other, meaning programs can use different control flows to process different data.
The orchestration of \acp{DPU} and their tasklets pose a major challenge during programming.
To put things into perspective:
UPMEM sells systems with up to 28 \ac{PIM} \acp{DIMM}, setting the total count of \acp{DPU} to \num{3584} and of tasklets to \num{57344} and \num{86016}, respectively.
Hence, for a task to run well on a \ac{PIM} system, it not only needs to frequently access the \ac{RAM}, it also needs to be highly parallelisable.
If such a highly parallelisable task is indeed on hand, speedups well in the double digits for memory-bound tasks, compared to an execution on a \ac{CPU} or \ac{GPU} are possible (compare \citeauthor{mutlu2022Benchmarking}~\cite{mutlu2022Benchmarking}).
Next to a faster execution, a gain in power efficiency is also to be expected, since data transfers between the \ac{RAM} and a host \ac{CPU} drive the energy consumption in regular systems significantly;
UPMEM claims a tenfold increase of the power efficiency.

The retention of the general \ac{DDR} architecture comes at a price.
A \ac{DPU} is implemented using only three layers of silicon, resulting in three times slower transistors compared to other transistors of the same process node.
Also, their density is considerably reduced.
In consequence, \acp{DPU} are not suitable for computing-intensive tasks (compare also \citeauthor{mutlu2022Benchmarking}~\cite{mutlu2022Benchmarking}).


\subsection{Structure of an \texorpdfstring{\upmem{}}{UPMEM} Chip}
\label{sec:prereq:arch:structure}

\begin{figure}
	\centering
	\tikzsetnextfilename{arch_chip}
	\begin{tikzpicture}[
		sketch,
		dpu/.style={ fill=black!5 },
		mem d/.style={ fill=accentcolor!10!white },
		mem u/.style={ fill=accentcolor!25!white },
		flow/.style={ {Straight Barb[width=1.5mm]}-{Straight Barb[width=1.5mm]} },
		flow left/.style={ {Straight Barb[width=1.5mm]}- },
		flow right/.style={ -{Straight Barb[width=1.5mm]} },
	]
		\def\lenx{24}
		\def\pad{0.5}
		\def\numdpu{4}
		\def\paddpu{0.35}
		\def\paddpuflow{(0.75)}
		\def\intermem{1}
		\def\lendpux{(\lenx-2*\pad-\numdpu*\paddpu)}
		\def\lendpuy{(5)}
		\def\lencontrolx{((\lenx-2*\pad-\intermem)/2)}
		\def\lencontroly{1}
		\def\leny{(\lendpuy+\lencontroly+2*\pad+\numdpu*\paddpu+1)}

		% DPUs.
		%			\draw (0, 0) rectangle +(\lenx, \leny);

		\coordinate (dpu) at (\pad, \pad);
		\draw[dpu] (dpu) rectangle +({\lendpux}, {\lendpuy});
		\foreach \i in {\numdpu,...,0}{
			\pgfmathsetmacro{\fade}{int( (\numdpu+1-\i)/(\numdpu+1) * 100 )}
			\pgfmathsetmacro{\fadefill}{int( (\numdpu+1-\i)/(\numdpu+1) * 0.05 * 100 )}
			\draw[dpu, draw=black!\fade, fill=black!\fadefill] ($(dpu) + (\i*\paddpu, \i*\paddpu)$) coordinate (dpu\i) rectangle +({\lendpux}, {\lendpuy});
		}
		\draw[latex-latex] ($(dpu) +  ({\lendpux+\pad/2}, 0)$) -- +(\numdpu*\paddpu, \numdpu*\paddpu) node[midway, below right=-1mm] {×8};
		\node[above, opacity=0] at (dpu) {×8};

		% Memories.
		\def\padmem{\paddpu}
		\def\lenmemtotalx{(\lendpux-2*\padmem-3*\intermem)}
		\def\lenmramx{(\lenmemtotalx/2)}
		\def\lenmemx{((\lenmemtotalx-\lenmramx)/3)}
		\def\lenmemy{(\lendpuy-2*\padmem)}
		\def\lenmemhalfy{((\lendpuy-2*\padmem-\intermem)/2)}

		\coordinate (mem) at ($(dpu) + (\padmem, \padmem)$);
		\draw[mem u] (mem)                                                        coordinate (pipe) rectangle +( {\lenmemx},     {\lenmemy}) node[midway] {Pipeline};
		\draw[mem u] ($(mem) + ({\lenmemx+\intermem}, 0)$)                        coordinate (wram) rectangle +( {\lenmemx}, {\lenmemhalfy}) node[midway] {WRAM};
		\draw[mem u] ($(mem) + ({\lenmemx+\intermem}, {\lenmemhalfy+\intermem})$) coordinate (iram) rectangle +( {\lenmemx}, {\lenmemhalfy}) node[midway] {IRAM};
		\draw[mem u] ($(mem) + ({2*(\lenmemx+\intermem)}, 0)$)                    coordinate (dma)  rectangle +( {\lenmemx},     {\lenmemy}) node[midway, align=center] {DMA\\Engine};
		\draw[mem d] ($(mem) + ({3*(\lenmemx+\intermem)}, 0)$)                    coordinate (mram) rectangle +({\lenmramx},     {\lenmemy}) node[midway] {MRAM};

		% Data flows.
		\draw[flow]       ($(iram) + (0, {\lenmemhalfy/2})$)          -- +(-\intermem, 0);  % Pipeline ↔ IRAM
		\draw[flow]       ($(wram) + (0, {\lenmemhalfy/2})$)          -- +(-\intermem, 0);  % Pipeline ↔ WRAM
		\draw[flow right] ($(pipe) + ({\lenmemx}, {\lenmemy/2})$)     -- +({2*\intermem+\lenmemx}, 0);  % Pipeline ↔ WRAM

		\draw[flow]       ($(wram) + ({\lenmemx}, {\lenmemhalfy/2})$) -- +(\intermem, 0);  % WRAM ↔ DMA
		\draw[flow left]  ($(iram) + ({\lenmemx}, {\lenmemhalfy/2})$) -- +(\intermem, 0);  % IRAM → DMA

		\draw[flow]       ($(dma)  + ({\lenmemx}, {\lenmemy/2})$)     -- +(\intermem, 0);  % DMA ↔ MRAM

		\draw[flow]       ($({\pad+\lencontrolx}, {\leny-\pad-\lencontroly/2})$)     -- +(\intermem, 0);  % CSI ↔ DDR
		\foreach \i in {\numdpu,...,0}{
			\pgfmathsetmacro{\fade}{int( (\numdpu+1-\i)/(\numdpu+1) * 100 )}
			\def\flowlen{\leny-2*\pad-\lencontroly-\i*\paddpu-\lendpuy+\padmem}
			\draw[flow, draw=black!\fade] ($(dpu\i) + ({\paddpu+\lenmemx/2+\i*\paddpuflow}, {\lendpuy-\padmem})$) -- +($(0, {\flowlen})$);
			\draw[flow, draw=black!\fade] ($(dpu\i) + ({\lendpux-\padmem-\lenmramx+\lenmemx*1.5+\i*\paddpuflow}, {\lendpuy-\padmem})$) -- +($(0, {\flowlen})$);
		}

		% Controllers.
		\draw[mem u] (\pad, {\leny-\pad})                      coordinate (csi) rectangle +({\lencontrolx}, -\lencontroly) node[midway] {Control/Status Interface};
		\draw[mem d] ($(csi) + ({\intermem+\lencontrolx}, 0)$) coordinate (ddr) rectangle +({\lencontrolx}, -\lencontroly) node[midway] {DDR\liningnums{4} Interface};
	\end{tikzpicture}

	\caption{
		The structure of a \ac{PIM} chip.
		The bright components are part of a standard \ac{DDR} package, the dark components are exclusive to \ac{PIM} chips.
	}
	\label{fig:arch:chip}
\end{figure}

A \ac{PIM} chip (\cref{fig:arch:chip}) contains eight \ac{DRAM} banks of \qty{64}{\mebi\byte} each.
These are connected with a regular memory controller through which a host \ac{CPU} can access the memory.
Next to each \ac{DRAM} bank is a \ac{DPU}.
The eight \acp{DPU} are connected with a special control interface which, in turn, is connected with the memory controller.
This allows the host to communicate with the \acp{DPU} but it does not allow \acp{DPU} to access \ac{DRAM} banks other than their own.
Instead, a \ac{DPU} possesses a direct connection to its \ac{DRAM} bank, thus bypassing the memory controller.
Such an access is also called \acfi{DMA} and is handled by the so-called \emph{\ac{DMA} engine}.
It is not possible for a \ac{DPU} and the host to access a \ac{DRAM} bank concurrently.

\Acp{DPU} contain several major and minor memories.
The memory of the \ac{DRAM} bank is also referred to as \acfi{MRAM}.
It is by far the largest memory of a \ac{DPU} and typically holds both the input provided by the host and the output calculated by the \ac{DPU}.
However, the \ac{MRAM} is also the slowest memory, for each access comes with non-negligible latency.

The \acfi{WRAM} is far smaller, comprising only \qty{64}{\kibi\byte}, yet the latency is practically zero.
A typical workflow is, hence, to load input data from the \ac{MRAM} into the \ac{WRAM}, process it, and write the output data back into the \ac{MRAM}.
The \ac{WRAM} also contains the stacks of the tasklets, where their local variables are stored, and global variables which are visible to all tasklets.
Moreover, tasklets can dynamically allocate further memory, which is also located in the \ac{WRAM}.
A \ac{DPU} does not dispose of a multilevel cache hierarchy moving data automatically like a \ac{CPU} does, and it is in the responsibility of the programmer to ensure that critical data are stored in the \ac{WRAM}.
Anyway, there is still a small number of automatically managed registers (see also \cref{sec:prereq:arch:isa}).
The host may access a specific section of the \ac{WRAM} only if the data has been specifically designated for this purpose, and such transfers are slower than transfers with the \ac{MRAM}.

Whilst the \ac{WRAM} holds the data which is processed, the \acfi{IRAM} contains the program (also called \emph{kernel}) which a \ac{DPU} executes.
The \ac{IRAM} has a size of \qty{24}{\kibi\byte} which translates to a maximum of \num{4096} instructions out of which a kernel has to be built.
This memory can be modified only by the host, as the \ac{DPU} can only read it, which is an automated process usually.

Next to these major memories, there is also a \qty{256}{\bit} large \emph{atomic memory} whose bits are accessible in a thread-safe way, allowing for mutual exclusion, barriers, and the like.
Furthermore, there is a \emph{run memory} through which threads can be booted, suspended, and resumed.

\begin{note}
	There are two \ac{DPU} models, v1A and v1B.
	The former runs at \qty{350}{\mega\hertz} and is equipped with 24 threads, whereas the latter runs at \qty{400}{\mega\hertz} and is equipped with 16 threads.
	Additionally, the model v1B can hold \qty{2}{\kibi\byte} of data less in its \ac{MRAM} and 128 instructions less in its \ac{IRAM}, since parts of those are \textquote{reserved for production and quality control purposes.}~\cite[Introduction~-- DPU chip characteristics]{upmemSDK}
	The model used for the measurements of this thesis is v1A.
\end{note}

