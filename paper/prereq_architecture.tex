\section{The \texorpdfstring{\upmem{}}{UPMEM} Architecture}
\label{sec:prereq:arch}

%This \lcnamecref{sec:prereq:arch} serves as an introduction to \ac{PIM} on an \upmem{} system in terms of both hardware and software, laying the foundation for understanding the design choices in \cref{sec:tasklet,sec:mram}.
\Cref{sec:prereq:arch:overview} provides a short conspectus of the composition of such a system, and \cref{sec:prereq:arch:structure} presents the components of an individual \ac{PIM} chip.
\Cref{sec:prereq:arch:pipeline} covers the instruction pipeline, which differs in key aspects from that of a modern \ac{CPU}, whilst \cref{sec:prereq:arch:isa} is concerned with the instruction set architecture, as acquaintance with it helps frequently in identifying optimisation potential.
Finally, \cref{sec:prereq:arch:code} gives an insight into developing programs for an \upmem{} system.
The four sources for this \lcnamecref{sec:prereq:arch} are a white paper by \upmem{}~\cite{upmem2021WhitePaper}, a talk given by \upmem{}'s founder, Fabrice Devaux, at the 31st Hot Chips Symposium~\cite{upmem2019HotChips}, the official documentation of the \upmem{} toolchain~\cite{upmemSDK}, and an extensive study by \citeauthor{mutlu2022Benchmarking}~\cite{mutlu2022Benchmarking}.
For the sake of clarity, most statements will not be given a specific source.

\subsection{Overview}
\label{sec:prereq:arch:overview}

The \ac{PIM} capabilities are realised on, at its base, sticks of regular \ac{RAM} or, to be more precise, on \acp{DIMM} of \aclu{DDR} \aclu{SDRAM} with a transfer rate of \qty{2400}{\mega\transfer\per\second} (\acs{DDR}-2400 \acs{SDRAM}).
Therefore, \ac{PIM} \acp{DIMM} can act as replacement for \acp{DIMM} already present in existing systems without repercussion for tasks which do not rely on in-memory processing.
With eight on each side, a \ac{PIM} \ac{DIMM} contains sixteen \emph{PIM chips}, that is modified \acsu{DRAM} packages, which contain the memory storage cells.
Each \ac{PIM} chip, in turn, contains eight \emph{\aclp{DPU}}\acused{DPU} (\acsp{DPU}), so there are 128 \acp{DPU} per \ac{DIMM}.
Each \ac{DPU} is closely situated to one of the memory banks of size \qty{64}{\mebi\byte}.
Due to the spatial proximity to its memory bank, a \ac{DPU} is capable of rapidly accessing data stored on a \ac{DIMM}.

A \ac{DPU} possesses either 16 or 24 hardware threads, whose software abstraction are called \emph{tasklets}, depending on the model.
Taklets work independently from each other, meaning programs can use different control flows to process different data.
The orchestration of \acp{DPU} and their tasklets pose a major challenge during programming.
To put things into perspective:
UPMEM sells systems with up to 28 \ac{PIM} \acp{DIMM}, setting the total count of \acp{DPU} to \num{3584} and of tasklets to \num{57344} and \num{86016}, respectively.
Hence, for a task to run well on a \ac{PIM} system, it not only needs to frequently access the \ac{RAM}, it also needs to be highly parallelisable.
If such a highly parallelisable task is indeed on hand, speedups well in the double digits for memory-bound tasks, compared to an execution on a \ac{CPU} or \ac{GPU} are possible (compare \citeauthor{mutlu2022Benchmarking}~\cite{mutlu2022Benchmarking}).
Next to a faster execution, a gain in power efficiency is also to be expected, since data transfers between the \ac{RAM} and a host \ac{CPU} drive the energy consumption in regular systems significantly;
UPMEM claims a tenfold increase of the power efficiency.

The retention of the general \ac{DDR} architecture comes at a price.
A \ac{DPU} is implemented using only three layers of silicon, resulting in three times slower transistors compared to other transistors of the same process node.
Also, their density is considerably reduced.
In consequence, \acp{DPU} are not suitable for computing-intensive tasks (compare also \citeauthor{mutlu2022Benchmarking}~\cite{mutlu2022Benchmarking}).


\subsection{Structure of an \texorpdfstring{\upmem{}}{UPMEM} Chip}
\label{sec:prereq:arch:structure}

\begin{figure}
	\centering
	\tikzsetnextfilename{arch_chip}
	\begin{tikzpicture}[
		sketch,
		dpu/.style={ fill=black!5 },
		mem d/.style={ fill=accentcolor!10!white },
		mem u/.style={ fill=accentcolor!25!white },
		flow/.style={ {Straight Barb[width=1.5mm]}-{Straight Barb[width=1.5mm]} },
		flow left/.style={ {Straight Barb[width=1.5mm]}- },
		flow right/.style={ -{Straight Barb[width=1.5mm]} },
	]
		\def\lenx{24}
		\def\pad{0.5}
		\def\numdpu{4}
		\def\paddpu{0.35}
		\def\paddpuflow{(0.75)}
		\def\intermem{1}
		\def\lendpux{(\lenx-2*\pad-\numdpu*\paddpu)}
		\def\lendpuy{(5)}
		\def\lencontrolx{((\lenx-2*\pad-\intermem)/2)}
		\def\lencontroly{1}
		\def\leny{(\lendpuy+\lencontroly+2*\pad+\numdpu*\paddpu+1)}

		% DPUs.
		%			\draw (0, 0) rectangle +(\lenx, \leny);

		\coordinate (dpu) at (\pad, \pad);
		\draw[dpu] (dpu) rectangle +({\lendpux}, {\lendpuy});
		\foreach \i in {\numdpu,...,0}{
			\pgfmathsetmacro{\fade}{int( (\numdpu+1-\i)/(\numdpu+1) * 100 )}
			\pgfmathsetmacro{\fadefill}{int( (\numdpu+1-\i)/(\numdpu+1) * 0.05 * 100 )}
			\draw[dpu, draw=black!\fade, fill=black!\fadefill] ($(dpu) + (\i*\paddpu, \i*\paddpu)$) coordinate (dpu\i) rectangle +({\lendpux}, {\lendpuy});
		}
		\draw[latex-latex] ($(dpu) +  ({\lendpux+\pad/2}, 0)$) -- +(\numdpu*\paddpu, \numdpu*\paddpu) node[midway, below right=-1mm] {×8};
		\node[above, opacity=0] at (dpu) {×8};

		% Memories.
		\def\padmem{\paddpu}
		\def\lenmemtotalx{(\lendpux-2*\padmem-3*\intermem)}
		\def\lenmramx{(\lenmemtotalx/2)}
		\def\lenmemx{((\lenmemtotalx-\lenmramx)/3)}
		\def\lenmemy{(\lendpuy-2*\padmem)}
		\def\lenmemhalfy{((\lendpuy-2*\padmem-\intermem)/2)}

		\coordinate (mem) at ($(dpu) + (\padmem, \padmem)$);
		\draw[mem u] (mem)                                                        coordinate (pipe) rectangle +( {\lenmemx},     {\lenmemy}) node[midway] {Pipeline};
		\draw[mem u] ($(mem) + ({\lenmemx+\intermem}, 0)$)                        coordinate (wram) rectangle +( {\lenmemx}, {\lenmemhalfy}) node[midway] {WRAM};
		\draw[mem u] ($(mem) + ({\lenmemx+\intermem}, {\lenmemhalfy+\intermem})$) coordinate (iram) rectangle +( {\lenmemx}, {\lenmemhalfy}) node[midway] {IRAM};
		\draw[mem u] ($(mem) + ({2*(\lenmemx+\intermem)}, 0)$)                    coordinate (dma)  rectangle +( {\lenmemx},     {\lenmemy}) node[midway, align=center] {DMA\\Engine};
		\draw[mem d] ($(mem) + ({3*(\lenmemx+\intermem)}, 0)$)                    coordinate (mram) rectangle +({\lenmramx},     {\lenmemy}) node[midway] {MRAM};

		% Data flows.
		\draw[flow]       ($(iram) + (0, {\lenmemhalfy/2})$)          -- +(-\intermem, 0);  % Pipeline ↔ IRAM
		\draw[flow]       ($(wram) + (0, {\lenmemhalfy/2})$)          -- +(-\intermem, 0);  % Pipeline ↔ WRAM
		\draw[flow right] ($(pipe) + ({\lenmemx}, {\lenmemy/2})$)     -- +({2*\intermem+\lenmemx}, 0);  % Pipeline ↔ WRAM

		\draw[flow]       ($(wram) + ({\lenmemx}, {\lenmemhalfy/2})$) -- +(\intermem, 0);  % WRAM ↔ DMA
		\draw[flow left]  ($(iram) + ({\lenmemx}, {\lenmemhalfy/2})$) -- +(\intermem, 0);  % IRAM → DMA

		\draw[flow]       ($(dma)  + ({\lenmemx}, {\lenmemy/2})$)     -- +(\intermem, 0);  % DMA ↔ MRAM

		\draw[flow]       ($({\pad+\lencontrolx}, {\leny-\pad-\lencontroly/2})$)     -- +(\intermem, 0);  % CSI ↔ DDR
		\foreach \i in {\numdpu,...,0}{
			\pgfmathsetmacro{\fade}{int( (\numdpu+1-\i)/(\numdpu+1) * 100 )}
			\def\flowlen{\leny-2*\pad-\lencontroly-\i*\paddpu-\lendpuy+\padmem}
			\draw[flow, draw=black!\fade] ($(dpu\i) + ({\paddpu+\lenmemx/2+\i*\paddpuflow}, {\lendpuy-\padmem})$) -- +($(0, {\flowlen})$);
			\draw[flow, draw=black!\fade] ($(dpu\i) + ({\lendpux-\padmem-\lenmramx+\lenmemx*1.5+\i*\paddpuflow}, {\lendpuy-\padmem})$) -- +($(0, {\flowlen})$);
		}

		% Controllers.
		\draw[mem u] (\pad, {\leny-\pad})                      coordinate (csi) rectangle +({\lencontrolx}, -\lencontroly) node[midway] {Control/Status Interface};
		\draw[mem d] ($(csi) + ({\intermem+\lencontrolx}, 0)$) coordinate (ddr) rectangle +({\lencontrolx}, -\lencontroly) node[midway] {DDR\liningnums{4} Interface};
	\end{tikzpicture}

	\caption{
		The structure of a \ac{PIM} chip.
		The bright components are part of a standard \ac{DDR} package, the dark components are exclusive to \ac{PIM} chips.
	}
	\label{fig:arch:chip}
\end{figure}

A \ac{PIM} chip (\cref{fig:arch:chip}) contains eight \ac{DRAM} banks of \qty{64}{\mebi\byte} each.
These are connected with a regular memory controller through which a host \ac{CPU} can access the memory.
Next to each \ac{DRAM} bank is a \ac{DPU}.
The eight \acp{DPU} are connected with a special control interface which, in turn, is connected with the memory controller.
This allows the host to communicate with the \acp{DPU} but it does not allow \acp{DPU} to access \ac{DRAM} banks other than their own.
Instead, a \ac{DPU} possesses a direct connection to its \ac{DRAM} bank, thus bypassing the memory controller.
Such an access is also called \acfi{DMA} and is handled by the so-called \emph{\ac{DMA} engine}.
It is not possible for a \ac{DPU} and the host to access a \ac{DRAM} bank concurrently.

\Acp{DPU} contain several major and minor memories.
The memory of the \ac{DRAM} bank is also referred to as \acfi{MRAM}.
It is by far the largest memory of a \ac{DPU} and typically holds both the input provided by the host and the output calculated by the \ac{DPU}.
However, the \ac{MRAM} is also the slowest memory, for each access comes with non-negligible latency.

The \acfi{WRAM} is far smaller, comprising only \qty{64}{\kibi\byte}, yet the latency is practically zero.
A typical workflow is, hence, to load input data from the \ac{MRAM} into the \ac{WRAM}, process it, and write the output data back into the \ac{MRAM}.
The \ac{WRAM} also contains the stacks of the tasklets, where their local variables are stored, and global variables which are visible to all tasklets.
Moreover, tasklets can dynamically allocate further memory, which is also located in the \ac{WRAM}.
A \ac{DPU} does not dispose of a multilevel cache hierarchy moving data automatically like a \ac{CPU} does, and it is in the responsibility of the programmer to ensure that critical data are stored in the \ac{WRAM}.
Anyway, there is still a small number of automatically managed registers (see also \cref{sec:prereq:arch:isa}).
The host may access a specific section of the \ac{WRAM} only if the data has been specifically designated for this purpose, and such transfers are slower than transfers with the \ac{MRAM}.

Whilst the \ac{WRAM} holds the data which is processed, the \acfi{IRAM} contains the program (also called \emph{kernel}) which a \ac{DPU} executes.
The \ac{IRAM} has a size of \qty{24}{\kibi\byte} which translates to a maximum of \num{4096} instructions out of which a kernel has to be built.
This memory can be modified only by the host, as the \ac{DPU} can only read it, which is an automated process usually.

Next to these major memories, there is also a \qty{256}{\bit} large \emph{atomic memory} whose bits are accessible in a thread-safe way, allowing for mutual exclusion, barriers, and the like.
Furthermore, there is a \emph{run memory} through which threads can be booted, suspended, and resumed.

\begin{note}
	There are two \ac{DPU} models, v1A and v1B.
	The former runs at \qty{350}{\mega\hertz} and is equipped with 24 threads, whereas the latter runs at \qty{400}{\mega\hertz} and is equipped with 16 threads.
	Additionally, the model v1B can hold \qty{2}{\kibi\byte} of data less in its \ac{MRAM} and 128 instructions less in its \ac{IRAM}, since parts of those are \textquote{reserved for production and quality control purposes.}~\cite[Introduction~-- DPU chip characteristics]{upmemSDK}
	The model used for the measurements of this thesis is v1A.
\end{note}


\subsection{The Instruction Pipeline}
\label{sec:prereq:arch:pipeline}

Instructions are executed using \emph{pipelining}, that is, instructions are divided into several stages which are performed one after another, with each stage taking exactly one cycle.
Once a stage has been completed, the respective transistors are free to process the next instruction even if the previous instruction has not reached the end of the pipeline yet.
The pipeline is \emph{scalar}, meaning there is at most one instruction per stage at any time, and \emph{executes in order}, meaning instructions are statically scheduled and executed in the order as indicated by the compilation.
Threads can have only one scheduled instruction in the pipeline.
However, all threads use the same pipeline, so a nominal throughput of one instruction per cycle is achieved if enough threads are active for all stages of the pipeline to be continuously performed (\cref{fig:arch:pipeline}).
The pipeline consists of fourteen stages, amongst these the fetching of the instruction from the \ac{IRAM}, the reading of the operands from the registers, and the performing of the operation itself while accessing the \ac{WRAM} if needed.
The last three stages can be interleaved, that is performed in parallel, with the first three stages.
Thence, the pipeline length is effectively reduced to eleven, meaning only eleven active threads are needed to exploit the full computing capabilities of a DPU.

\begin{figure}
	\centering
	\begingroup
	\addfontfeatures{RawFeature=+tnum}
	\NewDocumentCommand{\ins}{m}{\textcolor{accentcolor}{\underline{#1}}}
	\setlength{\tabcolsep}{3.17pt}
	\begin{tabular}{ccccccccccccccccccccc}
		\toprule
		Tasklet & \multicolumn{20}{c}{Cycle} \\
		\cline{2-21}
		& 21 & 22 & 23 & 24 & 24 & 25 & 26 & 27 & 28 & 29 & 30 & 31 & 32 & 33 & 34 & 35 & 36 & 37 & 38 & 39 \\
		\midrule

		0 & \lstinline|J| & \lstinline|K| & \lstinline|L|/\kern-1pt\lstinline|A| & \lstinline|M|/\lstinline|B| & \ins{\lstinline|N|}/\lstinline|C| & \lstinline|D| & \lstinline|E| & \lstinline|F| & \lstinline|G| & \lstinline|H| & \lstinline|I| & \lstinline|J| & \lstinline|K| & \lstinline|L|/\kern-1pt\lstinline|A| & \lstinline|M|/\lstinline|B| & \ins{\lstinline|N|}/\lstinline|C| & \lstinline|D| & \lstinline|E| & \lstinline|F| & \lstinline|G| \\

		1 & \lstinline|I| & \lstinline|J| & \lstinline|K| & \lstinline|L|/\kern-1pt\lstinline|A| & \lstinline|M|/\lstinline|B| & \ins{\lstinline|N|}/\lstinline|C| & \lstinline|D| & \lstinline|E| & \lstinline|F| & \lstinline|G| & \lstinline|H| & \lstinline|I| & \lstinline|J| & \lstinline|K| & \lstinline|L|/\kern-1pt\lstinline|A| & \lstinline|M|/\lstinline|B| & \ins{\lstinline|N|}/\lstinline|C| & \lstinline|D| & \lstinline|E| & \lstinline|F| \\

		2 & \lstinline|H| & \lstinline|I| & \lstinline|J| & \lstinline|K| & \lstinline|L|/\kern-1pt\lstinline|A| & \lstinline|M|/\lstinline|B| & \ins{\lstinline|N|}/\lstinline|C| & \lstinline|D| & \lstinline|E| & \lstinline|F| & \lstinline|G| & \lstinline|H| & \lstinline|I| & \lstinline|J| & \lstinline|K| & \lstinline|L|/\kern-1pt\lstinline|A| & \lstinline|M|/\lstinline|B| & \ins{\lstinline|N|}/\lstinline|C| & \lstinline|D| & \lstinline|E| \\

		3 & \lstinline|G| & \lstinline|H| & \lstinline|I| & \lstinline|J| & \lstinline|K| & \lstinline|L|/\kern-1pt\lstinline|A| & \lstinline|M|/\lstinline|B| & \ins{\lstinline|N|}/\lstinline|C| & \lstinline|D| & \lstinline|E| & \lstinline|F| & \lstinline|G| & \lstinline|H| & \lstinline|I| & \lstinline|J| & \lstinline|K| & \lstinline|L|/\kern-1pt\lstinline|A| & \lstinline|M|/\lstinline|B| & \ins{\lstinline|N|}/\lstinline|C| & \lstinline|D| \\
		\bottomrule
	\end{tabular}
	\endgroup

	\caption{
		An excerpt from cycles 21 to 39 of an exemplary pipeline with four threads which were scheduled one cycle apart.
		The fourteen letters \lstinline|A| to \lstinline|N| represent the fourteen stages of an instruction.
		Due to the interleave, every pair of subsequent, final stages \lstinline|N| is eleven cycles apart.
	}
	\label{fig:arch:pipeline}
\end{figure}

Nevertheless, having more than eleven threads active is not detrimental to the throughput, which remains at one instruction per cycle, it only means that individual threads are put into a queue and have to wait for some cycles.
This not only may make some parallel task easier to program, it can result in a performance gain when \acp{DMA} are involved.
\Acp{DMA} are mainly executed by the autonomous \ac{DMA} engine.
Whilst a thread is performing a \ac{DMA}, it is suspended and removed from the pipeline, freeing a slot up.
Therefrom, the employment of more than eleven threads allows to hide \ac{DMA} latency by keeping the pipeline full.

As concluding remark, it shall be mentioned that there are circumstances under which the execution of an instruction takes twelve cycles instead of eleven.
This is related to the identifiers of the registers used, however, the compiler usually manages to avoid these situations.
Hence, one can regard a \ac{DPU} as a \emph{uniform-cost machine} where each instruction takes eleven cycles to complete with the seldom exception of some taking twelve cycles and with the exception of \acp{DMA}.
Counting instructions is, therefore, a valid technique to assess the performance of some piece of code.


\subsection{Instruction Set Architecture}
\label{sec:prereq:arch:isa}

Notwithstanding \acp{DPU} kernels mainly being programmed using the high-level language \langC{}, it is beneficial to take a look at the instruction set architecture to gain a deeper understanding of the inner workings of a \ac{DPU}.
In this thesis, the investigation of the compilation is a recurring theme when identifying optimisation potential.
A \ac{DPU} is a \ac{RISC} with mainly 32-bit instructions \Dash most 64-bit instructions are pieced together from several 32-bit ones, thereby taking more than eleven cycles.
There is no hardware support for multiplication or division, so these are emulated by bitwise instructions, thereby taking even longer.
On top of that, there is no hardware support for floating point arithmetic, requiring costly emulation as well.

Each thread owns several private 32-bit registers, which can be passed as arguments to instructions.
The Registers~\lstinline|r0| to \lstinline|r23| are referred to as \emph{general-purpose registers}, although this may be seen as misnomer as some of these registers do have specific purposes by convention.
For example, before calling a \langC{} function, the Registers~\lstinline|r0| to \lstinline|r7| are filled with the up to eight arguments of the function.
Additionally, when the function has a return value, this value is written to Register~\lstinline|r0| and also to Register~\lstinline|r1| if the return value is \qty{64}{\bit} large.
Even though the Registers~\lstinline|r0| to \lstinline|r21| can be used for different purposes as well, the last two registers have practically exclusive uses:
Register~\lstinline|r22| contains the stack pointer, that is the address of the currently last element in the stack of the respective tasklet.
When a function is called and it needs store data on the stack, it saves the original value of the stack pointer on the stack itself before incrementing the stack pointer, therethrough allocating new memory.
When the function terminates, it loads the original stack pointer value back into Register~\lstinline|r22|, therethrough deallocating memory.
Register~\lstinline|r23| contains the return address, that is the \ac{IRAM} index of the instruction whither to jump back after the termination of a function.
All general-purpose buffers have in common that they can be combined to form 64-bit registers:
Registers~\lstinline|r0| and \lstinline|r1| form the 64-bit Register~\lstinline|d0|, Registers~\lstinline|r2| and \lstinline|r3| the Register~\lstinline|d2|, and so on.
This feature is precisely what is used when returning 64-bit values.

Next to the general-purpose registers, there are also four read-only registers which are shared by all threads:
the Registers~\lstinline|zero| and \lstinline|one| hold, as their names suggest, the constants \(0\) and \(1\), whereas the Registers~\lstinline|lneg| and \lstinline|mneg| hold the least negative and most negative 32-bit values, that is \(-1\) and \(-2^{31}\).
Furthermore, there are the also the private and read-only Registers \lstinline|id|, \lstinline|id2|, \lstinline|id4|, and \lstinline|id8|, which hold the identifier of the respective thread, multiplied by 1, 2, 4, and 8.
Last but not least, there are special registers for the program counter, that is the \ac{IRAM} index of the next instruction to execute, a performance counter used for measuring the time, a carry bit, and, finally, the zero flag.

Instructions follow a 3-operands design, meaning there can be up to three register arguments to an instruction, with the target register usually first.
Next to registers, it is also possible to pass \emph{immediate values}, that is constant values passed directly without a register, and \emph{labels}, which are effectively \ac{IRAM} indices of instructions.
The compilation of a \ac{DPU} kernel is reminiscent of ordinary assembler code.
Some examples:
\begin{itemize}
	\item
	\lstinline|lw r13, r12, -8| loads the 32-bit element which is one byte away from the \ac{WRAM} address stored in Register~\lstinline|r12| into Register~\lstinline|r13|.

	\item
	\lstinline|move r6, 4| stores the immediate value \lstinline|4| in Register~\lstinline|r6|.

	\item
	\lstinline|add r1, r5, r11| takes the 32-bit integers in Registers~\lstinline|r5| and \lstinline|r11|, adds them, stores the result in Register~\lstinline|r1|, and sets the carry bit accordingly.

	\item
	\lstinline|addc r0, r4, r10| performs an addition taking the carry bit into account, allowing to perform one 64-bit addition by invoking two 32-bit instructions.

	\item
	\lstinline|jump r23, .LABEL_FUNC| stores the current program counter in Register~\lstinline|r23| and then jumps to the beginning of a function designated by the label.
\end{itemize}
Of particular use are the instruction to load and store 64-bit large double words, which are some of the few instructions with \enquote{real}\todo{Diese Wortwahl! :(} 64-bit capabilities.
By invoking \lstinline|sd r22, <offset>, d22|, the content of both Registers~\lstinline|r22| and \lstinline|r23| is stored to some position relative to the current stack pointer, whence it can be recovered by invoking \lstinline|ld d22, -<offset>, r22| later on.
Thereby, the bandwidth of the \ac{WRAM} is effectively doubled and the instruction count is reduced.

The capabilities of \ac{DPU} instructions is substantially enhanced by the plethora of \emph{conditions}, of which there are a total of 51.
Conditions are binary flags which are passed as additional arguments to instructions so that they act as either test operation or combo operations.
A \emph{test operation} performs its usual purpose but stores the evaluation of the condition in the target register.
For example, the instruction \lstinline|add r0, r0, -1, pl| takes the content of Register~\lstinline|r0|, decrements it, and checks the condition \lstinline|pl|.
This condition evaluates to true if the result is greater than or equal to zero.
Therefore, Register~\lstinline|r0| will contain the value \lstinline|1| if and only if Register~\lstinline|r0| stored the number \lstinline|1| or greater, and will contain \lstinline|0| otherwise.
A \emph{combo operation} takes a label as yet another argument.
The instruction performs its usual purpose, checks whether the result fulfils the conditions, and performs a jump to a given label if yes.
An example is the instruction \lstinline|add r0, r0, -1, pl, .LABEL_LOOP|, where Register~\lstinline|r0| holds a loop index which get decremented.
Should Register~\lstinline|r0| now hold a value greater or equal to zero, a jump back to the beginning of the loop body marked by the label \lstinline|.LABEL_LOOP| is performed.
Otherwise, the next line of the compilation is executed.
This way, it takes just eleven cycles to update the loop index, check the loop condition, and perform the appropriate action.
Such techniques of saving instructions are especially valuable because \acp{DPU} are incapable of branch prediction.
Although conditions are employed automatically by the compiler for the most part, \cref{sec:mram} includes a manual use of conditions.


\subsection{Programming a Kernel}
\label{sec:prereq:arch:code}

Executing a task on \acp{DPU} requires both a program executed on the host \ac{CPU} and a kernel executed on the \acp{DPU}.
\Acp{DPU} are handled in groups, called \emph{ranks}, of up to 64 \acp{DPU} from the same side of a \ac{DIMM}.
The ranks, in turn, are aggregated in a \emph{\ac{DPU} set}.
A typical course of action is the following:
\begin{enumerate}
	\item
	Start the host program.

	\item
	Write the input to the \ac{MRAM} and/or \ac{WRAM} of all involved \acp{DPU}.

	\item
	Boot the \acp{DPU} and execute the kernel synchronously or asynchronously.

	\item
	Read the output from the \ac{MRAM} and/or \ac{WRAM}.

	\item
	Go back to the third step if needed.
\end{enumerate}
When kernels are executed synchronously, the host cannot access the memory until the execution on all ranks is finished.
Note that data is generally not deleted when a kernel finishes, so subsequent executions can hark back to previous results.
Also, any communication between the host and the \acp{DPU} must be initiated by the host.
The host program can be written in \langC{}, \langCpp{}, \langJava{}, or \langPython{}.
Apart from a few additional functionalities provided by the UPMEM \ac{API} for communication with the \acp{DPU}, the host program is a regular executable.

The kernel has to be written in either \langC{} or assembler, but we will focus on the former.
The software development kit includes a simulator which allows to run kernels on machines without UPMEM \acp{DIMM}.
The entry point for a kernel is the \lstinline|main| function, thence one can proceed as in any \langC{} program.
The \langC{} standard library is only partially available as some compute-intensive functionalities have been removed, for example the entire \lstinline|math| library.
All tasklets execute the same kernel but their control flow can be changed by simply including conditionals on the tasklet identifiers.
Synchronisation between tasklets can be achieved, amongst others, via barriers, mutual exclusion, and semaphores.
Communication between tasklets is achievable by defining global variables in the \ac{WRAM}.

For a performant kernel execution, it is generally recommended to restrict oneself to simple 32-bit logic as much as possible.
Some 64-bit functionalities are executed in eleven cycles, like loads and stores, but most take twice or even thrice as long.
Multiplication and division are emulated through functions operating on bits, so they should be avoided if necessary.
In their stead, addition, subtraction, and bitwise logic should be used.
Also, due to the uniform-cost model, almost any decrease in the count of instructions translates into a performance gain unless secondary effects like register overusage\todo{!} emerge.
Sometimes, one should explicitly save the result of a computation in a variable if the value is supposed to be reused at a later point, since the compiler includes a recalculation elsewise.
It also seems that pointer logic tends to be compiled better than index logic.
Lastly, we noticed during engineering that most of the time, inlining leads to a performance gain as the overhead for function calls is quite heavy.
Even though the call itself is a mere jump taking elven cycles, several registers must be saved and reloaded which takes some time;
for an empty function with two arguments, we determined a call overhead of \qty{144}{\cycle}, which can easily rise with heavier register usage.
This may also explain why turning arguments into global variables nets a performance gain in some cases as well.

The biggest changes to a regular program are in relation to the memory.
Any variable resides in the \ac{WRAM} per default, but creating an \ac{MRAM} variable is as easy as adding the qualifier \lstinline|__mram| to the variable declaration.
Per default, too, any pointer is assumed to point to data in the \ac{WRAM}, which can be changed by adding the qualifier \lstinline|__mram_ptr|.
The compiler correctly identifies confusion between pointers of different address spaces.

Local \ac{WRAM} variables are created on the stack of the respective tasklet, and the stack sizes can be changed on a case-by-case basis.
Nonetheless, it is possible for tasklets to dynamically allocate more space on the \ac{WRAM} via an incremental allocator similar to the standard \langC{} function \lstinline|malloc| but also via a buddy allocator or a block allocator.
Although this is called \emph{heap allocation}, the name is misleading.
The compiler organises the \ac{WRAM} such that all tasklet stacks and anything else statically allocated on the \ac{WRAM} is in the front, so that the free memory comprises a contiguous block in the back of the \ac{WRAM}.
Then, the so-called \emph{heap pointer} is set to the beginning of the free block.
When memory is allocated on the heap, the heap pointer is sufficiently incremented to mark the space as reserved.
Afterwards, the original position of the heap pointer is returned to the allocating tasklet, perhaps after some alignment.
In other words:
the heap memory is simply a stack memory shared by all tasklets.
Indeed, a \ac{DPU} lacks an equivalent to the standard \langC{} function \lstinline|free| to deallocate heap memory.
The only possibility is to reset the entire heap by setting the heap pointer back to its initial position.
There is also no model of ownership, so tasklets can write to any memory address, including the stack and heap memory of other tasklets.
In summary, one shall pay heed when structuring the scarce \ac{WRAM}.
This is also subject of \cref{sec:mram:triple}.

As hinted before, transferring data between the \ac{MRAM} and the \ac{WRAM} is in the responsibility of the programmer.
When only single elements are to be accessed, \ac{MRAM} data can be handled like normal variables.
For example, \lstinline|var = arr[i]| is valid code no matter whether the array \lstinline|arr|, the variable \lstinline|var|, or the index \lstinline|i| have been declared to reside in the WRAM or the MRAM.
However, this still constitutes one or more \acp{DMA} on each use, and each \ac{DMA} comes at a cost.
According to measurements~\cite{mutlu2022Benchmarking}, reading from the \ac{MRAM} has an overhead of \qty{77}{\cycle}, whilst writing to the \ac{MRAM} has an overhead of \qty{61}{\cycle}.
The transfer of each byte costs a further \qty{0.5}{\cycle}.
This means that \acp{DMA} for about \qty{128}{\byte} of data or less are dominated by the overhead.
Therefore, it recommended to move large blocks of \ac{MRAM} data into the \ac{WRAM}, perform calculations there, and move the modified data block back to the \ac{MRAM}.
This way, the overhead is mitigated.
Please note that, like for \ac{WRAM} data, the time to access \ac{MRAM} data is independent of the exact location \Dash only the memory type matters.

To perform such blockwise moves, one calls the \langC{} function \lstinline|mram_read| and \lstinline|mram_write|, which take a source address, a target address, and the number of bytes to use transfer.
There are, however, several limitations.
\begin{itemize}
	\item
	The \ac{WRAM} address must be aligned on 8 bytes.
	This can be ensured automatically by adding appropriate qualifiers to stack variables or by using heap memory.

	\item
	The \ac{MRAM} address must be aligned on 8 bytes.
	No special functionality exists to this end;
	it is up to the programmer to organise the \ac{MRAM} with this limitation in mind and to resort to \acp{DMA} to single elements if such an alignment is not given.

	\item
	The number of transferred bytes must be at least 8, at most 2048, and a multiple of 8.
\end{itemize}
Failing to fulfil these constraints can result in missing or corrupt data.
The \ac{DMA} engine works sequentially, meaning data for only tasklet can be transferred at a time.
If multiple tasklets call \lstinline|mram_read| or \lstinline|mram_write|, some of them will be suspended for longer as they wait for the other \acp{DMA} to finish.
If \acp{DMA} are very frequent, having many active tasklets is especially important to keep the pipeline full.

\todo[inline]{Schere Dich nicht darum, Kapitel 2 zu lesen. Es ist genauso unfertig wie beim letzten Male.}

