\subsection{Investigation of the Compilation}
\label{sec:tasklet:quick:compilation}

\def\quickpivots{Last,Median,Random,Median_of_random}
\expandafter\pgfplotsinvokeforeach\expandafter{\quickpivots}{
	\pgfplotstablereadnamed{data/quick/matrix/iterative/#1/shorter/uint32/uniform.txt}{tableQuickMatrixIt#1Shorter_32}
	\pgfplotstablereadnamed{data/quick/matrix/iterative/#1/left/uint32/uniform.txt}{tableQuickMatrixIt#1Left_32}
	\pgfplotstablereadnamed{data/quick/matrix/iterative/#1/right/uint32/uniform.txt}{tableQuickMatrixIt#1Right_32}

	\pgfplotstablereadnamed{data/quick/matrix/recursive/#1/shorter/uint32/uniform.txt}{tableQuickMatrixRec#1Shorter_32}
	\pgfplotstablereadnamed{data/quick/matrix/recursive/#1/left/uint32/uniform.txt}{tableQuickMatrixRec#1Left_32}
	\pgfplotstablereadnamed{data/quick/matrix/recursive/#1/right/uint32/uniform.txt}{tableQuickMatrixRec#1Right_32}
}
\def\quickalgos{Normal,TrivInThresh,NoTrivial,ThreshThenTriv,TrivialBC,ThreshBC,ThreshTrivBC,OneInsertion}

The quality of the compilation is highly erratic to such an extent that \Dash even with the same pivots \Dash one implementation variant may see a speedup of \num{1.33} over another one where none would be expected.
There are small details influencing the runtime.
For instance, storing the value of the pivot in a dedicated variable instead of accessing it through a pointer changes the runtime by a few percentage points in both directions, depending on the rest of implementation.
But as hinted at in \cref{sec:tasklet:quick:aspects}, there are four major parameters to examine:
\lcnamecref{imp:normal} of the base cases, recursion/iteration, pivot selection, and partition prioritisation.
Before the findings are discussed, the first parameter shall be explained in more depth.

Besides falling back to \IS{} if 18 elements or fewer remain (\enquote{treshold undercut}), another base case is imaginable, namely a termination if at most one element remains (\enquote{trivial length}).
Realistically speaking, it should not be needed to check for trivial lengths because even though it would take just one additional instruction, such tiny partitions are rare, and \IS{} would terminate after a few instructions anyway.
Nonetheless, its inclusion or exclusion can have significant impact as does the position at which the checks for the two base cases happen.
The following \lcnamecrefs{imp:normal} were benchmarked:
\begin{enumerate}[label=(\liningnums{\arabic*})]
	\item\label[implementation]{imp:normal}
	If the length is trivial, terminate immediately.
	Otherwise, if the threshold is undercut, sort with \IS{} and terminate.
	If neither, partition the input and invoke \QS{} on both partitions.
%	\textcolor{red}{[Normal]}

	\item\label[implementation]{imp:triviality_within_threshold}
	If the threshold is undercut, check if the length is trivial and either terminate immediately or sort with \IS{} and terminate afterwards.
	Otherwise, partition the input and invoke \QS{} on both partitions.
%	\textcolor{red}{[TrivInThresh]}
	\begin{itemize}
		\item
		This \lcnamecref{imp:triviality_within_threshold} significantly reduces the number of checks on triviality.
	\end{itemize}

	\item\label[implementation]{imp:no_triviality}
	If the threshold is undercut, sort with \IS{} and terminate.
	Otherwise, partition the input and invoke \QS{} on both partitions.
%	\textcolor{red}{[NoTrivial]}
	\begin{itemize}
		\item
		This \lcnamecref{imp:no_triviality} forgoes the check on triviality completely at the cost of some unneeded invocations of \IS{}.
	\end{itemize}

	\item\label[implementation]{imp:threshold_then_triviality}
	If the threshold is undercut, sort with \IS{} and terminate.
	Otherwise, if the length is trivial, terminate immediately.
	If neither, partition the input and invoke \QS{} on both partitions.
%	\textcolor{red}{[ThreshThenTriv]}
	\begin{itemize}
		\item
		This \lcnamecref{imp:threshold_then_triviality} is nonsensical from a logical point of view, since the length cannot be trivial if the threshold is not undercut.
		However, it gives the compiler an explicit guarantee that the loop for partitioning does not end immediately, eliminating a possible reason for the varying quality of the compilation.
	\end{itemize}

	\item\label[implementation]{imp:triviality_before_call}
	If the threshold is undercut, sort with \IS{} and terminate.
	Otherwise, partition the input.
	Then, check for either partition if its length is trivial and invoke \QS{} on it if not.
%	\textcolor{red}{[TrivialBC]}
	\begin{itemize}
		\item
		This \lcnamecref{imp:triviality_before_call}, as well as the next two, eliminates some unneeded invocations of \QS{}.
%		In the recursive case, these \lcnamecrefs{imp:triviality_before_call} lose the property of being tail-recursive.
	\end{itemize}

	\item\label[implementation]{imp:threshold_before_call}
	Partition the input.
	Check for either partition if the threshold is undercut and invoke either \IS{} or \QS{} on it.
%	\textcolor{red}{[ThreshBC]}

	\item\label[implementation]{imp:threshold_and_triviality_before_call}
	Partition the input.
	Check for either partition if its length is trivial or if the threshold is undercut and invoke either \IS{}, \QS{}, or nothing on it.
%	\textcolor{red}{[ThreshTrivBC]}

	\item\label[implementation]{imp:one_insertion}
	If the threshold is undercut, terminate immediately.
	Otherwise, partition the input and invoke \QS{} on both partitions.
	After all \QS*{} are done, sort the whole input array with \IS{}.
%	\textcolor{red}{[OneInsertion]}
	\begin{itemize}
		\item
		This \lcnamecref{imp:one_insertion} always does one pass of \IS{}.
		For example, the other \lcnamecrefs{imp:normal} invoke \IS{} roughly 91 times on 1024 uniformly distributed elements.
	\end{itemize}
\end{enumerate}
The performances of all benchmarked implementation on 32-bit integers are shown in \cref{fig:quick:implementations}.
The measurements were conducted on uniform input distributions, so the deterministic pivots are, in expectation, of the same quality as the random ones.

\pgfplotsset{
	quick matrix/.style={
		height=2.519cm,
		horizontal sep for naught,
		vertical sep for naught,
		adaptive group=3 by 4,
		groupplot ylabel={Cycles / \((n \lb n)\)},
		x from 16 to 1024 minor,
		ymin=55,
		ymax=80,
		/tikz/mark repeat=2,
	},
}

\begin{figure}[p]
	\captionsetup[subfigure]{aboveskip=0mm, belowskip=1mm}
	\begin{subfigure}{\textwidth}
		\tikzsetnextfilename{quick_implementations_rec}
		\begin{tikzpicture}[plot]
			\begin{groupplot}[quick matrix]
				\nextgroupplot[title/.add={}{Last Element}, xticklabels={}]
				\pgfplotsset{legend to name=leg:quick:implementations, legend entries={\ref{imp:normal}, \ref{imp:triviality_within_threshold}, \ref{imp:no_triviality}, \ref{imp:threshold_then_triviality}, \ref{imp:triviality_before_call}, \ref{imp:threshold_before_call}, \ref{imp:threshold_and_triviality_before_call}, \ref{imp:one_insertion}}}
				\expandafter\pgfplotsinvokeforeach\expandafter{\quickalgos}{
					\plotpernlogn{#1}{tableQuickMatrixRecLastLeft_32}
				}
				\nextgroupplot[title/.add={}{Determ.\ Median}, xticklabels={}, yticklabels={}]
				\expandafter\pgfplotsinvokeforeach\expandafter{\quickalgos}{
					\plotpernlogn{#1}{tableQuickMatrixRecMedianLeft_32}
				}
				\nextgroupplot[title/.add={}{Random Element}, xticklabels={}, yticklabels={}]
				\expandafter\pgfplotsinvokeforeach\expandafter{\quickalgos}{
					\plotpernlogn{#1}{tableQuickMatrixRecRandomLeft_32}
				}
				\nextgroupplot[title/.add={}{Random Median}, xticklabels={}, yticklabel pos=right]
				\expandafter\pgfplotsinvokeforeach\expandafter{\quickalgos}{
					\plotpernlogn{#1}{tableQuickMatrixRecMedian_of_randomLeft_32}
				}
				%
				\nextgroupplot[xticklabels={}]
				\expandafter\pgfplotsinvokeforeach\expandafter{\quickalgos}{
					\plotpernlogn{#1}{tableQuickMatrixRecLastRight_32}
				}
				\nextgroupplot[xticklabels={}, yticklabels={}]
				\expandafter\pgfplotsinvokeforeach\expandafter{\quickalgos}{
					\plotpernlogn{#1}{tableQuickMatrixRecMedianRight_32}
				}
				\nextgroupplot[xticklabels={}, yticklabels={}]
				\expandafter\pgfplotsinvokeforeach\expandafter{\quickalgos}{
					\plotpernlogn{#1}{tableQuickMatrixRecRandomRight_32}
				}
				\nextgroupplot[xticklabels={}, yticklabel pos=right]
				\expandafter\pgfplotsinvokeforeach\expandafter{\quickalgos}{
					\plotpernlogn{#1}{tableQuickMatrixRecMedian_of_randomRight_32}
				}
				%
				\nextgroupplot
				\expandafter\pgfplotsinvokeforeach\expandafter{\quickalgos}{
					\plotpernlogn{#1}{tableQuickMatrixRecLastShorter_32}
				}
				\nextgroupplot[yticklabels={}]
				\expandafter\pgfplotsinvokeforeach\expandafter{\quickalgos}{
					\plotpernlogn{#1}{tableQuickMatrixRecMedianShorter_32}
				}
				\nextgroupplot[yticklabels={}]
				\expandafter\pgfplotsinvokeforeach\expandafter{\quickalgos}{
					\plotpernlogn{#1}{tableQuickMatrixRecRandomShorter_32}
				}
				\nextgroupplot[yticklabel pos=right]
				\expandafter\pgfplotsinvokeforeach\expandafter{\quickalgos}{
					\plotpernlogn{#1}{tableQuickMatrixRecMedian_of_randomShorter_32}
				}
			\end{groupplot}
		\end{tikzpicture}
		\caption{
			Recursive Approach
		}
	\end{subfigure}

	\begin{subfigure}{\textwidth}
		\tikzsetnextfilename{quick_implementations_it}
		\begin{tikzpicture}[plot]
			\begin{groupplot}[quick matrix]
				\nextgroupplot[title/.add={}{Last Element}, xticklabels={}]
				\expandafter\pgfplotsinvokeforeach\expandafter{\quickalgos}{
					\plotpernlogn{#1}{tableQuickMatrixItLastLeft_32}
				}
				\nextgroupplot[title/.add={}{Determ.\ Median}, xticklabels={}, yticklabels={}]
				\expandafter\pgfplotsinvokeforeach\expandafter{\quickalgos}{
					\plotpernlogn{#1}{tableQuickMatrixItMedianLeft_32}
				}
				\nextgroupplot[title/.add={}{Random Element}, xticklabels={}, yticklabels={}]
				\expandafter\pgfplotsinvokeforeach\expandafter{\quickalgos}{
					\plotpernlogn{#1}{tableQuickMatrixItRandomLeft_32}
				}
				\nextgroupplot[title/.add={}{Random Median}, xticklabels={}, yticklabel pos=right]
				\expandafter\pgfplotsinvokeforeach\expandafter{\quickalgos}{
					\plotpernlogn{#1}{tableQuickMatrixItMedian_of_randomLeft_32}
				}
				%
				\nextgroupplot[xticklabels={}]
				\expandafter\pgfplotsinvokeforeach\expandafter{\quickalgos}{
					\plotpernlogn{#1}{tableQuickMatrixItLastRight_32}
				}
				\nextgroupplot[xticklabels={}, yticklabels={}]
				\expandafter\pgfplotsinvokeforeach\expandafter{\quickalgos}{
					\plotpernlogn{#1}{tableQuickMatrixItMedianRight_32}
				}
				\nextgroupplot[xticklabels={}, yticklabels={}]
				\expandafter\pgfplotsinvokeforeach\expandafter{\quickalgos}{
					\plotpernlogn{#1}{tableQuickMatrixItRandomRight_32}
				}
				\nextgroupplot[xticklabels={}, yticklabel pos=right]
				\expandafter\pgfplotsinvokeforeach\expandafter{\quickalgos}{
					\plotpernlogn{#1}{tableQuickMatrixItMedian_of_randomRight_32}
				}
				%
				\nextgroupplot
				\expandafter\pgfplotsinvokeforeach\expandafter{\quickalgos}{
					\plotpernlogn{#1}{tableQuickMatrixItLastShorter_32}
				}
				\nextgroupplot[yticklabels={}]
				\expandafter\pgfplotsinvokeforeach\expandafter{\quickalgos}{
					\plotpernlogn{#1}{tableQuickMatrixItMedianShorter_32}
				}
				\nextgroupplot[yticklabels={}]
				\expandafter\pgfplotsinvokeforeach\expandafter{\quickalgos}{
					\plotpernlogn{#1}{tableQuickMatrixItRandomShorter_32}
				}
				\nextgroupplot[yticklabel pos=right]
				\expandafter\pgfplotsinvokeforeach\expandafter{\quickalgos}{
					\plotpernlogn{#1}{tableQuickMatrixItMedian_of_randomShorter_32}
				}
			\end{groupplot}
		\end{tikzpicture}
		\caption{
			Iterative Approach
		}
	\end{subfigure}

	\tikzexternaldisable\hfil\pgfplotslegendfromname{leg:quick:implementations}\hfil\tikzexternalenable
	\caption{
		Mean runtimes of \QS*{} using \crefrange{imp:normal}{imp:one_insertion} and different pivot selection methods on uniformly distributed 32-bit integers.
		Left-hand partitions are prio\-ri\-tised in the first rows, right-hand ones in the second rows, and shorter ones in the third~rows.
	}
	\label{fig:quick:implementations}
\end{figure}

Even when ignoring the differences between specific \lcnamecrefs{imp:normal} for now, the high fluctuations between the plots become immediately apparent.
Plots within the same column share the same method to select pivots, plots within the same row share the same prioritisation of partitions.
In general, it would be expected that plots within the same column are fairly similar, yet prioritising the shorter partition is almost universally associated with an increase in runtime.
When focussing on the top-performing \lcnamecrefs{imp:normal}, the speedup can fall below \num{1} down to \num{0.87}.
There is no clear trend between the consistent prioritisations of either side, although the difference can be huge in individual cases as well.
However, recursive implementations are more susceptible to the partition prioritisation than iterative ones.

The correlation of recursive and iterative performance is weak.
On the one hand, there is, for example, \cref{imp:triviality_before_call} with deterministic medians and prioritisation of shorter partitions where the runtimes are essentially the same.
On the other hand, there is \cref{imp:one_insertion} with deterministic medians and prioritisation of right-hand partitions where recursion is slower by more than a third.
All in all, iterative implementations usually perform better, though, especially when focussing on the top-performers of each pivot selection method.

The ranking of the different \lcnamecrefs{imp:normal} is rather incoherent.
\Cref{imp:triviality_before_call}, which does not call \QS{} on trivial partitions, fares the best out of all \lcnamecrefs{imp:normal} decidedly, being amongst the top performers across all benchmarked implementations.
\Cref{imp:threshold_before_call,imp:threshold_and_triviality_before_call}, which call \QS{} even less often than \cref{imp:triviality_before_call}, are the polar opposite and bring up the rear of the ranking every single time.
Recursive implementations of \cref{imp:triviality_within_threshold}, where triviality is checked for only if the threshold is undercut, are often worse than recursive implementations of \cref{imp:normal}, where triviality is always checked for, whilst it is the other way around for iterative implementations.
Interestingly, for all investigated implementations, the compiler turns tail-recursive calls into jumps back to the beginning of the function, no matter how convoluted the logic surrounding the call is.

These observations, however, only apply to 32-bit integers.
\Cref{fig:quick:implementations_64} shows the same measurements for 64-bit integers.
Whilst the general trend for pivots, partition prioritisation, and recursion/iteration hold true, the rankings are vastly different.
\Cref{imp:triviality_before_call} is not undisputedly the best anymore.
\Cref{imp:threshold_before_call,imp:threshold_and_triviality_before_call} switch back and forth between being the worst and the best \lcnamecrefs{imp:normal}.
Even the nonsensical \cref{imp:threshold_then_triviality} manages to take the lead in a few cases.
Most notably, the two top-performing implementations using deterministic and random medians, respectively, are actually recursive.
Luckily, both use \cref{imp:triviality_before_call} so we decided to make the only difference between the default configurations of 32-bit and 64-bit \QS{} be the usage of iteration (32-bit) and recursion (64-bit).

What is causing these huge disparities?
There is a great variety in the compilations but some of the common occurrences are \dots{}
\begin{itemize}
	\item
	\dots{} one instruction more before (re-)starting to move the pointers, \dots{}

	\item
	\dots{} one instruction more while moving the left pointer by one element, \dots{}

	\item
	\dots{} one instruction more after the left pointer has stopped, \dots{}

	\item
	\dots{} more stores and loads when entering and exiting the function.
\end{itemize}
This focus on the left pointer \lstinline|i| is partially explainable by it being used to calculate the final position of the pivot and, thus, the inner boundaries of both new partitions.
As explained earlier under \enquote{\nameref{sec:tasklet:quick:imp:sentinels}}, the right pointer \lstinline|j| stops ultimately either on the left pointer \lstinline|i| directly (\lstinline|j == i|) or on the element behind it (\lstinline|j == (i - 1)|).
Since the pivot is moved to address \lstinline|i|, the inner boundaries of the partitions are \lstinline|i - 1| and \lstinline|i + 1|.
Implementations using the right pointer alone or both of them to calculate the boundaries were spot-checked to see whether it alleviates the problems, but the results were mixed:
from betterment over indifference to worsening, everything was observable.
