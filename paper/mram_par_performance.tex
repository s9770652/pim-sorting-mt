\subsection{Evaluation of the Performance}
\label{sec:mram:par:performance}

\expandafter\pgfplotsinvokeforeach\expandafter{\alldists}{
	% 32-bit | Instabil
	\pgfplotstablenewnamed[create on use/n/.style={}, create on use/µ_MergePar/.style={}, create on use/σ_MergePar/.style={}, columns={n,µ_MergePar,σ_MergePar}]{0}{tableMergeParUnstable_32#1}
	\pgfplotstablevertcatnamed{tableMergeParUnstable_32#1}{data/merge_par/NR_TASKLETS=1/STABLE=false/uint32/#1.txt}
	\pgfplotstablevertcatnamed{tableMergeParUnstable_32#1}{data/merge_par/NR_TASKLETS=2/STABLE=false/uint32/#1.txt}
	\pgfplotstablevertcatnamed{tableMergeParUnstable_32#1}{data/merge_par/NR_TASKLETS=4/STABLE=false/uint32/#1.txt}
	\pgfplotstablevertcatnamed{tableMergeParUnstable_32#1}{data/merge_par/NR_TASKLETS=8/STABLE=false/uint32/#1.txt}
	\pgfplotstablevertcatnamed{tableMergeParUnstable_32#1}{data/merge_par/NR_TASKLETS=16/STABLE=false/uint32/#1.txt}

	% 64-bit | Instabil
	\pgfplotstablenewnamed[create on use/n/.style={}, create on use/µ_MergePar/.style={}, create on use/σ_MergePar/.style={}, columns={n,µ_MergePar,σ_MergePar}]{0}{tableMergeParUnstable_64#1}
	\pgfplotstablevertcatnamed{tableMergeParUnstable_64#1}{data/merge_par/NR_TASKLETS=1/STABLE=false/uint64/#1.txt}
	\pgfplotstablevertcatnamed{tableMergeParUnstable_64#1}{data/merge_par/NR_TASKLETS=2/STABLE=false/uint64/#1.txt}
	\pgfplotstablevertcatnamed{tableMergeParUnstable_64#1}{data/merge_par/NR_TASKLETS=4/STABLE=false/uint64/#1.txt}
	\pgfplotstablevertcatnamed{tableMergeParUnstable_64#1}{data/merge_par/NR_TASKLETS=8/STABLE=false/uint64/#1.txt}
	\pgfplotstablevertcatnamed{tableMergeParUnstable_64#1}{data/merge_par/NR_TASKLETS=16/STABLE=false/uint64/#1.txt}
}

\NewDocumentCommand{\speeduppar}{m}{
	\pgfplotstablegetelem{0}{µ_MergePar}\of#1
	\pgfmathsetmacro{\messlatte}{\pgfplotsretval}
	\addplot+ table[
		create on use/tasklets/.style={create col/set list={1,2,4,8,16}},
		create on use/factor/.style={create col/expr={\messlatte / \thisrow{µ_MergePar}}},
		x=tasklets, y=factor
	] {#1};
}

\begin{figure}
	\tikzsetnextfilename{par_speedup}
	\begin{tikzpicture}[plot]
		\begin{groupplot}[
			adaptive group=1 by 2,
			groupplot xlabel={Tasklets},
			groupplot ylabel={Parallel Speedup},
			xtick={1,2,4,8,16},
			xmode=log,
			ymin=0,
			ymax=12,
			ytick distance=2,
		]
			\nextgroupplot[title/.add={}{32-bit}]
			\pgfplotsset{legend to name=leg:par:speedup, legend entries={Sorted, Reverse S., Almost S., Zero-One, Uniform, Zipf's}}
			\addplot[forget plot, very nearly transparent] coordinates {(1,1)(2,2)(3,3)(4,4)(5,5)(6,6)(7,7)(8,8)(9,9)(10,10)(11,11)(16,11)};
			\expandafter\pgfplotsinvokeforeach\expandafter{\alldists}{
				\expandafter\speeduppar\expandafter{\csname tableMergeParUnstable_32#1\endcsname}
			}
			%
			\nextgroupplot[title/.add={}{64-bit}]
			\addplot[forget plot, very nearly transparent] coordinates {(1,1)(2,2)(3,3)(4,4)(5,5)(6,6)(7,7)(8,8)(9,9)(10,10)(11,11)(16,11)};
			\expandafter\pgfplotsinvokeforeach\expandafter{\alldists}{
				\expandafter\speeduppar\expandafter{\csname tableMergeParUnstable_64#1\endcsname}
			}
		\end{groupplot}
	\end{tikzpicture}

	\tikzexternaldisable\hfil\pgfplotslegendfromname{leg:par:speedup}\hfil\tikzexternalenable
	\caption{
		Mean parallel speedups of \MS{} on all benchmarked input distributions and data types with \qty{32}{\mebi\byte} of data.
		The grey line indicates an ideal, linear speedup which is capped at~11.
	}
	\label{fig:par:speedup}
\end{figure}

The parallel speedup of the \MS{} on \qty{32}{\mebi\byte} of data is shown in \cref{fig:par:speedup}.
An ideal parallel speedup would be linear in the number of tasklets but capped at 11 or, rather, slightly above because of \ac{DMA} latency hiding.
For inputs with a uniform or Zipf's distribution, the measured speedup is very close to the optimum, reaching values above 10 for both 32-bit and 64-bit integers.
This is owed to workloads tending to be balanced naturally and tasklets being removed from the pipeline once they are finished.
For all other inputs, the parallel speedup is approximately between 7 and 9 with 32-bit integers and between~6 and~8 with 64-bit integers \Dash a consequence of workloads becoming more unbalanced.
This shall be illustrated by sorted inputs:
In the last round, two runs remain.
When Tasklet 0 performs the first partitioning step, the pivot divides the longer run into two equally long halves.
However, the pivot is strictly less or greater than any element in the shorter run, meaning the shorter run keeps its length of about \(n/2\) many elements as the division point is at one of its ends.
Such unbalanced divisions carry on to further partitioning steps.
When the number of tasklets is sixteen, the ratio between the least and the greatest number of assigned elements in the last round of parallel merging is about \num{2.3} for zero-one inputs and \num{4} for the three kinds of sorted inputs.

\Cref{fig:par:phases} shows the wall-clock times.
The measurements are subdivided into the three phases of the parallel \MS{}, that is the sequential \ac{WRAM} phase, the sequential \ac{MRAM} phase, and the parallel \ac{MRAM} phase.
On the one hand, most results are unsurprising.
The sorted, reverse sorted, and zero-one input distribution are sorted quickly as they lead to short first and second phases, whilst the uniform and Zipf's input distribution make these two phases last longer.
The third phase always takes approximately the same amount of time, implying that workload imbalances are cancelled out by earlier flushes, although the effect is weaker for 64-bit integers where computation is more costly.
%The greater parallel speedup of the uniform and Zipf's input distribution shines through by the third phase being short compared to the relatively long first and second phase.

Almost sorted inputs, on the other hand, are clear outsiders because of the remarkably long third phase, making them the worst case amongst all benchmarked ones.
Despite the long runtime, the parallel speedup is about the same as for sorted and reverse sorted inputs.
Indeed, they all share equal or nearly equal workload imbalances.
The explanation for the third phase being so long is the same as the one given in \cref{sec:mram:merge:performance} with respect to the sequential \MS{}:
A few great elements placed in a run of mostly little elements delays flushes, and the longer the runs become, the likelier it is for a run to contain an overly great element.

\pgfplotstablenew[create on use/n/.style={}, create on use/µ_MergePar/.style={}, create on use/σ_MergePar/.style={}, columns={n,µ_MergePar,σ_MergePar}]{0}{\tableMergeParFirst}
\expandafter\pgfplotsinvokeforeach\expandafter{\alldists}{
	\pgfplotstablevertcat{\tableMergeParFirst}{data/merge_par/phase_1/STABLE=false/uint32/#1.txt}
}
\pgfplotstablenew[create on use/n/.style={}, create on use/µ_MergeFS/.style={}, create on use/σ_MergeFS/.style={}, columns={n,µ_MergeFS,σ_MergeFS}]{0}{\tableMergeParSec}
\expandafter\pgfplotsinvokeforeach\expandafter{\alldists}{
	\pgfplotstablevertcat{\tableMergeParSec}{data/merge_mram/NR_TASKLETS=16/CACHE_SIZE=1024/SEQREAD_CACHE_SIZE=512/uint32/#1.txt}
}
\pgfplotstablenew[create on use/n/.style={}, create on use/µ_MergePar/.style={}, create on use/σ_MergePar/.style={}, columns={n,µ_MergePar,σ_MergePar}]{0}{\tableMergeParThird}
\expandafter\pgfplotsinvokeforeach\expandafter{\alldists}{
	\pgfplotstablevertcat{\tableMergeParThird}{data/merge_par/NR_TASKLETS=16/STABLE=false/uint32/#1.txt}
}
\pgfplotstablenew[
	create on use/dist/.style={create col/set list={Sorted,Reverse S.,Almost S.,Zero-One,Uniform,Zipf's}},
	create on use/first/.style={create col/copy column from table={\tableMergeParFirst}{µ_MergePar}},
	create on use/second/.style={create col/copy column from table={\tableMergeParSec}{µ_MergeFS}},
	create on use/third/.style={create col/copy column from table={\tableMergeParThird}{µ_MergePar}},
	columns={dist,first,second,third},
]{6}{\tableMergeParUnstableXxxii}

\pgfplotstablenew[create on use/n/.style={}, create on use/µ_MergePar/.style={}, create on use/σ_MergePar/.style={}, columns={n,µ_MergePar,σ_MergePar}]{0}{\tableMergeParFirst}
\expandafter\pgfplotsinvokeforeach\expandafter{\alldists}{
	\pgfplotstablevertcat{\tableMergeParFirst}{data/merge_par/phase_1/STABLE=false/uint64/#1.txt}
}
\pgfplotstablenew[create on use/n/.style={}, create on use/µ_MergeFS/.style={}, create on use/σ_MergeFS/.style={}, columns={n,µ_MergeFS,σ_MergeFS}]{0}{\tableMergeParSec}
\expandafter\pgfplotsinvokeforeach\expandafter{\alldists}{
	\pgfplotstablevertcat{\tableMergeParSec}{data/merge_mram/NR_TASKLETS=16/CACHE_SIZE=1024/SEQREAD_CACHE_SIZE=512/uint64/#1.txt}
}
\pgfplotstablenew[create on use/n/.style={}, create on use/µ_MergePar/.style={}, create on use/σ_MergePar/.style={}, columns={n,µ_MergePar,σ_MergePar}]{0}{\tableMergeParThird}
\expandafter\pgfplotsinvokeforeach\expandafter{\alldists}{
	\pgfplotstablevertcat{\tableMergeParThird}{data/merge_par/NR_TASKLETS=16/STABLE=false/uint64/#1.txt}
}
\pgfplotstablenew[
	create on use/dist/.style={create col/set list={Sorted,Reverse S.,Almost S.,Zero-One,Uniform,Zipf's}},
	create on use/first/.style={create col/copy column from table={\tableMergeParFirst}{µ_MergePar}},
	create on use/second/.style={create col/copy column from table={\tableMergeParSec}{µ_MergeFS}},
	create on use/third/.style={create col/copy column from table={\tableMergeParThird}{µ_MergePar}},
	columns={dist,first,second,third},
]{6}{\tableMergeParUnstableLxiv}

\begin{figure}
	\tikzsetnextfilename{par_phases}
	\begin{tikzpicture}[plot]
		\begin{groupplot}[
			adaptive group=1 by 2,
			barplot,
			ybar=-\pgfkeysvalueof{/pgf/bar width},
%			yticklabel style={/pgf/number format/.cd, precision=1, fixed, zerofill},  % Cool, it is bugged out, so I have to place all ticks and labels manually…
		]
			\nextgroupplot[ymax=1.5e9, ytick distance=5e8, yticklabels={,\(0.0\), \(0.5\), \(1.0\), \(1.5\)}, title/.add={}{32-bit}]
			\pgfplotsset{legend to name=leg:par:phases, legend entries={Third Phase,Second Phase,First Phase}, legend reversed}
			\pgfplotsset{cycle list shift=2}
			\addplot+ table[x=dist, y=third] {\tableMergeParUnstableXxxii};
			\pgfplotsset{cycle list shift=0}
			\addplot+ table[x=dist, y=second] {\tableMergeParUnstableXxxii};
			\pgfplotsset{cycle list shift=-2}
			\addplot+ table[x=dist, y=first] {\tableMergeParUnstableXxxii};
			%
			\nextgroupplot[ymax=1e9, yticklabels={,\(0.0\), \(0.2\), \(0.4\), \(0.6\), \(0.8\), \(1.0\)}, title/.add={}{64-bit}]
			\pgfplotsset{cycle list shift=2}
			\addplot+ table[x=dist, y=third] {\tableMergeParUnstableLxiv};
			\pgfplotsset{cycle list shift=0}
			\addplot+ table[x=dist, y=second] {\tableMergeParUnstableLxiv};
			\pgfplotsset{cycle list shift=-2}
			\addplot+ table[x=dist, y=first] {\tableMergeParUnstableLxiv};
		\end{groupplot}
	\end{tikzpicture}

	\tikzexternaldisable\hfil\pgfplotslegendfromname{leg:par:phases}\hfil\tikzexternalenable
	\caption{
		Mean wall-clock times of the parallel \MS{}, broken down into its three phases, on all benchmarked input distributions and data types with \qty{32}{\mebi\byte} of data.
		The first phase comprises sequential sorting in the \ac{WRAM}, the second one sequential sorting in the \ac{MRAM}, and the third one parallel sorting in the \ac{MRAM}.
	}
	\label{fig:par:phases}
\end{figure}

%The measurements for the stable parallel \MS{} are shown in \cref{fig:par:speedup_unstable,fig:par:phases_unstable}.
%For the uniform and Zipf's input distributions, the wall-clock time is only somewhat increased.
%Because the \ac{WRAM} \MS{} is hardly slower than the \ac{WRAM} \QS{}, the first phase takes hardly longer.
%The second phase is not altered algorithmically, but it makes up the majority of the total runtime increase due to its starting runs being only half as long as in the unstable case.
%For uniformly distributed input, the third phase takes just as long as before since duplicates are virtually non-existent.
%For inputs following Zipf's law, the third phase takes only slightly longer, as duplicates start to accumulate significantly only in later rounds.
%Since the first two phases affect also the sequentially executed \MS{} and the third phase remains largely the same, the speedups drop slightly but remain above~10.
%
%For the sorted and almost sorted input distributions, the picture is mostly the same.
%For the reverse sorted input distribution, the wall-clock time increases significantly.
%Because this input distribution constitutes the worst-case of the \ac{WRAM} \MS{}, the first phase takes about one-and-a-half times as long.
%Its speedup actually manages to rise slightly.
%We would expect the biggest speedup drop for the zero-one input distribution, where plentiful duplicates should worsen workload imbalances.
%Unfortunately, the buggy implementation renders an analysis of the runtime pointless.
