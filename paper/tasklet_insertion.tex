\section{\texorpdfstring{\IS{}}{InsertionSort}}
\label{sec:tasklet:insertion}

\IS{} works by moving the \(i\)th element leftwards as long as its left neighbour is greater, assuming that the elements with indices \(0\) to \(i - 1\) are already sorted \cites[83]{maurer1974datenstrukturen}[Chapter~2.2.1]{wirth1975algorithmen}.
Its asymptotic runtime is above the theoretical minimum of \(\bigoh{n \log n}\), reaching \(\bigoh*{n^2}\) not only in the worst case but also in the average case, since any of the \(\binom{n}{2}\) pairs of input elements is in wrong order, needing to be swapped at some point in the execution, with probability \qty{50}{\percent}.
Nonetheless, \IS{} does have some saving graces:
\begin{enumerate}
	\item
	If the input array is mostly or even fully sorted, the runtime drops down to \(\bigoh{n}\).

	\item
	It works in-place, needing only \(\bigoh{1}\) additional space.

	\item
	The sorting is stable.

	\item
	Its implementation is short, lending itself to inlining.

	\item
	The overhead is small.
\end{enumerate}
Especially the last two points make \IS{} a good fallback algorithm for asymptotically better sorting algorithms to use on short subarrays.

\subsection{Presentation of Key Aspects}
\label{sec:tasklet:insertion:aspects}

\paragraph{Sentinel Values}
When moving an element to the left, two checks are needed:
Does the left neighbour exist and is it less than the element to move?
The first check can be omitted through the use of \emph{sentinel values}~\cite[93]{wirth1975algorithmen}:
If the element with index \(-1\) is permanently set to the least possible value of the chosen data type, it is at least as little as any element in the input array, and the leftwards motion stops there at the latest.
Since a \ac{DPU} lacks branch prediction, the slowdown from performing twice as many checks as needed is quite high and goes up to \qty{30}{\percent} for short inputs with 24 uniformly distributed 32-bit elements.

Setting such an \emph{explicit} sentinel value can be omitted by using \emph{implicit} sentinel values.
At the start of round~\(i\), one can check whether the element with index \(0\) is at least as little as the element with index~\(i\).
If so, the former is a sufficient sentinel value, and \IS{} can proceed as normal.
If not, the latter must be the minimum of the first \(i + 1\) elements.
Therefore, one can shift the first \(i\) elements one position backwards and place the minimum in the front.
For simplicity, the words \enquote{explicit} and \enquote{implicit} are, henceforth, applied to the word \enquote{\IS{}} directly to imply the type of the sentinel value used.


\subsection*{Investigation of the Compilation}
\label{sec:tasklet:insertion:compilation}
\addcontentsline{toc}{subsection}{\nameref{sec:tasklet:insertion:compilation}}

A common theme when developing for DPUs is a nosediving quality of the compilation.
This is no different for \IS{} upon which shall be shed some light in this \lcnamecref{sec:tasklet:insertion:compilation}.

A naïve implementation of \IS{} begins sorting at the very start of the input and is shown in \cref{fig:insertion:impl:pred_first}.
Obviously, the first element alone is already sorted, so it is algorithmically sound to let \IS{} begin at the second element.
This optimisation is accomplished in \cref{fig:insertion:impl:pred_sec}.
Surprisingly, it leads to a nine instructions longer runtime at 16 integers!
The same happens if, in \cref{fig:insertion:impl:pred_sec}, one keeps \lstinline|*i = start| and instead uses \lstinline|curr = ++i|.

Looking at the compilation reveals the reason:
In the naïve version, the pointer \lstinline|pred| is optimised away and, in its stead, the pointer \lstinline|curr| is passed to all load operations together with a constant offset as second argument.
In the optimised version, the pointer \lstinline|pred| is used with an offset to fetch the values of \lstinline|to_sort| and \lstinline|*pred| at the beginning of each iteration of the outer loop.
Then, the pointer \lstinline|curr| is initialised using the pointer \lstinline|pred| before being used in the inner loop as in the naïve version.
This initialisation is done through one additional \lstinline|move| instruction.
% This is a consequence of reusing the register of the \lstinline|start| pointer for \lstinline|pred| instead of for \lstinline|i|, whose incremented value is put into another register.

These changes fully explain the prolongation of the runtime by nine instructions:
The optimised version loops 15 times in total, each time laboriously initialising the pointer \lstinline|curr|, and executes one \lstinline|add| instruction at the beginning of the function to advance the starting position.
The naïve version loops 16 times, the first time executing seven instructions for naught.

\begin{figure}
	\lstset{basicstyle=\ttfamily\small}
	\def\iscodewidth{0.47\linewidth}
	\begin{subfigure}{\iscodewidth}
		\begin{lstlisting}
void InsertionSort(int *start, int *end) {
	int *curr, *i = start;
	while ((curr = i++) <= end) {
		int to_sort = *curr;
		int *pred = curr - 1;
		while (*pred > to_sort) {
			*curr = *pred;
			curr = pred--;
		}
		*curr = to_sort;
	}
}
		\end{lstlisting}
		\caption{
			Start at the first element and with predecessor pointer.
		}
		\label{fig:insertion:impl:pred_first}
	\end{subfigure}
	\hfill
	\begin{subfigure}{\iscodewidth}
		\begin{lstlisting}
void InsertionSort(int *start, int *end) {
	int *curr, *i = start + 1;
	while ((curr = i++) <= end) {
		int to_sort = *curr;
		int *pred = curr - 1;
		while (*pred > to_sort) {
			*curr = *pred;
			curr = pred--;
		}
		*curr = to_sort;
	}
}
		\end{lstlisting}
		\caption{
			Start at the second element and with predecessor pointer.
		}
		\label{fig:insertion:impl:pred_sec}
	\end{subfigure}

	\begin{subfigure}{\iscodewidth}
		\begin{lstlisting}
void InsertionSort(int *start, int *end) {
	int *curr, *i = start;
	while ((curr = i++) <= end) {
		int to_sort = *curr;
		while (*(curr - 1) > to_sort) {
			*curr = *(curr - 1);
			curr--;
		}
		*curr = to_sort;
	}
}
		\end{lstlisting}
		\caption{
			Start at the first element and without predecessor pointer.
		}
		\label{fig:insertion:impl:offset_first}
	\end{subfigure}
	\hfill
	\begin{subfigure}{\iscodewidth}
		\begin{lstlisting}
void InsertionSort(int *start, int *end) {
	int *curr, *i = start + 1;
	while ((curr = i++) <= end) {
		int to_sort = *curr;
		while (*(curr - 1) > to_sort) {
			*curr = *(curr - 1);
			curr--;
		}
		*curr = to_sort;
	}
}
		\end{lstlisting}
		\caption{
			Start at the second element and without predecessor pointer.
		}
		\label{fig:insertion:impl:offset_sec}
	\end{subfigure}
	\caption{
		Four different implementations of \IS{} in C.
		\Cref{fig:insertion:impl:pred_first,fig:insertion:impl:offset_first} are compiled the same.
		\Cref{fig:insertion:impl:pred_sec,fig:insertion:impl:offset_sec} are compiled differently.
	}
\end{figure}

Multiple workarounds exist to sidestep this problem.
One workaround is to take the unoptimised code and change the starting position via inline assembler.
This is trivial for the explicit \IS{} since one can simply inject an \lstinline|add| instruction at the beginning of the function to increment the pointer \lstinline|start|.
The implicit and the sentinel-less \IS*{} need to know the original starting address \lstinline|start| later on, though, and initialise the actual starting point rather late;
injecting inline assembler proves more difficult as a consequence.
Moreover, as \IS{} is to be used as fallback algorithm by other functions which might also need to keep the original value of \lstinline|start|, inline assembler is a bad choice even for the explicit \IS{}.

Another workaround is the usage of a wrapper function calling \IS{} with the arguments \lstinline|start + 1| and \lstinline|end|.
This works quite well:
First, the register holding \lstinline|start| is incremented, and, then, the inlined code from the actual \IS{} follows.
Doing so makes the runtime drop as expected.

Recall how in the faster version (\cref{fig:insertion:impl:pred_first}), the pointer \lstinline|pred| is always deduced from the pointer \lstinline|curr| using an offset.
This gives the cue for yet another workaround:
In \cref{fig:insertion:impl:offset_first,fig:insertion:impl:offset_sec}, every occurrence of \lstinline|pred| is replaced with \lstinline|curr - 1|.
As a consequence, the code of \cref{fig:insertion:impl:offset_first} compiles the very same as the one of \cref{fig:insertion:impl:pred_first}, while \cref{fig:insertion:impl:offset_sec} yields the same compilation as the versions with the wrapper function or the inline assembly.
This workaround is clearly the best of the three and, hence, the one used in the rest of this thesis.

Alas, the eternal struggle with the compiler endeth not herewith.
A deeper look into the compilation reveals the following sequence:
\begin{center}
	\begin{tabular}{ll}
		\lstinline|move r3, r0| & \makebox[0pt][l]{\textit{// copy content of register \lstinline|r0| to \lstinline|r3|}}
		\\ \lstinline|jleu r4, r2, .LABEL| & \makebox[0pt][l]{\textit{// jump to \lstinline|.LABEL| if \lstinline|r4| \(\le\) \lstinline|r2|}}
		\\ \lstinline|move r3, r0| &
	\end{tabular}
\end{center}
Without delving further into its significance \Dash the second \lstinline|move r3, r0| is unneeded as it is impossible to jump directly to it nor to return via \lstinline|jleu|.
Also, \lstinline|move| does not set any flags like the zero flag or carry flag, as some other instructions do, so such a side effect can be excluded as justification.
Copying the whole assembler code and injecting it as inline assembler but with this second \lstinline|move r3, r0| removed pushes the runtime even further down whilst still sorting correctly.
New issues, especially for inlining, are introduced, though, and we deem a proper assembly implementation as out of scope for this thesis.


\subsection{Evaluation of the Performance}
\label{sec:tasklet:insertion:performance}

\def\insertionalgos{1NoSentinel,1,1Implicit,BubbleNonAdapt,BubbleAdapt,Selection}

\expandafter\pgfplotsinvokeforeach\expandafter{\alldists}{
	\pgfplotstablereadnamed{data/small sorts/uint32/#1.txt}{tableSmallSorts_32#1}
	\pgfplotstablereadnamed{data/small sorts/uint64/#1.txt}{tableSmallSorts_64#1}
}

The runtimes of the three \IS*{} can be compared in the \cref{fig:insertion:against_others,fig:insertion:against_others_uint32,fig:insertion:against_others_uint64}.
The sentinel-less \IS{} is consistently worse than the explicit one.
For most input distributions, the implicit \IS{} is also a bit slower, as it effectively performs one check more for each element.
Of course, the gap becomes less significant with increasing input length as the other operations of the loops dominate the runtime.

An outlier, however, are the reverse sorted inputs.
For 32-bit numbers (\cref{fig:insertion:against_others}), the speedup\footnote{
	The \emph{speedup} \(S\) of an algorithm~\(A\) over an algorithm~\(B\) is defined as the ratio \(\operatorname{t}\mkern1mu(B\mkern1mu) / \operatorname{t}\mkern1mu(A)\) of their runtimes \(\operatorname{t}\mkern1mu(A)\) and \(\operatorname{t}\mkern1mu(B\mkern1mu)\).
	Values below 1 indicate that algorithm~\(A\) runs slower than algorithm~\(B\).
} of the implicit \IS{} over the explicit one drops down to as little as \num{0.68}.
This comes as a surprise since both versions effectively execute the same loop body while shifting everything one position backwards, with only the loop condition being different.
Due to \acp{DPU} being unit-cost machines, a value check on whether the preceding element is less (explicit \IS{}) and an address check on whether the preceding position is the start of the array (implicit \IS{}) should take the same amount of time.
Yet, even the sentinel-less \IS{} surpasses the implicit \IS{}, despite doing both value checks and address checks.
For 64-bit numbers (\cref{fig:insertion:against_others_uint64}), the implicit \IS{} would be expected to perform better than the explicit one, considering that a value check now takes two instructions and an address check still only one.
Nonetheless, the two \IS*{} tie.
This constitutes another case of bad compilation.
We did not troubleshoot as the explicit \IS{} would still be expected to offer superior performance in most cases.
The explicit \IS{} is, therefore, used in the rest of this thesis and referred to plainly as \enquote{\IS{}} henceforth.

\begin{figure}
	\tikzsetnextfilename{insertion_against_others}
	\begin{tikzpicture}[plot]
		\begin{groupplot}[
			adaptive group=1 by 2,
			groupplot ylabel={Cycles / \(n^2\)},
			xtick distance=3,
			minor xtick=data,
			ymin=0,
			ymax=60,
			legend columns=3,
		]
			\nextgroupplot[title/.add={}{Reverse Sorted}]
			\pgfplotsset{legend to name=leg:insertion:against_others, legend entries={\IS{} (sentinel-less), \IS{} (explicit), \IS{} (implicit), \BS{} (not adaptive), \BS{} (adaptive), \SelS{}}}
			\expandafter\pgfplotsinvokeforeach\expandafter{\insertionalgos}{
				\plotpernn{#1}{tableSmallSorts_32reverse}
			}
			%
			\nextgroupplot[title/.add={}{Uniform}]
			\expandafter\pgfplotsinvokeforeach\expandafter{\insertionalgos}{
				\plotpernn{#1}{tableSmallSorts_32uniform}
			}
		\end{groupplot}
	\end{tikzpicture}

	\tikzexternaldisable\hfil\pgfplotslegendfromname{leg:insertion:against_others}\hfil\tikzexternalenable
	\caption{
		Mean runtimes of sorting algorithms with a runtime in \(\bigoh*{n^2}\) on 32-bit integers.
	}
	\label{fig:insertion:against_others}
\end{figure}

\begin{note}
	Other known simple sorting algorithm are \SelS{} and \BS{}.
	\emph{\SelS{}}~\cites[83]{maurer1974datenstrukturen}[Section~2.2.2]{wirth1975algorithmen} assumes, like \IS{}, that the elements with indices \(0\) to \(i - 1\) are already sorted in round~\(i\).
	It scans the elements with indices \(i\) to \(n\) and finds their minimum.
	Then, it swaps a minimum element with the element with index \(i\).
	\emph{\BS{}}~\cite[Section~2.2.3]{wirth1975algorithmen} scans the elements with indices \(0\) to \(n - i + 1\) and swaps each pair of neighbouring elements if they are in the wrong order.
	An easy extension is adaptive \BS{} which sorts only up to the position of the last swap.

	The average-runtime complexity of \SelS{} and \BS{} is the same as that of \IS{}.
	The asymptoticity, however, hides much higher constant factors such that \IS{} should always be preferred, as seen in \cref{fig:insertion:against_others,fig:insertion:against_others_uint32,fig:insertion:against_others_uint64}.
	Consequently, they will not be expanded on further in this thesis.
\end{note}

