\section{A First \texorpdfstring{\MS{}}{MergeSort} on MRAM Data}
\label{sec:mram:merge}

The MRAM \MS{} is based on the half-space WRAM \MS{} as presented in \cref{sec:tasklet:merge} so only the adaptations to the two-tier memory hierarchy are discussed.

\paragraph{Starting Runs}
It is yet again beneficial to form starting runs.
This is done by loading a section of the MRAM into the WRAM, sorting it through one of the algorithms presented in \cref{sec:tasklet}, and writing the sorted section back to the MRAM.
As those sorting algorithm rely on the data being present entirely within the WRAM, forming the starting runs does not make use of sequential readers.
Instead, the functions \lstinline|mram_read| and \lstinline|mram_write| are used directly.

Contrary to the WRAM \MS{} with starting runs of length 14, the starting runs of the MRAM \MS{} are much longer, easily containing a thousand elements and more.
The reason is that longer starting runs reduce the number of rounds of the MRAM \MS{} and, thus, reduce the DMAs to the MRAM, which are relatively costly compared to accesses to the WRAM.
According to \citeauthor{mutlu2022Benchmarking}~\cite[8\psq]{mutlu2022Benchmarking}, accessing 64-bit integers in the WRAM is about four and a half times faster than accessing 64-bit integers in the MRAM.
Since accessing 32-bit integers in the WRAM takes just as long as accessing 64-bit integers and since the performance of DMAs depends only on the total number of bytes, there is still a speedup of roughly 2 of WRAM accesses over MRAM accesses.
However, again similar to the WRAM \MS{}, it can be beneficial to slightly reduce the starting run length to achieve more balanced and faster rounds.
Nonetheless, the runtime difference between 1000, 1100, and 1200 elements is in the magnitude of one per mil.
For this reason, the starting run length is set simply to the maximum number of integers which the WRAM allotted to a tasklet can hold.

This does raise the question what said maximum is.
As calculated at the start of \cref{sec:tasklet}, each tasklet is allotted no more than \qty{5957}{\byte}.
Subtracting \qty{768}{\byte} for the stack of the tasklet and the call stack of \QS{} leaves \qty{5189}{\byte} or a little less than 1300 32-bit integers.\todo{Noch einmal später aktualisieren}
