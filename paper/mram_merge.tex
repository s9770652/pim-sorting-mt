\section{A \texorpdfstring{\MS{}}{MergeSort} for MRAM Data}
\label{sec:mram:merge}

The MRAM \MS{} is based on the half-space WRAM \MS{} as presented in \cref{sec:tasklet:merge} so only the adaptations of the merge process to the two-tier memory hierarchy are discussed.
Preconditions for the presented algorithm to work are that the size of every run is a multiple of 8 and that the address of each first byte also a multiple of 8.
These preconditions simplify DMAs and are trivially met for 64-bit elements.
To meet it for 32-bit elements, the parallel \MS{} (\cref{sec:par}) divides the work amongst tasklets accordingly and, if need be, introduces dummy elements.
First, the general MRAM merge process with basic optimisations is presented and, thereafter, several optimisations concerning sequential readers.


\subsection{Merging in the MRAM}

The underlying idea is the following:
First, initialise a sequential reader on either run.
Then, repeatedly compare the current elements, write the less element to the cache, and read the next element.
Whenever the cache is full, empty it by writing its content to the output location.
Once the end of the run with the less tail element is reached, stop comparing and empty the cache.
Since the sorting algorithm is based on half-space \MS{}, the merging is now done if the run with the less tail element was the first run, as the remainder of the second run is already in the correct position.
If the run with the less tail element was the second run, flush the first run by transferring its remainder from the MRAM to the output location with the help of the entire triple buffer.

During merging, checks on both the depletion\todo{Bezeichnung schon in \cref{sec:tasklet} einführen!} of the less run\todo{ebenso} and the fill level of the cache are needed.
For this reason, the merge process (\cref{alg:two-tier merge}) consists of two tiers, with the first one having a reduced number of depletion checks and with both making use of unrolled loops to reduce the number of fill level checks.
The first tier is in operation as long as there are at least \unrollfactor{} many elements left to merge in the less run.
This is verifiable through the function \lstinline|seqread_tell|, which returns the corresponding MRAM address of an element within a sequential-read buffer.
First, an unrolled loop with \unrollfactor{} many iterations is executed, with each iteration comparing the current elements of both runs, writing the less element to the cache, and advancing the respective pointer (\cref{alg:unrolled loop}).
Afterwards, it is checked whether the cache is filled with \unrolledcachelength{} many elements, with \unrolledcachelength{} being a multiple of \unrollfactor{} and \(\unrolledcachelength{} \times \text{\lstinline[keywords={}]|sizeof(T)|} \le \cachesize{}\) being a multiple of 8.
If the fill level is too low, it is jumped back to the beginning of the tier and the number of remaining elements checked anew.
If, however, the maximum fill level is indeed reached, the cache is emptied before jumping back to the beginning of the tier.
Because of \(\unrolledcachelength{} \times \text{\lstinline[keywords={}]|sizeof(T)|}\) being a multiple of 8, emptying the cache is possible through a simple call of \lstinline|mram_write|.

Once the first tier comes to an end because the elements left in the less run are too few, it is checked whether the less run is depleted.
If so, the cache is emptied and the remainder of the greater run is flushed.
To do the latter, the data is iteratively transferred from its original position in the MRAM to the output location using \lstinline|mram_read| and \lstinline|mram_write|.
Since the sequential readers are of no use anymore, the whole triple buffer may be used during the transfer in case that the cache size is below \qty{2048}{\byte}.
Should the number of elements within the cache be odd and the input be 32-bit elements, the size of the cache content is not a multiple of 8.
Furthermore, the size of the remainder of the greater run is also not a multiple of 8 then, given that both the sizes of two runs as well as \(\unrolledcachelength{} \times \text{\lstinline[keywords={}]|sizeof(T)|}\) were such.
For this reason, the current element of the greater run is moved to the cache to bring the size of its content to a multiple of 8, making calling \lstinline|mram_write| to empty the cache and flush the remainder unproblematic.

\NewDocumentCommand{\codeparttitle}{m}{\nonl\textsf{\textbf{\uppercase{#1}}}}
\SetArgSty{}
\SetFuncArgSty{}
\SetKw{KwAnd}{and}
\SetKw{KwContinue}{continue}
\SetKw{KwT}{T}

\SetKwArray{Cache}{cache}
\SetKwArray{Currs}{curr}
\SetKwArray{Ends}{ends}
\SetKwArray{Readers}{readers}

\SetKwData{Early}{early\_end}
\SetKwData{Factor}{\unrollfactor*}
\SetKwData{Length}{\unrolledcachelength*}
\SetKwData{Out}{out}

\SetKwFunction{Get}{seqread\_get}
\SetKwFunction{Read}{mram\_read}
\SetKwFunction{SizeOf}{sizeof}
\SetKwFunction{Tell}{seqread\_tell}
\SetKwFunction{Write}{mram\_write}

\begin{algorithm*}[!t]
	\KwData{%
		sequential readers \Readers{\(2\)},
		pointers \Currs{\(2\)} to current elements,
		pointers \Ends{\(2\)} to tail elements,
		output location \Out,
		cache \Cache
	}
	\KwResult{%
		both runs merged together and written to \Out
	}

	\(i\) ← \(0\)  \tct*{number of elements in cache}
	\Early ← \Ends{\(1\)} – \Factor + \(1\)\;
	\While(\tct*[f]{first tier}){\Tell{\Currs{\(1\)}, \Readers{\(1\)}} ≤ \Early}{
		Merge \Factor many elements \linebreak without checking for depletion (\cref{alg:unrolled loop}).\;
		\If{\(i\) ≤ \Length}{
			\KwContinue  \tct*{skips cache emptying}
		}
		\Write{\Cache, \Out, \Length × \SizeOf{\KwT}}\;
		\(i\) ← \(0\)\;
		\Out ← \Out + \Length\;
	}
	\If{\Tell{\Currs{\(1\)}, \Readers{\(1\)}} > \Ends{\(1\)}}{
		Empty the cache.\;
		Flush the first run.\;
		\Return\;
	}
	\While(\tct*[f]{second tier}){true}{
		Merge \Factor many elements \linebreak with checking for depletion (\cref{alg:unrolled loop}).\;
		\If{\(i\) ≤ \Length}{
			\KwContinue  \tct*{skips cache emptying}
		}
		\Write{\Cache, \Out, \Length × \SizeOf{\KwT}}\;
		\(i\) ← \(0\)\;
		\Out ← \Out + \Length\;
	}

	\caption{
		Two-tier merging of two MRAM runs, where the second one is the less run.
		In the event of the first run being less, flip all indices and omit flushing the other run.
	}
	\label{alg:two-tier merge}
\end{algorithm*}

If, however, the less run is not yet depleted, the second tier begins.
This one is structurally equal to the first tier with a single exception, for there is no guarantee that the unrolled loop will be executed in full:
The depletion check now happens whenever an element of the less run is written to the cache.
When it occurs, the cache is emptied and the greater run is flushed, completing the merging.

\begin{algorithm*}[!t]
	\KwData{%
		sequential readers \Readers{\(2\)},
		pointers \Currs{\(2\)} to current elements,
		pointers \Ends{\(2\)} to tail elements,
		output location \Out,
		cache \Cache,
		number \(i\) of elements in the cache
	}
	\KwResult{%
		\Factor many elements merged to \Cache{\(i\) .. \(i\) + \Factor – \(1\)}
	}

	\For(\tct*[f]{unrolled loop}){\(k\) ← \(1\) \KwTo \Factor}{
		\eIf{*\Currs{\(0\)} ≤ *\Currs{\(1\)}}{
			\Cache{\(i\)++} ← *\Currs{\(0\)}\;
			\Currs{\(0\)} ← \Get{\Currs{\(0\)}, \Readers{\(0\)}}\;
		}{
			\Cache{\(i\)++} ← *\Currs{\(1\)}\;
			\Currs{\(1\)} ← \Get{\Currs{\(1\)}, \Readers{\(1\)}}\;
			\If(\tct*[f]{omit in tier 1}){\Tell{\Currs{\(1\)}, \Readers{\(1\)}} = \Ends{\(1\)}}{
				Empty the cache.\;
				Flush the first run.\;
				\Return  \tct*{stops \cref{alg:two-tier merge}}
			}
		}
	}

	\caption{
		Merging \unrollfactor{} many elements.
		This algorithm is part of \cref{alg:two-tier merge}, meaning any change to a variable carries over.
		In the event of the first run being less, flip the indices in the inner \lstinline[keywords={}]|if| statement and move it up into the outer \lstinline[keywords={}]|if| block.
	}
	\label{alg:unrolled loop}
\end{algorithm*}

%\begin{algorithm*}
%	\KwData{%
%		sequential readers \Readers{\(2\)},
%		pointers \Currs{\(2\)} to current elements,
%		output location \Out,
%		cache \Cache,
%		number \(i\) of elements in the cache
%	}
%	\KwResult{%
%		all merged elements written to the output
%	}
%
%	\If(\tct*[f]{always false for 64-bit integers}){\SizeOf{\KwT} = \(4\) \KwAnd \(i\) \(\bmod\) \(2\) ≠ \(0\)}{
%		\Cache{\(i\)++} ← *\Currs{\(0\)}  \tct*{Now, the number of elements in the cache is even.}
%		\Currs{\(0\)} ← \Get{\Currs{\(0\)}, \Readers{\(0\)}}\;
%	}
%	\Write{\Cache, \Out, \(i\) × \SizeOf{\KwT}}  \tct*{The transfer size is always divisible by 8.}
%	\Out ← \Out + \(i\)\;
%
%	\caption{
%		Emptying the cache.
%		This algorithm is part of \cref{alg:two-tier merge,alg:unrolled merge}, meaning any change to a variable carries over.
%	}
%	\label{alg:flush cache}
%\end{algorithm*}
%
%\begin{algorithm*}
%	\SetKwData{From}{from}
%	\SetKwData{MaxTransferS}{MAX\_TRANSFER\_SIZE}
%	\SetKwData{MaxTransferL}{MAX\_TRANSFER\_LENGTH}
%	\SetKwData{Size}{size}
%	\SetKwData{TBS}{\triplebuffersize*}
%
%	\KwData{%
%		sequential readers \Readers{\(2\)},
%		pointers \Currs{\(2\)} to current elements,
%		pointers \Ends{\(2\)} to tail elements,
%		output location \Out,
%		cache \Cache
%	}
%	\KwResult{%
%		every non-merged element of the first run written behind the merged elements in the output
%	}
%
%	\MaxTransferS ← \(\min\{ \TBS, 2048 \}\)\;
%	\MaxTransferL ← \MaxTransferS \,/ \SizeOf{\KwT}\;
%	\Size ← \MaxTransferS  \tct*{size of current transfer}
%	\From ← \Tell{\Currs{\(0\)}, \Readers{\(0\)}}  \tct*{first byte to transfer}
%	\While{\From ≤ \Ends{\(0\)}}{
%		\If{\From + \MaxTransferL – \(1\) > \Ends{\(0\)}}{
%			\Size ← (\Ends{\(0\)} – \From + \(1\)) × \SizeOf{\KwT}  \tct*{smaller size on last transfer}
%		}
%		\Read{\From, \Cache, \Size}\;
%		\Write{\Cache, \Out, \Size}\;
%		\From ← \From + \MaxTransferL  \tct*{May be wrong on the last transfer …}
%		\Out ← \Out + \MaxTransferL  \tct*{… but pointers are useless then, anyway.}
%	}
%
%	\caption{
%		Flushing the remainder of the non-depleted Run 0.
%	}
%	\label{alg:flush run}
%\end{algorithm*}

\subsection{Making Sequential Readers Faster}
\label{sec:mram:merge:faster}

The most widely used sequential-reader function is \lstinline|seqread_get|, followed at some distance by \lstinline|seqread_tell| and, at even more distance, \lstinline|seqread_init|.
Those functions cannot be inlined, so each use of them constitutes a function call.
Each function call comes at non-negligible cost since arguments have to be loaded into the respective registers, the jump to the function itself be performed, the stack pointer and return address be saved and reloaded, modified registers be restored if need be, and the jump back to the return address be performed.
Since the DPU architecture is fundamentally compute-bound, function calls are a serious impediment to performance.
This has already been an argument in favour of the oft-used \IS{} whose short implementation lent itself to inlining.

Earlier attempts at reducing function calls included maintaining a counter on the number of elements left to make \lstinline|seqread_tell| obsolete.
This alone yielded prominent speedup while still being independent of the exact implementation of sequential readers and possible future changes to them.
Similarly, calls to \lstinline|seqread_get| can be reduced by advancing the pointers to current elements manually as long as the end of the first buffer halves are sufficiently far away.
Nevertheless, larger speedup is achievable by implementing an own sequential reader which can be inlined.
The simplest way to do so is to duplicate the driver source file \lstinline|seqread.inc| and make its content visible to the sorting algorithm;
the BSD-style licence of the driver permits such modification and redistribution given proper credits.
The speedup through inlining is significant.
For example, with \cachesize{} = 2880, \seqreadcachesize{} = 1024, \QS{} as WRAM sorting algorithm, and 2\textsuperscript{20} uniformly distributed 32-bit integers, \MS{} finishes after \qty[exponent-mode=engineering, round-mode=places, round-precision=2]{2143534915}{\cycle} if sequential readers are used as is, that is with function calls.
With inlining, the runtime drops down by \qty{32}{\percent} to \qty[exponent-mode=engineering, round-mode=places, round-precision=2]{1455341742}{\cycle}.

\begin{figure}[t]
	\begin{subfigure}{\textwidth}
		\begin{lstlisting}[language={[DPU]Assembler}, mathescape, keepspaces]
	add rcurr, rcurr, 8, nc10, .LABEL  // curr ← curr + 8; jump if no carry bit 10
	add rreader, rstack, –120          // get address of reader in the WRAM stack
	lw rmram, rreader, 4               // load MRAM address of reader
	add rmram, rmram, 1024             // MRAM address ← MRAM address + 1024
	sw rrdr, 4, rreader                // store new MRAM address in reader
	lw rwram, rreader, 0               // load buffer address of reader
	ldma rwram, rmram, 255             // load (255 + 1) $×$ 8 bytes from the MRAM
	add rcurr, rcurr, –1024            // curr ← curr – 1024
.LABEL:\end{lstlisting}
		\caption{
			The assembler code as generated by \lstinline|__builtin_dpu_seqread_get|.
			In some cases, line 2 is omitted, namely when the address of the reader is kept in an own register.
		}
		\label{fig:mram:assembler:auto}
	\end{subfigure}

	\begin{subfigure}{\textwidth}
		\begin{lstlisting}[language={[DPU]Assembler}, mathescape, keepspaces]
	add rcurr, rcurr, 8, nc11, .LABEL  // curr ← curr + 8; jump if no carry bit 11
	add rmram, rmram, 2048             // MRAM address ← MRAM address + 2048
	ldma rwram, rmram, 255             // load (255 + 1) $×$ 8 bytes from the MRAM
	add rcurr, rcurr, –2048            // curr ← curr – 2048
.LABEL:\end{lstlisting}
		\caption{
			The handwritten assembler code.
		}
		\label{fig:mram:assembler:manual}
	\end{subfigure}
	\caption{
		Comparison of the assembler codes of the regular function \lstinline|seqread_get|, which calls \lstinline|__builtin_dpu_seqread_get|, and the improved one, which is handwritten.
		In both cases, elements are \qty{8}{\byte} large and \seqreadcachesize{} is set to 1024.
		If no update to the sequential-read buffers is needed, the instruction \lstinline|add| jumps to the label at the end, skipping the rest of the code.
	}
	\label{fig:mram:assembler}
\end{figure}

A downside to the two-buffer system is that data are loaded twice.
Since one of the preconditions demands that the address of the first byte of each run is a multiple of 8 and since elements are either \qty{32}{\bit} or \qty{64}{\bit} large, it is assured that the last element in the first buffer can never extend into the second half.
Therefore, a natural optimisation is to regard two consecutive sequential-read buffers as one and to load new data only when the pointer to the current element reaches the end of the second original buffer.
However, the two-buffer system is intrinsic to \lstinline|__builtin_dpu_seqread_get|, that is the function used by \lstinline|seqread_get|, and we know of no alternative to it.
For this reason, inline assembly is employed to imitate its compilation but with a check on the next greater carry bit and appropriate pointer modifications.

A closer look at the original compilation (\cref{fig:mram:assembler:auto}) reveals more savings potential.
Whenever new data need to be loaded, the address of the sequential-read buffer is loaded from the \lstinline[keywords={}]|struct| representing the reader (ln.~6) despite being a constant.
The MRAM address stored in the reader is not only loaded (ln.~3) but also stored (ln.~5) after being set to the new value.
In half of the cases, even the address of the reader \lstinline[keywords={}]|struct| itself needs to be loaded first from the stack (ln.~2), because the register into which it is loaded gets overwritten later
These four load and store instruction can be saved by abandoning \lstinline[keywords={}]|struct|s to represent readers and using two array of length 2, one for the buffer addresses and one for the MRAM addresses.
As a consequence, these values are kept permanently within registers without ever being overwritten, making the inline assembler code (\cref{fig:mram:assembler:manual}) significantly shorter \Dash admittedly, the savings are less than the reduced number of lines suggests as the DMA dominates the runtime in this piece of code.

Next to \lstinline|seqread_get|, more optimisation potential is hidden in the function \lstinline|seqread_init|, which is called twice before a pair of runs is merged.
The original function checks whether the MRAM address to which a sequential reader is set is already in the buffer.
Since sequential readers are always initialised to the beginnings of runs and the runs are too long, this check is always negative and can be omitted.
Moreover, recall that the original function divides the MRAM into pages, which always begin at a multiple of \seqreadcachesize{}.
This means that a run may begin in the middle of a page so the preceding, uninteresting data must be loaded as well.
Since runs begin at multiples of 8, the function \lstinline|seqread_init| can load from the first byte of the run onwards directly without issues because of \lstinline|mram_read|.
The reason why the original function \lstinline|seqread_init| did not simply round the given MRAM address down to the next multiple of 8 but instead bothered with computing the page boundaries can only be speculated.

Where the previous two versions take \qty[exponent-mode=engineering, round-mode=places, round-precision=2]{2143534915}{\cycle} and \qty[exponent-mode=engineering, round-mode=places, round-precision=2]{1455341742}{\cycle}, respectively, the optimised version takes \qty[exponent-mode=engineering, round-mode=places, round-precision=2]{1411887248}{\cycle}, that is \qty{34}{\percent} less than the regular sequential reader and \qty{3}{\percent} less than the inlined sequential reader.
This small gain is a testament to how overwhelmingly compute-bound \MS{} is.
The share of time spend on memory accesses shall be assessed in more detail in the following section.

