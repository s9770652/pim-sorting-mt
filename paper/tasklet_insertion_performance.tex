\subsection*{Evaluation of the Performance}
\label{sec:tasklet:insertion:performance}

\def\insertionalgos{1NoSentinel,1,1Implicit,BubbleNonAdapt,BubbleAdapt,Selection}

\pgfplotsinvokeforeach{sorted,reverse,almost,uniform,zipf,normal}{
	\pgfplotstablereadnamed{data/small sorts/uint32/#1.txt}{tableSmallSorts_32#1}
	\pgfplotstablereadnamed{data/small sorts/uint64/#1.txt}{tableSmallSorts_64#1}
}

The runtimes of the explicit and implicit \IS*{} can be compared in the \cref{fig:insertion:against_others,fig:insertion:against_others_uint32,fig:insertion:against_others_uint64}.
For most input distributions, the implicit \IS{} is logically a bit slower, effectively adding one check instruction to each round.
An outlier, however, are the reverse sorted inputs.
For 32-bit numbers (\cref{fig:insertion:against_others,fig:insertion:against_others_uint32}), the implicit version is up to 45\% slower than the explicit one.
This comes as a surprise since both versions effectively execute the same loop body while shifting everything one index backwards, with only the loop condition being different.
Due to the uni-cost model, a value check on whether the preceding element is smaller (explicit) and an address check on whether the preceding position is the start of the array (implicit) should take the same amount of time.
Yet, even the \IS{} not relying on sentinel values surpasses the implicit \IS{}, although doing both value checks and address checks!
For 64-bit numbers (\cref{fig:insertion:against_others_uint64}), the implicit \IS{} would be expected to perform better than the explicit one, considering that a value check now takes two instructions and an address check still only one.
Nonetheless, the two \IS*{} tie.

\begin{figure}
	\tikzsetnextfilename{insertion_against_others}
	\begin{tikzpicture}[plot]
		\begin{groupplot}[
			horizontal sep for labels,
			adaptive group=1 by 2,
			groupplot xlabel={Input Length \(n\)},
			xtick distance=3,
			minor xtick=data,
			legend columns=3,
			title={},
		]
			\nextgroupplot[ylabel=Cycles / \(n^2\), ymin=0, ymax=60]
			\pgfplotsset{legend to name=leg:insertion:against_others, legend entries={\IS{} (no sentinel), \IS{} (explicit), \IS{} (implicit), \BS{} (not adaptive), \BS{} (adaptive), \SelS{}}}
			\expandafter\pgfplotsinvokeforeach\expandafter{\insertionalgos}{
				\plotpernn{#1}{tableSmallSorts_32uniform}
			}
			%
			\nextgroupplot[ylabel=Speed-up, ymin=0.2, ymax=1, yticklabel style={/pgf/number format/.cd, precision=1, fixed, zerofill}]
			\expandafter\pgfplotsinvokeforeach\expandafter{\insertionalgos}{
				\ifstrequal{#1}{1}{
					\pgfplotsset{cycle list shift=1}
				}{
					\plotspeedup{#1}{1}{tableSmallSorts_32uniform}
				}
			}
		\end{groupplot}
	\end{tikzpicture}

	\hfil\pgfplotslegendfromname{leg:insertion:against_others}\hfil
	\caption{
		Comparison of sorting algorithms with a runtime in \(\bigoh{n^2}\).
		The adaptive \BS{} terminates prematurely if no changes were made to the input array during an iteration.
		The speed-ups are with respect to the \IS{} relying on explicit sentinel values.
	}
	\label{fig:insertion:against_others}
\end{figure}
