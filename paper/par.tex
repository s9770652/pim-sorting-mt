\chapter{Parallel Sorting in the MRAM}
\label{sec:par}

\begin{itemize}
	\item
	Falls Einzelelemente verwendet werden, geschieht das um den Pivot herum.

	\item
	Ausrechnen, dass die binäre Suche eine Promyriade der Laufzeit oder so ausmacht.

	\item
	Bzgl.\ des Fehlers bei sortierten und fast sortierten Eingaben
	\begin{itemize}
		\item
		Tritt anscheinend noch nicht bei der sequentiellen Phase auf.

		\item
		Effekt wird mit steigendem \lstinline|NR_TASKLETS| größer, aber erst bei \lstinline|NR_TASKLETS| = 16 dreht sich die Reihenfolge um.

		\item
		Wenn man Anweisungen statt Takte zählt, stimmt die Reihenfolge wieder und zwar deutlich.
		Unabhängig davon, wo \lstinline|NR_TASKLETS| steht, gilt, dass die sortierte Eingabe 5,65× weniger Anweisungen braucht, d.\,h., es muss an der Datenübertragung liegen.
	\end{itemize}

	\item
	Warum der \MS{} gut ist:
	\begin{itemize}
		\item
		Es gibt nur vier Synchronisationspunkte.

		\item
		Binäres Mischen heißt, es gibt nur zwei sequentielle Leser. Drei wären wohl noch machbar (nötig bei zwölf Tasklets), aber 16-Wege-Verschmelzen wäre unurst.

		\item
		Die besonders frühen Tasklets nehmen keine Rechenzeit bei der Suspendierung ein.
	\end{itemize}

	\item
	\lstinline|resume| braucht ungefähr 100 Takte

	\item
	sehr ungleiche Laufzeiten für Binärsuche:
	stabil, sortiert -> Min: 8000 T, Max: 320\,000 T (Sortiert, 16 Tasklets, erste Runde)
	\begin{itemize}
		\item
		viel Warterei auf andere -> DMA dominieren Laufzeit der Binärsuche
	\end{itemize}
\end{itemize}

\pgfplotsinvokeforeach{sorted,reverse,almost,uniform,zipf,normal}{
	% 32-bit | Instabil
	\pgfplotstablenewnamed[create on use/n/.style={}, create on use/µ_MergePar/.style={}, create on use/σ_MergePar/.style={}, columns={n,µ_MergePar,σ_MergePar}]{0}{tableMergeParUnstable_32#1}
	\pgfplotstablevertcatnamed{tableMergeParUnstable_32#1}{data/merge_par/NR_TASKLETS=1/STABLE=false/uint32/#1.txt}
	\pgfplotstablevertcatnamed{tableMergeParUnstable_32#1}{data/merge_par/NR_TASKLETS=2/STABLE=false/uint32/#1.txt}
	\pgfplotstablevertcatnamed{tableMergeParUnstable_32#1}{data/merge_par/NR_TASKLETS=4/STABLE=false/uint32/#1.txt}
	\pgfplotstablevertcatnamed{tableMergeParUnstable_32#1}{data/merge_par/NR_TASKLETS=8/STABLE=false/uint32/#1.txt}
	\pgfplotstablevertcatnamed{tableMergeParUnstable_32#1}{data/merge_par/NR_TASKLETS=16/STABLE=false/uint32/#1.txt}

	% 64-bit | Instabil
	\pgfplotstablenewnamed[create on use/n/.style={}, create on use/µ_MergePar/.style={}, create on use/σ_MergePar/.style={}, columns={n,µ_MergePar,σ_MergePar}]{0}{tableMergeParUnstable_64#1}
	\pgfplotstablevertcatnamed{tableMergeParUnstable_64#1}{data/merge_par/NR_TASKLETS=1/STABLE=false/uint64/#1.txt}
	\pgfplotstablevertcatnamed{tableMergeParUnstable_64#1}{data/merge_par/NR_TASKLETS=2/STABLE=false/uint64/#1.txt}
	\pgfplotstablevertcatnamed{tableMergeParUnstable_64#1}{data/merge_par/NR_TASKLETS=4/STABLE=false/uint64/#1.txt}
	\pgfplotstablevertcatnamed{tableMergeParUnstable_64#1}{data/merge_par/NR_TASKLETS=8/STABLE=false/uint64/#1.txt}
	\pgfplotstablevertcatnamed{tableMergeParUnstable_64#1}{data/merge_par/NR_TASKLETS=16/STABLE=false/uint64/#1.txt}

	% 32-bit | Stabil
	\pgfplotstablenewnamed[create on use/n/.style={}, create on use/µ_MergePar/.style={}, create on use/σ_MergePar/.style={}, columns={n,µ_MergePar,σ_MergePar}]{0}{tableMergeParStable_32#1}
	\pgfplotstablevertcatnamed{tableMergeParStable_32#1}{data/merge_par/NR_TASKLETS=1/STABLE=true/uint32/#1.txt}
	\pgfplotstablevertcatnamed{tableMergeParStable_32#1}{data/merge_par/NR_TASKLETS=2/STABLE=true/uint32/#1.txt}
	\pgfplotstablevertcatnamed{tableMergeParStable_32#1}{data/merge_par/NR_TASKLETS=4/STABLE=true/uint32/#1.txt}
	\pgfplotstablevertcatnamed{tableMergeParStable_32#1}{data/merge_par/NR_TASKLETS=8/STABLE=true/uint32/#1.txt}
	\pgfplotstablevertcatnamed{tableMergeParStable_32#1}{data/merge_par/NR_TASKLETS=16/STABLE=true/uint32/#1.txt}

	% 64-bit | Stabil
	\pgfplotstablenewnamed[create on use/n/.style={}, create on use/µ_MergePar/.style={}, create on use/σ_MergePar/.style={}, columns={n,µ_MergePar,σ_MergePar}]{0}{tableMergeParStable_64#1}
	\pgfplotstablevertcatnamed{tableMergeParStable_64#1}{data/merge_par/NR_TASKLETS=1/STABLE=true/uint64/#1.txt}
	\pgfplotstablevertcatnamed{tableMergeParStable_64#1}{data/merge_par/NR_TASKLETS=2/STABLE=true/uint64/#1.txt}
	\pgfplotstablevertcatnamed{tableMergeParStable_64#1}{data/merge_par/NR_TASKLETS=4/STABLE=true/uint64/#1.txt}
	\pgfplotstablevertcatnamed{tableMergeParStable_64#1}{data/merge_par/NR_TASKLETS=8/STABLE=true/uint64/#1.txt}
	\pgfplotstablevertcatnamed{tableMergeParStable_64#1}{data/merge_par/NR_TASKLETS=16/STABLE=true/uint64/#1.txt}
}

\NewDocumentCommand{\speedup}{m}{
	\pgfplotstablegetelem{0}{µ_MergePar}\of#1
	\pgfmathsetmacro{\messlatte}{\pgfplotsretval}
		\addplot+ table[
		create on use/tasklets/.style={create col/set list={1,2,4,8,16}},
		create on use/factor/.style={create col/expr={\messlatte / \thisrow{µ_MergePar}}},
		x=tasklets, y=factor
	] {#1};
}

\begin{figure}
	\tikzsetnextfilename{speedup}
	\begin{tikzpicture}[plot]
		\begin{groupplot}[
			adaptive group=2 by 2,
			groupplot xlabel={Tasklets},
			groupplot ylabel={Speedup},
			xtick={1,2,4,8,16},
			xmode=log,
			ymin=0,
			ymax=12,
			ytick distance=2,
		]
			\nextgroupplot[title/.add={}{32-bit | Instabil}]
			\pgfplotsset{legend to name=leg:par:speedup, legend entries={Sorted, Reverse S., Almost S., Uniform, Zipf's, Normal}}
			\addplot[forget plot, very nearly transparent] coordinates {(1,1)(2,2)(3,3)(4,4)(5,5)(6,6)(7,7)(8,8)(9,9)(10,10)(11,11)(16,11)};
			\pgfplotsinvokeforeach{sorted,reverse,almost,uniform,zipf,normal}{
				\expandafter\speedup\expandafter{\csname tableMergeParUnstable_32#1\endcsname}
			}
			%
			\nextgroupplot[title/.add={}{64-bit | Instabil}]
			\addplot[forget plot, very nearly transparent] coordinates {(1,1)(2,2)(3,3)(4,4)(5,5)(6,6)(7,7)(8,8)(9,9)(10,10)(11,11)(16,11)};
			\pgfplotsinvokeforeach{sorted,reverse,almost,uniform,zipf,normal}{
				\expandafter\speedup\expandafter{\csname tableMergeParUnstable_64#1\endcsname}
			}
			%
			\nextgroupplot[title/.add={}{32-bit | Stabil}]
			\addplot[forget plot, very nearly transparent] coordinates {(1,1)(2,2)(3,3)(4,4)(5,5)(6,6)(7,7)(8,8)(9,9)(10,10)(11,11)(16,11)};
			\pgfplotsinvokeforeach{sorted,reverse,almost,uniform,zipf,normal}{
				\expandafter\speedup\expandafter{\csname tableMergeParStable_32#1\endcsname}
			}
			%
			\nextgroupplot[title/.add={}{64-bit | Stabil}]
			\addplot[forget plot, very nearly transparent] coordinates {(1,1)(2,2)(3,3)(4,4)(5,5)(6,6)(7,7)(8,8)(9,9)(10,10)(11,11)(16,11)};
			\pgfplotsinvokeforeach{sorted,reverse,almost,uniform,zipf,normal}{
				\expandafter\speedup\expandafter{\csname tableMergeParStable_64#1\endcsname}
			}
		\end{groupplot}
	\end{tikzpicture}

	\hfil\pgfplotslegendfromname{leg:par:speedup}\hfil
\end{figure}
