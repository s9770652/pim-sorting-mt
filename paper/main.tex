\documentclass[
	book,  % underlying KOMA document class to load
	chapterprefix,  % word ‘Chapter’/‘Appendix’ in line before heading
	overfullrule,  % draw block boxes where lines overrun?
	british,  % language
	oneside,  % discrimination between left and right pages?
	groupplots,  % loading code for group plots
]{../garticle}

% Maths & Algorithms.
\DeclareMathOperator{\lb}{lb}

% Figures.
% taken from: https://tex.stackexchange.com/a/199396
\pgfplotsset{
	select coords between index/.style 2 args={
		x filter/.code={
			\ifnum\coordindex<#1\def\pgfmathresult{}\fi
			\ifnum\coordindex>#2\def\pgfmathresult{}\fi
		},
	},
}

\NewDocumentCommand{\plotruntime}{O{} m m}{
	\addplot+
	plot [#1]
	table [x=n, y=µ_#2] {#3};
}

\NewDocumentCommand{\plotwithbars}{O{} m m}{
	\addplot+
	plot [#1, error bars/.cd, y dir=both, y explicit]
%	table [x=n, y=µ_#2, y error=σ_#2] {#3};
	table [x=n, y=µ_#2, y error expr={2 * \thisrow{σ_#2}}] {#3};
}

\NewDocumentCommand{\plotpernlogn}{O{} m m}{
	\addplot+
	plot [#1]
	table [x=n, y expr={\thisrow{µ_#2} / (\thisrow{n} * log2(\thisrow{n}))}] {#3};
}

\NewDocumentCommand{\plotpernn}{O{} m m}{
	\addplot+
	plot [#1]
	table [x=n, y expr={\thisrow{µ_#2} / \thisrow{n}^2}] {#3};
}

\NewDocumentCommand{\plotspeedup}{O{} m m m}{
	\addplot+
	plot [#1]
	table [x=n, y expr={\thisrow{µ_#3} / \thisrow{µ_#2}}] {#4};
}

% Texts.
%\RequirePackage{showframe}
\RequirePackage[final]{listings}
\lstset{
	basicstyle=\ttfamily,
	numbers=left,
	numbersep=4pt,
	language=C,
	tabsize=2,
}

\DefineNamedColor{named}{accentcolor}{RGB}{118, 13, 28}  % Mordred’s red

\RedeclareSectionCommand[tocentryformat=\itshape]{subsection}

\pgfplotscreateplotcyclelist{ZAW colour list}{
	RoyalBlue!75!Green,  % 1
	BurntOrange!90!Black,  % 2
	Lavender!85!White,  % 3
	LimeGreen,  % 4
	Mulberry,  % 5
	CornflowerBlue,  % 6
	Goldenrod!95!Black,  % 7
	OliveGreen,  % 8
	Red,  % 9
	Black,  % 10
	Gray,  % 11
}

\pgfplotsset{
	height=4cm,
	cycle multiindex list={ZAW colour list \nextlist mark list*},  % automatic colouring of plots
	x from 16 to 1024/.style={
		xmode=log,
		xtick={16, 32, 64, 128, 256, 512, 1024},
		xticklabels={\(16\), \(32\), \(64\), \(128\), \(256\), \(512\), \(1024\)},
	},
	x from 16 to 1024 minor/.style={
		xmode=log,
		xtick={16, 64, 256, 1024},
		xticklabels={\(16\), \(64\), \(256\), \(1024\)},
		minor xtick={32, 128, 512},
	},
	log ticks with fixed point,  % uses 0.1, 0.001, … on logarithmix axes instead of 10^-1, 10^-2, …
	horizontal sep for labels/.style={group/horizontal sep=18mm},  % padding between groupplots (with ticks + labels)
	horizontal sep for ticks/.style={group/horizontal sep=10mm},  % padding between groupplots (with ticks)
	horizontal sep for naught/.style={group/horizontal sep=3mm/2},  % padding between groupplots (without anything)
	horizontal sep for ticks,
	vertical sep for ticks/.style={group/vertical sep=14mm},  % padding between rows of groupplots (with ticks)
	vertical sep for yticks/.style={group/vertical sep=5mm},  % padding between rows of groupplots (with ticks only on the x axis)
	vertical sep for naught/.style={group/vertical sep=3mm/2},  % padding between rows of groupplots (without anything)
	vertical sep for ticks,
	groupplot xlabel={Input Length \(n\)},
	title dummy/.style={title=\vphantom{gh|}},  % invisible title that increases the margin; to be used in conjuction with title/.add={…}{…}
	title dummy,
	legend columns=-1,
	every axis plot/.append style={thick},
}

% Automatic export and import of every TikZ picture as PDF.
\usetikzlibrary{external}
\tikzsetexternalprefix{figs/}
%\tikzexternalize
\makeatletter
\renewcommand{\todo}[2][]{\tikzexternaldisable\@todo[#1]{#2}\tikzexternalenable}  % excluding to-do notes from externalisation
\makeatother

\titlehead{Group of Algorithm Engineering\hfill Summer Semester 2024 \\ Institute for Computer Science \\ Goethe University Frankfurt}
\subject{Master's Thesis}
%\title{Sorting \\ for a \\ Processing-in-Memory \\ Architecture}
%\title{Sorting \\ on a \\ Processing-in-Memory \\ Architecture}
%\title{On Sorting for Processing-in-Memory}
\title{On Efficient Sorting Through In-Memory Processing}
%\title{Engineering Sorting Algorithms \\ for a Processing-in-Memory Architecture}
%\title{Engineering Sorting Algorithms \\ for Processing-in-Memory}
%\subtitle{Some cool subtitle if need be}
\subtitle{Implementation and Evaluation \dots{} / Engineering \dots{} / Exploring \dots{}}
%\subtitle{Engineering Algorithms for UPMEM-based DRAM Processing Units}
\author{\texorpdfstring{Ƶ}{Z}eno Adrian \texorpdfstring{\Lss05{W\kern-1.5pt}}{W}eil}
\publishers{\begin{tabular}{r @{~}l}
	Supervisor: & Dr Manuel Penschuck
\end{tabular}}

\makeatletter
\hypersetup{
	pdfauthor=\@author,
	pdftitle=\@title,
	pdfsubject=\@subject,
}
\makeatother

\begin{document}
	\frontmatter

	\maketitle

%	\begin{abstract}
	\noindent
	The growing disparity between processing and memory speed, coupled with increasing data demands, has led to memory accesses being a bottleneck for many modern workflows.
	An example are sorting algorithms, which are often designed around the constraints set by memory subsytems.
	\Acl*{PIM} (also known as processing in memory, \acs*{PIM}) is an umbrella term encompassing several approaches which offload computational tasks to accelerators in or near the memory itself.
	In \acs*{PIM} systems designed and manufactured by \upmem{}, traditional dynamic random-access memory (\acs*{DRAM}) modules are augmented with general-purpose processors called \acfp*{DPU}.
	These are located next to the memory banks themselves, whereby high memory access speed is accomplished.
	An \upmem{}-based \acs*{PIM} system may contain thousands of \acsp*{DPU}, each capable of additional thread-level parallelism.
	Although designed for general use, the \acs*{DPU} architecture does come with limitations to its computational prowess.

	The scope of this thesis is the design, implementation, and evaluation of sorting algorithms which run on a single \acs*{DPU}.
	For several sequential and parallel sorting algorithms, we document the engineering process and adaptations to the merits and shortcomings of the \acs*{DPU} architecture.
	We find that sorting is a suitable task for a \acs*{DPU}, which can be sped up nearly ideally through multithreading.
	This paves the way for more large-scale sorting algorithms which run on multiple \acsp*{DPU}.
\end{abstract}


	\tableofcontents

	\listoftodos

	\mainmatter

	\bigskip
	\todo[inline]{Architektur}
	\todo[inline]{Speicherzugriffe (memcpy, mram\_read \dots)}
	\todo[inline]{triple buffer}

	We took our cue from \citeauthor{axtmann2020engineering}~\cite{axtmann2020engineering} for the choice of distributions.
	\begin{description}
		\item[Sorted]
		The numbers from \(0\) to \(n - 1\) are generated in ascending order.

		\item[Reverse Sorted]
		The numbers from \(0\) to \(n - 1\) are generated in descending order.

		\item[Almost Sorted]
		First, the numbers from \(0\) to \(n - 1\) are generated in ascending order, then, \(\floor{\sqrt{n}}\) random pairs are sequentially drawn and swapped.
		There are no checks on whether pairs have common elements.

		\item[Uniform]
		Each number is drawn independently and uniformly from the range \([0, 2^{31} - 1]\).

		\item[Narrow Uniform]
		Each number is drawn independently and uniformly from the range \([0, n - 1]\).
		\todo{Bisher nicht. Als Ersatz für die Normalverteilung?}

		\item[Zipf's]
		Each number is drawn independently from the range \([1, 100]\), with each value \(k\) drawn with a probability proportional to \(1/k^{0.75}\).

		\item[Normal]
		Each number is drawn independently according to a normal distribution with mean \(\mu = 2^{31}\) and standard deviation \(\sigma = \min\paren*{1, \floor*{n/8}}\).
	\end{description}

	\section{Sorting with One Tasklet}

This section covers the very first phase where each tasklet sorts on its own, \ie{} sequentially.
Unless specified otherwise, every measurement shown in this section was repeated a thousand times on a uniform input distribution with 32-bit integers, and the default configurations of the sorting algorithms were as follows:
\begin{description}
	\item[\IS{}]
	using one explicit sentinel value

	\item[\ShS{}]
	using \(h_1\) sentinel values

	\item[\QS{}]
	iterative implementation;
	switching to \IS{} whenever 13 elements or less remain in a partition;
	median of three as pivot;
	prioritising the right-hand partition over the left-hand partition

	\item[\MS{}]
	half space;
	starting run length of 32 elements

	\item[\HS{}]
	top-down for 32-bit integers;
	bottom-up with swap disparity for 64-bit integers
\end{description}
Further measurements can be found in \cref{apx:tasklet}.

\section{\texorpdfstring{\IS{}}{InsertionSort}}
\label{sec:tasklet:insertion}

\IS{} works by moving the \(i\)th element leftwards as long as its left neighbour is greater, assuming that the elements with indices \(0\) to \(i - 1\) are already sorted \cites[83]{maurer1974datenstrukturen}[Chapter~2.2.1]{wirth1975algorithmen}.
Its asymptotic runtime is above the theoretical minimum of \(\bigoh{n \log n}\), reaching \(\bigoh*{n^2}\) not only in the worst case but also in the average case, since any of the \(\binom{n}{2}\) pairs of input elements is in wrong order, needing to be swapped at some point in the execution, with probability \qty{50}{\percent}.
Nonetheless, \IS{} does have some saving graces:
\begin{enumerate}
	\item
	If the input array is mostly or even fully sorted, the runtime drops down to \(\bigoh{n}\).

	\item
	It works in place, needing only \(\bigoh{1}\) additional space.

	\item
	The sorting is stable.

	\item
	Its implementation is short, lending itself to inlining.

	\item
	The overhead is small.
\end{enumerate}
Especially the last two points make \IS{} a good fallback algorithm for asymptotically better sorting algorithms to use on short subarrays.

\subsection{Presentation of Key Aspects}
\label{sec:tasklet:insertion:aspects}

\paragraph{Sentinel Values}
When moving an element to the left, two checks are needed:
Does the left neighbour exist and is it less than the element to move?
The first check can be omitted through the use of \emph{sentinel values}~\cite[93]{wirth1975algorithmen}:
If the element with index \(-1\) is permanently set to the least possible value of the chosen data type, it is at least as little as any element in the input array, and the leftwards motion stops there at the latest.
Since a \ac{DPU} lacks branch prediction, the slowdown from performing twice as many checks as needed is quite high and goes up to \qty{30}{\percent} for short inputs with 24 uniformly distributed 32-bit elements.

Setting such an \emph{explicit} sentinel value can be omitted by using \emph{implicit} sentinel values~\cite{sanders2019sequential}.
At the start of round~\(i\), one can check whether the element with index \(0\) is at least as little as the element with index~\(i\).
If so, the former is a sufficient sentinel value, and \IS{} can proceed as normal.
If not, the latter must be the minimum of the first \(i + 1\) elements.
Therefore, one can shift the first \(i\) elements one position backwards and place the minimum in the front.
For simplicity, the words \enquote{explicit} and \enquote{implicit} are, henceforth, applied to the word \enquote{\IS{}} directly to imply the type of the sentinel value used.


\subsection*{Investigation of the Compilation}
\label{sec:tasklet:insertion:compilation}
\addcontentsline{toc}{subsection}{\nameref{sec:tasklet:insertion:compilation}}

A common theme when developing for DPUs is a nosediving quality of the compilation.
This is no different for \IS{} upon which shall be shed some light in this \lcnamecref{sec:tasklet:insertion:compilation}.

A naïve implementation of \IS{} begins sorting at the very start of the input and is shown in \cref{fig:insertion:impl:pred_first}.
Obviously, the first element alone is already sorted, so it is algorithmically sound to let \IS{} begin at the second element.
This optimisation is accomplished in \cref{fig:insertion:impl:pred_sec}.
Surprisingly, it leads to a nine instructions longer runtime at 16 integers!
The same happens if, in \cref{fig:insertion:impl:pred_sec}, one keeps \lstinline|*i = start| and instead uses \lstinline|curr = ++i|.

Looking at the compilation reveals the reason:
In the naïve version, the pointer \lstinline|pred| is optimised away and, in its stead, the pointer \lstinline|curr| is passed to all load operations together with a constant offset as second argument.
In the optimised version, the pointer \lstinline|pred| is used with an offset to fetch the values of \lstinline|to_sort| and \lstinline|*pred| at the beginning of each iteration of the outer loop.
Then, the pointer \lstinline|curr| is initialised using the pointer \lstinline|pred| before being used in the inner loop as in the naïve version.
This initialisation is done through one additional \lstinline|move| instruction.
% This is a consequence of reusing the register of the \lstinline|start| pointer for \lstinline|pred| instead of for \lstinline|i|, whose incremented value is put into another register.

These changes fully explain the prolongation of the runtime by nine instructions:
The optimised version loops 15 times in total, each time laboriously initialising the pointer \lstinline|curr|, and executes one \lstinline|add| instruction at the beginning of the function to advance the starting position.
The naïve version loops 16 times, the first time executing seven instructions for naught.

\begin{figure}
	\lstset{basicstyle=\ttfamily\small}
	\def\iscodewidth{0.47\linewidth}
	\begin{subfigure}{\iscodewidth}
		\begin{lstlisting}
void InsertionSort(int *start, int *end) {
	int *curr, *i = start;
	while ((curr = i++) <= end) {
		int to_sort = *curr;
		int *pred = curr - 1;
		while (*pred > to_sort) {
			*curr = *pred;
			curr = pred--;
		}
		*curr = to_sort;
	}
}
		\end{lstlisting}
		\caption{
			Start at the first element and with predecessor pointer.
		}
		\label{fig:insertion:impl:pred_first}
	\end{subfigure}
	\hfill
	\begin{subfigure}{\iscodewidth}
		\begin{lstlisting}
void InsertionSort(int *start, int *end) {
	int *curr, *i = start + 1;
	while ((curr = i++) <= end) {
		int to_sort = *curr;
		int *pred = curr - 1;
		while (*pred > to_sort) {
			*curr = *pred;
			curr = pred--;
		}
		*curr = to_sort;
	}
}
		\end{lstlisting}
		\caption{
			Start at the second element and with predecessor pointer.
		}
		\label{fig:insertion:impl:pred_sec}
	\end{subfigure}

	\begin{subfigure}{\iscodewidth}
		\begin{lstlisting}
void InsertionSort(int *start, int *end) {
	int *curr, *i = start;
	while ((curr = i++) <= end) {
		int to_sort = *curr;
		while (*(curr - 1) > to_sort) {
			*curr = *(curr - 1);
			curr--;
		}
		*curr = to_sort;
	}
}
		\end{lstlisting}
		\caption{
			Start at the first element and without predecessor pointer.
		}
		\label{fig:insertion:impl:offset_first}
	\end{subfigure}
	\hfill
	\begin{subfigure}{\iscodewidth}
		\begin{lstlisting}
void InsertionSort(int *start, int *end) {
	int *curr, *i = start + 1;
	while ((curr = i++) <= end) {
		int to_sort = *curr;
		while (*(curr - 1) > to_sort) {
			*curr = *(curr - 1);
			curr--;
		}
		*curr = to_sort;
	}
}
		\end{lstlisting}
		\caption{
			Start at the second element and without predecessor pointer.
		}
		\label{fig:insertion:impl:offset_sec}
	\end{subfigure}
	\caption{
		Four different implementations of \IS{} in C.
		\Cref{fig:insertion:impl:pred_first,fig:insertion:impl:offset_first} are compiled the same.
		\Cref{fig:insertion:impl:pred_sec,fig:insertion:impl:offset_sec} are compiled differently.
	}
\end{figure}

Multiple workarounds exist to sidestep this problem.
One workaround is to take the unoptimised code and change the starting position via inline assembler.
This is trivial for the explicit \IS{} since one can simply inject an \lstinline|add| instruction at the beginning of the function to increment the pointer \lstinline|start|.
The implicit and the sentinel-less \IS*{} need to know the original starting address \lstinline|start| later on, though, and initialise the actual starting point rather late;
injecting inline assembler proves more difficult as a consequence.
Moreover, as \IS{} is to be used as fallback algorithm by other functions which might also need to keep the original value of \lstinline|start|, inline assembler is a bad choice even for the explicit \IS{}.

Another workaround is the usage of a wrapper function calling \IS{} with the arguments \lstinline|start + 1| and \lstinline|end|.
This works quite well:
First, the register holding \lstinline|start| is incremented, and, then, the inlined code from the actual \IS{} follows.
Doing so makes the runtime drop as expected.

Recall how in the faster version (\cref{fig:insertion:impl:pred_first}), the pointer \lstinline|pred| is always deduced from the pointer \lstinline|curr| using an offset.
This gives the cue for yet another workaround:
In \cref{fig:insertion:impl:offset_first,fig:insertion:impl:offset_sec}, every occurrence of \lstinline|pred| is replaced with \lstinline|curr - 1|.
As a consequence, the code of \cref{fig:insertion:impl:offset_first} compiles the very same as the one of \cref{fig:insertion:impl:pred_first}, while \cref{fig:insertion:impl:offset_sec} yields the same compilation as the versions with the wrapper function or the inline assembly.
This workaround is clearly the best of the three and, hence, the one used in the rest of this thesis.

Alas, the eternal struggle with the compiler endeth not herewith.
A deeper look into the compilation reveals the following sequence:
\begin{center}
	\begin{tabular}{ll}
		\lstinline|move r3, r0| & \makebox[0pt][l]{\textit{// copy content of register \lstinline|r0| to \lstinline|r3|}}
		\\ \lstinline|jleu r4, r2, .LABEL| & \makebox[0pt][l]{\textit{// jump to \lstinline|.LABEL| if \lstinline|r4| \(\le\) \lstinline|r2|}}
		\\ \lstinline|move r3, r0| &
	\end{tabular}
\end{center}
Without delving further into its significance \Dash the second \lstinline|move r3, r0| is unneeded as it is impossible to jump directly to it nor to return via \lstinline|jleu|.
Also, \lstinline|move| does not set any flags like the zero flag or carry flag, as some other instructions do, so such a side effect can be excluded as justification.
Copying the whole assembler code and injecting it as inline assembler but with this second \lstinline|move r3, r0| removed pushes the runtime even further down whilst still sorting correctly.
New issues, especially for inlining, are introduced, though, and we deem a proper assembly implementation as out of scope for this thesis.


\subsection*{Evaluation of the Performance}
\label{sec:tasklet:insertion:performance}
\addcontentsline{toc}{subsection}{\nameref{sec:tasklet:insertion:performance}}

\def\insertionalgos{1NoSentinel,1,1Implicit,BubbleNonAdapt,BubbleAdapt,Selection}

\pgfplotsinvokeforeach{sorted,reverse,almost,uniform,zipf,normal}{
	\pgfplotstablereadnamed{data/small sorts/uint32/#1.txt}{tableSmallSorts_32#1}
	\pgfplotstablereadnamed{data/small sorts/uint64/#1.txt}{tableSmallSorts_64#1}
}

The runtimes of the three \IS*{} can be compared in the \cref{fig:insertion:against_others,fig:insertion:against_others_uint32,fig:insertion:against_others_uint64}.
The sentinel-less \IS{} is consistently worse than the explicit one.
For most input distributions, the implicit \IS{} is also a bit slower, as it effectively performs one check more for each element.
Of course, the gap becomes less significant with increasing input lengths as the loops dominate the runtime.

An outlier, however, are the reverse sorted inputs.
For 32-bit numbers (\cref{fig:insertion:against_others}), the implicit \IS{} is up to 45\% slower than the explicit one.
This comes as a surprise since both versions effectively execute the same loop body while shifting everything one position backwards, with only the loop condition being different.
Due to the uni-cost model, a value check on whether the preceding element is smaller (explicit) and an address check on whether the preceding position is the start of the array (implicit) should take the same amount of time.
Yet, even the \IS{} not relying on sentinel values surpasses the implicit \IS{}, although doing both value checks and address checks!
For 64-bit numbers (\cref{fig:insertion:against_others_uint64}), the implicit \IS{} would be expected to perform better than the explicit one, considering that a value check now takes two instructions and an address check still only one.
Nonetheless, the two \IS*{} tie.
This constitutes another case of bad compilation.
We did not bother with troubleshooting, as the explicit \IS{} would still be expected to offer superior performance in most cases.
The explicit \IS{} is, therefore, used in the rest of this thesis and referred to as plain \enquote{\IS{}} henceforth.

\begin{figure}
	\tikzsetnextfilename{insertion_against_others}
	\begin{tikzpicture}[plot]
		\begin{groupplot}[
			adaptive group=1 by 2,
			groupplot xlabel={Input Length \(n\)},
			groupplot ylabel={Cycles / \(n^2\)},
			xtick distance=3,
			minor xtick=data,
			ymin=0,
			ymax=60,
			legend columns=3,
		]
			\nextgroupplot[title/.add={}{Reverse Sorted}]
			\pgfplotsset{legend to name=leg:insertion:against_others, legend entries={\IS{} (sentinel-less), \IS{} (explicit), \IS{} (implicit), \BS{} (not adaptive), \BS{} (adaptive), \SelS{}}}
			\expandafter\pgfplotsinvokeforeach\expandafter{\insertionalgos}{
				\plotpernn{#1}{tableSmallSorts_32reverse}
			}
			%
			\nextgroupplot[title/.add={}{Uniform}]
			\expandafter\pgfplotsinvokeforeach\expandafter{\insertionalgos}{
				\plotpernn{#1}{tableSmallSorts_32uniform}
			}
		\end{groupplot}
	\end{tikzpicture}

	\hfil\pgfplotslegendfromname{leg:insertion:against_others}\hfil
	\caption{
		Average runtimes of sorting algorithms with a runtime in \(\bigoh{n^2}\) on 32-bit integers.
		The speed-ups are with respect to the \IS{} relying on explicit sentinel values.
	}
	\label{fig:insertion:against_others}
\end{figure}

\begin{note}
	Other known simple sorting algorithm are \SelS{} and \BS{}.
	\emph{\SelS{}} assumes like \IS{} that the elements with indices \(0\) to \(i - 1\) are already sorted in round~\(i\).
	It scans the elements with indices \(i\) to \(n\) and finds their minimum.
	Then, it swaps a minimum element with the element with index \(i\).
	\emph{\BS{}} scans the elements with indices \(0\) to \(n - i + 1\) and swaps each pair of neighbouring elements if they are in the wrong order.
	An easy extension is adaptive \BS{} which stops if no swaps were done during a round.

	The average runtime complexity of \SelS{} and \BS{} is the same as that of \IS{}.
	The asymptoticity, however, hides much higher constant factors such that \IS{} should always be preferred, as seen in \cref{fig:insertion:against_others,fig:insertion:against_others_uint32,fig:insertion:against_others_uint64}.
	Consequently, they will not be expanded on further in this thesis.
\end{note}


\section{\texorpdfstring{\ShS{}}{ShellSort}}
\label{sec:tasklet:shell}

\IS{} suffers from little elements in the back of the input, since those have to be brought to the front through \(\bigtheta{n}\) comparisons and swaps.
\ShS{}~\cites{Shell1959AHS}[Chapter~2.2.4]{wirth1975algorithmen} circumvents this by doing \(k\) passes of \IS{} with decreasing step sizes:
In pass~\(p = 1, \dots, k\) with step size \(\stepsizes_{k - p}\), the input array is divided into \(\stepsizes_{k - p}\) subarrays so that the \(i\)th subarray contains the elements with indices \(\paren{i, \: i + \stepsizes_{k - p}, \: i + 2 \stepsizes_{k - p}, \: \dots}\), for \(0 \le i < \stepsizes_{k - p}\).
These subarrays then get sorted individually through \IS{}.
The final step size is \(\stepsizes_0 = 1\) such that a regular \IS{} is performed.
Intuitively, early \IS*{} are fast as they touch only few elements and little elements in the back are brought forward in large strides.
Later \IS*{} are also fast as elements are close to being sorted.
Like regular \IS{}, \ShS{} also works in place but loses the stability property.

Finding the right balance between the heightened overhead through multiple \IS{} passes and the shortened runtime of each \IS{} pass is subject to research to this day \cite{skean2023optimization,lee2021empirically} and depends on the cost of the operation types (comparing, swapping, looping).
Traditionally, step sizes were constructed mathematically, allowing to determine \ShS{}'s runtime to be, for example, \(\bigoh[\big]{n^{1.2}}\)~\cite[106]{wirth1975algorithmen} or \(\bigoh[\big]{n \log^2 n}\)~\cite[Section 2]{skean2023optimization}, that is better than \IS{}.
Nowadays, well-performing step sizes are identified empirically~\cite{10.1007/3-540-44669-9_12,skean2023optimization,lee2021empirically}, making a generalisation and, thus, asymptotic analysis more difficult.

\subsubsection*{Evaluation of the Performance}
\label{subsubsec:tasklet:shell:performance}

\pgfplotstablereadnamed{data/shell/two-tier/uint32/reverse.txt}{tableShellTwo_32reverse}
\pgfplotstablereadnamed{data/shell/two-tier/uint32/uniform.txt}{tableShellTwo_32uniform}
\pgfplotsinvokeforeach{7,...,17}{
	\pgfplotstablereadnamed{data/shell/h1=#1/uint32/reverse.txt}{tableShell#1_32reverse}
	\pgfplotstablereadnamed{data/shell/h1=#1/uint32/uniform.txt}{tableShell#1_32uniform}
}

Let us first focus on short input arrays where only two passes with step sizes~\(\stepsizes_1\) and \(1\) suffice.
The previous results on \IS{} suggest that the fastest \ShS{} should make use of~\(\stepsizes_1\) sentinel values.
\Cref{fig:shell:two-tier,fig:shell:two-tier_uint32,fig:shell:two-tier_uint64} show that, with the exception of the \ShS{} with step size \(\stepsizes_1 = 2\), the additional passes start to pay off at around 16 elements for both 32-bit and 64-bit values with the fully random input distributions, reaching a speed-up of around 15\% at around 24 elements.
In case of the reverse sorted input, the speed-up is practically always positive even for very short inputs, reaching between 50\% and 110\% at around 24 elements.

\begin{figure}
	\tikzsetnextfilename{shell_two-tier}
	\begin{tikzpicture}[plot]
		\begin{groupplot}[
			horizontal sep for labels,
			adaptive group=1 by 2,
			groupplot xlabel={Input Length \(n\)},
			xtick distance=3,
			minor xtick=data,
		]
			\nextgroupplot[ylabel=Cycles / \(n^2\), ymin=0, ymax=50, ytick distance=10, legend to name=leg:shell:two-tier]
			\legend{\(1\), \(...\), \(9\)}
			\pgfplotsinvokeforeach{1,...,9}{
				\plotpernn[x filter/.expression={x > #1 ? x : nan}]{#1}{tableSmallSorts_32uniform}
			}
			%
			\nextgroupplot[ylabel=Speed-up, ymin=0.7, ymax=1.2, ytick distance=0.1, yticklabel style={/pgf/number format/.cd, precision=1, fixed, zerofill}]
			\pgfplotsset{cycle list shift=1}
			\pgfplotsinvokeforeach{2,...,9}{
				\plotspeedup[x filter/.expression={x > #1 ? x : nan}]{#1}{1}{tableSmallSorts_32uniform}
			}
		\end{groupplot}
	\end{tikzpicture}

	\hfil\pgfplotslegendfromname{leg:shell:two-tier}\hfil
	\caption{
		Comparison of \IS{} (1) and various two-tier \ShS*{} (2--9), whose step sizes \(\stepsizes_1\) are indicated by the labels.
		The speed-ups are with respect to the \IS{}.
	}
	\label{fig:shell:two-tier}
\end{figure}

\NewDocumentCommand{\shellscatter}{m m m}{
	\addplotnamedtable[select row={#1}, forget plot][x=µ_#2, y expr={6}]{tableShellTwo_#3};
	\ifnumless{#2}{7}{
		\addplotnamedtable[select row={#1}, forget plot][x=µ_#2, y expr={7}]{tableShell7_#3};
	}{}
	\ifnumless{#2}{8}{
		\addplotnamedtable[select row={#1}, forget plot][x=µ_#2, y expr={8}]{tableShell8_#3};
	}{}
	\ifnumless{#2}{9}{
		\addplotnamedtable[select row={#1}, forget plot][x=µ_#2, y expr={9}]{tableShell9_#3};
	}{}
	\addplotnamedtable[select row={#1}][x=µ_#2, y expr={10}, forget plot]{tableShell10_#3};
	\addplotnamedtable[select row={#1}][x=µ_#2, y expr={11}, forget plot]{tableShell11_#3};
	\addplotnamedtable[select row={#1}][x=µ_#2, y expr={12}, forget plot]{tableShell12_#3};
	\addplotnamedtable[select row={#1}][x=µ_#2, y expr={13}, forget plot]{tableShell13_#3};
	\addplotnamedtable[select row={#1}][x=µ_#2, y expr={14}, forget plot]{tableShell14_#3};
	\addplotnamedtable[select row={#1}][x=µ_#2, y expr={15}, forget plot]{tableShell15_#3};
	\ifnumgreater{#1}{16}{
		\addplotnamedtable[select row={#1}][x=µ_#2, y expr={16}, forget plot]{tableShell16_#3};
		\addplotnamedtable[select row={#1}][x=µ_#2, y expr={17}]{tableShell17_#3};
	}{
		\addplotnamedtable[select row={#1},opacity=0][x=µ_#2, y expr={17}]{tableShell15_#3};
	}
}

\pgfplotsset{
	shell scatter plot/.style={
		adaptive group=3 by 2,
		groupplot xlabel={Mean [\(10^4\) Cycles]},
		groupplot ylabel={\(\stepsizes_1\)},
		scaled x ticks=base 10:-4,
		xtick scale label code/.code={},  % removes exponent underneath the axis
		ytick={6, 7, 9, ..., 17},
		yticklabels={/, \(7\), \(9\), \(...\), \(17\)},
		/tikz/only marks,
		cycle list shift=2,  % for sharing colours with the previous figure
		ymajorgrids,
		title={Input Length \textit{n} = },
		legend columns=-1,
	},
}

\begin{figure}[p]
	\tikzsetnextfilename{shell_three-tier}
	\begin{tikzpicture}[plot]
		\newcommand{\type}{32uniform}
		\begin{groupplot}[shell scatter plot]
			\pgfplotsinvokeforeach{16,32,48,64,96,128}{
				\nextgroupplot[title/.add={}{ #1}]
				\pgfplotsforeachungrouped\h in {3,...,9}{
					\shellscatter{#1}{\h}{\type}
				}
			}
			\pgfplotsset{legend to name=leg:shell:three-tier}
			\legend{\(3\), \(4\), \(5\), \(6\), \(7\), \(8\), \(9\)}
		\end{groupplot}
	\end{tikzpicture}

	\hfil\pgfplotslegendfromname{leg:shell:three-tier}\hfil
	\caption{
		Runtimes of \ShS*{} with two passes (/) and three passes (7--17).
		The coloured symbols encode the step size \(\stepsizes_1\) for two-tier \ShS*{} and the step size~\(\stepsizes_2\) for three-tier \ShS*{}.
		For the latter, the step size \(\stepsizes_1\) is noted on the y-axes.
	}
	\label{fig:shell:three-tier}
\end{figure}

When moving to greater input lengths (\cref{fig:shell:three-tier,fig:shell:three-tier_uint32reverse,fig:shell:three-tier_uint32uniform,fig:shell:three-tier_uint64reverse,fig:shell:three-tier_uint64uniform}), the differences in performance between the two-tier \ShS*{} become more pronounced;
%especially the ones with \(\stepsizes_1 = 3\) and \(\stepsizes_1 = 4\) fall off whereas the one with \(\stepsizes_1 = 6\) holds its ground quite well.
Between 48 and 64 elements, three passes get worthwhile to consider.
%Interestingly, many \ShS*{} with \(\stepsizes_2 = 4\) take the lead whilst the ones with \(\stepsizes_2 = 6\) are mid-table.
On the one hand, the results are in accordance with \citeauthor{10.1007/3-540-44669-9_12}~\cite{10.1007/3-540-44669-9_12} who, for 128 elements, determined \(\stepsizes = (1, 9)\) to be the optimal pair and \(\stepsizes = (17, 4, 1)\) to be the optimal triplet.
On the other hand, the gain from doing three passes is far smaller:
While \citeauthor{10.1007/3-540-44669-9_12} calculated an average speed-up of 33\% over doing two passes, while it is only 16\% on a DPU.
In opposition to his results, this also makes it unlikely that doing four passes would already net any gain at this input length.
Without access to \citeauthor{10.1007/3-540-44669-9_12}'s original code, giving a satisfactory explanation for the discrepancy is hard, however.

But would pushing the limits of \ShS{} even be rewarding?
Firstly, greater input lengths require greater steps \Dash well into the three digits for \(n \approx 1000\) \cite{skean2023optimization, 10.1007/3-540-44669-9_12} \Dash and those in turn require more sentinel values.
Implicit sentinel values could provide relief since the slow-down from implicitness should trend to zero for longer inputs, as was the case for \IS{}.
Secondly, finding the best step sizes for longer inputs requires a lot more work because the length and, thus, the number of reasonable combinations of step sizes become larger and, as also seen in this subsection, longer optimal tuples cannot be constructed straightforwardly from shorter optimal ones.

Its application, on the other hand, would likely be niche.
\ShS{} is outperformed by other algorithms presented hereafter, and those have no use for a \ShS{} adjusted to longer inputs.
Its only saving grace could be its in-place property combined with medium speed, although even with relying solely on implicit sentinel value, not much space would be saved compared to the fastest sorting algorithm, namely \QS{}.


\section{\texorpdfstring{\QS{}}{QuickSort}}
\label{sec:tasklet:quick}

\QS{} \cite{hoare1962quicksort} uses partitioning to sort in an expected average runtime of \(\bigoh{n \log n}\) and a worst-case runtime of \(\bigoh{n^2}\):
A pivot element is chosen from the input array, then the input array gets scanned and elements greater or lesser than the pivot are moved to the right or left side of the array, respectively.
Finally, \QS{} is used on the left and right side (the \enquote{partitions}).
\QS{} does not sort in-place, as additional space of size \(\bigoh{\log n}\) is needed for a call stack.
Furthermore, \QS{} is not stable.


\paragraph{Sentinel Values}
The partitioning is implemented using \citeauthor{hoare1962quicksort}'s original scheme \cite{hoare1962quicksort}:
At the start of each partitioning step, a pivot \lstinline|p| is chosen and swapped with the last element.
Then, two pointers are set to either end of the partition.
The left pointer \lstinline|i| moves rightwards until finding an element at least as great as the pivot (\lstinline|*i >= p|), while the right pointer \lstinline|j| moves leftwards until finding an element at most as great as the pivot (\lstinline|*l <= p|).
The two elements found are in the wrong order so they are swapped, and the pointers move onwards.
This process continues until the pointers meet.
Finally, the pivot is swapped with the first element of the right partition.

Only an explicit check for whether the pointers have met after stopping is needed.
Since the elements of the partitions to the left are at most as great as the elements of the current partition, they naturally act as bounds check for the pointer moving rightwards.
The pivot at the end acts as bounds check for the pointer moving leftwards.
Since the leftmost partitions have no neighbour to the left, one explicit sentinel values set to the minimum possible value must be placed at the start of the input.
The downside to this approach is that elements equal to the pivot are also swapped.

\paragraph{Base Cases}

\begin{figure}
	\pgfplotstableset{
		create on use/n/.style={create col/copy column from table={data/quick/fallback/uint32/16.txt}{n}},
	}
	\pgfplotsinvokeforeach{14,15,16,17,18,19,20}{
		\pgfplotstableset{create on use/µ_#1_32/.style={create col/copy column from table={data/quick/fallback/uint32/#1.txt}{µ_TrivialBC}}}
		\pgfplotstableset{create on use/µ_#1_64/.style={create col/copy column from table={data/quick/fallback/uint64/#1.txt}{µ_TrivialBC}}}
	}
	\pgfplotstablenew[columns={n,µ_14_32,µ_15_32,µ_16_32,µ_17_32,µ_18_32,µ_19_32,µ_20_32,µ_14_64,µ_15_64,µ_16_64,µ_17_64,µ_18_64,µ_19_64,µ_20_64}]{\pgfplotstablegetrowsof{data/quick/fallback/uint32/16.txt}}{\tableQuickFallback}

	\tikzsetnextfilename{quick_fallback}
	\begin{tikzpicture}[plot]
		\begin{groupplot}[
			horizontal sep for labels,
			adaptive group=1 by 2,
			groupplot ylabel={Speed-up},
			x from 16 to 1024,
			ymin=0.993,
			ymax=1.001,
			extra y ticks={0.993,1.001},
			yticklabel style={/pgf/number format/.cd, precision=3, fixed, zerofill},
		]
			\nextgroupplot[title=32-bit\strut]
			\pgfplotsset{legend to name=leg:quick:fallback, legend entries={15,...,20}}
			\pgfplotsset{update limits=false} \addplot coordinates {(15,0.99)}; \pgfplotsset{update limits=true}
			\pgfplotsinvokeforeach{16,17,18,19,20}{
				\ifnumequal{#1}{18}{
					\addplot coordinates {(18,0.99)};
				}{
					\plotspeedup{#1_32}{18_32}{tableQuickFallback}
				}
			}
			%
			\nextgroupplot[title=64-bit\strut]
			\pgfplotsinvokeforeach{15,16,17,18,19}{
				\ifnumequal{#1}{17}{
					\addplot coordinates {(17,0.99)};
				}{
					\plotspeedup{#1_64}{17_64}{tableQuickFallback}
				}
			}
		\end{groupplot}
	\end{tikzpicture}

	\hfil\pgfplotslegendfromname{leg:quick:fallback}\hfil
	\caption{
		Speed-ups of \QS*{} with different thresholds (15--20) for when to fall back to \IS{} over a threshold of 18 elements (32-bit) and 17 elements (64-bit).
		Using \ShS{} was not beneficial overall, likely because many partitions fall below the thresholds.
	}
	\label{fig:quick:fallback}
\end{figure}

When only a few elements remain in a partition, \QS{}'s overhead predominates such that \IS{} lends itself as fallback algorithm.
As seen in \cref{fig:quick:fallback}, the optimal threshold for switching the sorting algorithm is 18 elements for 32-bit integers on uniform inputs and likely similar on inputs following Zipf's or normal distributions.
For 64-bit integers, the optimal threshold is 17 elements, but we set 18 elements to be the default threshold for both data types to simplify matters since the impact is minuscule.
Up to 40\% of the runtime is saved compared to a \QS{} never falling back.
For sorted and almost sorted inputs, the threshold is higher since \IS{} is very fast on them so falling back earlier and, thus, ending the sorting process is better.
The same is true for reverse sorted inputs even though these are the worst-case inputs for \IS{} because \QS{}'s two pointers invert large swaths of the input.
However, these input distributions should be catered for by a pattern-defeating \QS{} as laid out in \cref{sec:tasklet:conclusion}, hence the 18 elements as default threshold.

To avoid unnecessary uses of \IS{}, another base case is imaginable, namely terminating when a partition contains at most 1 elements.
There are tremendous consequences for the runtime depending on the exact implementation of the base cases, as shown later in \enquote{\nameref{sec:tasklet:quick:compilation}}.


\paragraph{Recursion vs.\ Iteration}
In theory, the question of whether a DPU algorithm should be implemented recursively or iteratively comes down to convenience.
Due to the uniform costs of instructions, jumping to the start of a loop or to the start of a function essentially costs the same, as does managing arguments automatically through the regular call stack and manually through a simulated one.
Furthermore, in case of \QS{}, the compiler turns tail-recursive calls into jumps back to the function start, so that one partition is sorted recursively and the other iteratively.
All this would suggest a recursive implementation due to the reduced maintenance.

In practice, it comes down to the compilation.
Even parts of the algorithms which are independent from the choice between recursion and iteration can be compiled differently, such that there are implementations where iteration is faster than recursion and the other way around.
Overall though, iterative implementations \emph{tend} to be compiled better with superior register usage and less instructions used for the actual \QS{} algorithm.


\paragraph{Partition Prioritisation}
Whether the left-hand or the right-hand partition is sorted first should not make any difference for the runtime but actually does so because of different compilation, as shown later in \enquote{\nameref{sec:tasklet:quick:compilation}}.
Always sorting the shorter partition first and putting the longer partition on the call stack guarantees that the problem size is at least halved each step, so that the call stack stores \(\bigoh{\log n}\) elements at most.
This approach, however, is linked to huge speed penalties, which is why it is advisable to always prioritise the same side;
in this Thesis, the right-hand partitions are prioritised.
An overflow of the call stack becomes unlikely with the right pivot choice.


\paragraph{Pivot Choice}
Another parameter to tune is the way in which the pivot is chosen.
The following were implemented and tested:
\begin{itemize}
	\item
	Using the \emph{last element} is the fastest way, requiring zero additional instructions.

	\item
	Taking the \emph{deterministic median} of three elements, namely the first, middle, and last one, is far more computationally expensive since the position of the middle element must be calculated, the median be determined, and the pivot be swapped with the last element of the array, where it acts as sentinel.

	\item
	A \emph{random element} is most efficiently drawn using an xorshift random number generator and rejection sampling \cite{lukas_geis}.

	\item
	The \emph{random median} is a combination of the previous two methods, where the median of three random elements is taken.
	For simplicity, there is no check on whether an element is drawn twice or thrice.
	Since the partitions are rather long, this should happen seldom, anyhow.
\end{itemize}
A median increases the chances of choosing a pivot that is neither particularly high nor particularly low.
This leads to more balanced partitions such that the call stack is less likely to overflow and the base cases are reached faster.
But even then it is still possible to construct inputs where the runtime climbs up to \(\bigtheta{n^2}\) \cite{erkiö1984worstcase}, as everything is moved to the same partition so that the problem size is reduced by only one element (namely the pivot) after each partitioning step.

The random pivots circumvent this problem.
Whilst the pivots could, by ill luck, also lead to the same unbalanced partitions as the deterministic pivots, the worst-case expected runtime is \(\bigoh{n \log n}\) \cite{blum2011probabilistic}.
Using the median of medians \cite{blum1973median} could guarantee a runtime of \(\bigoh{n \log n}\) but was not implemented because a performant implementation would probably be quite complex and its benefit minuscule for this Thesis.

The general trend, as seen in \enquote{\nameref{sec:tasklet:quick:compilation}}, is the following:
A median gets more beneficial for the average runtime, the longer the input becomes, and leads to small pay-offs in the end.
Moreover, the standard deviations of the runtimes are cut roughly in half, although not shown in the figures of \enquote{\nameref{sec:tasklet:quick:compilation}} for reasons of clarity.
If the input is known to be fairly random, a deterministic choice yields a noticeably speed-up.
However, the gain remains in the single digits percentage-wise, supporting the findings by \citeauthor{lukas_geis}~\cite{lukas_geis} that drawing random numbers is quite cheap.
For this reason, the median of three random elements is used as default configuration throughout this Thesis.


\subsubsection*{Investigation of the Compilation}
\label{subsubsec:tasklet:quick:compilation}

%\pgfplotstablereadnamed{data/quick/rec_vs_it.txt}{tableQuickRecVsIter}
%\pgfplotstablereadnamed{data/quick/recursive/no switched sides/uniform/end.txt}{tableQuickRecNssUniEnd}
%\pgfplotstablereadnamed{data/quick/recursive/no switched sides/uniform/middle.txt}{tableQuickRecNssUniMiddle}
%\pgfplotstablereadnamed{data/quick/recursive/no switched sides/uniform/median_of_three.txt}{tableQuickRecNssUniMedian}
%\pgfplotstablereadnamed{data/quick/recursive/no switched sides/uniform/random.txt}{tableQuickRecNssUniRandom}
%\pgfplotstablereadnamed{data/quick/recursive/switched sides/uniform/end.txt}{tableQuickRecSsUniEnd}
%\pgfplotstablereadnamed{data/quick/recursive/switched sides/uniform/middle.txt}{tableQuickRecSsUniMiddle}
%\pgfplotstablereadnamed{data/quick/recursive/switched sides/uniform/median_of_three.txt}{tableQuickRecSsUniMedian}
%\pgfplotstablereadnamed{data/quick/recursive/switched sides/uniform/random.txt}{tableQuickRecSsUniRandom}
%\pgfplotstablereadnamed{data/quick/iterative/no switched sides/uniform/end.txt}{tableQuickIterNssUniEnd}
%\pgfplotstablereadnamed{data/quick/iterative/no switched sides/uniform/middle.txt}{tableQuickIterNssUniMiddle}
%\pgfplotstablereadnamed{data/quick/iterative/no switched sides/uniform/median_of_three.txt}{tableQuickIterNssUniMedian}
%\pgfplotstablereadnamed{data/quick/iterative/no switched sides/uniform/random.txt}{tableQuickIterNssUniRandom}
%\pgfplotstablereadnamed{data/quick/iterative/switched sides/uniform/end.txt}{tableQuickIterSsUniEnd}
%\pgfplotstablereadnamed{data/quick/iterative/switched sides/uniform/middle.txt}{tableQuickIterSsUniMiddle}
%\pgfplotstablereadnamed{data/quick/iterative/switched sides/uniform/median_of_three.txt}{tableQuickIterSsUniMedian}
%\pgfplotstablereadnamed{data/quick/iterative/switched sides/uniform/random.txt}{tableQuickIterSsUniRandom}

The quality of the compilation and thus the real performance of \QS{} is erratic to such an extent that one implementation variant may see a speed-up of 25\% over another one even with the same pivot choice although virtually none would be expected.
As hinted in the preceding paragraphs, this raises the need for a benchmark suite with the following parameters:
base case handling, recursion/iteration, pivot choice, and partition prioritisation.
Before the results are discussed, the first parameter shall be explained in more depth.

Besides falling back to \IS{} if 13 elements remain (\enquote{treshold undercut}), another base case is imaginable, namely a full termination if 1, 0, or --1 elements remain (\enquote{trivial length}).
Theoretically, it should not be needed to check for trivial lengths because even though it is doable with just one additional instruction, such short partitions are rare and the \IS{} would terminate after a few instructions anyway.
Nonetheless, its inclusion or exclusion can have significant impacts.
The following \nameCrefs{imp:normal} were tested:
\begin{enumerate}
	\item\label[implementation]{imp:normal}
	If the length is trivial, terminate.
	If not and if the threshold is undercut, sort with \IS{}.
	Otherwise, sort with \QS{} and use \QS{} on both partitions.
%	\textcolor{red}{[Normal]}

	\item\label[implementation]{imp:triviality_within_threshold}
	If the threshold is undercut, check if the length is trivial and terminate or sort with \IS{}, respectively.
	Otherwise, sort with \QS{} and use \QS{} on both partitions.
%	\textcolor{red}{[TrivInThresh]}
	\begin{itemize}
		\item
		This \nameCref{imp:triviality_within_threshold} significantly reduces the number of checks for trivial length.
	\end{itemize}

	\item\label[implementation]{imp:no_triviality}
	If the threshold is undercut, sort with \IS{}.
	Otherwise, sort with \QS{} and use \QS{} on both partitions.
%	\textcolor{red}{[NoTrivial]}
	\begin{itemize}
		\item
		This \nameCref{imp:no_triviality} forgoes the check for a trivial length completely, at the cost of unneeded \IS*{}.
	\end{itemize}

	\item\label[implementation]{imp:threshold_then_triviality}
	If the threshold is undercut, sort with \IS{}.
	If not and if the length is trivial, terminate.
	Otherwise, sort with \QS{} and use \QS{} on both partitions.
%	\textcolor{red}{[ThreshThenTriv]}
	\begin{itemize}
		\item
		This \nameCref{imp:threshold_then_triviality}, while nonsensical from a logical point of view, gives the compiler an explicit guarantee that the partitioning loop does not end immediately.
	\end{itemize}

	\item\label[implementation]{imp:triviality_before_call}
	If the threshold is undercut, sort with \IS{}.
	Otherwise, sort with \QS{}.
	Then check for either partition if its length is trivial and use \QS{} if not.
%	\textcolor{red}{[TrivialBC]}
	\begin{itemize}
		\item
		This \nameCref{imp:triviality_before_call}, as well as the next two, gets rid of some unneeded uses of \QS{}.
		In the recursive case, these \nameCrefs{imp:triviality_before_call} lose the property of being tail-recursive.
	\end{itemize}

	\item\label[implementation]{imp:threshold_before_call}
	Sort with \QS{}.
	Check for either partition if the threshold is undercut and use \IS{} or \QS{}, respectively.
%	\textcolor{red}{[ThreshBC]}

	\item\label[implementation]{imp:threshold_and_triviality_before_call}
	Sort with \QS{}.
	Check for either partition if its length is trivial or if the threshold is undercut and use \IS{}, \QS{}, or nothing, respectively.
%	\textcolor{red}{[ThreshTrivBC]}

	\item\label[implementation]{imp:one_insertion}
	If the threshold is undercut, terminate.
	Otherwise, sort with \QS{} and use \QS{} on both partitions.
	After all \QS*{} are done, sort the whole input array with \IS{}.
%	\textcolor{red}{[OneInsertion]}
	\begin{itemize}
		\item
		This \nameCref{imp:one_insertion} always does one pass of \IS{}.
		For example, the other \nameCrefs{imp:normal} do roughly 90 at 1024 elements.
	\end{itemize}
\end{enumerate}

All results are shown in \cref{fig:quick:implementations}.
When using recursion, \cref{imp:normal,imp:triviality_before_call} perform the best, especially for longer inputs.
Their compilations are fundamentally the same, including the conversion of the second recursive call into a jump back to the function start.
All other \nameCrefs{imp:normal} fare vastly worse.
Common occurrences are \dots{}
\begin{itemize}
	\item
	\dots{} one more instruction in the loop finding the next element to move to the right, \dots{}

	\item
	\dots{} one more instruction after such an element has been found, \dots{}

	\item
	\dots{} more stores and loads when entering and leaving the function.
\end{itemize}

\todo[inline]{%
	Ich kann mir leider nicht alles erklären.
	Als Beispiel habe ich die Kompilate von \cref{imp:normal} / Recursive / Last für Left First und Right First hochgeladen (die verkürzten Varianten besitzen nur noch Befehle und Sprungmarken).
	Ersteres ist ja die schnellste rekursive Variante, während letzteres deutlich schlechter abschneidet.
	Dennoch sehe ich bei der langsameren Variante keinen fundamental anderen Algorithmus.
	Je Funktionsaufruf kommen ≈3 Extra-Aufrufe hinzu (bei insgesamt ≈104 rekursiven und ≈104 \enquote{Endaufrufen} bei 1024 Elementen), was fast 70\,000 Takte Unterschied nicht erklären kann.
}

%\begin{figure}[p]
%	\def\algos{Normal,TrivInThresh,NoTrivial,ThreshThenTriv,TrivialBC,ThreshBC,ThreshTrivBC,OneInsertion}
%	\pgfplotsset{
%		height=3.5cm,
%		horizontal sep for naught,
%		adaptive group=2 by 4,
%		groupplot xlabel={Input Length \(n\)},
%		groupplot ylabel={Cycles / \((n \lb n)\)},
%		xmode=log,
%		xtick={20, 64, 256, 1024},
%		xticklabels={\(20\), \(64\), \(256\), \(1024\)},
%		minor xtick={32, 128, 512},
%		ymin=55,
%		ymax=80,
%		ytick distance=5,
%		legend columns=-1,
%	}
%	\begin{subfigure}{\textwidth}
%		\tikzsetnextfilename{quick_implementations_rec}
%		\begin{tikzpicture}[plot]
%			\begin{groupplot}
%				\nextgroupplot[title={Last | Left First}, legend to name=leg:quick:implementations]
%				\legend{\ref{imp:normal}, \ref{imp:triviality_within_threshold}, \ref{imp:no_triviality}, \ref{imp:threshold_then_triviality}, \ref{imp:triviality_before_call}, \ref{imp:threshold_before_call}, \ref{imp:threshold_and_triviality_before_call}, \ref{imp:one_insertion}}
%				\expandafter\pgfplotsinvokeforeach\expandafter{\algos}{
%					\plotpernlogn{#1}{tableQuickRecNssUniEnd}
%				}
%				%
%				\nextgroupplot[title={Middle | Left First}, yticklabels={}]
%				\expandafter\pgfplotsinvokeforeach\expandafter{\algos}{
%					\plotpernlogn{#1}{tableQuickRecNssUniMiddle}
%				}
%				%
%				\nextgroupplot[title={Median | Left First}, yticklabels={}]
%				\expandafter\pgfplotsinvokeforeach\expandafter{\algos}{
%					\plotpernlogn{#1}{tableQuickRecNssUniMedian}
%				}
%				%
%				\nextgroupplot[title={Random | Left First}, yticklabel pos=right]
%				\expandafter\pgfplotsinvokeforeach\expandafter{\algos}{
%					\plotpernlogn{#1}{tableQuickRecNssUniRandom}
%				}
%				%
%				\nextgroupplot[title={Last | Right First}]
%				\expandafter\pgfplotsinvokeforeach\expandafter{\algos}{
%					\plotpernlogn{#1}{tableQuickRecSsUniEnd}
%				}
%				%
%				\nextgroupplot[title={Middle | Right First}, yticklabels={}]
%				\expandafter\pgfplotsinvokeforeach\expandafter{\algos}{
%					\plotpernlogn{#1}{tableQuickRecSsUniMiddle}
%				}
%				%
%				\nextgroupplot[title={Median | Right First}, yticklabels={}]
%				\expandafter\pgfplotsinvokeforeach\expandafter{\algos}{
%					\plotpernlogn{#1}{tableQuickRecSsUniMedian}
%				}
%				%
%				\nextgroupplot[title={Random | Right First}, yticklabel pos=right]
%				\expandafter\pgfplotsinvokeforeach\expandafter{\algos}{
%					\plotpernlogn{#1}{tableQuickRecSsUniRandom}
%				}
%			\end{groupplot}
%		\end{tikzpicture}
%		\caption{
%			Recursive Approach
%		}
%		\bigskip
%	\end{subfigure}
%	%
%	\begin{subfigure}{\textwidth}
%		\tikzsetnextfilename{quick_implementations_it}
%		\begin{tikzpicture}[plot]
%			\begin{groupplot}
%				\nextgroupplot[title={Last | Left First}]
%				\expandafter\pgfplotsinvokeforeach\expandafter{\algos}{
%					\plotpernlogn{#1}{tableQuickIterNssUniEnd}
%				}
%				%
%				\nextgroupplot[title={Middle | Left First}, yticklabels={}]
%				\expandafter\pgfplotsinvokeforeach\expandafter{\algos}{
%					\plotpernlogn{#1}{tableQuickIterNssUniMiddle}
%				}
%				%
%				\nextgroupplot[title={Median | Left First}, yticklabels={}]
%				\expandafter\pgfplotsinvokeforeach\expandafter{\algos}{
%					\plotpernlogn{#1}{tableQuickIterNssUniMedian}
%				}
%				%
%				\nextgroupplot[title={Random | Left First}, yticklabel pos=right]
%				\expandafter\pgfplotsinvokeforeach\expandafter{\algos}{
%					\plotpernlogn{#1}{tableQuickIterNssUniRandom}
%				}
%				%
%				\nextgroupplot[title={Last | Right First}]
%				\expandafter\pgfplotsinvokeforeach\expandafter{\algos}{
%					\plotpernlogn{#1}{tableQuickIterSsUniEnd}
%				}
%				%
%				\nextgroupplot[title={Middle | Right First}, yticklabels={}]
%				\expandafter\pgfplotsinvokeforeach\expandafter{\algos}{
%					\plotpernlogn{#1}{tableQuickIterSsUniMiddle}
%				}
%				%
%				\nextgroupplot[title={Median | Right First}, yticklabels={}]
%				\expandafter\pgfplotsinvokeforeach\expandafter{\algos}{
%					\plotpernlogn{#1}{tableQuickIterSsUniMedian}
%				}
%				%
%				\nextgroupplot[title={Random | Right First}, yticklabel pos=right]
%				\expandafter\pgfplotsinvokeforeach\expandafter{\algos}{
%					\plotpernlogn{#1}{tableQuickIterSsUniRandom}
%				}
%			\end{groupplot}
%		\end{tikzpicture}
%		\caption{
%			Iterative approach
%		}
%	\end{subfigure}
%
%	\bigskip
%	\hfil\pgfplotslegendfromname{leg:quick:implementations}\hfil
%	\caption{
%		Comparison of the different implementations (1--8) of \QS{} for all possible pivot choices.
%		In the first rows, the left-hand partitions are sorted before the right-hand ones, while it is the reverse in the second rows.
%	}
%	\label{fig:quick:implementations}
%\end{figure}
%
%\begin{figure}
%	\tikzsetnextfilename{quick_rec_vs_it}
%	\begin{tikzpicture}[plot]
%		\begin{groupplot}[
%			horizontal sep for labels,
%			adaptive group=1 by 2,
%			groupplot xlabel={Input Length \(n\)},
%			xmode=log,
%			xtick={20, 32, 64, 128, 256, 512, 1024},
%			xticklabels={\(20\), \(32\), \(64\), \(128\), \(256\), \(512\), \(1024\)},
%			legend columns=-1,
%		]
%			\nextgroupplot[ylabel=Cycles / \((n \lb n)\), ymin=55, ymax=65, extra y ticks={55, 65}, legend to name=leg:rec_vs_it]
%			\legend{Iterative, Recursive}
%			\plotpernlogn{It}{tableQuickRecVsIter}
%			\plotpernlogn{Rec}{tableQuickRecVsIter}
%			%
%			\nextgroupplot[ylabel=Speed-up, ymin=0.96, ymax=1.04, /pgf/number format/.cd, precision=2, fixed zerofill=true]
%			\plotspeedup{It}{Rec}{tableQuickRecVsIter}
%		\end{groupplot}
%	\end{tikzpicture}
%
%	\hfil\pgfplotslegendfromname{leg:rec_vs_it}\hfil
%	\caption{
%		Comparison of the fastest recursive and iterative \QS*{} (cf. \cref{subsubsec:tasklet:quick:compiler}).
%		The actual algorithm is compiled the very same in both cases, so that time differences are only due to the way \QS{} is applied to the partitions.
%		\todo{nicht mehr wegen des Pivots!}
%	}
%	\label{fig:rec_vs_it}
%\end{figure}

\clearpage

\def\quickpivots{LAST,MEDIAN,RANDOM,MEDIAN_OF_RANDOM}
\expandafter\pgfplotsinvokeforeach\expandafter{\quickpivots}{
	\pgfplotstablereadnamed{data/quick/matrix/iterative/#1/shorter/uniform.txt}{tableQuickMatrixIt#1Shorter}
	\pgfplotstablereadnamed{data/quick/matrix/iterative/#1/left/uniform.txt}{tableQuickMatrixIt#1Left}
	\pgfplotstablereadnamed{data/quick/matrix/iterative/#1/right/uniform.txt}{tableQuickMatrixIt#1Right}

	\pgfplotstablereadnamed{data/quick/matrix/recursive/#1/shorter/uniform.txt}{tableQuickMatrixRec#1Shorter}
	\pgfplotstablereadnamed{data/quick/matrix/recursive/#1/left/uniform.txt}{tableQuickMatrixRec#1Left}
	\pgfplotstablereadnamed{data/quick/matrix/recursive/#1/right/uniform.txt}{tableQuickMatrixRec#1Right}
}

\tikzexternaldisable

\begin{figure}[p]
	\pgfplotsset{
		height=2.6cm,
		horizontal sep for naught,
		vertical sep for naught,
		adaptive group=3 by 4,
		groupplot xlabel={Input Length \(n\)},
		groupplot ylabel={Cycles / \((n \lb n)\)},
		xmode=log,
		xtick={16, 64, 256, 1024},
		xticklabels={\(16\), \(64\), \(256\), \(1024\)},
		minor xtick={32, 128, 512},
		ymin=55,
		ymax=80,
		legend columns=-1,
	}
	\def\quickalgos{Normal,TrivInThresh,NoTrivial,ThreshThenTriv,TrivialBC,ThreshBC,ThreshTrivBC,OneInsertion}
	\captionsetup[subfigure]{aboveskip=0mm,belowskip=1mm}
	\begin{subfigure}{\textwidth}
		\tikzsetnextfilename{quick_implementations_rec}
		\begin{tikzpicture}[plot]
			\begin{groupplot}
				\nextgroupplot[title=Last, xticklabels={}, legend to name=leg:quick:implementations]
				\legend{\ref{imp:normal}, \ref{imp:triviality_within_threshold}, \ref{imp:no_triviality}, \ref{imp:threshold_then_triviality}, \ref{imp:triviality_before_call}, \ref{imp:threshold_before_call}, \ref{imp:threshold_and_triviality_before_call}, \ref{imp:one_insertion}}
				\expandafter\pgfplotsinvokeforeach\expandafter{\quickalgos}{
					\plotpernlogn{#1}{tableQuickMatrixRecLASTLeft}
				}
				\nextgroupplot[title=Median, xticklabels={}, yticklabels={}]
				\expandafter\pgfplotsinvokeforeach\expandafter{\quickalgos}{
					\plotpernlogn{#1}{tableQuickMatrixRecMEDIANLeft}
				}
				\nextgroupplot[title=Random, xticklabels={}, yticklabels={}]
				\expandafter\pgfplotsinvokeforeach\expandafter{\quickalgos}{
					\plotpernlogn{#1}{tableQuickMatrixRecRANDOMLeft}
				}
				\nextgroupplot[title=Median (Random), xticklabels={}, yticklabel pos=right]
				\expandafter\pgfplotsinvokeforeach\expandafter{\quickalgos}{
					\plotpernlogn{#1}{tableQuickMatrixRecMEDIAN_OF_RANDOMLeft}
				}
				%
				\nextgroupplot[xticklabels={}]
				\expandafter\pgfplotsinvokeforeach\expandafter{\quickalgos}{
					\plotpernlogn{#1}{tableQuickMatrixRecLASTRight}
				}
				\nextgroupplot[xticklabels={}, yticklabels={}]
				\expandafter\pgfplotsinvokeforeach\expandafter{\quickalgos}{
					\plotpernlogn{#1}{tableQuickMatrixRecMEDIANRight}
				}
				\nextgroupplot[xticklabels={}, yticklabels={}]
				\expandafter\pgfplotsinvokeforeach\expandafter{\quickalgos}{
					\plotpernlogn{#1}{tableQuickMatrixRecRANDOMRight}
				}
				\nextgroupplot[xticklabels={}, yticklabel pos=right]
				\expandafter\pgfplotsinvokeforeach\expandafter{\quickalgos}{
					\plotpernlogn{#1}{tableQuickMatrixRecMEDIAN_OF_RANDOMRight}
				}
				%
				\nextgroupplot
				\expandafter\pgfplotsinvokeforeach\expandafter{\quickalgos}{
					\plotpernlogn{#1}{tableQuickMatrixRecLASTShorter}
				}
				\nextgroupplot[yticklabels={}]
				\expandafter\pgfplotsinvokeforeach\expandafter{\quickalgos}{
					\plotpernlogn{#1}{tableQuickMatrixRecMEDIANShorter}
				}
				\nextgroupplot[yticklabels={}]
				\expandafter\pgfplotsinvokeforeach\expandafter{\quickalgos}{
					\plotpernlogn{#1}{tableQuickMatrixRecRANDOMShorter}
				}
				\nextgroupplot[yticklabel pos=right]
				\expandafter\pgfplotsinvokeforeach\expandafter{\quickalgos}{
					\plotpernlogn{#1}{tableQuickMatrixRecMEDIAN_OF_RANDOMShorter}
				}
			\end{groupplot}
		\end{tikzpicture}
		\caption{
			Recursive Approach
		}
	\end{subfigure}
	\begin{subfigure}{\textwidth}
		\tikzsetnextfilename{quick_implementations_it}
		\begin{tikzpicture}[plot]
			\begin{groupplot}
				\nextgroupplot[title=Last, xticklabels={}]
				\expandafter\pgfplotsinvokeforeach\expandafter{\quickalgos}{
					\plotpernlogn{#1}{tableQuickMatrixItLASTLeft}
				}
				\nextgroupplot[title=Median, xticklabels={}, yticklabels={}]
				\expandafter\pgfplotsinvokeforeach\expandafter{\quickalgos}{
					\plotpernlogn{#1}{tableQuickMatrixItMEDIANLeft}
				}
				\nextgroupplot[title=Random, xticklabels={}, yticklabels={}]
				\expandafter\pgfplotsinvokeforeach\expandafter{\quickalgos}{
					\plotpernlogn{#1}{tableQuickMatrixItRANDOMLeft}
				}
				\nextgroupplot[title=Median (Random), xticklabels={}, yticklabel pos=right]
				\expandafter\pgfplotsinvokeforeach\expandafter{\quickalgos}{
					\plotpernlogn{#1}{tableQuickMatrixItMEDIAN_OF_RANDOMLeft}
				}
				%
				\nextgroupplot[xticklabels={}]
				\expandafter\pgfplotsinvokeforeach\expandafter{\quickalgos}{
					\plotpernlogn{#1}{tableQuickMatrixItLASTRight}
				}
				\nextgroupplot[xticklabels={}, yticklabels={}]
				\expandafter\pgfplotsinvokeforeach\expandafter{\quickalgos}{
					\plotpernlogn{#1}{tableQuickMatrixItMEDIANRight}
				}
				\nextgroupplot[xticklabels={}, yticklabels={}]
				\expandafter\pgfplotsinvokeforeach\expandafter{\quickalgos}{
					\plotpernlogn{#1}{tableQuickMatrixItRANDOMRight}
				}
				\nextgroupplot[xticklabels={}, yticklabel pos=right]
				\expandafter\pgfplotsinvokeforeach\expandafter{\quickalgos}{
					\plotpernlogn{#1}{tableQuickMatrixItMEDIAN_OF_RANDOMRight}
				}
				%
				\nextgroupplot
				\expandafter\pgfplotsinvokeforeach\expandafter{\quickalgos}{
					\plotpernlogn{#1}{tableQuickMatrixItLASTShorter}
				}
				\nextgroupplot[yticklabels={}]
				\expandafter\pgfplotsinvokeforeach\expandafter{\quickalgos}{
					\plotpernlogn{#1}{tableQuickMatrixItMEDIANShorter}
				}
				\nextgroupplot[yticklabels={}]
				\expandafter\pgfplotsinvokeforeach\expandafter{\quickalgos}{
					\plotpernlogn{#1}{tableQuickMatrixItRANDOMShorter}
				}
				\nextgroupplot[yticklabel pos=right]
				\expandafter\pgfplotsinvokeforeach\expandafter{\quickalgos}{
					\plotpernlogn{#1}{tableQuickMatrixItMEDIAN_OF_RANDOMShorter}
				}
			\end{groupplot}
		\end{tikzpicture}
		\caption{
			Iterative Approach
		}
	\end{subfigure}

	\hfil\pgfplotslegendfromname{leg:quick:implementations}\hfil
	\caption{
		Comparison of \crefrange{imp:normal}{imp:one_insertion}.
		The left-hand partition is preferred in the first rows, the right-hand one in the second rows, and the shorter one in the third rows.
	}
	\label{fig:quick:implementations}
\end{figure}

\tikzexternalenable

\clearpage

\subsection*{Evaluation of the Performance}
\label{sec:tasklet:quick:performance}
\addcontentsline{toc}{subsection}{\nameref{sec:tasklet:quick:performance}}

\pgfplotsinvokeforeach{sorted,reverse,almost,uniform,zipf,normal}{
	\pgfplotstablereadnamed{data/quick/matrix/iterative/Median_of_random/right/uint32/#1.txt}{tableQuickRand_32#1}
	\pgfplotstablereadnamed{data/quick/matrix/recursive/Median_of_random/right/uint64/#1.txt}{tableQuickRand_64#1}
}

Before turning to the performance of \QS{} on specific input distributions, the ratio between costs and benefits of the pivot choices shall be evaluated.
Looking again at \cref{fig:quick:implementations,fig:heap:runtime_uint64} shows that a median gets more beneficial, the longer the input becomes, achieving small pay-offs for the longest ones.
Moreover, the standard deviations of the runtimes, although not shown in the figures for reasons of clarity, are cut roughly in half.
Randomisation slows down noticeably, so random pivots are disadvantageous if the input is known to be fairly random.
However, the decrease remains in the single digits percentage-wise, supporting the findings by \citeauthor{lukas_geis}~\cite{lukas_geis} that drawing random numbers is quite cheap.
For this reason, the random median is used as default method throughout this thesis.

\begin{figure}
	\tikzsetnextfilename{quick_runtime}
	\begin{tikzpicture}[plot]
		\begin{groupplot}[
			adaptive group=1 by 2,
			groupplot ylabel={Cycles / \((n \lb n)\)},
			x from 16 to 1024,
			ytick distance=10,
		]
			\nextgroupplot[title/.add={}{32-bit}, ymin=30, ymax=80]
			\pgfplotsset{legend to name=leg:quick:runtime, legend entries={Sorted, Reverse S., Almost S., Uniform, Zipf's, Normal}}
			\pgfplotsinvokeforeach{sorted,reverse,almost,uniform,zipf,normal}{
				\plotpernlogn{TrivialBC}{tableQuickRand_32#1}
			}
			%
			\nextgroupplot[title/.add={}{64-bit}, ymin=40,ymax=90]
			\pgfplotsinvokeforeach{sorted,reverse,almost,uniform,zipf,normal}{
				\plotpernlogn{TrivialBC}{tableQuickRand_64#1}
			}
		\end{groupplot}
	\end{tikzpicture}

	\hfil\pgfplotslegendfromname{leg:quick:runtime}\hfil
	\caption{
		Mean runtime of \QS{} on all tested input distributions and data types.
	}
	\label{fig:quick:runtime}
\end{figure}

\Cref{fig:quick:runtime} shows the runtime of \QS{} in it default configuration, that is, with random medians.
\Cref{fig:quick:runtime_uint32,fig:quick:runtime_uint64} additionally contain the runtimes with deterministic medians as well as the standard deviations of the measurements.
The mean runtimes are rather close across all input distributions, a consequence of using random medians and of considering elements equal to the pivot as different.
In fact, it is \IS{} that primarily causes the discrepancies, as setting the threshold to one element proves.
This also explains why \QS{} performs so well on large inputs with Zipf's distribution:
This distributions generates many duplicates, which are put into the same partitions, so \IS{} performs many simple scans.

One might expect \QS{} to perform even better on sorted and reverse sorted input, since everything is either already in the correct position or because the two pointers quickly invert large swaths of the inputs.
However, a side effect of swapping the pivot twice can be that many elements are displaced by one position from where they should be in the sorted input.
Take reverse sorted inputs and the deterministic median as an example:
The element \(n/2\) is chosen as pivot out of the elements \(n\), \(n/2\), and \(0\) and then gets swapped with the last element, that is, with \(0\).
Thereupon, the pointers invert the rest of the input such that the start of the input looks something like \(1, 2, \dots, n/2-1, 0, n/2, \dots\) after the first partitioning step.
Indeed, this pattern makes \QS{} with deterministic medians degrade and eventually overflow the call stack, which is why the respective plots in \cref{fig:quick:runtime_uint32,fig:quick:runtime_uint64} leave the charts.
An implementation without swapping the pivot promises better performance for such cases, but in exploratory ones, the performance on more random input distributions suffered drastically.


\section{\texorpdfstring{\MS{}}{MergeSort}}
\label{sec:tasklet:merge}

Two-way bottom-up \MS{}~\cites{katajainen1997meticulous}[85\psq]{maurer1974datenstrukturen}[Chapter~2.3.1]{wirth1975algorithmen} repeatedly compares two elements and \emph{merges} them to form sorted pairs.
Once only pairs (and perhaps one single element) remain, the pairs are merged into quadruplets, the quadruples into octuplets and so on until a single sorted sequence remains.
Sorted sequences are also referred to as \emph{runs}.
\MS{} has a guaranteed runtime of \(\bigoh{n \log n}\) and is the only stable sorting algorithm with subquadratic runtime presented in this \lcnamecref{sec:tasklet}.
Its biggest drawback is that additional space of size \(\bigtheta{n}\) is needed.


\paragraph{Starting Runs}
Instead of starting by merging runs of length 1, that is individual elements, it is beneficial to first subdivide the input and use either \IS{} or \ShS{} on the individual subdivisions.
These larger \emph{starting runs} allow to skip a few of the early rounds of merging.
For simplicity, all starting runs have the same, predefined length with possible exception of the last one which can be shorter.
A substantial downside to \ShS{} is that whilst it does allow to sort bigger starting runs quicker, it is not stable unlike \IS{}.
If \MS{} is supposed to sort stably, then \IS{} has to be used.

Unlike \QS{}, where each partition naturally acts as sentinel for the subsequent one, it is necessary to temporarily place sentinels values in front of each starting run and later restore the overwritten values of the preceding run.
The step sizes used for \ShS{} \Dash namely \(\stepsizes = (1, 6)\) for up to 48 elements, and \(\stepsizes = (1, 5, 12)\) for any more \Dash have been chosen based on the findings in \cref{sec:tasklet:shell}, according to which these step sizes offer top performance for uniformly distributed inputs and medial performance for reverse sorted inputs.
Spot-check inspection suggest no deterioration of \IS{}'s and \ShS{}'s compilation due to inlining.


\paragraph{Memory Footprint}
A simple implementation of \MS{} (denoted by \emph{full-space}) writes all merged runs to an auxiliary array, raising the need for space for \(n\) additional elements.
After a round is finished and all pairs of runs have been merged, the input array and the auxiliary array switch roles, and the merging begins anew.
Are the final sorted elements supposed to be saved in the original input array, a final round with a write-back from the auxiliary array to the input array is needed if the number of rounds is odd.

A slightly more sophisticated implementation (denoted by \emph{half-space}) needs space for only \(n/2\) additional elements:
When two adjacent runs are to be merged, the first one is copied to an auxiliary array.
Then, the copy and the second run are merged to the beginning of the first run.
As a side effect, no write-back is ever needed and, additionally, the merging of two runs can be terminated prematurely once the last element of the copied run is merged, since the last elements of the other run are already in place.
Strictly speaking, the auxiliary array holds~\(n-1\) elements in the worst-case.
By way of example, if the starting run length is \(n - 1\), there would be two runs of lengths \(n - 1\) and \(1\), respectively, and the first one would be copied away.
However, both the maximum number of elements to sort and the starting run length are predetermined, so the memory footprint can indeed be halved compared to the full-space \MS{}.

Further optimised, half-space \MS{} would not need to copy the first runs immediately.
It suffices to search for the foremost element of the first run which is greater than the first element of the second element.
All previous elements are already in the correct position so only the following elements need to be copied to the auxiliary array.
This \emph{deferred} copying, although examined during development, was not in use when measuring runtimes since it unfortunately complicates the following optimisation.


\paragraph{Unrolling}
There are four common reasons for \emph{flushing}, that is, writing \Dash many oftwhiles \Dash consecutive elements:
\begin{enumerate}
	\item
	When two runs are merged and the end of one of them is reached, the remaining elements of the other one can be moved safely to the output location.
	Especially with the sorted, reverse sorted, and almost sorted input distributions, the number of remaining elements will be high.

	\item
	The number of runs is odd, so the full-space \MS{} moves the last run to the output location unconditionally.

	\item
	The full-space \MS{} may write all elements from the auxiliary array back to the input array if the former contains the final sorted sequence.

	\item
	Before each merger of a pair of runs, the half-space \MS{} copies the first run to the auxiliary array.
\end{enumerate}
Therefore, flushing accounts for a considerable part of the runtime, and reducing the loop overhead (variable incrementation and bounds checking) is helpful.
This can be done via \emph{unrolling}:
As long as at least, let us say, \(x\) elements still need to be flushed, a loop with step size \(x\) is executed, and in each iteration, \(x\) elements are moved.
Is \(x\) a compile-time constant, the compiler implements the moving of the elements through \(x\) instruction which use constant, pre-calculated offsets.
Once less than \(x\) elements remain, an ordinary loop with step size \(1\), which moves elements individually, is used.
In good cases, this approach reduces the loop overhead to an \(x\)th, whilst in bad cases, where less than \(x\) elements are to be flushed, the overhead is increased by one additional check.

Due to time reasons, we refrained from doing automatic and extensive tests and relied on manual and exploratory tests to come up with the following strategy:
When the full-space \MS{} performs a write-back or when the half-space \MS{} copies the first run, \(x\) is set to the starting run length.
In all other cases, \(x\) is set to 24.
This strategy, albeit not optimal, makes the \MS*{} significantly faster:
Sorting sorted, reverse sorted, and almost sorted inputs gets faster by up to 30\%, while sorting more random inputs still get faster for the most part and slower by low single-digits at worst, depending on the starting run length.

\subsection{Investigation of the Compilation}
\label{sec:tasklet:merge:compilation}

\begin{figure}[p]
	\lstset{basicstyle=\ttfamily\small}
	\def\codewidth{0.34\linewidth}
	\newlength\assemblerwidth \setlength\assemblerwidth{0.595\linewidth}
	\begin{subfigure}{\textwidth}
		\begin{minipage}{\codewidth}
			\begin{lstlisting}[belowskip=3\baselineskip+\medskipamount]
#pragma unroll
for (int k = 0; k < 16; k++) {
	if (*i <= *j) {
		out[k] = *i++;
	} else {
		out[k] = *j++;
	}
}
			\end{lstlisting}
		\end{minipage}
		\hfill
		\begin{minipage}{\assemblerwidth}
			\begin{minipage}{ \widthof{\lstinline|	move rj, rtmp, true, .LABEL_k_out|} }
				\begin{lstlisting}[language={[DPU]Assembler}, mathescape, keepspaces]
$\textnormal{\textit{// iteration \texttt{k}}}$
	lw ri${}_{*}$, ri, 0
	lw rj${}_{*}$, rj, 0
	add rtmp, rj, 4
	jle ri${}_{*}$, rj${}_{*}$, .LABEL_k_i
	move ri${}_{*}$, rj${}_{*}$
	move rj, rtmp, true, .LABEL_k_out
.LABEL_k_i:
	add ri, ri, 4
.LABEL_k_out:
	sw rout, 4${}×{}$k, ri${}_{*}$
\end{lstlisting}
			\end{minipage}
			\hfill
			\begin{minipage}{ \widthof{\itshape// jump if \lstinline|*i| ≤ \lstinline|*j|} }
				\itshape\small
				\phantom{lg}

				// load \lstinline|*(i + 0)|

				// load \lstinline|*(j + 0)|

				// \lstinline|tmp| ← \lstinline|j| + \lstinline|1|

				// jump if \lstinline|*i| ≤ \lstinline|*j|

				// overwrite \lstinline[mathescape]|ri${}_{*}$|

				// \lstinline|j| ← \lstinline|tmp|; jump

				\phantom{lg}

				// \lstinline|i| ← \lstinline|i| + \lstinline|1|

				\phantom{lg}

				// \lstinline|out[k]| ← \lstinline|*i|
			\end{minipage}
		\end{minipage}
		\caption[]{
			This code takes 8 instructions per iteration.
			First, the pointers are dereferenced (lns.~2, 3).
			Then, the resulting address from incrementing pointer \lstinline|j| is calculated (ln.~4).
			If the first run contains the less current element, it is jumped to line~9, where pointer \lstinline|i| is incremented.
			Lastly, the less element \lstinline|*i| is written to the output (ln.~11).
			If the second run contains the less current element, the register holding \lstinline|*i| is overwritten with~\lstinline|*j| (ln.~7).
			Then, a combo operation (ln.~7) finally applies the result from incrementing pointer \lstinline|j| and jumps to the line where the output is set.

			\hspace*{1em}
			We do not know why pointer \lstinline|j| gets temporarily incremented.
			According to the documentation, an \lstinline|add| instruction is compatible with the \lstinline|true| flag, meaning the \lstinline|add| instruction in line~4 and the \lstinline|move| instruction in line~7 could be fused.
		}
		\label{fig:merge:load:twice}
	\end{subfigure}

	\begin{subfigure}{\textwidth}
		\begin{minipage}{\codewidth}
			\begin{lstlisting}[belowskip=2\baselineskip+\medskipamount+\smallskipamount]
int val_i = *i, val_j = *j;
#pragma unroll
for (int k = 0; k < 16; k++) {
	if (val_i <= val_j) {
		out[k] = val_i;
		val_i = *++i;
	} else {
		out[k] = val_j;
		val_j = *++j;
	}
}
			\end{lstlisting}
		\end{minipage}
		\hfill
		\begin{minipage}{\assemblerwidth}
			\begin{minipage}{ \widthof{\lstinline|	jgt ri*, rj*, .LABEL_k_j|~} }
				\begin{lstlisting}[language={[DPU]Assembler}, mathescape, keepspaces]
$\textnormal{\textit{// iteration \texttt{k} (\texttt{val\_i} ≤ \texttt{val\_j})}}$
	jgt ri${}_{*}$, rj${}_{*}$, .LABEL_k_j
.LABEL_k_i:
	sw rout, 4${}×{}$k, ri${}_{*}$
	add ri, ri, 4
	lw ri${}_{*}$, ri, 0
\end{lstlisting}
			\end{minipage}
			\hfill
			\begin{minipage}{ \widthof{\itshape// jump if \lstinline|val_i| > \lstinline|val_j|} }
				\itshape\small
				\phantom{lg}

				// jump if \lstinline|val_i| > \lstinline|val_j|

				\phantom{lg}

				// \lstinline|out[k]| ← \lstinline|val_i|

				// \lstinline|i| ← \lstinline|i| + \lstinline|1|

				// \lstinline|val_i| ← \lstinline|*(i + 0)|
			\end{minipage}
			\smallskip
			\begin{minipage}{ \widthof{\lstinline|	jgt ri*, rj*, .LABEL_k_j|~} }
				\begin{lstlisting}[language={[DPU]Assembler}, mathescape, keepspaces]
$\textnormal{\textit{// iteration \texttt{k} (\texttt{val\_i} > \texttt{val\_j})}}$
	jle ri${}_{*}$, rj${}_{*}$, .LABEL_k_i
.LABEL_k_j:
	sw rout, 4${}×{}$k, rj${}_{*}$
	add rj, rj, 4
	lw rj${}_{*}$, rj, 0
\end{lstlisting}
			\end{minipage}
			\hfill
			\begin{minipage}{ \widthof{\itshape// jump if \lstinline|val_i| > \lstinline|val_j|} }
				\itshape\small
				\phantom{lg}

				// jump if \lstinline|val_i| ≤ \lstinline|val_j|

				\phantom{lg}

				// \lstinline|out[k]| ← \lstinline|val_j|

				// \lstinline|j| ← \lstinline|j| + \lstinline|1|

				// \lstinline|val_j| ← \lstinline|*(j + 0)|
			\end{minipage}
		\end{minipage}
		\caption{
			This code takes 4 instructions per iteration.
			There are 16 cascaded iterations in the assembler code, all of them writing the elements of the first run to the output (top).
			There is an analogue cascade writing only elements of the second run to the output (bottom).
			Labels allow to switch between the cascades.
			First, it is checked whether the cascade should be changed (ln.~2).
			Then, the output is set (ln.~4), the respective pointer incremented (ln.~5), and the new value from dereferencing loaded (ln.~6).
		}
		\label{fig:merge:load:once}
	\end{subfigure}
	\caption{
		Two \langC{} implementations of an unrolled loop which merges 16 elements, contrasted with their compilations.
		Only the assembler codes of one iteration are shown, as all iterations follow the same scheme;
		a sixteenfold cascade of the given assembler codes yields the whole assembler codes of the loops.
		The pointers \lstinline|i| and \lstinline|j| point initially to the first elements of the runs.
		The serially numbered registers (\enquote{\lstinline|r|\dots}) and jump labels (\enquote{\lstinline|.LABEL|\dots}) were renamed to aid understanding.
		Note that the data type \lstinline[keywords={}]|int| is \qty{4}{\byte} large, which is why all offsets are multiples of four.
	}
	\label{fig:merge:load}
\end{figure}

\noindent
A significant portion of the runtime is spent on the repeated comparison of elements in a pair of runs, followed by a write of the less element to the output.
\Cref{fig:merge:load:twice} shows a straightforward implementation of an unrolled loop performing such comparisons and writes.
The code makes use of two pointers \lstinline|i| and \lstinline|j| which are initially set to the first elements of the runs.
To get their values, they are simply dereferenced.
After the output \lstinline|out[k]| has been set in iteration \lstinline|k|, the respective pointer of the less element is incremented.

Despite the succinctness of the \langC{} code, the resulting assembler code is of subpar quality.
Depending on the run from which an element got merged in the previous iteration, an iteration takes either 7 or 8 instructions.
This is a consequence of loading the values of both dereferenced pointers at the beginning of each iteration despite one of the values not having changed since the last iteration.
\Cref{fig:merge:load:once} shows an alternative implementation, whose compilation results in four instructions per iteration.
This was achieved by dereferencing the pointers \lstinline|i| and \lstinline|j| before the loop begins and storing the values in dedicated variables.
The comparisons and writes use only these dedicated variables, of which only one gets updated per iteration.
A more detailed description of the compilations is given in the caption of \cref{fig:merge:load}.

It can only be speculated as to the reason for the poor compilation of the simpler implementation.
Perhaps the constant reloads are related to the ability of tasklets to write to any \ac{WRAM} address.
Theoretically, it could be that the value obtained by dereferencing the non-incremented pointer changes between two subsequent iterations.
This explanation is not fully satisfactory as it would imply yet another load instruction before writing \lstinline|out[k]|.

Unluckily, only with the full-space \MS{} does this change to dedicated variables make iterations take 4 instructions unanimously.
With the half-space \MS{}, some merge iterations take 5 instructions.
The reason is that the second pointer \lstinline|j| is never incremented directly.
Instead, whenever an element of the second run is merged, a counter is incremented and, then, the new address of pointer \lstinline|j| is calculated by taking the address of the first element of the second run and adding the counter.
The fix is to change the loop which iterates over the pairs of runs to merge.
Rather than using the ends of the second runs as natural loop index, it has to be iterated over the ends of the first runs.

A last mention shall be given to the merge function used by the half-space \MS{}.
Passing the copied run as second argument and the uncopied run as the first one nets a noticeably speedup over an implementation with flipped arguments and, of course, flipped logic.
Sadly, we could not pinpoint the fundamental cause for this phenomenon.


\subsubsection*{Evaluation of the Performance}
\label{subsubsec:tasklet:merge:performance}

\def\mergealgos{16,24,32,48,64,96}

\pgfplotstablereadnamed{data/wram_sorts.txt}{tableWramSorts}
\expandafter\pgfplotsinvokeforeach\expandafter{\mergealgos}{
	\pgfplotstablereadnamed{data/merge/threshold=#1/uint32/sorted.txt}{tableMergeStart#1_32sorted}
	\pgfplotstablereadnamed{data/merge/threshold=#1/uint32/reverse.txt}{tableMergeStart#1_32reverse}
	\pgfplotstablereadnamed{data/merge/threshold=#1/uint32/almost.txt}{tableMergeStart#1_32almost}
	\pgfplotstablereadnamed{data/merge/threshold=#1/uint32/uniform.txt}{tableMergeStart#1_32uniform}
	\pgfplotstablereadnamed{data/merge/threshold=#1/uint32/zipf.txt}{tableMergeStart#1_32zipf}
	\pgfplotstablereadnamed{data/merge/threshold=#1/uint32/normal.txt}{tableMergeStart#1_32normal}
}

\pgfplotsset{
	merge/.style={
		horizontal sep for ticks,
		adaptive group=1 by 3,
		groupplot ylabel={Cycles / \((n \lb n)\)},
		x from 16 to 1024 minor,
		xmax=1024,
		enlarge x limits={abs=3mm, true},
		every legend image post={mark=none},
	},
	merge filter 16/.style={x filter/.expression={(\thisrow{n} == 16) || (\thisrow{n} ==  24) || (\thisrow{n} ==  96) || (\thisrow{n} == 384) || (\thisrow{n} == 1536) ? \pgfmathresult : nan}},
	merge filter 24/.style={x filter/.expression={(\thisrow{n} == 16) || (\thisrow{n} ==  32) || (\thisrow{n} == 128) || (\thisrow{n} == 512) || (\thisrow{n} == 2048) ? \pgfmathresult : nan}},
	merge filter 32/.style={x filter/.expression={(\thisrow{n} == 16) || (\thisrow{n} ==  48) || (\thisrow{n} == 192) || (\thisrow{n} == 768) || (\thisrow{n} == 3072) ? \pgfmathresult : nan}},
	merge filter 48/.style={x filter/.expression={(\thisrow{n} == 16) || (\thisrow{n} ==  64) || (\thisrow{n} == 256) || (\thisrow{n} == 1024) ? \pgfmathresult : nan}},
	merge filter 64/.style={x filter/.expression={(\thisrow{n} == 16) || (\thisrow{n} ==  96) || (\thisrow{n} == 384) || (\thisrow{n} == 1536) ? \pgfmathresult : nan}},
	merge filter 96/.style={x filter/.expression={(\thisrow{n} == 16) || (\thisrow{n} == 128) || (\thisrow{n} == 512) || (\thisrow{n} == 2048) ? \pgfmathresult : nan}},
}

\begin{figure}
	\tikzsetnextfilename{merge_starting_runs}
	\begin{tikzpicture}[plot]
		\begin{groupplot}[
			merge,
			ymin=65,
			ymax=90,
			ytick distance=5,
		]
			\nextgroupplot[title={No Write-back\strut}, legend to name=leg:merge:starting_runs]
			\expandafter\legend\expandafter{\mergealgos}
			\clip (0, 0) rectangle (1024, 200);
			\expandafter\pgfplotsinvokeforeach\expandafter{\mergealgos}{
				\plotpernlogn[merge filter #1]{Merge}{tableMergeStart#1_32uniform}
			}
			%
			\nextgroupplot[title={Write-back\strut}]
			\clip (0, 0) rectangle (1024, 200);
			\expandafter\pgfplotsinvokeforeach\expandafter{\mergealgos}{
				\plotpernlogn[merge filter #1]{MergeWriteBack}{tableMergeStart#1_32uniform}
			}
			%
			\nextgroupplot[title={Half Space}]
			\clip (0, 0) rectangle (1024, 200);
			\expandafter\pgfplotsinvokeforeach\expandafter{\mergealgos}{
				\plotpernlogn[merge filter #1]{MergeHalfSpace}{tableMergeStart#1_32uniform}
			}
		\end{groupplot}
	\end{tikzpicture}

	\hfil\pgfplotslegendfromname{leg:merge:starting_runs}\hfil
	\caption{
		Comparison of \MS*{}, which need an auxiliary array of length either \(n\) (\enquote{No Write-back} / \enquote{Write-back}) or \(\sfrac{n}{2}\) (\enquote{Half Space}), for different lengths of the starting runs.
		The \MS*{} use a \ShS{} with the step sizes \(\stepsizes = (1)\) for length 16, \(\stepsizes = (6, 1)\) for lengths 24 to 48, and \(\stepsizes = (12, 5, 1)\) for lengths 64 and 96, respectively.
	}
	\label{fig:merge:starting_runs}
\end{figure}

Three implementations have been tested:
full space \MS{} without write-backs, full space \MS{} with write-backs, and half space \MS{}.
\Cref{fig:merge:starting_runs,fig:merge:starting_runs_uint32sorted,fig:merge:starting_runs_uint32uniform,fig:merge:starting_runs_uint64sorted,fig:merge:starting_runs_uint64uniform} show their performance for various starting run lengths.
Please note that the plots are smoothed:
Whenever the number of rounds increments, the runtimes hike, making the zigzagging plots cross each other unswervingly and, thereby, hard to read.
Thence, the figures contain marks for select measurements only in such a way that the resulting plots act as an upper bound on the runtime.

The measurements show that the \MS*{} guarantee a runtime of \(\bigoh{n \lb n}\) as expected.
The differences in runtime between the different input distributions are small compared to \QS{} and are ascribable to \ShS{} and to the differing suitability of the unrolling;
cases where the usage of \ShS{} worsened the runtime are unbeknown.

Even though the tested starting run lengths range from 16 to 96 elements, the mean runtime differences are surprisingly small.
Notwithstanding that the optimal choice depends on the specific input length because of the zigzagging, a starting run length of 32 elements fares decidedly well on average across all tested scenarios.

The half-space \MS{} delivers a strong performance despite its vastly lower memory footprint.
With 32-bit integers, it beats the full-space \MS{} without write-backs by 11\% on sorted inputs and effectively ties on all other inputs but the reverse sorted ones where it narrowly falls behind.
Naturally, the full-space \MS{} with write-backs is consistently (with the exception of reverse sorted inputs) at a disadvantage, despite seeing some light with inferior starting run lengths.
With 64-bit integers, the full-space \MS{} without write-backs manages to turn the ties into scant leads in the range from 1\% to 3\%.
Using the \MS{} with write-backs is still unprofitable.

In summary, a proper implementation of half-space \MS{} with deferred copying and fine-tuned unrolling would require some work but has the potential to be the overall best stable sorting algorithm.


\subsection{\texorpdfstring{\HS{}}{HeapSort}}
\label{subsec:tasklet:heap}

Another sorting algorithm with a guaranteed runtime of \(\bigtheta{n \log n}\) is \HS{}, which is unstable but in-place.
A max-heap is a binary tree of logarithmic depth whose layers are fully filled, possibly with the exception of the last layer, which must be filled from left to right.
Each vertex contains a key, and the key of each father must be at least as great as those of his sons.
As a consequence of this heap order, the root contains the greatest key.

A heap with \(n\) keys can be represented as an array of length \(n\) using a bijective mapping between the vertex positions and the array indices (see later).
After the heap has been built in-place from the input array in time \(\bigoh{n}\), the sorting works as follows:
At the start of round \(r = 1, \dots, n\), the first \(n - (r - 1)\) elements of the array represent the heap and the last \(r - 1\) elements the end of the sorted output.
Upon removal of the root, which contains the \(r\)th greatest element of the input, the heap structure must be restored in time \(\bigoh{\log n}\).
Since the heap has shrunken by one key, the key of the removed root can be stored at the freed-up position directly after the end of the heap.

\paragraph{Sifting Direction}
After the heap is built, the \emph{top-down} \HS{} proceeds as follows:
At the start of each round, the root and the rightmost leaf (\enquote{last leaf}) in the bottom layer swap places.
The root is now in the right position, but the formerly last leaf may violate the heap order, that is, the root may have a lesser key than one or both of its sons.
The greater of the two sons is determined, and the root and the greater son swap places.
This downwards-sifting of the former leaf continues iteratively until the heap order is restored.

In contrast, the \emph{bottom-up} \HS{} \cite{wegener1993heapsort} works as follows:
At the start of each round, the key of the root is removed so that a hole is now at the top of the heap.
Then, the greater of the two sons of the hole is determined, and they swap places.
This downwards-sifting of the hole continues iteratively until it becomes a leaf.
Now, the last leaf is moved to the position of the hole, which could violate the heap order if the moved leaf is greater than its father.
If so, it needs to be sifted upwards by iteratively swapping positions with its respective father until the heap order is restored.
At last, the original root key can be put where the formerly last leaf used to be.

The motivation behind these variants is at follows:
In each step where the top-down \HS{} sifts the formerly last leaf downwards, two value checks (Which son is greater? Is the father lesser than the greater son?) need to be done.
The leaves of a heap tend to be small so the downwards-sifting lasts awhile.
As opposed to this, each step of the bottom-up \HS{} needs only one value check (Is the father lesser than the greater son?).
Both \HS*{} sift downwards similarly long so many checks can be saved.
Since the last leaf effectively takes the place of another leaf and since both are likely small, the upwards-sifting should be short-lived and, hopefully, not eat the gain up.

The upwards-sifting reverts some of the changes done by the downwards-sifting.
The bottom-up \HS{} can be brought to swap parity with the top-down \HS{} with the following change:
The downwards-sifting is traced but the keys are not actually moved.
Once the leaf where the hole would end up is reached, the sifting is backtracked until the bottommost key which is at least as great as the last leaf.
This is the position where the last leaf would end up after the upwards-sifting, so all keys below can stay put and all keys above move to their fathers' positions, that is, thither the swaps from the downwards-sifting would have put them.
This makes the downwards-sifting even cheaper but the upwards-sifting must now go all the way up to the root.

\paragraph{Indexing}
With a zero-based indexing, the sons of a vertex \(i\) can be calculated with the well-known formulas \(2i + 1\) and \(2n + 2\).
With a one-based indexing, the formulas turn into \(2i\) and \(2i + 1\).
The compiler automatically turns the multiplication by two into a left-shift by one.
Since DPUs can execute an instruction called \lstinline|lsl_add| which first shifts leftwards and then adds an offset (useful \eg{} for array indexing), the formulas \(2i + 1\) and \(2i\) take the same amount of time to compute.

Nevertheless, the zero-based indexing is about 7\% slower despite \lstinline|lsl_add| being indeed in use.
The reason is that only the number of bits to shift can be passed as immediate value, that is as plain number, but not the offset, which must be passed via a register.
While DPUs have a read-only register storing the number \(1\) at disposal, read-only registers can only ever be the first register argument, not the second one, which, for \lstinline|lsl_add|, would be the offset.
As a consequence, the compiler moves the number \(1\) to a register whenever \(2i + 1\) is to be computed, only to immediately overwrite the \(1\) with the result from \lstinline|lsl_add|.
Hence, the calculation of \(2i + 1\) does take one more instruction than \(2n\) after all.

\paragraph{Sentinel Values}
When \HS{} sifts a vertex downwards, it needs to determine the greater of its two sons before deciding whether and whither to move.
If and only if the heap has an even number of vertices, there is a left son without a right brother:
the rightmost leaf in the bottom layer.
Instead of adding some check on whether the right brother exists, one can rather add the missing leaf and give it the smallest possible key each time the heap reaches an even size.
Thus, if it has been confirmed that a left son exists, a right one does also exist, and if two brothers contain the same key, the left one should be considered greater.

Likewise, whenever \HS{} sifts upwards and considers the father \(i/2\) of a vertex \(i\), it will only proceed if the father is lesser.
Since the fatherless root has index \(1\) and the result of an integer division is truncated towards \(0\) in C, the formula yields \(0\), so it makes sense to set the element at index \(0\) to the greatest possible key to stop any upwards-sifting.
The savings from these approaches were around the 13\% mark.

\paragraph{Code Duplication}
A strategy particularly useful for \HS{}, although also employed in \MS{}, is code duplication.
Handling the greater of two sons is the fastest if the logic is written twice, once for either son, and then executed conditionally;
logic written once for a generalised variable holding the greater son is compiled considerably worse.
The savings from this approach were around the 7\% mark.

%\paragraph{Fallback Algorithm}
%Once 15 elements remain in the heap, they are sorted with \IS{}.
%This threshold is a good compromise, although the impact of \IS{} is rather forgettable, admittedly.

\subsection{Evaluation of the Performance}
\label{sec:tasklet:heap:performance}

\expandafter\pgfplotsinvokeforeach\expandafter{\alldists}{
	\pgfplotstablereadnamed{data/heap/uint32/#1.txt}{tableHeap_32#1}
	\pgfplotstablereadnamed{data/heap/uint64/#1.txt}{tableHeap_64#1}
}

\pgfplotsset{
	heap/.style={
		adaptive group=1 by 2,
		groupplot ylabel={Cycles / \((n \lb n)\)},
		x from 16 to 1024,
	},
}

\def\heapalgos{HeapOnlyDown,HeapUpDown,HeapSwapParity}

\begin{figure}
	\tikzsetnextfilename{heap_runtime}
	\begin{tikzpicture}[plot]
		\begin{groupplot}[heap, ytick distance=5]
			\nextgroupplot[title/.add={}{32-bit}, ymin=130, ymax=155, legend to name=leg:heap:runtime]
			\legend{\HS{} (top-down), \HS{} (bottom-up), \HS{} (swap parity)}
			\expandafter\pgfplotsinvokeforeach\expandafter{\heapalgos}{
				\plotpernlogn{#1}{tableHeap_32uniform}
			}
			%
			\nextgroupplot[title/.add={}{64-bit}, ymin=155, ymax=180]
			\expandafter\pgfplotsinvokeforeach\expandafter{\heapalgos}{
				\plotpernlogn{#1}{tableHeap_64uniform}
			}
		\end{groupplot}
	\end{tikzpicture}

	\tikzexternaldisable\hfil\pgfplotslegendfromname{leg:heap:runtime}\hfil\tikzexternalenable
	\caption{
		Mean runtimes of all \HS{} implementations on uniformly distributed integers.
	}
	\label{fig:heap:runtime}
\end{figure}

The measurements are visualised in \cref{fig:heap:runtime,fig:heap:runtime_uint32,fig:heap:runtime_uint64}.
In general, the performance of \HS{} is hardly volatile and mostly independent from the input distribution.
Reverse sorted inputs are sorted faster than most since they are already max-heaps so the heap-building phase is brief, whereas sorted inputs are sorted the slowest since they are min-heaps essentially needing inversion.
Nonetheless, the reverse sorted inputs get sorted at most \qty{10}{\percent} faster than the sorted ones.
An extraördinary outlier are zero-one inputs on which the top-down \HS{} achieves a speedup of \num{1.8} over the bottom-up \HS{}.
With zero-one inputs, roughly half of the elements are zeroes and roughly half are ones.
After building the min-heap, the first half of the array consists of ones and the second half of zeroes.
About half of the ones and about half of the zeroes are already in these respective halves of the input, so the heap-building phase is somewhat brief, too.
More importantly, however, is the foreshortening of the downwards-sifting.
After about \(n/2\) many rounds, only zeroes remain in the heap.
Therefore, no last leaf violates the heap order when moved to the top, so the downwards-sifting terminates immediately.
But the downwards-sifting becomes briefer even earlier as many ones turn into leaves once the zeroes which are their children have been moved to the top and were sifted down to somewhere else.
Such ones do not violate the heap order when moved to the top at a later point, too.


Ignoring this outlier, the normalised runtimes of the top-down \HS{} and the bottom-up \HS{} with swap parity show a slight upwards trends, whereas that of the bottom-up \HS{} with swap disparity mostly shows a slight downwards trends.
The exception are reverse sorted inputs, where the latter also shows a slight upwards trend.
Of interest is their ranking:
Value checks on 64-bit integers take two instructions, so that the savings of the bottom-up \HS{} with swap disparity allow it to outperform the top-down \HS{} even for short inputs.
Its advantage grows with the input length.
This makes sense as roughly \qty{50}{\percent} of the vertices are leaves and \qty{25}{\percent} are parents of leaves, no matter the total heap size.
Therefore, the percentage of former last leaves being sifted down from the top to the bottom remains steady but the travelled distance increases.
Value checks on 32-bit integers, on the other hand, take only one instruction, so that the reduction of these is overshadowed by the increased overhead from the longer downwards-sifting and the added upwards-sifting.
Indeed, at around 2000 elements, the bottom-up \HS{} with swap disparity overtakes the top-down \HS{} because of their inverse trends, but the lead stays meagre even at \num{6000} elements.
%However, these long inputs are uninteresting due to the limited \ac{WRAM} and the multiple tasklets used in most applications.

The bottom-up \HS{} with swap parity consistently trails behind.
This comes as no surprise since the overhead of its considerably prolonged upwards-sifting bears no proportion to the few swaps saved.
This holds true even for 64-bit integers as moves still cost only one instruction so the savings do not increase.
Unrolling the upwards-sifting (cf.\ \cref{sec:tasklet:merge}) proved to be unhelpful.


\subsection*{Investigation of the Compilation}
\label{sec:tasklet:heap:compilation}

While engineering, some strange observations were made.
For example, the runtime difference between stopping \HS{} when only one element remains in the heap and stopping \HS{} when only three elements remain (which then get sorted by \IS{}) can be in the tens of thousands of cycles in both directions, depending on the sifting direction and the input distribution.
Stopping \HS{} even earlier has comparatively little effect.
\IS{} has ultimately been removed from all implementations.

The undisputedly strangest observation was the following:
Before building a heap, a single sentinel leaf must be inserted if the input length is even.
Adding this leaf if the input length is odd makes no difference algorithmically, as it would be a left leaf never to be accessed due to the bounds checks.
However, adding an if-statement determining whether the sentinel leaf has to be placed has dramatic effects compared to placing the sentinel value unconditionally.
Since the parity of the input length is needed later anyway, the conditional version is expected to gain one instruction.
Yet, when measuring the runtimes on 1024 elements, one can observe anything from a reduction by 5000 cycles over changes within the margin of error to increases by 25\,000 cycles, depending on the sifting direction and the input distribution.
Adding one or more sentinel leaves outside of the \HS{} functions has no impact on this behaviour.
A comparison of the compilations reveals minute differences at the beginnings of the \HS{} functions, none of which affect anything repeatedly executed.
The register usage does also not change in such a manner that the execution time of instructions is prolonged to 12 cycles, as described in the DPU SDK documentation \cite[Instruction Set Architecture -- Efficient scheduling]{upmemSDK}.




	\appendix

	\section{Further Measurements on Sorting with One Tasklet}
\label{apx:tasklet}

\subsection{\texorpdfstring{\IS{}}{InsertionSort}}
\label{subapx:single:insertion}

\pgfplotstableread{data/small sorts/uint32/sorted.txt}{\tableSmallSortsXxxiiSorted}
\pgfplotstableread{data/small sorts/uint32/reverse.txt}{\tableSmallSortsXxxiiReverse}
\pgfplotstableread{data/small sorts/uint32/almost.txt}{\tableSmallSortsXxxiiAlmost}
\pgfplotstableread{data/small sorts/uint32/zipf.txt}{\tableSmallSortsXxxiiZipf}
\pgfplotstableread{data/small sorts/uint32/normal.txt}{\tableSmallSortsXxxiiNormal}
\pgfplotstableread{data/small sorts/uint64/sorted.txt}{\tableSmallSortsLxivSorted}
\pgfplotstableread{data/small sorts/uint64/reverse.txt}{\tableSmallSortsLxivReverse}
\pgfplotstableread{data/small sorts/uint64/almost.txt}{\tableSmallSortsLxivAlmost}
\pgfplotstableread{data/small sorts/uint64/uniform.txt}{\tableSmallSortsLxivUniform}
\pgfplotstableread{data/small sorts/uint64/zipf.txt}{\tableSmallSortsLxivZipf}
\pgfplotstableread{data/small sorts/uint64/normal.txt}{\tableSmallSortsLxivNormal}

\begin{figure}[!h]
	\def\algos{1NoSentinel,1,1Implicit,BubbleNonAdapt,BubbleAdapt,Selection}
	\tikzsetnextfilename{insertion_against_others_uint32}
	\begin{tikzpicture}[plot]
		\begin{groupplot}[
			adaptive group=3 by 2,
			groupplot xlabel={Input Length \(n\)},
			groupplot ylabel={Cycles / \(n^2\)},
			xtick distance=3,
			minor xtick=data,
			ymin=0,
			ymax=60,
			legend columns=3,
		]
			\nextgroupplot[title={Sorted}, legend to name=leg:insertion:against_others_uint32]
			\legend{\IS{} (no sentinel), \IS{} (sentinel), \IS{} (implicit), \BS{} (not adaptive), \BS{} (adaptive), \SelS{}}
			\expandafter\pgfplotsinvokeforeach\expandafter{\algos}{
				\plotpernn{#1}{\tableSmallSortsXxxiiSorted}
			}
			%
			\nextgroupplot[title={Reverse Sorted}]
			\expandafter\pgfplotsinvokeforeach\expandafter{\algos}{
				\plotpernn{#1}{\tableSmallSortsXxxiiReverse}
			}
			%
			\nextgroupplot[title={Almost Sorted}]
			\expandafter\pgfplotsinvokeforeach\expandafter{\algos}{
				\plotpernn{#1}{\tableSmallSortsXxxiiAlmost}
			}
			%
			\nextgroupplot[title={Uniform}]
			\expandafter\pgfplotsinvokeforeach\expandafter{\algos}{
				\plotpernn{#1}{\tableSmallSortsXxxiiUniform}
			}
			%
			\nextgroupplot[title={Zipf's}]
			\expandafter\pgfplotsinvokeforeach\expandafter{\algos}{
				\plotpernn{#1}{\tableSmallSortsXxxiiZipf}
			}
			%
			\nextgroupplot[title={Normal}]
			\expandafter\pgfplotsinvokeforeach\expandafter{\algos}{
				\plotpernn{#1}{\tableSmallSortsXxxiiNormal}
			}
		\end{groupplot}
	\end{tikzpicture}

	\hfil\pgfplotslegendfromname{leg:insertion:against_others_uint32}\hfil
	\caption{
		An extension to \cref{fig:insertion:against_others}.
		The data type is 32-bit unsigned integers.
	}
	\label{fig:insertion:against_others_uint32}
\end{figure}

\begin{figure}
	\def\algos{1NoSentinel,1,1Implicit,BubbleNonAdapt,BubbleAdapt,Selection}
	\tikzsetnextfilename{insertion_against_others_uint64}
	\begin{tikzpicture}[plot]
		\begin{groupplot}[
			adaptive group=3 by 2,
			groupplot xlabel={Input Length \(n\)},
			groupplot ylabel={Cycles / \(n^2\)},
			xtick distance=3,
			minor xtick=data,
			ymin=0,
			ymax=70,
			extra y ticks={70},
			legend columns=3,
		]
			\nextgroupplot[title={Sorted}, legend to name=leg:insertion:against_others_uint64]
			\legend{\IS{} (no sentinel), \IS{} (sentinel), \IS{} (implicit), \BS{} (not adaptive), \BS{} (adaptive), \SelS{}}
			\expandafter\pgfplotsinvokeforeach\expandafter{\algos}{
				\plotpernn{#1}{\tableSmallSortsLxivSorted}
			}
			%
			\nextgroupplot[title={Reverse Sorted}]
			\expandafter\pgfplotsinvokeforeach\expandafter{\algos}{
				\plotpernn{#1}{\tableSmallSortsLxivReverse}
			}
			%
			\nextgroupplot[title={Almost Sorted}]
			\expandafter\pgfplotsinvokeforeach\expandafter{\algos}{
				\plotpernn{#1}{\tableSmallSortsLxivAlmost}
			}
			%
			\nextgroupplot[title={Uniform}]
			\expandafter\pgfplotsinvokeforeach\expandafter{\algos}{
				\plotpernn{#1}{\tableSmallSortsLxivUniform}
			}
			%
			\nextgroupplot[title={Zipf's}]
			\expandafter\pgfplotsinvokeforeach\expandafter{\algos}{
				\plotpernn{#1}{\tableSmallSortsLxivZipf}
			}
			%
			\nextgroupplot[title={Normal}]
			\expandafter\pgfplotsinvokeforeach\expandafter{\algos}{
				\plotpernn{#1}{\tableSmallSortsLxivNormal}
			}
		\end{groupplot}
	\end{tikzpicture}

	\hfil\pgfplotslegendfromname{leg:insertion:against_others_uint64}\hfil
	\caption{
		An extension to \cref{fig:insertion:against_others}.
		The data type is 64-bit unsigned integers.
	}
	\label{fig:insertion:against_others_uint64}
\end{figure}


\clearpage

\subsection{\texorpdfstring{\ShS{}}{ShellSort}}
\label{subapp:single:shell}

\begin{figure}
	\begin{tikzpicture}[plot]
		\begin{groupplot}[
			adaptive group=3 by 2,
			groupplot xlabel={Input Length \(n\)},
			groupplot ylabel={Cycles / \(n^2\)},
			xtick distance=3,
			minor xtick=data,
			legend columns=-1,
		]
			\nextgroupplot[title={Sorted}, legend to name=leg:shell:two-tier_uint32]
			\legend{\(1\), \(...\), \(9\)}
			\pgfplotsinvokeforeach{1,...,9}{
				\plotpernn{#1}{\tableSmallSortsXxxiiSorted}
			}
			%
			\nextgroupplot[title={Reverse Sorted}]
			\pgfplotsinvokeforeach{1,...,9}{
				\plotpernn{#1}{\tableSmallSortsXxxiiReverse}
			}
			%
			\nextgroupplot[title={Almost Sorted}]
			\pgfplotsinvokeforeach{1,...,9}{
				\plotpernn{#1}{\tableSmallSortsXxxiiAlmost}
			}
			%
			\nextgroupplot[title={Uniform}]
			\pgfplotsinvokeforeach{1,...,9}{
				\plotpernn{#1}{\tableSmallSortsXxxiiUniform}
			}
			%
			\nextgroupplot[title={Zipf's}]
			\pgfplotsinvokeforeach{1,...,9}{
				\plotpernn{#1}{\tableSmallSortsXxxiiZipf}
			}
			%
			\nextgroupplot[title={Normal}]
			\pgfplotsinvokeforeach{1,...,9}{
				\plotpernn{#1}{\tableSmallSortsXxxiiNormal}
			}
		\end{groupplot}
	\end{tikzpicture}

	\hfil\pgfplotslegendfromname{leg:shell:two-tier_uint32}\hfil
	\caption{
		A continuation of \cref{fig:shell:two-tier} with more input distributions.
		The data type is 32-bit unsigned integers.
	}
	\label{fig:shell:two-tier_uint32}
\end{figure}

\begin{figure}
	\begin{tikzpicture}[plot]
		\begin{groupplot}[
			adaptive group=3 by 2,
			groupplot xlabel={Input Length \(n\)},
			groupplot ylabel={Cycles / \(n^2\)},
			xtick distance=3,
			minor xtick=data,
			legend columns=-1,
		]
			\nextgroupplot[title={Sorted}, legend to name=leg:shell:two-tier_uint64]
			\legend{\(1\), \(...\), \(9\)}
			\pgfplotsinvokeforeach{1,...,9}{
				\plotpernn{#1}{\tableSmallSortsLxivSorted}
			}
			%
			\nextgroupplot[title={Reverse Sorted}]
			\pgfplotsinvokeforeach{1,...,9}{
				\plotpernn{#1}{\tableSmallSortsLxivReverse}
			}
			%
			\nextgroupplot[title={Almost Sorted}]
			\pgfplotsinvokeforeach{1,...,9}{
				\plotpernn{#1}{\tableSmallSortsLxivAlmost}
			}
			%
			\nextgroupplot[title={Uniform}]
			\pgfplotsinvokeforeach{1,...,9}{
				\plotpernn{#1}{\tableSmallSortsLxivUniform}
			}
			%
			\nextgroupplot[title={Zipf's}]
			\pgfplotsinvokeforeach{1,...,9}{
				\plotpernn{#1}{\tableSmallSortsLxivZipf}
			}
			%
			\nextgroupplot[title={Normal}]
			\pgfplotsinvokeforeach{1,...,9}{
				\plotpernn{#1}{\tableSmallSortsLxivNormal}
			}
		\end{groupplot}
	\end{tikzpicture}

	\hfil\pgfplotslegendfromname{leg:shell:two-tier_uint64}\hfil
	\caption{
		A continuation of \cref{fig:shell:two-tier} with more input distributions.
		The data type is 64-bit unsigned integers.
	}
	\label{fig:shell:two-tier_uint64}
\end{figure}


\clearpage

\subsection{\texorpdfstring{\MS{}}{MergeSort}}
\label{subapx:single:merge}

\expandafter\pgfplotsinvokeforeach\expandafter{\mergealgos}{
	\pgfplotstablereadnamed{data/merge/fallback=#1/uint64/sorted.txt}{tableMergeStart#1_64sorted}
	\pgfplotstablereadnamed{data/merge/fallback=#1/uint64/reverse.txt}{tableMergeStart#1_64reverse}
	\pgfplotstablereadnamed{data/merge/fallback=#1/uint64/almost.txt}{tableMergeStart#1_64almost}
	\pgfplotstablereadnamed{data/merge/fallback=#1/uint64/uniform.txt}{tableMergeStart#1_64uniform}
	\pgfplotstablereadnamed{data/merge/fallback=#1/uint64/zipf.txt}{tableMergeStart#1_64zipf}
	\pgfplotstablereadnamed{data/merge/fallback=#1/uint64/normal.txt}{tableMergeStart#1_64normal}
}

\begin{figure}[ht]
	\pgfplotsset{height=3cm}
	\begin{subfigure}{\textwidth}
		\tikzsetnextfilename{merge_starting_runs_uint32sorted}
		\begin{tikzpicture}[plot]
			\begin{groupplot}[merge fallback]
				\nextgroupplot[title={No Write-back\strut}]
				\expandafter\pgfplotsinvokeforeach\expandafter{\mergealgos}{
					\plotpernlognnew[merge sort filter #1]{Merge}{tableMergeStart#1_32sorted}
				}
				%
				\nextgroupplot[title={Write-back\strut}]
				\expandafter\pgfplotsinvokeforeach\expandafter{\mergealgos}{
					\plotpernlognnew[merge sort filter #1]{MergeWriteBack}{tableMergeStart#1_32sorted}
				}
				%
				\nextgroupplot[title={Half Space}]
				\expandafter\pgfplotsinvokeforeach\expandafter{\mergealgos}{
					\plotpernlognnew[merge sort filter #1]{MergeHalfSpace}{tableMergeStart#1_32sorted}
				}
			\end{groupplot}
		\end{tikzpicture}
		\caption{
			Sorted
		}
		\medskip
	\end{subfigure}
	\begin{subfigure}{\textwidth}
		\tikzsetnextfilename{merge_starting_runs_uint32reverse}
		\begin{tikzpicture}[plot]
			\begin{groupplot}[merge fallback]
				\nextgroupplot[title={No Write-back\strut}]
				\expandafter\pgfplotsinvokeforeach\expandafter{\mergealgos}{
					\plotpernlognnew[merge sort filter #1]{Merge}{tableMergeStart#1_32reverse}
				}
				%
				\nextgroupplot[title={Write-back\strut}]
				\expandafter\pgfplotsinvokeforeach\expandafter{\mergealgos}{
					\plotpernlognnew[merge sort filter #1]{MergeWriteBack}{tableMergeStart#1_32reverse}
				}
				%
				\nextgroupplot[title={Half Space}]
				\expandafter\pgfplotsinvokeforeach\expandafter{\mergealgos}{
					\plotpernlognnew[merge sort filter #1]{MergeHalfSpace}{tableMergeStart#1_32reverse}
				}
			\end{groupplot}
		\end{tikzpicture}
		\caption{
			Reverse Sorted
		}
		\medskip
	\end{subfigure}
	\begin{subfigure}{\textwidth}
		\tikzsetnextfilename{merge_starting_runs_uint32almost}
		\begin{tikzpicture}[plot]
			\begin{groupplot}[merge fallback]
				\nextgroupplot[title={No Write-back\strut}]
				\expandafter\pgfplotsinvokeforeach\expandafter{\mergealgos}{
					\plotpernlognnew[merge sort filter #1]{Merge}{tableMergeStart#1_32almost}
				}
				%
				\nextgroupplot[title={Write-back\strut}]
				\expandafter\pgfplotsinvokeforeach\expandafter{\mergealgos}{
					\plotpernlognnew[merge sort filter #1]{MergeWriteBack}{tableMergeStart#1_32almost}
				}
				%
				\nextgroupplot[title={Half Space}]
				\expandafter\pgfplotsinvokeforeach\expandafter{\mergealgos}{
					\plotpernlognnew[merge sort filter #1]{MergeHalfSpace}{tableMergeStart#1_32almost}
				}
			\end{groupplot}
		\end{tikzpicture}
		\caption{
			Almost Sorted
		}
		\bigskip
	\end{subfigure}

	\hfil\pgfplotslegendfromname{leg:merge:starting_runs}\hfil
	\caption{
		An extension to \cref{fig:merge:starting_runs}.
		The data type is 32-bit unsigned integers.
	}
	\label{fig:merge:starting_runs_uint32sorted}
\end{figure}

\begin{figure}[ht]
	\pgfplotsset{height=3cm}
	\begin{subfigure}{\textwidth}
		\tikzsetnextfilename{merge_starting_runs_uint32uniform}
		\begin{tikzpicture}[plot]
			\begin{groupplot}[merge fallback]
				\nextgroupplot[title={No Write-back\strut}]
				\expandafter\pgfplotsinvokeforeach\expandafter{\mergealgos}{
					\plotpernlognnew[merge sort filter #1]{Merge}{tableMergeStart#1_32uniform}
				}
				%
				\nextgroupplot[title={Write-back\strut}]
				\expandafter\pgfplotsinvokeforeach\expandafter{\mergealgos}{
					\plotpernlognnew[merge sort filter #1]{MergeWriteBack}{tableMergeStart#1_32uniform}
				}
				%
				\nextgroupplot[title={Half Space}]
				\expandafter\pgfplotsinvokeforeach\expandafter{\mergealgos}{
					\plotpernlognnew[merge sort filter #1]{MergeHalfSpace}{tableMergeStart#1_32uniform}
				}
			\end{groupplot}
		\end{tikzpicture}
		\caption{
			Uniform
		}
		\medskip
	\end{subfigure}
	\begin{subfigure}{\textwidth}
		\tikzsetnextfilename{merge_starting_runs_uint32zipf}
		\begin{tikzpicture}[plot]
			\begin{groupplot}[merge fallback]
				\nextgroupplot[title={No Write-back\strut}]
				\expandafter\pgfplotsinvokeforeach\expandafter{\mergealgos}{
					\plotpernlognnew[merge sort filter #1]{Merge}{tableMergeStart#1_32zipf}
				}
				%
				\nextgroupplot[title={Write-back\strut}]
				\expandafter\pgfplotsinvokeforeach\expandafter{\mergealgos}{
					\plotpernlognnew[merge sort filter #1]{MergeWriteBack}{tableMergeStart#1_32zipf}
				}
				%
				\nextgroupplot[title={Half Space}]
				\expandafter\pgfplotsinvokeforeach\expandafter{\mergealgos}{
					\plotpernlognnew[merge sort filter #1]{MergeHalfSpace}{tableMergeStart#1_32zipf}
				}
			\end{groupplot}
		\end{tikzpicture}
		\caption{
			Zipf's
		}
		\medskip
	\end{subfigure}
	\begin{subfigure}{\textwidth}
		\tikzsetnextfilename{merge_starting_runs_uint32normal}
		\begin{tikzpicture}[plot]
			\begin{groupplot}[merge fallback]
				\nextgroupplot[title={No Write-back\strut}]
				\expandafter\pgfplotsinvokeforeach\expandafter{\mergealgos}{
					\plotpernlognnew[merge sort filter #1]{Merge}{tableMergeStart#1_32normal}
				}
				%
				\nextgroupplot[title={Write-back\strut}]
				\expandafter\pgfplotsinvokeforeach\expandafter{\mergealgos}{
					\plotpernlognnew[merge sort filter #1]{MergeWriteBack}{tableMergeStart#1_32normal}
				}
				%
				\nextgroupplot[title={Half Space}]
				\expandafter\pgfplotsinvokeforeach\expandafter{\mergealgos}{
					\plotpernlognnew[merge sort filter #1]{MergeHalfSpace}{tableMergeStart#1_32normal}
				}
			\end{groupplot}
		\end{tikzpicture}
		\caption{
			Normal
		}
		\bigskip
	\end{subfigure}

	\hfil\pgfplotslegendfromname{leg:merge:starting_runs}\hfil
	\caption{
		An extension to \cref{fig:merge:starting_runs}.
		The data type is 32-bit unsigned integers.
	}
	\label{fig:merge:starting_runs_uint32uniform}
\end{figure}

\begin{figure}[ht]
	\pgfplotsset{height=3cm}
	\begin{subfigure}{\textwidth}
		\tikzsetnextfilename{merge_starting_runs_uint64sorted}
		\begin{tikzpicture}[plot]
			\begin{groupplot}[merge fallback]
				\nextgroupplot[title={No Write-back\strut}]
				\expandafter\pgfplotsinvokeforeach\expandafter{\mergealgos}{
					\plotpernlognnew[merge sort filter #1]{Merge}{tableMergeStart#1_64sorted}
				}
				%
				\nextgroupplot[title={Write-back\strut}]
				\expandafter\pgfplotsinvokeforeach\expandafter{\mergealgos}{
					\plotpernlognnew[merge sort filter #1]{MergeWriteBack}{tableMergeStart#1_64sorted}
				}
				%
				\nextgroupplot[title={Half Space}]
				\expandafter\pgfplotsinvokeforeach\expandafter{\mergealgos}{
					\plotpernlognnew[merge sort filter #1]{MergeHalfSpace}{tableMergeStart#1_64sorted}
				}
			\end{groupplot}
		\end{tikzpicture}
		\caption{
			Sorted
		}
		\medskip
	\end{subfigure}
	\begin{subfigure}{\textwidth}
		\tikzsetnextfilename{merge_starting_runs_uint64reverse}
		\begin{tikzpicture}[plot]
			\begin{groupplot}[merge fallback]
				\nextgroupplot[title={No Write-back\strut}]
				\expandafter\pgfplotsinvokeforeach\expandafter{\mergealgos}{
					\plotpernlognnew[merge sort filter #1]{Merge}{tableMergeStart#1_64reverse}
				}
				%
				\nextgroupplot[title={Write-back\strut}]
				\expandafter\pgfplotsinvokeforeach\expandafter{\mergealgos}{
					\plotpernlognnew[merge sort filter #1]{MergeWriteBack}{tableMergeStart#1_64reverse}
				}
				%
				\nextgroupplot[title={Half Space}]
				\expandafter\pgfplotsinvokeforeach\expandafter{\mergealgos}{
					\plotpernlognnew[merge sort filter #1]{MergeHalfSpace}{tableMergeStart#1_64reverse}
				}
			\end{groupplot}
		\end{tikzpicture}
		\caption{
			Reverse Sorted
		}
		\medskip
	\end{subfigure}
	\begin{subfigure}{\textwidth}
		\tikzsetnextfilename{merge_starting_runs_uint64almost}
		\begin{tikzpicture}[plot]
			\begin{groupplot}[merge fallback]
				\nextgroupplot[title={No Write-back\strut}]
				\expandafter\pgfplotsinvokeforeach\expandafter{\mergealgos}{
					\plotpernlognnew[merge sort filter #1]{Merge}{tableMergeStart#1_64almost}
				}
				%
				\nextgroupplot[title={Write-back\strut}]
				\expandafter\pgfplotsinvokeforeach\expandafter{\mergealgos}{
					\plotpernlognnew[merge sort filter #1]{MergeWriteBack}{tableMergeStart#1_64almost}
				}
				%
				\nextgroupplot[title={Half Space}]
				\expandafter\pgfplotsinvokeforeach\expandafter{\mergealgos}{
					\plotpernlognnew[merge sort filter #1]{MergeHalfSpace}{tableMergeStart#1_64almost}
				}
			\end{groupplot}
		\end{tikzpicture}
		\caption{
			Almost Sorted
		}
		\bigskip
	\end{subfigure}

	\hfil\pgfplotslegendfromname{leg:merge:starting_runs}\hfil
	\caption{
		An extension to \cref{fig:merge:starting_runs}.
		The data type is 64-bit unsigned integers.
	}
	\label{fig:merge:starting_runs_uint64sorted}
\end{figure}

\begin{figure}[ht]
	\pgfplotsset{height=3cm}
	\begin{subfigure}{\textwidth}
		\tikzsetnextfilename{merge_starting_runs_uint64uniform}
		\begin{tikzpicture}[plot]
			\begin{groupplot}[merge fallback]
				\nextgroupplot[title={No Write-back\strut}]
				\expandafter\pgfplotsinvokeforeach\expandafter{\mergealgos}{
					\plotpernlognnew[merge sort filter #1]{Merge}{tableMergeStart#1_64uniform}
				}
				%
				\nextgroupplot[title={Write-back\strut}]
				\expandafter\pgfplotsinvokeforeach\expandafter{\mergealgos}{
					\plotpernlognnew[merge sort filter #1]{MergeWriteBack}{tableMergeStart#1_64uniform}
				}
				%
				\nextgroupplot[title={Half Space}]
				\expandafter\pgfplotsinvokeforeach\expandafter{\mergealgos}{
					\plotpernlognnew[merge sort filter #1]{MergeHalfSpace}{tableMergeStart#1_64uniform}
				}
			\end{groupplot}
		\end{tikzpicture}
		\caption{
			Uniform
		}
		\medskip
	\end{subfigure}
	\begin{subfigure}{\textwidth}
		\tikzsetnextfilename{merge_starting_runs_uint64zipf}
		\begin{tikzpicture}[plot]
			\begin{groupplot}[merge fallback]
				\nextgroupplot[title={No Write-back\strut}]
				\expandafter\pgfplotsinvokeforeach\expandafter{\mergealgos}{
					\plotpernlognnew[merge sort filter #1]{Merge}{tableMergeStart#1_64zipf}
				}
				%
				\nextgroupplot[title={Write-back\strut}]
				\expandafter\pgfplotsinvokeforeach\expandafter{\mergealgos}{
					\plotpernlognnew[merge sort filter #1]{MergeWriteBack}{tableMergeStart#1_64zipf}
				}
				%
				\nextgroupplot[title={Half Space}]
				\expandafter\pgfplotsinvokeforeach\expandafter{\mergealgos}{
					\plotpernlognnew[merge sort filter #1]{MergeHalfSpace}{tableMergeStart#1_64zipf}
				}
			\end{groupplot}
		\end{tikzpicture}
		\caption{
			Zipf's
		}
		\medskip
	\end{subfigure}
	\begin{subfigure}{\textwidth}
		\tikzsetnextfilename{merge_starting_runs_uint64normal}
		\begin{tikzpicture}[plot]
			\begin{groupplot}[merge fallback]
				\nextgroupplot[title={No Write-back\strut}]
				\expandafter\pgfplotsinvokeforeach\expandafter{\mergealgos}{
					\plotpernlognnew[merge sort filter #1]{Merge}{tableMergeStart#1_64normal}
				}
				%
				\nextgroupplot[title={Write-back\strut}]
				\expandafter\pgfplotsinvokeforeach\expandafter{\mergealgos}{
					\plotpernlognnew[merge sort filter #1]{MergeWriteBack}{tableMergeStart#1_64normal}
				}
				%
				\nextgroupplot[title={Half Space}]
				\expandafter\pgfplotsinvokeforeach\expandafter{\mergealgos}{
					\plotpernlognnew[merge sort filter #1]{MergeHalfSpace}{tableMergeStart#1_64normal}
				}
			\end{groupplot}
		\end{tikzpicture}
		\caption{
			Normal
		}
		\bigskip
	\end{subfigure}

	\hfil\pgfplotslegendfromname{leg:merge:starting_runs}\hfil
	\caption{
		An extension to \cref{fig:merge:starting_runs}.
		The data type is 64-bit unsigned integers.
	}
	\label{fig:merge:starting_runs_uint64uniform}
\end{figure}


\clearpage

\subsection{\texorpdfstring{\HS{}}{HeapSort}}
\label{subapx:tasklet:heap}

\pgfplotsset{heap/.append style={adaptive group=3 by 2}}

\begin{figure}[!ht]
	\tikzsetnextfilename{heap_runtime_uint32}
	\begin{tikzpicture}[plot]
		\begin{groupplot}[
			heap,
			ymin=120,
			ymax=150,
		]
			\nextgroupplot[title={Sorted}]
			\expandafter\pgfplotsinvokeforeach\expandafter{\heapalgos}{
				\plotpernlogn{#1}{tableHeap_32sorted}
			}
			%
			\nextgroupplot[title={Reverse Sorted}]
			\expandafter\pgfplotsinvokeforeach\expandafter{\heapalgos}{
				\plotpernlogn{#1}{tableHeap_32reverse}
			}
			%
			\nextgroupplot[title={Almost Sorted}]
			\expandafter\pgfplotsinvokeforeach\expandafter{\heapalgos}{
				\plotpernlogn{#1}{tableHeap_32almost}
			}
			%
			\nextgroupplot[title={Uniform}]
			\expandafter\pgfplotsinvokeforeach\expandafter{\heapalgos}{
				\plotpernlogn{#1}{tableHeap_32uniform}
			}
			%
			\nextgroupplot[title={Zipf's}]
			\expandafter\pgfplotsinvokeforeach\expandafter{\heapalgos}{
				\plotpernlogn{#1}{tableHeap_32zipf}
			}
			%
			\nextgroupplot[title={Normal}]
			\expandafter\pgfplotsinvokeforeach\expandafter{\heapalgos}{
				\plotpernlogn{#1}{tableHeap_32normal}
			}
		\end{groupplot}
	\end{tikzpicture}

	\hfil\pgfplotslegendfromname{leg:heap:runtime}\hfil
	\caption{
		An extension to \cref{fig:heap:runtime}.
		The data type is 32-bit unsigned integers.
	}
	\label{fig:heap:runtime_uint32}
\end{figure}

\begin{figure}
	\tikzsetnextfilename{heap_runtime_uint64}
	\begin{tikzpicture}[plot]
		\begin{groupplot}[
			heap,
			ymin=145,
			ymax=185,
			extra y ticks={145,185},
		]
			\nextgroupplot[title={Sorted}]
			\expandafter\pgfplotsinvokeforeach\expandafter{\heapalgos}{
				\plotpernlogn{#1}{tableHeap_64sorted}
			}
			%
			\nextgroupplot[title={Reverse Sorted}]
			\expandafter\pgfplotsinvokeforeach\expandafter{\heapalgos}{
				\plotpernlogn{#1}{tableHeap_64reverse}
			}
			%
			\nextgroupplot[title={Almost Sorted}]
			\expandafter\pgfplotsinvokeforeach\expandafter{\heapalgos}{
				\plotpernlogn{#1}{tableHeap_64almost}
			}
			%
			\nextgroupplot[title={Uniform}]
			\expandafter\pgfplotsinvokeforeach\expandafter{\heapalgos}{
				\plotpernlogn{#1}{tableHeap_64uniform}
			}
			%
			\nextgroupplot[title={Zipf's}]
			\expandafter\pgfplotsinvokeforeach\expandafter{\heapalgos}{
				\plotpernlogn{#1}{tableHeap_64zipf}
			}
			%
			\nextgroupplot[title={Normal}]
			\expandafter\pgfplotsinvokeforeach\expandafter{\heapalgos}{
				\plotpernlogn{#1}{tableHeap_64normal}
			}
		\end{groupplot}
	\end{tikzpicture}

	\hfil\pgfplotslegendfromname{leg:heap:runtime}\hfil
	\caption{
		An extension to \cref{fig:heap:runtime}.
		The data type is 64-bit unsigned integers.
	}
	\label{fig:heap:runtime_uint64}
\end{figure}



	\mybibliography[heading=bibintoc]
\end{document}