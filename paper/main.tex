\documentclass[draft, british]{../garticle}

\RequirePackage{mfirstuc}  % used for capitalising names of sorting algorithms
% Maths & Algorithms.
\DeclareMathOperator{\lb}{lb}

% Figures.
% taken from: https://tex.stackexchange.com/a/199396
\pgfplotsset{
	select coords between index/.style 2 args={
		x filter/.code={
			\ifnum\coordindex<#1\def\pgfmathresult{}\fi
			\ifnum\coordindex>#2\def\pgfmathresult{}\fi
		},
	},
}

% Draws a plot of data in column µ_#2 of the table #3.
% A column n must be present. Applies custom options #1.
\NewDocumentCommand{\plotruntime}{O{} m m}{
	\addplot+
	plot [#1]
	table [x=n, y=µ_#2] {#3};
}

% Draws a plot of column µ_#2 with error bars loaded from σ_#2 of table #3.
% A column n must be present. Applies custom options #1.
\NewDocumentCommand{\plotwithbars}{O{} m m}{
	\addplot+
	plot [#1, error bars/.cd, y dir=both, y explicit]
%	table [x=n, y=µ_#2, y error=σ_#2] {#3};
	table [x=n, y=µ_#2, y error expr={2 * \thisrow{σ_#2}}] {#3};
}

% Draws a plot of column µ_#2 of table #3, divided by n log n.
% A column n must be present. Applies custom options #1.
\NewDocumentCommand{\plotpernlogn}{O{} m m}{
	\addplot+
	plot [#1]
	table [x=n, y expr={\thisrow{µ_#2} / (\thisrow{n} * log2(\thisrow{n}))}] {#3};
}

% Draws a plot of column µ_#2 of table #3, divided by n² (a column n must be present).
% A column n must be present. Applies custom options #1.
\NewDocumentCommand{\plotpernn}{O{} m m}{
	\addplot+
	plot [#1]
	table [x=n, y expr={\thisrow{µ_#2} / \thisrow{n}^2}] {#3};
}

% Draws a plot of column µ_#3 divided by column µ_#2 of table #4.
% A column n must be present. Applies custom options #1.
\NewDocumentCommand{\plotspeedup}{O{} m m m}{
	\addplot+
	plot [#1]
	table [x=n, y expr={\thisrow{µ_#3} / \thisrow{µ_#2}}] {#4};
}

% Texts.
\NewDocumentCommand{\BS}{s}{Bubble\-Sort\IfBooleanT{#1}{s}}
\NewDocumentCommand{\HS}{s}{Heap\-Sort\IfBooleanT{#1}{s}}
\NewDocumentCommand{\IS}{s}{Insertion\-Sort\IfBooleanT{#1}{s}}
\NewDocumentCommand{\QS}{s}{Quick\-Sort\IfBooleanT{#1}{s}}
\NewDocumentCommand{\SelS}{s}{Selection\-Sort\IfBooleanT{#1}{s}}
\NewDocumentCommand{\ShS}{s}{Shell\-Sort\IfBooleanT{#1}{s}}

\makeatletter
\@ifpackagewith{cleveref}{capitalize}{
	\crefname{implementation}{Impl.}{Impl.}
	\Crefname{implementation}{Implementation}{Implementations}
}{
	\crefname{implementation}{impl.}{impl.}
	\Crefname{implementation}{Implementation}{Implementations}
}

\makeatother


\titlehead{Group of Algorithm Engineering\hfill Summer Semester 2024 \\ Institute for Computer Science \\ Goethe University Frankfurt}
\subject{Master's Thesis}
%\title{Sorting \\ for a \\ Processing-in-Memory \\ Architecture}
%\title{Sorting \\ on a \\ Processing-in-Memory \\ Architecture}
%\title{On Sorting for Processing-in-Memory}
\title{On Efficient Sorting Through In-Memory Processing}
%\title{Engineering Sorting Algorithms \\ for a Processing-in-Memory Architecture}
%\title{Engineering Sorting Algorithms \\ for Processing-in-Memory}
%\subtitle{Some cool subtitle if need be}
\subtitle{Implementation and Evaluation \dots{} / Engineering \dots{} / Exploring \dots{}}
%\subtitle{Engineering Algorithms for UPMEM-based DRAM Processing Units}
\author{\texorpdfstring{Ƶ}{Z}eno Adrian \texorpdfstring{\Lss05{W\kern-1.5pt}}{W}eil}
\publishers{\begin{tabular}{r @{~}l}
	Supervisor: & Dr Manuel Penschuck
\end{tabular}}

\makeatletter
\hypersetup{
	pdfauthor=\@author,
	pdftitle=\@title,
	pdfsubject=\@subject,
}
\makeatother

\usepgfplotslibrary{groupplots}
\tikzset{
	plot/.style={  % Not set globally lest other packages break.
		trim axis group left, trim axis group right,  % Only axes define the bounding box. Thus, lables can extend into the margins.
	},
}
\pgfplotsset{
	height=4cm,
	cycle list name=exotic,  % automatic colouring of plots
	scale only axis,  % height and width apply only to the axes, not the labels
	enlargelimits={abs=3mm, auto},  % for axes without specified limits, the limits are a bit bigger than neeeded (→ padding)
	log ticks with fixed point,  % uses 0.1, 0.001, … on logarithmix axes instead of 10^-1, 10^-2, …
	ymajorgrids,  % grey, horizontal background lines at each y-tick
	legend style={  % adds padding to legends with multiple columns
		cells={anchor=west},  % left aligned labels
		/tikz/column 2/.style={column sep=5pt}, /tikz/column 4/.style={column sep=5pt}, /tikz/column 6/.style={column sep=5pt}, /tikz/column 8/.style={column sep=5pt}, /tikz/column 10/.style={column sep=5pt}, /tikz/column 12/.style={column sep=5pt}, /tikz/column 14/.style={column sep=5pt}, /tikz/column 16/.style={column sep=5pt},
	},
	group style={horizontal sep=5em, vertical sep=5em},
}

% Adds groupplot xlabel and groupplot ylabel, i.e. common axis labels for groupplots.
% groupplot xlabel should also be used if there is only one plot in the group plot for margin reasons.
% taken from: https://tex.stackexchange.com/a/117935
\makeatletter
\pgfplotsset{
	groupplot xlabel/.initial={},
	every groupplot x label/.style={
		at={($({\pgfplots@group@name\space c1r\pgfplots@group@rows.west}|-{\pgfplots@group@name\space c1r\pgfplots@group@rows.outer south})!0.5!({\pgfplots@group@name\space c\pgfplots@group@columns r\pgfplots@group@rows.east}|-{\pgfplots@group@name\space c\pgfplots@group@columns r\pgfplots@group@rows.outer south})$)},
		anchor=north,
	},
	groupplot ylabel/.initial={},
	every groupplot y label/.style={
		rotate=90,
		at={($({\pgfplots@group@name\space c1r1.north}-|{\pgfplots@group@name\space c1r1.outer
				west})!0.5!({\pgfplots@group@name\space c1r\pgfplots@group@rows.south}-|{\pgfplots@group@name\space c1r\pgfplots@group@rows.outer west})$)},
		anchor=south
	},
	execute at end groupplot/.code={%
		\node [/pgfplots/every groupplot x label]
		{\pgfkeysvalueof{/pgfplots/groupplot xlabel}};
		\node [/pgfplots/every groupplot y label]
		{\pgfkeysvalueof{/pgfplots/groupplot ylabel}};
	}
}

\def\endpgfplots@environment@groupplot{%
	\endpgfplots@environment@opt%
	\pgfkeys{/pgfplots/execute at end groupplot}%
	\endgroup%
}
\makeatother


\begin{document}
	\pagenumbering{gobble}  % turn off pagenumbering

%	\maketitle

%	\begin{abstract}
	\noindent
	The growing disparity between processing and memory speed, coupled with increasing data demands, has led to memory accesses being a bottleneck for many modern workflows.
	An example are sorting algorithms, which are often designed around the constraints set by memory subsytems.
	\Acl*{PIM} (also known as processing in memory, \acs*{PIM}) is an umbrella term encompassing several approaches which offload computational tasks to accelerators in or near the memory itself.
	In \acs*{PIM} systems designed and manufactured by \upmem{}, traditional dynamic random-access memory (\acs*{DRAM}) modules are augmented with general-purpose processors called \acfp*{DPU}.
	These are located next to the memory banks themselves, whereby high memory access speed is accomplished.
	An \upmem{}-based \acs*{PIM} system may contain thousands of \acsp*{DPU}, each capable of additional thread-level parallelism.
	Although designed for general use, the \acs*{DPU} architecture does come with limitations to its computational prowess.

	The scope of this thesis is the design, implementation, and evaluation of sorting algorithms which run on a single \acs*{DPU}.
	We investigate several sequential and parallel sorting algorithms, documenting the engineering process and adaptations to the merits and shortcomings of the \acs*{DPU} architecture.
	We find that sorting is a suitable task for a \acs*{DPU}, which can be sped up nearly ideally through multithreading.
	This paves the way for more large-scale sorting algorithms which run on multiple \acsp*{DPU}.
\end{abstract}


	\tableofcontents

	\listoftodos

	\begingroup
%	\clearpage
	\pagenumbering{arabic}  % turn on pagenumbering
	\endgroup

	\bigskip
	\todo[inline]{Architektur}
	\todo[inline]{Speicherzugriffe (memcpy, mram\_read \dots)}
	\todo[inline]{triple buffer}

	\pgfplotstableread{data/small_sorts.txt}{\tablesmallsorts}

\begin{figure}
	\begin{tikzpicture}[plot]
		\begin{groupplot}[
			width=0.4358\linewidth,
			group/group size=2 by 1,
			groupplot xlabel={Input Length \(n\)},
			xtick distance=3,
			minor xtick=data,
			legend entries={Insertion (sentinel), Insertion (no sentinel), Bubble, Bubble (adaptive), Selection},
			legend columns=3,
			legend to name={leg:speed-up_over_is},
		]
			\nextgroupplot[ylabel=Cycles / \(n^2\), ymin=0, ymax=80]
			\plotpernn{1}{\tablesmallsorts}
			\plotpernn{1NoSentinel}{\tablesmallsorts}
			\plotpernn{BubbleNonAdapt}{\tablesmallsorts}
			\plotpernn{BubbleAdapt}{\tablesmallsorts}
			\plotpernn{Selection}{\tablesmallsorts}
			%
			\nextgroupplot[ylabel=Speed-up, ymin=0.2, ymax=1]
			\plotonlyforticks{0}{\tablesmallsorts}  % Hacky: Invisible but needed to *properly* skip over first colour (even in the legend!)
			\plotspeedup{1NoSentinel}{1}{\tablesmallsorts}
			\plotspeedup{BubbleNonAdapt}{1}{\tablesmallsorts}
			\plotspeedup{BubbleAdapt}{1}{\tablesmallsorts}
			\plotspeedup{Selection}{1}{\tablesmallsorts}
		\end{groupplot}
	\end{tikzpicture}

	\hfil\pgfplotslegendfromname{leg:speed-up_over_is}\hfil
	\caption{
		Comparison of sorting algorithms with \(\bigoh{n^2}\) runtime on a uniform input distribution.
		The \IS*{} differ in whether they rely on sentinel values.
		The adaptive \BS{} terminates prematurely if no changes were made to the input array during an iteration.
		The speed-ups are with respect to the \IS{} relying on sentinel values.
	}
	\label{fig:speed-up_over_is}
\end{figure}

\paragraph{\IS{}}
This stable sorting algorithm works by moving the \(i\)th element to the left as long as its left neighbour is bigger, assuming that the elements \(0\) to \(i - 1\) are already sorted.
Even though in both the average case and the worst case, \IS{} has a runtime of \(\bigoh{n^2}\)\todo{Beleg?}, it features quite some advantages:
\begin{enumerate*}
	\item
	It works in-place, needing only \(\bigoh{1}\) additional space.

	\item
	It is inherently adaptive:
	If the input array is mostly or even fully sorted, the runtime drops down to \(\bigoh{n}\).

	\item
	Its program code is short, lending itself to inlining.

	\item
	The overhead is small.
\end{enumerate*}
Especially the last two points make \IS{} a good base algorithm for asymptotically better sorting algorithms to use on very small subarrays.

When moving an element to the left, two checks are needed:
Does the left neighbour exist and is it smaller than the element to move?
The first check can be omitted through the use of \emph{sentinel values}:
If the element at index \(-1\) is at least as small as any value in the input array, the leftwards motion stops there at the latest.
Since a DPU has no branch predictor, the slowdown from performing twice as many checks as needed is quite high and lies between 20\% and 40\% in the relevant input range (\cref{fig:speed-up_over_is}).%
\todo{ex- und implizite Wächterwerte benennen}
\todo[inline]{auf Compilersperenzchen eingehen?}

Other known simple sorting algorithm with similar runtime complexity are \SelS{} and \BS{}.
The asymptoticity, however, hides much higher constant factors such that even for as little as three elements \IS{} is superior (\cref{fig:speed-up_over_is}) and should always be used.



\begin{figure}
	\begin{tikzpicture}[plot]
		\begin{groupplot}[
			width=0.4358\linewidth,
			group/group size=2 by 1,
			groupplot xlabel={Input Length \(n\)},
			xtick distance=3,
			minor xtick=data,
			legend entries={\(1\), \(...\), \(9\)},
			legend columns=-1,
			legend to name={leg:shell_sort},
		]
			\nextgroupplot[ylabel=Cycles / \(n^2\), ymin=0, ymax=80]
			\pgfplotsinvokeforeach{1,...,9}{
				\plotpernn{#1}{\tablesmallsorts}
			}
			%
			\nextgroupplot[ylabel=Speed-up, ymin=0.6, ymax=1.2]
			\plotonlyforticks{0}{\tablesmallsorts}  % Hacky: Invisible but needed to *properly* skip over first colour (even in the legend!)
			\pgfplotsinvokeforeach{2,...,9}{
				\plotspeedup{#1}{1}{\tablesmallsorts}
			}
		\end{groupplot}
	\end{tikzpicture}

	\hfil\pgfplotslegendfromname{leg:shell_sort}\hfil
	\caption{
		Comparison of \IS{} and various \ShS*{} on a uniform input distribution.
		Each \ShS{} does one \IS{} pass with a step size between 2 and 9 before doing a pass of regular \IS{}.
		The speed-ups are with respect to the \IS{}.
	}
\end{figure}

\paragraph{\ShS{}}
\IS{} suffers from small elements at the end of the input array, since those have to be brought to the front through \(\bigoh{n}\) comparisons and swaps.
\ShS{}, proposed by Donald L. Shell in 1959 \cite{Shell1959AHS},


	\begin{description}
	\item[r0] start
	\item[r1] end
	\item[r23] return address
\end{description}
\begin{verbatim}
insertion_sort_nosentinel:
    jleu r0, r1, .LBB2_1  // Continue if array of positive length …
.LBB2_8:
    jump r23  // … else leave the function.
.LBB2_1:
    move r2, r0, true, .LBB2_2  // i ← start; Jump to beginning of outer loop.
.LBB2_5:
    move r4, r5  // ?
.LBB2_7:
    add r2, r2, 4  // i++
    sw r4, 0, r3  // *curr ← to_sort
    jgtu r2, r1, .LBB2_8  // If i > end, terminate.
.LBB2_2:  // Beginning of outer loop.
    lw r3, r2, 0  // to_sort ← *i;
    add r5, r2, -4  // prev ← i - 1
    move r4, r2  // curr ← i
    jltu r5, r0, .LBB2_7  // If prev < start, skip to the next iteration of the outer loop.
    move r5, r2  // (prev + 1) ← i
.LBB2_4:
    lw r6, r5, -4  // *prev
    jleu r6, r3, .LBB2_5  // If *prev > to_sort, terminate inner loop.
    add r4, r5, -4  // Store prev.
    add r7, r5, -8  // Store prev--.
    sw r5, 0, r6  // *curr ← *prev
    move r5, r4  // curr ← prev
    jgeu r7, r0, .LBB2_4  // If prev >= start, continue with the next iteration of the inner loop.
    jump .LBB2_7  // Continue with the next iteration of the outer loop.
\end{verbatim}

\begin{verbatim}
insertion_sort_sentinel:
    jleu r0, r1, .LBB3_1  // Continue if array of positive length …
.LBB3_5:
    jump r23  // … else leave the function.
.LBB3_4:
    add r0, r0, 4  // i++
    sw r3, 0, r2  // *curr ← to_sort
    jgtu r0, r1, .LBB3_5  // If i > end, leave the function.
.LBB3_1:
    lw r2, r0, 0  // to_sort ← *i
    lw r4, r0, -4  // *prev
    move r3, r0  // curr ← i
    jleu r4, r2, .LBB3_4  // If *prev > to_sort, terminate inner loop.
    move r3, r0  // ???
.LBB3_3:
    sw r3, 0, r4  // *curr ← *prev
    lw r4, r3, -8  // *(prev - 1)
    add r3, r3, -4  // curr ← prev
    jgtu r4, r2, .LBB3_3  // If *(prev - 1) > to_sort, continue with the next iteration of the inner loop.
    jump .LBB3_4  // Leave inner loop.
\end{verbatim}

%	\clearpage

	\mybibliography
\end{document}