\documentclass[draft, british, oneside]{../garticle}

% Maths & Algorithms.
\DeclareMathOperator{\lb}{lb}

% Figures.
% taken from: https://tex.stackexchange.com/a/199396
\pgfplotsset{
	select coords between index/.style 2 args={
		x filter/.code={
			\ifnum\coordindex<#1\def\pgfmathresult{}\fi
			\ifnum\coordindex>#2\def\pgfmathresult{}\fi
		},
	},
}

\NewDocumentCommand{\plotruntime}{O{} m m}{
	\addplot+
	plot [#1]
	table [x=n, y=µ_#2] {#3};
}

\NewDocumentCommand{\plotwithbars}{O{} m m}{
	\addplot+
	plot [#1, error bars/.cd, y dir=both, y explicit]
%	table [x=n, y=µ_#2, y error=σ_#2] {#3};
	table [x=n, y=µ_#2, y error expr={2 * \thisrow{σ_#2}}] {#3};
}

\NewDocumentCommand{\plotpernlogn}{O{} m m}{
	\addplot+
	plot [#1]
	table [x=n, y expr={\thisrow{µ_#2} / (\thisrow{n} * log2(\thisrow{n}))}] {#3};
}

\NewDocumentCommand{\plotpernn}{O{} m m}{
	\addplot+
	plot [#1]
	table [x=n, y expr={\thisrow{µ_#2} / \thisrow{n}^2}] {#3};
}

\NewDocumentCommand{\plotspeedup}{O{} m m m}{
	\addplot+
	plot [#1]
	table [x=n, y expr={\thisrow{µ_#3} / \thisrow{µ_#2}}] {#4};
}

% Texts.
\RequirePackage[final]{listings}
\lstset{
	basicstyle=\ttfamily\small,
	numbers=left,
	numbersep=4pt,
	language=C,
	tabsize=2,
}

\titlehead{Group of Algorithm Engineering\hfill Summer Semester 2024 \\ Institute for Computer Science \\ Goethe University Frankfurt}
\subject{Master's Thesis}
%\title{Sorting \\ for a \\ Processing-in-Memory \\ Architecture}
%\title{Sorting \\ on a \\ Processing-in-Memory \\ Architecture}
%\title{On Sorting for Processing-in-Memory}
\title{On Efficient Sorting Through In-Memory Processing}
%\title{Engineering Sorting Algorithms \\ for a Processing-in-Memory Architecture}
%\title{Engineering Sorting Algorithms \\ for Processing-in-Memory}
%\subtitle{Some cool subtitle if need be}
\subtitle{Implementation and Evaluation \dots{} / Engineering \dots{} / Exploring \dots{}}
%\subtitle{Engineering Algorithms for UPMEM-based DRAM Processing Units}
\author{\texorpdfstring{Ƶ}{Z}eno Adrian \texorpdfstring{\Lss05{W\kern-1.5pt}}{W}eil}
\publishers{\begin{tabular}{r @{~}l}
	Supervisor: & Dr Manuel Penschuck
\end{tabular}}

\makeatletter
\hypersetup{
	pdfauthor=\@author,
	pdftitle=\@title,
	pdfsubject=\@subject,
}
\makeatother

\usepgfplotslibrary{groupplots}
\tikzset{
	plot/.style={  % Not set globally lest other packages break.
		trim axis group left, trim axis group right,  % Only axes define the bounding box. Thus, lables can extend into the margins.
	},
}
\pgfplotsset{
	filter discard warning=false,
	height=4cm,
	cycle multiindex list={exotic \nextlist mark list*},  % automatic colouring of plots
	scale only axis,  % height and width apply only to the axes, not the labels
	enlargelimits={abs=3mm, auto},  % for axes without specified limits, the limits are a bit bigger than neeeded (→ padding)
	log ticks with fixed point,  % uses 0.1, 0.001, … on logarithmix axes instead of 10^-1, 10^-2, …
	ymajorgrids,  % grey, horizontal background lines at each y-tick
	set layers,  % needed as otherwise extra ticks are drawn above the border
	legend style={  % adds padding to legends with multiple columns
		cells={anchor=west},  % left aligned labels
		/tikz/column 2/.style={column sep=5pt}, /tikz/column 4/.style={column sep=5pt}, /tikz/column 6/.style={column sep=5pt}, /tikz/column 8/.style={column sep=5pt}, /tikz/column 10/.style={column sep=5pt}, /tikz/column 12/.style={column sep=5pt}, /tikz/column 14/.style={column sep=5pt}, /tikz/column 16/.style={column sep=5pt},
	},
	group/horizontal sep=18mm,  % padding between groupplots
	group/vertical sep=14mm,  % padding between groupplots
	title style={font=\sffamily\bfseries},  % changing the style of every plot title
	every axis title shift=0pt,  % removing default shift upwards of 6pt
}

% Adds groupplot xlabel and groupplot ylabel, i.e. common axis labels for groupplots.
% groupplot xlabel should also be used if there is only one plot in the group plot for margin reasons.
% taken from: https://tex.stackexchange.com/a/117935
\makeatletter
\pgfplotsset{
	groupplot xlabel/.initial={},
	every groupplot x label/.style={
		at={($({\pgfplots@group@name\space c1r\pgfplots@group@rows.west}|-{\pgfplots@group@name\space c1r\pgfplots@group@rows.outer south})!0.5!({\pgfplots@group@name\space c\pgfplots@group@columns r\pgfplots@group@rows.east}|-{\pgfplots@group@name\space c\pgfplots@group@columns r\pgfplots@group@rows.outer south})$)},
		anchor=north,
	},
	groupplot ylabel/.initial={},
	every groupplot y label/.style={
		rotate=90,
		at={($({\pgfplots@group@name\space c1r1.north}-|{\pgfplots@group@name\space c1r1.outer
				west})!0.5!({\pgfplots@group@name\space c1r\pgfplots@group@rows.south}-|{\pgfplots@group@name\space c1r\pgfplots@group@rows.outer west})$)},
		anchor=south
	},
	execute at end groupplot/.code={%
		\node [/pgfplots/every groupplot x label]
		{\pgfkeysvalueof{/pgfplots/groupplot xlabel}};
		\node [/pgfplots/every groupplot y label]
		{\pgfkeysvalueof{/pgfplots/groupplot ylabel}};
	}
}

\def\endpgfplots@environment@groupplot{%
	\endpgfplots@environment@opt%
	\pgfkeys{/pgfplots/execute at end groupplot}%
	\endgroup%
}
\makeatother

% Allows to read a table in and save it with any name (e.g. \pgfplotstablereadnamed{int32.dat}{tableInt32}).
% taken from: my brain
\ExplSyntaxOn
\NewDocumentCommand{\pgfplotstablereadnamed}{m m}{%
	\use:e { \exp_not:n { \pgfplotstableread { #1 } } { \exp_not:c { #2 } } }
}
\ExplSyntaxOff

% Allows to plot a table with a parametrised name (e.g.: \addplotnamedtable{table#1};).
% taken from: https://tex.stackexchange.com/a/603677
\ExplSyntaxOn
\NewDocumentCommand{\addplotnamedtable}{oO{}m}{%
	\IfNoValueTF{#1}{  % the call is like \addplot table[...}{...}
	\use:e { \exp_not:n { \addplot+ table~[#2] } { \exp_not:c { #3 } } }
}{  % the call is like \addplot [...] table [...] {...}
	\use:e { \exp_not:n { \addplot+ [#1]~table~[#2] } { \exp_not:c { #3 } } }
}
}
\ExplSyntaxOff

% Lets the user specify the group plot size in the usual row-by-column format and expands the group plot adequately horizontally.
% taken from: my brain
\makeatletter
\pgfplotsset{
	adaptive group/.style args={#1 by #2}{  % #rows by #cols
		group/group size=#2 by #1,  % #cols by #rows
		width=\dimexpr(\linewidth - \numexpr(\pgfplots@group@columns-1) * \pgfkeysvalueof{/pgfplots/group/horizontal sep}) / \pgfplots@group@columns,
	}
}
\makeatother

\DefineNamedColor{named}{accentcolor}{RGB}{118, 13, 28}  % Mordred's red

\renewcommand{\topfraction}{.8}
\renewcommand{\floatpagefraction}{.8}

\begin{document}
	\pagenumbering{gobble}  % turn off pagenumbering

%	\maketitle

%	\begin{abstract}
	\noindent
	The growing disparity between processing and memory speed, coupled with increasing data demands, has led to memory accesses being a bottleneck for many modern workflows.
	An example are sorting algorithms, which are often designed around the constraints set by memory subsytems.
	\Acl*{PIM} (also known as processing in memory, \acs*{PIM}) is an umbrella term encompassing several approaches which offload computational tasks to accelerators in or near the memory itself.
	In \acs*{PIM} systems designed and manufactured by \upmem{}, traditional dynamic random-access memory (\acs*{DRAM}) modules are augmented with general-purpose processors called \acfp*{DPU}.
	These are located next to the memory banks themselves, whereby high memory access speed is accomplished.
	An \upmem{}-based \acs*{PIM} system may contain thousands of \acsp*{DPU}, each capable of additional thread-level parallelism.
	Although designed for general use, the \acs*{DPU} architecture does come with limitations to its computational prowess.

	The scope of this thesis is the design, implementation, and evaluation of sorting algorithms which run on a single \acs*{DPU}.
	For several sequential and parallel sorting algorithms, we document the engineering process and adaptations to the merits and shortcomings of the \acs*{DPU} architecture.
	We find that sorting is a suitable task for a \acs*{DPU}, which can be sped up nearly ideally through multithreading.
	This paves the way for more large-scale sorting algorithms which run on multiple \acsp*{DPU}.
\end{abstract}


	\tableofcontents

	\listoftodos

	\begingroup
%	\clearpage
	\pagenumbering{arabic}  % turn on pagenumbering
	\endgroup

	\bigskip
	\todo[inline]{Architektur}
	\todo[inline]{Speicherzugriffe (memcpy, mram\_read \dots)}
	\todo[inline]{triple buffer}

	We took our cue from \citeauthor{axtmann2020engineering}~\cite{axtmann2020engineering} for the choice of distributions.
	\begin{description}
		\item[Sorted]
		The numbers from \(0\) to \(n - 1\) are generated in ascending order.

		\item[Reverse Sorted]
		The numbers from \(0\) to \(n - 1\) are generated in descending order.

		\item[Almost Sorted]
		First, the numbers from \(0\) to \(n - 1\) are generated in ascending order, then, \(\floor{\sqrt{n}}\) random pairs are sequentially drawn and swapped.
		There are no checks on whether pairs have common elements.

		\item[Uniform]
		Each number is drawn independently and uniformly from the range \([0, 2^{31} - 1]\).

		\item[Narrow Uniform]
		Each number is drawn independently and uniformly from the range \([0, n - 1]\).
		\todo{Bisher nicht. Als Ersatz für die Normalverteilung?}

		\item[Zipf's]
		Each number is drawn independently from the range \([1, 100]\), with each value \(k\) drawn with a probability proportional to \(\sfrac{1}{k^{0.75}}\).

		\item[Normal]
		Each number is drawn independently according to a normal distribution with mean \(\mu = 2^{31}\) and standard deviation \(\sigma = \min\paren*{1, \floor*{\sfrac{n}{8}}}\).
	\end{description}

	\section{Sorting with One Tasklet}

\subsection{\IS{}}

\pgfplotstableread{data/small_sorts.txt}{\tablesmallsorts}

\begin{figure}
	\begin{tikzpicture}[plot]
		\begin{groupplot}[
			width=0.4358\linewidth,
			group/group size=2 by 1,
			groupplot xlabel={Input Length \(n\)},
			xtick distance=3,
			minor xtick=data,
			legend columns=-1,
		]
			\nextgroupplot[ylabel=Cycles / \(n^2\), ymin=0, ymax=70, extra y ticks={70}, legend to name=leg:insertion_sort]
			\legend{Insertion, Insertion (no sentinel), Bubble, Bubble(adaptive), Selection}
			\plotpernn{1}{\tablesmallsorts}
			\plotpernn{1NoSentinel}{\tablesmallsorts}
			\plotpernn{BubbleNonAdapt}{\tablesmallsorts}
			\plotpernn{BubbleAdapt}{\tablesmallsorts}
			\plotpernn{Selection}{\tablesmallsorts}
			%
			\nextgroupplot[ylabel=Speed-up, ymin=0.2, ymax=1]
			\pgfplotsset{cycle list shift=1}
			\plotspeedup{1NoSentinel}{1}{\tablesmallsorts}
			\plotspeedup{BubbleNonAdapt}{1}{\tablesmallsorts}
			\plotspeedup{BubbleAdapt}{1}{\tablesmallsorts}
			\plotspeedup{Selection}{1}{\tablesmallsorts}
		\end{groupplot}
	\end{tikzpicture}

	\hfil\pgfplotslegendfromname{leg:insertion_sort}\hfil
	\caption{
		Comparison of sorting algorithms with \(\bigoh{n^2}\) runtime.
		The \IS*{} differ in whether they rely on sentinel values.
		The adaptive \BS{} terminates prematurely if no changes were made to the input array during an iteration.
		The speed-ups are with respect to the \IS{} relying on sentinel values.
	}
	\label{fig:insertion_sort}
\end{figure}

This stable sorting algorithm works by moving the \(i\)th element to the left as long as its left neighbour is bigger, assuming that the elements \(0\) to \(i - 1\) are already sorted.
Even though in both the average case and the worst case, \IS{} has a runtime of \(\bigoh{n^2}\)\todo{Beleg?}, it features quite some advantages:
\begin{enumerate*}
	\item
	It works in-place, needing only \(\bigoh{1}\) additional space.

	\item
	It is inherently adaptive:
	If the input array is mostly or even fully sorted, the runtime drops down to \(\bigoh{n}\).

	\item
	Its program code is short, lending itself to inlining.

	\item
	The overhead is small.
\end{enumerate*}
Especially the last two points make \IS{} a good base algorithm for asymptotically better sorting algorithms to use on very small subarrays.

When moving an element to the left, two checks are needed:
Does the left neighbour exist and is it smaller than the element to move?
The first check can be omitted through the use of \emph{sentinel values}:
If the element at index \(-1\) is at least as small as any value in the input array, the leftwards motion stops there at the latest.
Since a DPU has no branch predictor, the slowdown from performing twice as many checks as needed is quite high and lies between 20\% and 40\%\todo{auf Kompilat eingehen?} in the relevant input range (\cref{fig:insertion_sort}).%
\todo{ex- und implizite Wächterwerte benennen}
Thence, \enquote{\IS{}} refers to the version relying on sentinel values henceforth.
\todo[inline]{auf Compilersperenzchen eingehen?}

\begin{note}
	Other known simple sorting algorithm with similar runtime complexity are \SelS{} and \BS{}.
	The asymptoticity, however, hides much higher constant factors such that even for as little as three elements \IS{} is superior (\cref{fig:insertion_sort}) and should always be preferred.
\end{note}

\subsection{\texorpdfstring{\ShS{}}{ShellSort}}
\label{subsec:single:shell}

\pgfplotstableread{data/shell/no_third_step.txt}{\tableShellNoThirdStep}
\pgfplotstableread{data/shell/h1=7.txt}{\tableShellVII}
\pgfplotstableread{data/shell/h1=8.txt}{\tableShellVIII}
\pgfplotstableread{data/shell/h1=9.txt}{\tableShellIX}
\pgfplotstableread{data/shell/h1=10.txt}{\tableShellX}
\pgfplotstableread{data/shell/h1=11.txt}{\tableShellXI}
\pgfplotstableread{data/shell/h1=12.txt}{\tableShellXII}
\pgfplotstableread{data/shell/h1=13.txt}{\tableShellXIII}
\pgfplotstableread{data/shell/h1=14.txt}{\tableShelLxiv}
\pgfplotstableread{data/shell/h1=15.txt}{\tableShellXV}
\pgfplotstableread{data/shell/h1=16.txt}{\tableShellXVI}
\pgfplotstableread{data/shell/h1=17.txt}{\tableShellXVII}
\pgfplotstableread{data/shell/custom.txt}{\tableShellCustom}

\begin{figure}
	\begin{tikzpicture}[plot]
		\begin{groupplot}[
			adaptive group=1 by 2,
			groupplot xlabel={Input Length \(n\)},
			xtick distance=3,
			minor xtick=data,
			legend columns=-1,
		]
			\nextgroupplot[ylabel=Cycles / \(n^2\), ymin=0, ymax=60, legend to name=leg:shell:two_rounds]
			\legend{\(1\), \(...\), \(9\)}
			\pgfplotsinvokeforeach{1,...,9}{
				\plotpernn{#1}{\tableSmallSortsXxxiiUniform}
			}
			%
			\nextgroupplot[ylabel=Speed-up, ymin=0.6, ymax=1.2]
			\pgfplotsset{cycle list shift=1}
			\pgfplotsinvokeforeach{2,...,9}{
				\plotspeedup{#1}{1}{\tableSmallSortsXxxiiUniform}
			}
		\end{groupplot}
	\end{tikzpicture}

	\hfil\pgfplotslegendfromname{leg:shell:two_rounds}\hfil
	\caption{
		Comparison of \IS{} (1) and various \ShS*{} (2--9).
		Each \ShS{} does one \IS{} pass with a step size between 2 and 9 before doing a pass of regular \IS{}.
		The speed-ups are with respect to the \IS{}.
	}
	\label{fig:shell:two_rounds}
\end{figure}

\IS{} suffers from small elements at the end of the input, since those have to be brought to the front through \(\bigoh{n}\) comparisons and swaps.
\ShS{}, proposed by Donald L. Shell in 1959~\cite{Shell1959AHS}, remedies this by doing multiple passes of \IS{} with different step sizes:
In round~\(r\) with step size \(\stepsizes_r\), the input array is divided into the subarrays of indices \((i, \stepsizes_r + i, 2 \stepsizes_r + i, \dots)\) for \(i = 0, \dots, \stepsizes_r - 1\) which then get sorted individually through \IS{}.
The step size get smaller each round, with the final step size being \(1\) such that a regular \IS{} is performed.
Intuitively, the individual \IS*{} are fast since elements which need to travel long distances already did big jumps.
Finding the right balance between the heightened overhead through multiple \IS{} passes and the shortened runtime of each \IS{} pass is subject to research to this day \cite{skean2023optimization, lee2021empirically} and depends on the cost of the operations (comparing, swapping, looping).

Let us first focus on small input arrays where only two rounds with step sizes~\(\stepsizes_1\) and \(1\) suffice.
The previous results on \IS{} suggest that \ShS{} should make use of~\(\stepsizes_1\) sentinel values lest bounds checking eats any gain up.
\Cref{fig:shell:two_rounds} shows that the additional rounds starts to pay off at around 20 elements for \(\stepsizes_1 \ge 3\).
Bear in mind that these measurements were conducted on a uniform input distribution;
if \ShS{} is used by another algorithm on a subarray, these thresholds may be higher or even non-existent due to some degree of presorting.

\NewDocumentCommand{\shellscatter}{m m}{
	\addplot+ plot [select coords between index={#1}{#1}, forget plot] table [x=µ_#2, y expr={6}] {\tableShellNoThirdStep};
	\ifnumless{#2}{7}{
		\addplot+ plot [select coords between index={#1}{#1}, forget plot] table [x=µ_#2, y expr={7}] {\tableShellVII};
	}{}
	\ifnumless{#2}{8}{
		\addplot+ plot [select coords between index={#1}{#1}, forget plot] table [x=µ_#2, y expr={8}] {\tableShellVIII};
	}{}
	\ifnumless{#2}{9}{
		\addplot+ plot [select coords between index={#1}{#1}, forget plot] table [x=µ_#2, y expr={9}] {\tableShellIX};
	}{}
	\addplot+ plot [select coords between index={#1}{#1}] table [x=µ_#2, y expr={10}, forget plot] {\tableShellX};
	\addplot+ plot [select coords between index={#1}{#1}] table [x=µ_#2, y expr={11}, forget plot] {\tableShellXI};
	\addplot+ plot [select coords between index={#1}{#1}] table [x=µ_#2, y expr={12}, forget plot] {\tableShellXII};
	\addplot+ plot [select coords between index={#1}{#1}] table [x=µ_#2, y expr={13}, forget plot] {\tableShellXIII};
	\addplot+ plot [select coords between index={#1}{#1}] table [x=µ_#2, y expr={14}, forget plot] {\tableShelLxiv};
	\addplot+ plot [select coords between index={#1}{#1}] table [x=µ_#2, y expr={15}, forget plot] {\tableShellXV};
	\addplot+ plot [select coords between index={#1}{#1}] table [x=µ_#2, y expr={16}, forget plot] {\tableShellXVI};
	\addplot+ plot [select coords between index={#1}{#1}] table [x=µ_#2, y expr={17}] {\tableShellXVII};
}

\begin{figure}[p]
	\begin{tikzpicture}[plot]
		\newcommand{\setn}[1]{\textit{n} = #1}
		\begin{groupplot}[
			group/horizontal sep=12mm,
			adaptive group=3 by 2,
			groupplot xlabel={Mean [\(10^4\) Cycles]},
			groupplot ylabel={\(\stepsizes_1\)},
			scaled x ticks=base 10:-4,
			xtick scale label code/.code={},  % removes exponent underneath the axis
			ytick={6, 7, 9, ..., 17},
			yticklabels={/, \(7\), \(9\), \(...\), \(17\)},
			/tikz/only marks,
			cycle list shift=2,  % for sharing colours with the previous figure
			legend columns=-1,
		]
			\nextgroupplot[title={Input Length \setn{24}}]
			\pgfplotsinvokeforeach{3,...,9}{
				\shellscatter{0}{#1}
			}
			%
			\nextgroupplot[title={Input Length \setn{32}}]
			\pgfplotsinvokeforeach{3,...,9}{
				\shellscatter{1}{#1}
			}
			%
			\nextgroupplot[title={Input Length \setn{48}}]
			\pgfplotsinvokeforeach{3,...,9}{
				\shellscatter{2}{#1}
			}
			%
			\nextgroupplot[title={Input Length \setn{64}}]
			\pgfplotsinvokeforeach{3,...,9}{
				\shellscatter{3}{#1}
			}
			%
			\nextgroupplot[title={Input Length \setn{96}}]
			\pgfplotsinvokeforeach{3,...,9}{
				\shellscatter{4}{#1}
			}
			%
			\nextgroupplot[title={Input Length \setn{128}}, legend to name=leg:shell:three_rounds]
			\legend{\(3\), \(4\), \(5\), \(6\), \(7\), \(8\), \(9\)}
			\pgfplotsinvokeforeach{3,...,9}{
				\shellscatter{5}{#1}
			}
		\end{groupplot}
	\end{tikzpicture}

	\hfil\pgfplotslegendfromname{leg:shell:three_rounds}\hfil
	\caption{
		Runtimes of \ShS*{} with two rounds (/) and three rounds (7--17).
		The coloured symbols encode the step size \(\stepsizes_1\) for two-round \ShS*{} and the step size~\(\stepsizes_2\) for three-round \ShS*{}.
		For the latter, the step size \(\stepsizes_1\) is noted on the y-axes.
	}
	\label{fig:shell:three_rounds}
\end{figure}

\begin{figure}
	\begin{tikzpicture}[plot]
		\begin{groupplot}[
			adaptive group=1 by 2,
			groupplot xlabel={Input Length \(n\)},
			xtick distance=16,
			minor xtick=data,
			legend columns=-1,
		]
			\nextgroupplot[ylabel=Cycles / \((n \lb n)\), ymin=0, ymax=225, extra y ticks={50, 150, 225}, legend to name=leg:shell:against_others]
			\legend{Insertion, Shell, Quick}
			\plotpernlogn{InsertSent}{\tableShellCustom}
			\plotpernlogn{ShellCiura}{\tableShellCustom}
			\plotpernlogn{Quick}{\tableShellCustom}
			%
			\nextgroupplot[ylabel=Speed-up, ymin=0.2, ymax=1.2]
			\plotspeedup{InsertSent}{Quick}{\tableShellCustom}
			\plotspeedup{ShellCiura}{Quick}{\tableShellCustom}
		\end{groupplot}
	\end{tikzpicture}

	\hfil\pgfplotslegendfromname{leg:shell:against_others}\hfil
	\caption{
		Comparison of \IS{}, \ShS{} and \QS{}.
		The \ShS{} uses the step sizes \(\stepsizes = (6, 1)\) for \(n \le 64\) and \(\stepsizes = (17, 4, 1)\) elsewise.
		The speed-ups are with respect to the \QS{}.
		\todo[inline]{\QS{} mit Zufallspivot auch noch einbauen?}
		\todo[inline]{verschiedene Verteilungen?}
	}
	\label{fig:shell:against_others}
\end{figure}

When moving to greater input lengths (\cref{fig:shell:three_rounds}), the differences in performance between the two-round \ShS*{} become more pronounced;
especially the ones with \(\stepsizes_1 = 3\) and \(\stepsizes_1 = 4\) fall off whereas the one with \(\stepsizes_1 = 6\) holds its ground quite well.
With more than 64 elements, three round get worthwhile to consider.
Interestingly, many \ShS*{} with \(\stepsizes_2 = 4\) take the lead whilst the ones with \(\stepsizes_2 = 6\) are mid-table.
This is in accordance with \citeauthor{10.1007/3-540-44669-9_12} \cite{10.1007/3-540-44669-9_12} who noted~\(\stepsizes = (17, 4, 1)\) to be the optimal triplet for 128 elements.
It is noteworthy, though, that he measured the quadruplet~\(\stepsizes = (38, 9, 4, 1)\) to be about 5\% faster in the MIX machine model.
On a DPU, this sequence leads to a runtime of nearly exactly 74.000 cycles, placing it only mid-table.
Without access to Ciura's original code, giving a satisfactory explanation for the discrepancy is hard, however.

But would pushing the limits of \ShS{} even be rewarding?
Two issues come up.
Firstly, greater input lengths require greater steps \Dash well into the three digits for \(n \approx 1000\) \cite{skean2023optimization, 10.1007/3-540-44669-9_12} \Dash and those in turn require more sentinel values.
But the more sentinel values are stored, the less space is available for the actual input array, leading to smaller runs and thus hurting the overall sorting algorithm.
Explorative testing suggests that falling back to bounds checking for big steps is too punishing.
Secondly, there simply are better alternatives, namely \QS{} (which will be discussed in more detail in the next section).
\Cref{fig:shell:against_others} shows that even though \ShS{} takes just a fraction of the time \IS{} takes \Dash apparently achieving a runtime between \(\bigomega{n \lb n}\) and \(\bigoh{n \lb^2 n}\) \Dash\negthinspace, \QS{} beats both from 20 elements onwards.
Even \QS{}'s standard deviation of 1432 cycles at 128 elements is superior to \ShS{}'s 2670 cycles.
Together with \cref{fig:shell:two_rounds}, this means that \ShS{} is not worth using at all and will, consequently, not be improved upon in this thesis.


	\appendix

	\section{Further Measurements on Sorting with One Tasklet}
\label{app:single}

\subsection{\texorpdfstring{\IS{}}{InsertionSort}}
\label{subapx:single:insertion}

\pgfplotstableread{data/small sorts/uint32/sorted.txt}{\tableSmallSortsXxxiiSorted}
\pgfplotstableread{data/small sorts/uint32/reverse.txt}{\tableSmallSortsXxxiiReverse}
\pgfplotstableread{data/small sorts/uint32/almost.txt}{\tableSmallSortsXxxiiAlmost}
\pgfplotstableread{data/small sorts/uint32/zipf.txt}{\tableSmallSortsXxxiiZipf}
\pgfplotstableread{data/small sorts/uint32/normal.txt}{\tableSmallSortsXxxiiNormal}
\pgfplotstableread{data/small sorts/uint64/sorted.txt}{\tableSmallSortsLxivSorted}
\pgfplotstableread{data/small sorts/uint64/reverse.txt}{\tableSmallSortsLxivReverse}
\pgfplotstableread{data/small sorts/uint64/almost.txt}{\tableSmallSortsLxivAlmost}
\pgfplotstableread{data/small sorts/uint64/uniform.txt}{\tableSmallSortsLxivUniform}
\pgfplotstableread{data/small sorts/uint64/zipf.txt}{\tableSmallSortsLxivZipf}
\pgfplotstableread{data/small sorts/uint64/normal.txt}{\tableSmallSortsLxivNormal}

\begin{figure}[!h]
	\def\algos{1NoSentinel,1,1Implicit,BubbleNonAdapt,BubbleAdapt,Selection}
	\tikzsetnextfilename{insertion_against_others_uint32}
	\begin{tikzpicture}[plot]
		\begin{groupplot}[
			adaptive group=3 by 2,
			groupplot xlabel={Input Length \(n\)},
			groupplot ylabel={Cycles / \(n^2\)},
			xtick distance=3,
			minor xtick=data,
			ymin=0,
			ymax=60,
			legend columns=3,
		]
			\nextgroupplot[title={Sorted}, legend to name=leg:insertion:against_others_uint32]
			\legend{\IS{} (no sentinel), \IS{} (sentinel), \IS{} (implicit), \BS{} (not adaptive), \BS{} (adaptive), \SelS{}}
			\expandafter\pgfplotsinvokeforeach\expandafter{\algos}{
				\plotpernn{#1}{\tableSmallSortsXxxiiSorted}
			}
			%
			\nextgroupplot[title={Reverse Sorted}]
			\expandafter\pgfplotsinvokeforeach\expandafter{\algos}{
				\plotpernn{#1}{\tableSmallSortsXxxiiReverse}
			}
			%
			\nextgroupplot[title={Almost Sorted}]
			\expandafter\pgfplotsinvokeforeach\expandafter{\algos}{
				\plotpernn{#1}{\tableSmallSortsXxxiiAlmost}
			}
			%
			\nextgroupplot[title={Uniform}]
			\expandafter\pgfplotsinvokeforeach\expandafter{\algos}{
				\plotpernn{#1}{\tableSmallSortsXxxiiUniform}
			}
			%
			\nextgroupplot[title={Zipf's}]
			\expandafter\pgfplotsinvokeforeach\expandafter{\algos}{
				\plotpernn{#1}{\tableSmallSortsXxxiiZipf}
			}
			%
			\nextgroupplot[title={Normal}]
			\expandafter\pgfplotsinvokeforeach\expandafter{\algos}{
				\plotpernn{#1}{\tableSmallSortsXxxiiNormal}
			}
		\end{groupplot}
	\end{tikzpicture}

	\hfil\pgfplotslegendfromname{leg:insertion:against_others_uint32}\hfil
	\caption{
		An extension to \cref{fig:insertion:against_others}.
		The data type is 32-bit unsigned integers.
	}
	\label{fig:insertion:against_others_uint32}
\end{figure}

\begin{figure}
	\def\algos{1NoSentinel,1,1Implicit,BubbleNonAdapt,BubbleAdapt,Selection}
	\tikzsetnextfilename{insertion_against_others_uint64}
	\begin{tikzpicture}[plot]
		\begin{groupplot}[
			adaptive group=3 by 2,
			groupplot xlabel={Input Length \(n\)},
			groupplot ylabel={Cycles / \(n^2\)},
			xtick distance=3,
			minor xtick=data,
			ymin=0,
			ymax=70,
			extra y ticks={70},
			legend columns=3,
		]
			\nextgroupplot[title={Sorted}, legend to name=leg:insertion:against_others_uint64]
			\legend{\IS{} (no sentinel), \IS{} (sentinel), \IS{} (implicit), \BS{} (not adaptive), \BS{} (adaptive), \SelS{}}
			\expandafter\pgfplotsinvokeforeach\expandafter{\algos}{
				\plotpernn{#1}{\tableSmallSortsLxivSorted}
			}
			%
			\nextgroupplot[title={Reverse Sorted}]
			\expandafter\pgfplotsinvokeforeach\expandafter{\algos}{
				\plotpernn{#1}{\tableSmallSortsLxivReverse}
			}
			%
			\nextgroupplot[title={Almost Sorted}]
			\expandafter\pgfplotsinvokeforeach\expandafter{\algos}{
				\plotpernn{#1}{\tableSmallSortsLxivAlmost}
			}
			%
			\nextgroupplot[title={Uniform}]
			\expandafter\pgfplotsinvokeforeach\expandafter{\algos}{
				\plotpernn{#1}{\tableSmallSortsLxivUniform}
			}
			%
			\nextgroupplot[title={Zipf's}]
			\expandafter\pgfplotsinvokeforeach\expandafter{\algos}{
				\plotpernn{#1}{\tableSmallSortsLxivZipf}
			}
			%
			\nextgroupplot[title={Normal}]
			\expandafter\pgfplotsinvokeforeach\expandafter{\algos}{
				\plotpernn{#1}{\tableSmallSortsLxivNormal}
			}
		\end{groupplot}
	\end{tikzpicture}

	\hfil\pgfplotslegendfromname{leg:insertion:against_others_uint64}\hfil
	\caption{
		An extension to \cref{fig:insertion:against_others}.
		The data type is 64-bit unsigned integers.
	}
	\label{fig:insertion:against_others_uint64}
\end{figure}

\subsection{\texorpdfstring{\ShS{}}{ShellSort}}
\label{subapp:single:shell}

\begin{figure}
	\begin{tikzpicture}[plot]
		\begin{groupplot}[
			adaptive group=3 by 2,
			groupplot xlabel={Input Length \(n\)},
			groupplot ylabel={Cycles / \(n^2\)},
			xtick distance=3,
			minor xtick=data,
			legend columns=-1,
		]
			\nextgroupplot[title={Sorted}, legend to name=leg:shell:two-tier_uint32]
			\legend{\(1\), \(...\), \(9\)}
			\pgfplotsinvokeforeach{1,...,9}{
				\plotpernn{#1}{\tableSmallSortsXxxiiSorted}
			}
			%
			\nextgroupplot[title={Reverse Sorted}]
			\pgfplotsinvokeforeach{1,...,9}{
				\plotpernn{#1}{\tableSmallSortsXxxiiReverse}
			}
			%
			\nextgroupplot[title={Almost Sorted}]
			\pgfplotsinvokeforeach{1,...,9}{
				\plotpernn{#1}{\tableSmallSortsXxxiiAlmost}
			}
			%
			\nextgroupplot[title={Uniform}]
			\pgfplotsinvokeforeach{1,...,9}{
				\plotpernn{#1}{\tableSmallSortsXxxiiUniform}
			}
			%
			\nextgroupplot[title={Zipf's}]
			\pgfplotsinvokeforeach{1,...,9}{
				\plotpernn{#1}{\tableSmallSortsXxxiiZipf}
			}
			%
			\nextgroupplot[title={Normal}]
			\pgfplotsinvokeforeach{1,...,9}{
				\plotpernn{#1}{\tableSmallSortsXxxiiNormal}
			}
		\end{groupplot}
	\end{tikzpicture}

	\hfil\pgfplotslegendfromname{leg:shell:two-tier_uint32}\hfil
	\caption{
		A continuation of \cref{fig:shell:two-tier} with more input distributions.
		The data type is 32-bit unsigned integers.
	}
	\label{fig:shell:two-tier_uint32}
\end{figure}

\begin{figure}
	\begin{tikzpicture}[plot]
		\begin{groupplot}[
			adaptive group=3 by 2,
			groupplot xlabel={Input Length \(n\)},
			groupplot ylabel={Cycles / \(n^2\)},
			xtick distance=3,
			minor xtick=data,
			legend columns=-1,
		]
			\nextgroupplot[title={Sorted}, legend to name=leg:shell:two-tier_uint64]
			\legend{\(1\), \(...\), \(9\)}
			\pgfplotsinvokeforeach{1,...,9}{
				\plotpernn{#1}{\tableSmallSortsLxivSorted}
			}
			%
			\nextgroupplot[title={Reverse Sorted}]
			\pgfplotsinvokeforeach{1,...,9}{
				\plotpernn{#1}{\tableSmallSortsLxivReverse}
			}
			%
			\nextgroupplot[title={Almost Sorted}]
			\pgfplotsinvokeforeach{1,...,9}{
				\plotpernn{#1}{\tableSmallSortsLxivAlmost}
			}
			%
			\nextgroupplot[title={Uniform}]
			\pgfplotsinvokeforeach{1,...,9}{
				\plotpernn{#1}{\tableSmallSortsLxivUniform}
			}
			%
			\nextgroupplot[title={Zipf's}]
			\pgfplotsinvokeforeach{1,...,9}{
				\plotpernn{#1}{\tableSmallSortsLxivZipf}
			}
			%
			\nextgroupplot[title={Normal}]
			\pgfplotsinvokeforeach{1,...,9}{
				\plotpernn{#1}{\tableSmallSortsLxivNormal}
			}
		\end{groupplot}
	\end{tikzpicture}

	\hfil\pgfplotslegendfromname{leg:shell:two-tier_uint64}\hfil
	\caption{
		A continuation of \cref{fig:shell:two-tier} with more input distributions.
		The data type is 64-bit unsigned integers.
	}
	\label{fig:shell:two-tier_uint64}
\end{figure}



	\mybibliography[heading=bibintoc]
\end{document}