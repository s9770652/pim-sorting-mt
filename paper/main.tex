\documentclass[draft, british]{../garticle}

% Maths & Algorithms.
\DeclareMathOperator{\lb}{lb}

% Figures.
% taken from: https://tex.stackexchange.com/a/199396
\pgfplotsset{
	select coords between index/.style 2 args={
		x filter/.code={
			\ifnum\coordindex<#1\def\pgfmathresult{}\fi
			\ifnum\coordindex>#2\def\pgfmathresult{}\fi
		},
	},
}

% Draws a plot of data in column µ_#2 of the table #3.
% A column n must be present. Applies custom options #1.
\NewDocumentCommand{\plotruntime}{O{} m m}{
	\addplot+
	plot [#1]
	table [x=n, y=µ_#2] {#3};
}

% Draws a plot of column µ_#2 with error bars loaded from σ_#2 of table #3.
% A column n must be present. Applies custom options #1.
\NewDocumentCommand{\plotwithbars}{O{} m m}{
	\addplot+
	plot [#1, error bars/.cd, y dir=both, y explicit]
%	table [x=n, y=µ_#2, y error=σ_#2] {#3};
	table [x=n, y=µ_#2, y error expr={2 * \thisrow{σ_#2}}] {#3};
}

% Draws a plot of column µ_#2 of table #3, divided by n log n.
% A column n must be present. Applies custom options #1.
\NewDocumentCommand{\plotpernlogn}{O{} m m}{
	\addplot+
	plot [#1]
	table [x=n, y expr={\thisrow{µ_#2} / (\thisrow{n} * log2(\thisrow{n}))}] {#3};
}

% Draws a plot of column µ_#2 of table #3, divided by n² (a column n must be present).
% A column n must be present. Applies custom options #1.
\NewDocumentCommand{\plotpernn}{O{} m m}{
	\addplot+
	plot [#1]
	table [x=n, y expr={\thisrow{µ_#2} / \thisrow{n}^2}] {#3};
}

% Draws a plot of column µ_#3 divided by column µ_#2 of table #4.
% A column n must be present. Applies custom options #1.
\NewDocumentCommand{\plotspeedup}{O{} m m m}{
	\addplot+
	plot [#1]
	table [x=n, y expr={\thisrow{µ_#3} / \thisrow{µ_#2}}] {#4};
}

% Texts.
\NewDocumentCommand{\BS}{s}{Bubble\-Sort\IfBooleanT{#1}{s}}
\NewDocumentCommand{\HS}{s}{Heap\-Sort\IfBooleanT{#1}{s}}
\NewDocumentCommand{\IS}{s}{Insertion\-Sort\IfBooleanT{#1}{s}}
\NewDocumentCommand{\QS}{s}{Quick\-Sort\IfBooleanT{#1}{s}}
\NewDocumentCommand{\SelS}{s}{Selection\-Sort\IfBooleanT{#1}{s}}
\NewDocumentCommand{\ShS}{s}{Shell\-Sort\IfBooleanT{#1}{s}}

\makeatletter
\@ifpackagewith{cleveref}{capitalize}{
	\crefname{implementation}{Impl.}{Impl.}
	\Crefname{implementation}{Implementation}{Implementations}
}{
	\crefname{implementation}{impl.}{impl.}
	\Crefname{implementation}{Implementation}{Implementations}
}

\makeatother


\titlehead{Group of Algorithm Engineering\hfill Summer Semester 2024 \\ Institute for Computer Science \\ Goethe University Frankfurt}
\subject{Master's Thesis}
%\title{Sorting \\ for a \\ Processing-in-Memory \\ Architecture}
%\title{Sorting \\ on a \\ Processing-in-Memory \\ Architecture}
%\title{On Sorting for Processing-in-Memory}
\title{On Efficient Sorting Through In-Memory Processing}
%\title{Engineering Sorting Algorithms \\ for a Processing-in-Memory Architecture}
%\title{Engineering Sorting Algorithms \\ for Processing-in-Memory}
%\subtitle{Some cool subtitle if need be}
\subtitle{Implementation and Evaluation \dots{} / Engineering \dots{} / Exploring \dots{}}
%\subtitle{Engineering Algorithms for UPMEM-based DRAM Processing Units}
\author{\texorpdfstring{Ƶ}{Z}eno Adrian \texorpdfstring{\Lss05{W\kern-1.5pt}}{W}eil}
\publishers{\begin{tabular}{r @{~}l}
	Supervisor: & Dr Manuel Penschuck
\end{tabular}}

\makeatletter
\hypersetup{
	pdfauthor=\@author,
	pdftitle=\@title,
	pdfsubject=\@subject,
}
\makeatother

\usepgfplotslibrary{groupplots}
\tikzset{
	plot/.style={  % Not set globally lest other packages break.
		trim axis group left, trim axis group right,  % Only axes define the bounding box. Thus, lables can extend into the margins.
	},
}
\pgfplotsset{
	height=4cm,
	cycle multiindex list={exotic \nextlist mark list*},  % automatic colouring of plots
	scale only axis,  % height and width apply only to the axes, not the labels
	enlargelimits={abs=3mm, auto},  % for axes without specified limits, the limits are a bit bigger than neeeded (→ padding)
	log ticks with fixed point,  % uses 0.1, 0.001, … on logarithmix axes instead of 10^-1, 10^-2, …
	ymajorgrids,  % grey, horizontal background lines at each y-tick
	set layers,  % needed as otherwise extra ticks are drawn above the border
	legend style={  % adds padding to legends with multiple columns
		cells={anchor=west},  % left aligned labels
		/tikz/column 2/.style={column sep=5pt}, /tikz/column 4/.style={column sep=5pt}, /tikz/column 6/.style={column sep=5pt}, /tikz/column 8/.style={column sep=5pt}, /tikz/column 10/.style={column sep=5pt}, /tikz/column 12/.style={column sep=5pt}, /tikz/column 14/.style={column sep=5pt}, /tikz/column 16/.style={column sep=5pt},
	},
	group style={horizontal sep=5em, vertical sep=5em},
}

% Adds groupplot xlabel and groupplot ylabel, i.e. common axis labels for groupplots.
% groupplot xlabel should also be used if there is only one plot in the group plot for margin reasons.
% taken from: https://tex.stackexchange.com/a/117935
\makeatletter
\pgfplotsset{
	groupplot xlabel/.initial={},
	every groupplot x label/.style={
		at={($({\pgfplots@group@name\space c1r\pgfplots@group@rows.west}|-{\pgfplots@group@name\space c1r\pgfplots@group@rows.outer south})!0.5!({\pgfplots@group@name\space c\pgfplots@group@columns r\pgfplots@group@rows.east}|-{\pgfplots@group@name\space c\pgfplots@group@columns r\pgfplots@group@rows.outer south})$)},
		anchor=north,
	},
	groupplot ylabel/.initial={},
	every groupplot y label/.style={
		rotate=90,
		at={($({\pgfplots@group@name\space c1r1.north}-|{\pgfplots@group@name\space c1r1.outer
				west})!0.5!({\pgfplots@group@name\space c1r\pgfplots@group@rows.south}-|{\pgfplots@group@name\space c1r\pgfplots@group@rows.outer west})$)},
		anchor=south
	},
	execute at end groupplot/.code={%
		\node [/pgfplots/every groupplot x label]
		{\pgfkeysvalueof{/pgfplots/groupplot xlabel}};
		\node [/pgfplots/every groupplot y label]
		{\pgfkeysvalueof{/pgfplots/groupplot ylabel}};
	}
}

\def\endpgfplots@environment@groupplot{%
	\endpgfplots@environment@opt%
	\pgfkeys{/pgfplots/execute at end groupplot}%
	\endgroup%
}
\makeatother

\DefineNamedColor{named}{accentcolor}{RGB}{118, 13, 28}  % Mordred's red


\begin{document}
	\pagenumbering{gobble}  % turn off pagenumbering

%	\maketitle

%	\begin{abstract}
	\noindent
	The growing disparity between processing and memory speed, coupled with increasing data demands, has led to memory accesses being a bottleneck for many modern workflows.
	An example are sorting algorithms, which are often designed around the constraints set by memory subsytems.
	\Acl*{PIM} (also known as processing in memory, \acs*{PIM}) is an umbrella term encompassing several approaches which offload computational tasks to accelerators in or near the memory itself.
	In \acs*{PIM} systems designed and manufactured by \upmem{}, traditional dynamic random-access memory (\acs*{DRAM}) modules are augmented with general-purpose processors called \acfp*{DPU}.
	These are located next to the memory banks themselves, whereby high memory access speed is accomplished.
	An \upmem{}-based \acs*{PIM} system may contain thousands of \acsp*{DPU}, each capable of additional thread-level parallelism.
	Although designed for general use, the \acs*{DPU} architecture does come with limitations to its computational prowess.

	The scope of this thesis is the design, implementation, and evaluation of sorting algorithms which run on a single \acs*{DPU}.
	We investigate several sequential and parallel sorting algorithms, documenting the engineering process and adaptations to the merits and shortcomings of the \acs*{DPU} architecture.
	We find that sorting is a suitable task for a \acs*{DPU}, which can be sped up nearly ideally through multithreading.
	This paves the way for more large-scale sorting algorithms which run on multiple \acsp*{DPU}.
\end{abstract}


	\tableofcontents

	\listoftodos

	\begingroup
%	\clearpage
	\pagenumbering{arabic}  % turn on pagenumbering
	\endgroup

	\bigskip
	\todo[inline]{Architektur}
	\todo[inline]{Speicherzugriffe (memcpy, mram\_read \dots)}
	\todo[inline]{triple buffer}

	\section{Sorting with One Tasklet}

This section covers the very first phase where each tasklet sorts on its own, \ie{} sequentially.
Unless specified otherwise, every measurement shown in this section was conducted on a uniform input distribution with each 32-bit integer, and the default configurations of the sorting algorithms were as follows:
\begin{description}
	\item[\IS{}]
	using one explicit sentinel value

	\item[\ShS{}]
	using \(h_1\) sentinel values

	\item[\QS{}]
	iterative implementation;
	switching to \IS{} whenever 13 elements or less remain in a partition;
	median of three as pivot;
	prioritising the right-hand partition over the left-hand partition
\end{description}
Further measurements can be found in \cref{apx:single}.

\subsection{\IS{}}

\pgfplotstableread{data/small_sorts.txt}{\tablesmallsorts}

\begin{figure}
	\begin{tikzpicture}[plot]
		\begin{groupplot}[
			width=0.4358\linewidth,
			group/group size=2 by 1,
			groupplot xlabel={Input Length \(n\)},
			xtick distance=3,
			minor xtick=data,
			legend columns=-1,
		]
			\nextgroupplot[ylabel=Cycles / \(n^2\), ymin=0, ymax=80, legend to name=leg:insertion_sort]
			\legend{Insertion, Insertion (no sentinel), Bubble, Bubble(adaptive), Selection}
			\plotpernn{1}{\tablesmallsorts}
			\plotpernn{1NoSentinel}{\tablesmallsorts}
			\plotpernn{BubbleNonAdapt}{\tablesmallsorts}
			\plotpernn{BubbleAdapt}{\tablesmallsorts}
			\plotpernn{Selection}{\tablesmallsorts}
			%
			\nextgroupplot[ylabel=Speed-up, ymin=0.2, ymax=1]
			\pgfplotsset{cycle list shift=1}
			\plotspeedup{1NoSentinel}{1}{\tablesmallsorts}
			\plotspeedup{BubbleNonAdapt}{1}{\tablesmallsorts}
			\plotspeedup{BubbleAdapt}{1}{\tablesmallsorts}
			\plotspeedup{Selection}{1}{\tablesmallsorts}
		\end{groupplot}
	\end{tikzpicture}

	\hfil\pgfplotslegendfromname{leg:insertion_sort}\hfil
	\caption{
		Comparison of sorting algorithms with \(\bigoh{n^2}\) runtime on a uniform input distribution.
		The \IS*{} differ in whether they rely on sentinel values.
		The adaptive \BS{} terminates prematurely if no changes were made to the input array during an iteration.
		The speed-ups are with respect to the \IS{} relying on sentinel values.
	}
	\label{fig:insertion_sort}
\end{figure}

This stable sorting algorithm works by moving the \(i\)th element to the left as long as its left neighbour is bigger, assuming that the elements \(0\) to \(i - 1\) are already sorted.
Even though in both the average case and the worst case, \IS{} has a runtime of \(\bigoh{n^2}\)\todo{Beleg?}, it features quite some advantages:
\begin{enumerate*}
	\item
	It works in-place, needing only \(\bigoh{1}\) additional space.

	\item
	It is inherently adaptive:
	If the input array is mostly or even fully sorted, the runtime drops down to \(\bigoh{n}\).

	\item
	Its program code is short, lending itself to inlining.

	\item
	The overhead is small.
\end{enumerate*}
Especially the last two points make \IS{} a good base algorithm for asymptotically better sorting algorithms to use on very small subarrays.

When moving an element to the left, two checks are needed:
Does the left neighbour exist and is it smaller than the element to move?
The first check can be omitted through the use of \emph{sentinel values}:
If the element at index \(-1\) is at least as small as any value in the input array, the leftwards motion stops there at the latest.
Since a DPU has no branch predictor, the slowdown from performing twice as many checks as needed is quite high and lies between 20\% and 40\%\todo{auf Kompilat eingehen?} in the relevant input range (\cref{fig:insertion_sort}).%
\todo{ex- und implizite Wächterwerte benennen}
\todo[inline]{auf Compilersperenzchen eingehen?}

Other known simple sorting algorithm with similar runtime complexity are \SelS{} and \BS{}.
The asymptoticity, however, hides much higher constant factors such that even for as little as three elements \IS{} is superior (\cref{fig:insertion_sort}) and should always be used.

\subsection{\texorpdfstring{\ShS{}}{ShellSort}}
\label{subsec:single:shell}

\pgfplotsinvokeforeach{reverse,uniform}{
	\pgfplotstablereadnamed{data/shell/two-tier/uint32/#1.txt}{tableShellTwo_32#1}
	\pgfplotstablereadnamed{data/shell/h1=7/uint32/#1.txt}{tableShell7_32#1}
	\pgfplotstablereadnamed{data/shell/h1=8/uint32/#1.txt}{tableShell8_32#1}
	\pgfplotstablereadnamed{data/shell/h1=9/uint32/#1.txt}{tableShell9_32#1}
	\pgfplotstablereadnamed{data/shell/h1=10/uint32/#1.txt}{tableShell10_32#1}
	\pgfplotstablereadnamed{data/shell/h1=11/uint32/#1.txt}{tableShell11_32#1}
	\pgfplotstablereadnamed{data/shell/h1=12/uint32/#1.txt}{tableShell12_32#1}
	\pgfplotstablereadnamed{data/shell/h1=13/uint32/#1.txt}{tableShell13_32#1}
	\pgfplotstablereadnamed{data/shell/h1=14/uint32/#1.txt}{tableShell14_32#1}
	\pgfplotstablereadnamed{data/shell/h1=15/uint32/#1.txt}{tableShell15_32#1}
	\pgfplotstablereadnamed{data/shell/h1=16/uint32/#1.txt}{tableShell16_32#1}
	\pgfplotstablereadnamed{data/shell/h1=17/uint32/#1.txt}{tableShell17_32#1}
}

\IS{} suffers from small elements at the end of the input, since those have to be brought to the front through \(\bigoh{n}\) comparisons and swaps.
\ShS{}, proposed by Donald L. Shell in 1959~\cite{Shell1959AHS}, gets around this by doing multiple passes of \IS{} with different step sizes:
In pass~\(p\) with step size \(\stepsizes_p\), the input array is divided into the subarrays of indices \((i, \stepsizes_p + i, 2 \stepsizes_p + i, \dots)\) for \(i = 0, \dots, \stepsizes_p - 1\) which then get sorted individually through \IS{}.
The step sizes get smaller each pass, with the final step size being \(1\) such that a regular \IS{} is performed.
Intuitively, the individual \IS*{} are fast since elements which need to travel long distances do big jumps.
Finding the right balance between the heightened overhead through multiple \IS{} passes and the shortened runtime of each \IS{} pass is subject to research to this day \cite{skean2023optimization,lee2021empirically} and depends on the cost of the operation types (comparing, swapping, looping).

Let us first focus on small input arrays where only two passes with step sizes~\(\stepsizes_1\) and \(1\) suffice.
The previous results on \IS{} suggest that \ShS{} should make use of~\(\stepsizes_1\) sentinel values lest bounds checking eats any gain up.
\Cref{fig:shell:two-tier} shows that the additional passes start to pay off at around 16 elements for \(\stepsizes_1 \ge 3\).
Bear in mind that these measurements were conducted on a uniform input distribution;
if \ShS{} is used by another algorithm on a subarray, these thresholds may be higher or even non-existent due to some degree of presorting.

\begin{figure}
	\begin{tikzpicture}[plot]
		\begin{groupplot}[
			adaptive group=1 by 2,
			groupplot xlabel={Input Length \(n\)},
			xtick distance=3,
			minor xtick=data,
			legend columns=-1,
		]
			\nextgroupplot[ylabel=Cycles / \(n^2\), ymin=0, ymax=60, legend to name=leg:shell:two-tier]
			\legend{\(1\), \(...\), \(9\)}
			\pgfplotsinvokeforeach{1,...,9}{
				\plotpernn{#1}{\tableSmallSortsXxxiiUniform}
			}
			%
			\nextgroupplot[ylabel=Speed-up, ymin=0.6, ymax=1.2]
			\pgfplotsset{cycle list shift=1}
			\pgfplotsinvokeforeach{2,...,9}{
				\plotspeedup{#1}{1}{\tableSmallSortsXxxiiUniform}
			}
		\end{groupplot}
	\end{tikzpicture}

	\hfil\pgfplotslegendfromname{leg:shell:two-tier}\hfil
	\caption{
		Comparison of \IS{} (1) and various two-tier \ShS*{} (2--9), whose step sizes \(\stepsizes_1\) are indicated by their label.
		The speed-ups are with respect to the \IS{}.
	}
	\label{fig:shell:two-tier}
\end{figure}

\NewDocumentCommand{\shellscatter}{m m m}{
	\addplotnamedtable[select row={#1}, forget plot][x=µ_#2, y expr={6}]{tableShellTwo_#3};
	\ifnumless{#2}{7}{
		\addplotnamedtable[select row={#1}, forget plot][x=µ_#2, y expr={7}]{tableShell7_#3};
	}{}
	\ifnumless{#2}{8}{
		\addplotnamedtable[select row={#1}, forget plot][x=µ_#2, y expr={8}]{tableShell8_#3};
	}{}
	\ifnumless{#2}{9}{
		\addplotnamedtable[select row={#1}, forget plot][x=µ_#2, y expr={9}]{tableShell9_#3};
	}{}
	\addplotnamedtable[select row={#1}][x=µ_#2, y expr={10}, forget plot]{tableShell10_#3};
	\addplotnamedtable[select row={#1}][x=µ_#2, y expr={11}, forget plot]{tableShell11_#3};
	\addplotnamedtable[select row={#1}][x=µ_#2, y expr={12}, forget plot]{tableShell12_#3};
	\addplotnamedtable[select row={#1}][x=µ_#2, y expr={13}, forget plot]{tableShell13_#3};
	\addplotnamedtable[select row={#1}][x=µ_#2, y expr={14}, forget plot]{tableShell14_#3};
	\addplotnamedtable[select row={#1}][x=µ_#2, y expr={15}, forget plot]{tableShell15_#3};
	\addplotnamedtable[select row={#1}][x=µ_#2, y expr={16}, forget plot]{tableShell16_#3};
	\addplotnamedtable[select row={#1}][x=µ_#2, y expr={17}]{tableShell17_#3};
}

\pgfplotsset{
	shell scatter plot/.style={
		group/horizontal sep=12mm,
		adaptive group=3 by 2,
		groupplot xlabel={Mean [\(10^4\) Cycles]},
		groupplot ylabel={\(\stepsizes_1\)},
		scaled x ticks=base 10:-4,
		xtick scale label code/.code={},  % removes exponent underneath the axis
		ytick={6, 7, 9, ..., 17},
		yticklabels={/, \(7\), \(9\), \(...\), \(17\)},
		/tikz/only marks,
		cycle list shift=2,  % for sharing colours with the previous figure
		legend columns=-1,
	},
}

\begin{figure}[p]
	\begin{tikzpicture}[plot]
		\newcommand{\setn}[1]{\textit{n} = #1}
		\newcommand{\type}{32uniform}
		\begin{groupplot}[shell scatter plot]
			\pgfplotsinvokeforeach{16,32,48,64,96,128}{
				\nextgroupplot[title={Input Length \setn{#1}}]
				\pgfplotsforeachungrouped\h in {3,...,9}{
					\shellscatter{#1}{\h}{\type}
				}
			}
			\pgfplotsset{legend to name=leg:shell:three-tier}
			\legend{\(3\), \(4\), \(5\), \(6\), \(7\), \(8\), \(9\)}
		\end{groupplot}
	\end{tikzpicture}

	\hfil\pgfplotslegendfromname{leg:shell:three-tier}\hfil
	\caption{
		Runtimes of \ShS*{} with two passes (/) and three passes (7--17).
		The coloured symbols encode the step size \(\stepsizes_1\) for two-tier \ShS*{} and the step size~\(\stepsizes_2\) for three-tier \ShS*{}.
		For the latter, the step size \(\stepsizes_1\) is noted on the y-axes.
	}
	\label{fig:shell:three-tier}
\end{figure}

When moving to greater input lengths (\cref{fig:shell:three-tier}), the differences in performance between the two-tier \ShS*{} become more pronounced;
especially the ones with \(\stepsizes_1 = 3\) and \(\stepsizes_1 = 4\) fall off whereas the one with \(\stepsizes_1 = 6\) holds its ground quite well.
With more than 64 elements, three passes get worthwhile to consider.
Interestingly, many \ShS*{} with \(\stepsizes_2 = 4\) take the lead whilst the ones with \(\stepsizes_2 = 6\) are mid-table.
This is in accordance with \citeauthor{10.1007/3-540-44669-9_12} \cite{10.1007/3-540-44669-9_12} who noted~\(\stepsizes = (17, 4, 1)\) to be the optimal triplet for 128 elements.
It is noteworthy, though, that he measured the quadruplet~\(\stepsizes = (38, 9, 4, 1)\) to be about 5\% faster in the MIX machine model.
On a DPU, this sequence leads to a runtime of nearly exactly 74.000 cycles, placing it only mid-table.
Without access to Ciura's original code, giving a satisfactory explanation for the discrepancy is hard, however.

But would pushing the limits of \ShS{} even be rewarding?
Two issues come up.
Firstly, greater input lengths require greater steps \Dash well into the three digits for \(n \approx 1000\) \cite{skean2023optimization, 10.1007/3-540-44669-9_12} \Dash and those in turn require more sentinel values.
But the more sentinel values are stored, the less space is available for the actual input array, leading to smaller runs and thus hurting the overall sorting algorithm.
Explorative testing suggests that falling back to bounds checking for big steps is too punishing.
Secondly, there simply are better alternatives, namely \QS{} (which will be discussed in more detail in the next section).
\Cref{fig:shell:against_others} shows that even though \ShS{} takes just a fraction of the time \IS{} takes \Dash apparently achieving a runtime between \(\bigomega{n \lb n}\) and \(\bigoh{n \lb^2 n}\) \Dash\negthinspace, \QS{} beats both from 20 elements onwards.
Even \QS{}'s standard deviation of 1432 cycles at 128 elements is superior to \ShS{}'s 2670 cycles.
Together with \cref{fig:shell:two-tier}, this means that \ShS{} is not worth using at all and will, consequently, not be improved upon in this thesis.

\subsection{\texorpdfstring{\QS{}}{QuickSort}}
\label{subsec:tasklet:quick}

\pgfplotstablereadnamed{data/quick/fallback.txt}{tableQuickFallback}
\pgfplotstablereadnamed{data/quick/pivot.txt}{tableQuickPivot}

\QS{} uses partitioning to sort in an expected average runtime of \(\bigoh{n \log n}\):
A pivot element is chosen from the input array, then the input array gets scanned and elements bigger or smaller than the pivot are moved to the right or left of the pivot element, respectively.
Finally, \QS{} is used on the left and right partitions.


\paragraph{Base Cases}
When only a few elements remain in a partition, \QS{}'s overhead predominates such that \IS{} lends itself as fallback algorithm.
As \cref{fig:quick:fallback} demonstrates, the optimal threshold for switching the sorting algorithm is around 13 elements, netting a speed-up of 30\% and more over a \QS{} without fallback algorithm.
This low threshold also means that even a simple two-round \ShS{} is not worth considering.

\begin{figure}
	\tikzsetnextfilename{quick_fallback}
	\begin{tikzpicture}[plot]
		\begin{groupplot}[
			horizontal sep for labels,
			adaptive group=1 by 2,
			groupplot xlabel={Input Length \(n\)},
			groupplot ylabel={Speed-up},
			xmode=log,
			xtick={20, 32, 64, 128, 256, 512, 1024},
			xticklabels={\(20\), \(32\), \(64\), \(128\), \(256\), \(512\), \(1024\)},
			legend columns=-1,
		]
			\nextgroupplot[title={Over No Fallback\strut}, legend to name=leg:quick:fallback]
			\legend{\(10\), \(11\), \(...\), \(16\)}
			\pgfplotsinvokeforeach{10,...,16}{
				\plotspeedup{#1}{None}{tableQuickFallback}
			}
			%
			\nextgroupplot[title={Over a Threshold of 13\strut}, /pgf/number format/.cd, precision=3, fixed zerofill=true]
			\pgfplotsinvokeforeach{10,...,16}{
				\ifnumequal{#1}{13}{
					\pgfplotsset{cycle list shift=1}
				}{
					\plotspeedup{#1}{13}{tableQuickFallback}
				}
			}
		\end{groupplot}
	\end{tikzpicture}

	\hfil\pgfplotslegendfromname{leg:quick:fallback}\hfil
	\caption{
		Comparison of \QS*{} with different thresholds for the fallback to \IS{}, with a \QS{} without fallback algorithm and the fastest \QS{} with a threshold of 13 elements.
	}
	\label{fig:quick:fallback}
\end{figure}

Besides falling back to \IS{}, another base case is imaginable, namely terminating when the partition has a length of 1, 0, or even --1 elements.
Realistically speaking, this should not be necessary, because even though the extra check is done with just one additional instruction, it is a rare occurrence and the \IS{} would terminate after a few instructions anyway.
Yet, there are tremendous consequences for the runtime depending on the exact implementation of the base cases.
Since these are likely caused by the compiler, they are laid out in \cref{subsec:appendix:quick}.
\todo{zurückkehren nach Unterunterabschnitt}


\paragraph{Recursion vs. Iteration}
In theory, the question of whether an algorithm should be implemented recursively or iteratively comes down to convenience.
Due to the uniform cost of instructions, putting arguments automatically on the call stack or manually in an array essentially costs the same, as does jumping to the start of a loop and to the start of a function.
Furthermore, in case of \QS{}, the compiler turns tail-recursive calls into jumps back to the function start, so that one partition is sorted recursively and the other iteratively.
All this would suggest a recursive implementation with less code complexity.

In practice, it comes down to the compilation.
Selcouthly, even parts of the algorithms which are independent from the choice between recursion and iteration can be compiled differently, such that there are implementations where iteration is faster than recursion and the other way around.
Overall though, iterative implementations tend to be compiled better with superior register usage and less instructions used for the actual \QS{} algorithm.
The fastest implementation is indeed an iterative one, even if it beats the fastest recursive implementations \Dash outliers, admittedly \Dash by less than 4\%.
More details are given in \cref{subsubsec:tasklet:quick:compiler}.


\paragraph{Pivot Choice}
Another parameter to tune is the way in which the pivot is chosen.
The following were implemented and tested:
\begin{itemize}
	\item
	Using the \emph{last element} is the fastest way, requiring zero instructions.

	\item
	Choosing the \emph{middle element} is slower than choosing the last one, requiring a calculation of its address and swapping it with the last element so that it can act as sentinel value during partitioning.
	The upside is that it is more suited for sorted and nearly sorted inputs.

	\item
	Taking the \emph{median of three elements}, namely the first, middle, and last one, is even more computationally expensive but increases the chances of choosing a pivot that is neither particularly high nor particularly low.

	\item
	A \emph{random element} is most efficiently drawn using an xorshift random number generator and rejection sampling \cite{lukas_geis}.
\end{itemize}
Luckily, the pivot choice seldom has bearing on the overall compilation, making a comparison easier.
\todo{Stimmt nicht!}
The results are shown in \cref{fig:quick:pivot}.
Choosing the middle element is cheap enough for the runtime to be slowed down by a low single-digit percentage, and the increased pivot quality from choosing the median of three elements more than offsets the cost increase, thus making it the best choice.
At 1024 elements, the runtime with a random pivot is 10\% worse than with the median of three elements.
Since drawing the random index is more than thrice as costly as computing the middle index, a median of three random elements would likely yield even worse times, should one need randomisation.
Again, more details are given in \cref{subsubsec:tasklet:quick:compiler}.

\begin{figure}
	\tikzsetnextfilename{quick_pivot}
	\begin{tikzpicture}[plot]
		\begin{groupplot}[
			horizontal sep for labels,
			adaptive group=1 by 2,
			groupplot xlabel={Input Length \(n\)},
			xmode=log,
			xtick={20, 32, 64, 128, 256, 512, 1024},
			xticklabels={\(20\), \(32\), \(64\), \(128\), \(256\), \(512\), \(1024\)},
			legend columns=-1,
		]
			\nextgroupplot[ylabel=Cycles / \((n \lb n)\), ymin=55, ymax=70, legend to name=leg:quick:pivot]
			\legend{Last, Middle, Median of Three, Random}
			\plotpernlogn{End}{tableQuickPivot}
			\plotpernlogn{Middle}{tableQuickPivot}
			\plotpernlogn{MedianOfThree}{tableQuickPivot}
			\plotpernlogn{Random}{tableQuickPivot}
			%
			\nextgroupplot[ylabel=Speed-up, ymin=0.9, ymax=1.01, extra y ticks={1.01}, /pgf/number format/.cd, precision=2, fixed zerofill=true]
			\plotspeedup{End}{MedianOfThree}{tableQuickPivot}
			\plotspeedup{Middle}{MedianOfThree}{tableQuickPivot}
			\pgfplotsset{cycle list shift=1}
			\plotspeedup{Random}{MedianOfThree}{tableQuickPivot}
		\end{groupplot}
	\end{tikzpicture}

	\hfil\pgfplotslegendfromname{leg:quick:pivot}\hfil
	\caption{
		Comparison of \QS{} with different pivot choices.
		The speed-ups are with respect to the \QS{} with the median of three as pivot choice.
	}
	\label{fig:quick:pivot}
\end{figure}


\paragraph{Prioritisation of Partitions}
After partitioning, in order to minimise the call stack, \QS{} should be used on the smaller of the two partitions first.
For code simplicity and to reduce the overhead, no such mechanism was implemented.
As shown in \cref{subsubsec:tasklet:quick:compiler}, the choice between always sorting the left-hand partition or the right-hand partition first can have tremendous effects nevertheless.



\subsubsection{Investigating the Compilation}
\label{subsubsec:tasklet:quick:compiler}

\pgfplotstableread{data/quick/rec_vs_it.txt}{\tableQuickRecVsIter}
\pgfplotstableread{data/quick/recursive/no switched sides/uniform/end.txt}{\tableQuickRecNssUniEnd}
\pgfplotstableread{data/quick/recursive/no switched sides/uniform/middle.txt}{\tableQuickRecNssUniMiddle}
\pgfplotstableread{data/quick/recursive/no switched sides/uniform/median_of_three.txt}{\tableQuickRecNssUniMedian}
\pgfplotstableread{data/quick/recursive/no switched sides/uniform/random.txt}{\tableQuickRecNssUniRandom}
\pgfplotstableread{data/quick/recursive/switched sides/uniform/end.txt}{\tableQuickRecSsUniEnd}
\pgfplotstableread{data/quick/recursive/switched sides/uniform/middle.txt}{\tableQuickRecSsUniMiddle}
\pgfplotstableread{data/quick/recursive/switched sides/uniform/median_of_three.txt}{\tableQuickRecSsUniMedian}
\pgfplotstableread{data/quick/recursive/switched sides/uniform/random.txt}{\tableQuickRecSsUniRandom}
\pgfplotstableread{data/quick/iterative/no switched sides/uniform/end.txt}{\tableQuickIterNssUniEnd}
\pgfplotstableread{data/quick/iterative/no switched sides/uniform/middle.txt}{\tableQuickIterNssUniMiddle}
\pgfplotstableread{data/quick/iterative/no switched sides/uniform/median_of_three.txt}{\tableQuickIterNssUniMedian}
\pgfplotstableread{data/quick/iterative/no switched sides/uniform/random.txt}{\tableQuickIterNssUniRandom}
\pgfplotstableread{data/quick/iterative/switched sides/uniform/end.txt}{\tableQuickIterSsUniEnd}
\pgfplotstableread{data/quick/iterative/switched sides/uniform/middle.txt}{\tableQuickIterSsUniMiddle}
\pgfplotstableread{data/quick/iterative/switched sides/uniform/median_of_three.txt}{\tableQuickIterSsUniMedian}
\pgfplotstableread{data/quick/iterative/switched sides/uniform/random.txt}{\tableQuickIterSsUniRandom}

The quality of the compilation and thus the real performance of \QS{} is erratic to such an extent that one implementation variant may see a speed-up of 25\% over another one even with the same pivot choice although virtually none would be expected.
As hinted in the preceding paragraphs, this raises the need for a benchmark suite with the following parameters:
base case handling, recursion/iteration, pivot choice, and partition prioritisation.
Before the results are discussed, the first parameter shall be explained in more depth.

Besides falling back to \IS{} if 13 elements remain (\enquote{treshold undercut}), another base case is imaginable, namely a full termination if 1, 0, or --1 elements remain (\enquote{trivial length}).
Theoretically, it should not be needed to check for trivial lengths because even though it is doable with just one additional instruction, such small partitions are rare and the \IS{} would terminate after a few instructions anyway.
Nonetheless, its inclusion or exclusion can have significant impacts.
The following Implementations were tested:
\begin{enumerate}
	\item\label[implementation]{imp:normal}
	If the length is trivial, terminate.
	If not and if the threshold is undercut, sort with \IS{}.
	Otherwise, sort with \QS{} and use \QS{} on both partitions.
%	\textcolor{red}{[Normal]}

	\item\label[implementation]{imp:triviality_within_threshold}
	If the threshold is undercut, check if the length is trivial and terminate or sort with \IS{}, respectively.
	Otherwise, sort with \QS{} and use \QS{} on both partitions.
%	\textcolor{red}{[TrivInThresh]}
	\begin{itemize}
		\item
		This Implementation significantly reduces the number of checks for trivial length.
	\end{itemize}

	\item\label[implementation]{imp:no_triviality}
	If the threshold is undercut, sort with \IS{}.
	Otherwise, sort with \QS{} and use \QS{} on both partitions.
%	\textcolor{red}{[NoTrivial]}
	\begin{itemize}
		\item
		This Implementation forgoes the check for a trivial length completely, at the cost of unneeded \IS*{}.
	\end{itemize}

	\item\label[implementation]{imp:threshold_then_triviality}
	If the threshold is undercut, sort with \IS{}.
	If not and if the length is trivial, terminate.
	Otherwise, sort with \QS{} and use \QS{} on both partitions.
%	\textcolor{red}{[ThreshThenTriv]}
	\begin{itemize}
		\item
		This Implementation, while nonsensical from a logical point of view, gives the compiler an explicit guarantee that the partitioning loop does not end immediately.
	\end{itemize}

	\item\label[implementation]{imp:triviality_before_call}
	If the threshold is undercut, sort with \IS{}.
	Otherwise, sort with \QS{}.
	Then check for either partition if its length is trivial and use \QS{} if not.
%	\textcolor{red}{[TrivialBC]}
	\begin{itemize}
		\item
		This Implementation, as well as the next two, gets rid of some unneeded uses of \QS{}.
		In the recursive case, these Implementations lose the property of being tail-recursive.
	\end{itemize}

	\item\label[implementation]{imp:threshold_before_call}
	Sort with \QS{}.
	Check for either partition if the threshold is undercut and use \IS{} or \QS{}, respectively.
%	\textcolor{red}{[ThreshBC]}

	\item\label[implementation]{imp:threshold_and_triviality_before_call}
	Sort with \QS{}.
	Check for either partition if its length is trivial or if the threshold is undercut and use \IS{}, \QS{}, or nothing, respectively.
%	\textcolor{red}{[ThreshTrivBC]}

	\item\label[implementation]{imp:one_insertion}
	If the threshold is undercut, terminate.
	Otherwise, sort with \QS{} and use \QS{} on both partitions.
	After all \QS*{} are done, sort the whole input array with \IS{}.
%	\textcolor{red}{[OneInsertion]}
	\begin{itemize}
		\item
		This Implementation always does one pass of \IS{}.
		For example, the other Implementations do roughly 90 at 1024 elements.
	\end{itemize}
\end{enumerate}

All results are shown in \cref{fig:quick_implementations}.
When using recursion, \cref{imp:normal,imp:triviality_before_call} perform the best, especially for large inputs.
Their compilations are fundamentally the same, including the conversion of the second recursive call into a jump back to the function start \Dash even though \cref{imp:triviality_before_call} is not tail-recursive.

\begin{figure}[p]
	\def\algos{Normal,TrivInThresh,NoTrivial,ThreshThenTriv,TrivialBC,ThreshBC,ThreshTrivBC,OneInsertion}
	\pgfplotsset{
		width=\dimexpr(\linewidth-9mm)/4,  % 9mm = 3 × group/horizontal sep
		height=3.5cm,
		group/group size=4 by 2,
		group/horizontal sep=3mm,
		groupplot xlabel={Input Length \(n\)},
		groupplot ylabel={Cycles / \((n \lb n)\)},
		xmode=log,
		xtick={20, 64, 256, 1024},
		xticklabels={\(20\), \(64\), \(256\), \(1024\)},
		minor xtick={32, 128, 512},
		ymin=55,
		ymax=80,
		ytick distance=5,
		legend columns=-1,
	}
	\begin{subfigure}{\textwidth}
		\begin{tikzpicture}[plot]
			\begin{groupplot}
				\nextgroupplot[title={Last | Left First}, legend to name=leg:quick_implementations]
				\legend{\ref{imp:normal}, \ref{imp:triviality_within_threshold}, \ref{imp:no_triviality}, \ref{imp:threshold_then_triviality}, \ref{imp:triviality_before_call}, \ref{imp:threshold_before_call}, \ref{imp:threshold_and_triviality_before_call}, \ref{imp:one_insertion}}
				\expandafter\pgfplotsinvokeforeach\expandafter{\algos}{
					\plotpernlogn{#1}{\tableQuickRecNssUniEnd}
				}
				%
				\nextgroupplot[title={Middle | Left First}, yticklabels={}]
				\expandafter\pgfplotsinvokeforeach\expandafter{\algos}{
					\plotpernlogn{#1}{\tableQuickRecNssUniMiddle}
				}
				%
				\nextgroupplot[title={Median | Left First}, yticklabels={}]
				\expandafter\pgfplotsinvokeforeach\expandafter{\algos}{
					\plotpernlogn{#1}{\tableQuickRecNssUniMedian}
				}
				%
				\nextgroupplot[title={Random | Left First}, yticklabel pos=right]
				\expandafter\pgfplotsinvokeforeach\expandafter{\algos}{
					\plotpernlogn{#1}{\tableQuickRecNssUniRandom}
				}
				%
				\nextgroupplot[title={Last | Right First}]
				\expandafter\pgfplotsinvokeforeach\expandafter{\algos}{
					\plotpernlogn{#1}{\tableQuickRecSsUniEnd}
				}
				%
				\nextgroupplot[title={Middle | Right First}, yticklabels={}]
				\expandafter\pgfplotsinvokeforeach\expandafter{\algos}{
					\plotpernlogn{#1}{\tableQuickRecSsUniMiddle}
				}
				%
				\nextgroupplot[title={Median | Right First}, yticklabels={}]
				\expandafter\pgfplotsinvokeforeach\expandafter{\algos}{
					\plotpernlogn{#1}{\tableQuickRecSsUniMedian}
				}
				%
				\nextgroupplot[title={Random | Right First}, yticklabel pos=right]
				\expandafter\pgfplotsinvokeforeach\expandafter{\algos}{
					\plotpernlogn{#1}{\tableQuickRecSsUniRandom}
				}
			\end{groupplot}
		\end{tikzpicture}
		\caption{
			Recursive Approach
		}
		\bigskip
	\end{subfigure}
	%
	\begin{subfigure}{\textwidth}
		\begin{tikzpicture}[plot]
			\begin{groupplot}
				\nextgroupplot[title={Last | Left First}]
				\expandafter\pgfplotsinvokeforeach\expandafter{\algos}{
					\plotpernlogn{#1}{\tableQuickIterNssUniEnd}
				}
				%
				\nextgroupplot[title={Middle | Left First}, yticklabels={}]
				\expandafter\pgfplotsinvokeforeach\expandafter{\algos}{
					\plotpernlogn{#1}{\tableQuickIterNssUniMiddle}
				}
				%
				\nextgroupplot[title={Median | Left First}, yticklabels={}]
				\expandafter\pgfplotsinvokeforeach\expandafter{\algos}{
					\plotpernlogn{#1}{\tableQuickIterNssUniMedian}
				}
				%
				\nextgroupplot[title={Random | Left First}, yticklabel pos=right]
				\expandafter\pgfplotsinvokeforeach\expandafter{\algos}{
					\plotpernlogn{#1}{\tableQuickIterNssUniRandom}
				}
				%
				\nextgroupplot[title={Last | Right First}]
				\expandafter\pgfplotsinvokeforeach\expandafter{\algos}{
					\plotpernlogn{#1}{\tableQuickIterSsUniEnd}
				}
				%
				\nextgroupplot[title={Middle | Right First}, yticklabels={}]
				\expandafter\pgfplotsinvokeforeach\expandafter{\algos}{
					\plotpernlogn{#1}{\tableQuickIterSsUniMiddle}
				}
				%
				\nextgroupplot[title={Median | Right First}, yticklabels={}]
				\expandafter\pgfplotsinvokeforeach\expandafter{\algos}{
					\plotpernlogn{#1}{\tableQuickIterSsUniMedian}
				}
				%
				\nextgroupplot[title={Random | Right First}, yticklabel pos=right]
				\expandafter\pgfplotsinvokeforeach\expandafter{\algos}{
					\plotpernlogn{#1}{\tableQuickIterSsUniRandom}
				}
			\end{groupplot}
		\end{tikzpicture}
		\caption{
			Iterative approach
		}
	\end{subfigure}

	\bigskip
	\hfil\pgfplotslegendfromname{leg:quick_implementations}\hfil
	\caption{
		Comparison of the different implementations (1--8) of \QS{} for all possible pivot choices.
		In the first rows, the left partitions are sorted before the right ones, while it is the reverse in the second rows.
	}
	\label{fig:quick_implementations}
\end{figure}

\begin{figure}
	\begin{tikzpicture}[plot]
		\begin{groupplot}[
			width=0.4401\linewidth,
			group/group size=2 by 1,
			groupplot xlabel={Input Length \(n\)},
			xmode=log,
			xtick={20, 32, 64, 128, 256, 512, 1024},
			xticklabels={\(20\), \(32\), \(64\), \(128\), \(256\), \(512\), \(1024\)},
			legend columns=-1,
		]
			\nextgroupplot[ylabel=Cycles / \((n \lb n)\), ymin=55, ymax=65, extra y ticks={55, 65}, legend to name=leg:rec_vs_it]
			\legend{Iterative, Recursive}
			\plotpernlogn{It}{\tableQuickRecVsIter}
			\plotpernlogn{Rec}{\tableQuickRecVsIter}
			%
			\nextgroupplot[ylabel=Speed-up, ymin=0.96, ymax=1.04, /pgf/number format/.cd, precision=2, fixed zerofill=true]
			\plotspeedup{It}{Rec}{\tableQuickRecVsIter}
		\end{groupplot}
	\end{tikzpicture}

	\hfil\pgfplotslegendfromname{leg:rec_vs_it}\hfil
	\caption{
		Comparison of the fastest recursive and iterative \QS*{} (cf. \cref{subsubsec:tasklet:quick:compiler}).
		The actual algorithm is compiled the very same in both cases, so that time differences are only due to the way \QS{} is applied to the partitions.
		\todo{nicht mehr wegen des Pivots!}
	}
	\label{fig:rec_vs_it}
\end{figure}


\subsection{\texorpdfstring{\MS{}}{MergeSort}}
\label{subsec:tasklet:merge}

\MS{} repeatedly compares two sorted subarrays and merges them into a longer sorted array in time \(\bigtheta{n \log n}\).
Unlike \QS{}, this runtime is guaranteed.
Furthermore, the sorting is naturally stable but at the cost of not happening in-place.

\paragraph{Starting Runs}
Instead of starting by merging runs of length 1, it is beneficial to first create longer starting runs using \ShS{}.
Unlike \QS{}, where each partition naturally acted as sentinel for the subsequent one, it is necessary to temporarily place sentinels values in front of each starting run and later restore the original values of the preceding run.
The step sizes used for \ShS{} \Dash namely \(\stepsizes = (1)\) for lengths up to 16, \(\stepsizes = (6, 1)\) for lengths up to 48, and \(\stepsizes = (12, 5, 1)\) for everything above \Dash have been chosen based on the results in \cref{subsec:tasklet:shell}, according to which these step sizes offer top performance for uniformly distributed inputs and medial performance for the reverse sorted inputs.
Spot-check inspection suggest no deterioration of \ShS{}'s compilation due to inlining.

\paragraph{Memory Footprint}
A simple but fast implementation of \MS{} writes all merged runs to an auxiliary array, raising the need for space for \(n\) additional elements (\enquote{full space}).
After a round is finished and all pairs of runs have been merged, the input array and the auxiliary array switch roles, and the merging starts anew.
Are the final sorted elements supposed to be saved in the original input array, a final round with a write-back from the auxiliary array to the input array is needed for some input lengths.

A slightly more sophisticated implementation needs space for only \(n/2\) additional elements (\enquote{half space}):
When two adjacent runs are to be merged, the first one can be copied to an auxiliary array.
Then, the copy and the second run are merged to the start of the first run.
As a side effect, no write-back is ever needed and, additionally, the merging of two runs can be terminated prematurely once the last element of the copied run is merged, since the last elements of the other run are already in place.
%As a consequence, flushes will only be performed on at most half of the runs.
Further optimised, \MS{} would not need to copy the first runs immediately.
It suffices to search for the foremost element of the first run which is greater than the first element of the second element.
All previous elements are already in the correct position so only the following elements need to be copied to the auxiliary array.
This optimisation, although examined during development, was not in use when measuring runtimes since it unfortunately complicates another optimisation, namely unrolling.

\paragraph{Unrolling}
There are four common reasons for \emph{flushing}, that is, writing \Dash many oftwhiles \Dash consecutive elements:
\begin{enumerate}
	\item
	When two runs are merged and the end of one of them is reached, the remaining elements of the other one can be moved safely to the output location.
	Especially with the sorted, reverse sorted, and almost sorted input distributions, the number of remaining elements will be high.

	\item
	The number of runs is odd, so the full-space \MS{} moves the last run to the output location immediately.

	\item
	The full-space \MS{} may write all elements from the auxiliary array back to the input array if the former contains the final sorted sequence.

	\item
	The half-space \MS{} copies runs, whose length are always a multiple of the the starting run length, before each merger of pairs.
\end{enumerate}
Therefore, flushing account for a considerable part of the runtime, and reducing the loop overhead (variable incrementation and bounds checking) is helpful.
This can be done via \emph{unrolling}:
As long as at least, let us say, \(x\) elements still need to be flushed, the \(x\) foremost elements are moved first and then all necessary variables are incremented by \(x\).
Is \(x\) a compile-time constant, the compiler implements the moving of the elements through \(x\) instruction which use constant, pre-calculated offsets.
Once less than \(x\) elements remain, an ordinary loop which moves elements individually is used.
In good cases, this approach reduces the loop overhead to an \(x\)th, whilst in bad cases, where less than \(x\) are to be flushed, the overhead is increased by one additional check.

Due to time reasons, we refrained from doing automatic and extensive tests and relied on manual and exploratory tests to come up with the following strategy:
When the full-space \MS{} performs a write-back or when the half-space \MS{} copies the first run, \(x\) is set to the starting run length.
In all other cases, \(x\) is set to 24.
This strategy, albeit not optimal, makes the \MS*{} significantly faster:
Sorting sorted, reverse sorted, and almost sorted inputs sees speed-ups up to 30\%, whereas sorting more random inputs still sees speed-ups for the most part and slow-downs into low single-digits at worst, depending on the starting run length.

\subsubsection*{Investigation of the Compilation}
\label{subsubsec:tasklet:merge:compilation}

Yet again, the compiler shows unforeseen behaviour.
The following is a collocation of some of the issues found while engineering \MS{}.
They will not be discussed in detail here but still provide a point of reference for future work:
\begin{itemize}
	\item
	As already mentioned, sorting the starting runs via \ShS{} requires the placement and later removal of temporary sentinel values.
	For the very first starting run, one can omit storing and restoring the overwritten elements by using permanent sentinel values;
	this optimisation was in use when measuring runtimes.
	On the downside, this leads to a bigger compilation as \ShS{} is inlined twice.
	If the size of the whole compilation is already close to the maximum, one might be inclined to handle the first starting run just like the others.
	Realistically, this should slow down the total runtime by just a few hundreds of cycles, yet the real slow-down is in the thousands.

	\item
	If the input is so short that it fits entirely within the first starting run, one can immediately end the execution after \ShS{} is done.
	Several implementations were tested, with unsatisfactory results:
	Some increased the runtime for longer inputs, others decreased it but also increased it for shorter inputs.
	The settled-on implementation is of the former variety since the increases hit shorter inputs harder relatively and a more thorough solution would not further the purpose of this section.

	\item
	Concerning the half-space \MS{}:
	Treating the copied run logically as the second run and the uncopied run as the first one nets a noticeable decrease in runtime compared to an implementation with flipped logic.
	Even worse, only with the former does unrolling improve the speed, being an impairment with the latter!
	This behaviour occurs with both immediate and deferred copying of the first runs.
	An inspection of the issue unearthed marvels like code of the form
	\begin{center}
		\vspace{-\baselineskip}
		\texttt{*i++ = *j; a = b - i; c = i; i = d;}
	\end{center}
	leading to 5\% longer runtimes compared to
	\begin{center}
		\texttt{*i = *j; a = b - (i + 1); c = i + 1; i = d;}
	\end{center}
	even though executed at most once per merger, but we could sadly not pinpoint the fundamental cause for the behaviour.
\end{itemize}


\subsubsection*{Evaluation of the Performance}
\label{subsubsec:tasklet:merge:performance}

\def\mergealgos{16,24,32,48,64,96}

\pgfplotstablereadnamed{data/wram_sorts.txt}{tableWramSorts}
\expandafter\pgfplotsinvokeforeach\expandafter{\mergealgos}{
	\pgfplotstablereadnamed{data/merge/fallback=#1/uint32/sorted.txt}{tableMergeStart#1_32sorted}
	\pgfplotstablereadnamed{data/merge/fallback=#1/uint32/reverse.txt}{tableMergeStart#1_32reverse}
	\pgfplotstablereadnamed{data/merge/fallback=#1/uint32/almost.txt}{tableMergeStart#1_32almost}
	\pgfplotstablereadnamed{data/merge/fallback=#1/uint32/uniform.txt}{tableMergeStart#1_32uniform}
	\pgfplotstablereadnamed{data/merge/fallback=#1/uint32/zipf.txt}{tableMergeStart#1_32zipf}
	\pgfplotstablereadnamed{data/merge/fallback=#1/uint32/normal.txt}{tableMergeStart#1_32normal}
}

\pgfplotsset{
	merge/.style={
		horizontal sep for ticks,
		adaptive group=1 by 3,
		groupplot xlabel={Input Length \(n\)},
		groupplot ylabel={Cycles / \((n \lb n)\)},
		xmode=log,
		xmax=1024,
		xtick={16, 64, 256, 1024},
		xticklabels={\(16\), \(64\), \(256\), \(1024\)},
		minor xtick={32, 128, 512},
		enlarge x limits={abs=3mm, true},
		legend columns=-1,
		every legend image post={mark=none},
	},
	merge filter 16/.style={x filter/.expression={(\thisrow{n} == 16) || (\thisrow{n} ==  24) || (\thisrow{n} ==  96) || (\thisrow{n} == 384) || (\thisrow{n} == 1536) ? \pgfmathresult : nan}},
	merge filter 24/.style={x filter/.expression={(\thisrow{n} == 16) || (\thisrow{n} ==  32) || (\thisrow{n} == 128) || (\thisrow{n} == 512) || (\thisrow{n} == 2048) ? \pgfmathresult : nan}},
	merge filter 32/.style={x filter/.expression={(\thisrow{n} == 16) || (\thisrow{n} ==  48) || (\thisrow{n} == 192) || (\thisrow{n} == 768) || (\thisrow{n} == 3072) ? \pgfmathresult : nan}},
	merge filter 48/.style={x filter/.expression={(\thisrow{n} == 16) || (\thisrow{n} ==  64) || (\thisrow{n} == 256) || (\thisrow{n} == 1024) ? \pgfmathresult : nan}},
	merge filter 64/.style={x filter/.expression={(\thisrow{n} == 16) || (\thisrow{n} ==  96) || (\thisrow{n} == 384) || (\thisrow{n} == 1536) ? \pgfmathresult : nan}},
	merge filter 96/.style={x filter/.expression={(\thisrow{n} == 16) || (\thisrow{n} == 128) || (\thisrow{n} == 512) || (\thisrow{n} == 2048) ? \pgfmathresult : nan}},
}

\begin{figure}
	\tikzsetnextfilename{merge_starting_runs}
	\begin{tikzpicture}[plot]
		\begin{groupplot}[
			merge,
			ymin=65,
			ymax=90,
			ytick distance=5,
			]
			\nextgroupplot[title={No Write-back\strut}, legend to name=leg:merge:starting_runs]
			\expandafter\legend\expandafter{\mergealgos}
			\clip (0, 0) rectangle (1024, 200);
			\expandafter\pgfplotsinvokeforeach\expandafter{\mergealgos}{
				\plotpernlogn[merge filter #1]{Merge}{tableMergeStart#1_32uniform}
			}
			%
			\nextgroupplot[title={Write-back\strut}]
			\clip (0, 0) rectangle (1024, 200);
			\expandafter\pgfplotsinvokeforeach\expandafter{\mergealgos}{
				\plotpernlogn[merge filter #1]{MergeWriteBack}{tableMergeStart#1_32uniform}
			}
			%
			\nextgroupplot[title={Half Space}]
			\clip (0, 0) rectangle (1024, 200);
			\expandafter\pgfplotsinvokeforeach\expandafter{\mergealgos}{
				\plotpernlogn[merge filter #1]{MergeHalfSpace}{tableMergeStart#1_32uniform}
			}
		\end{groupplot}
	\end{tikzpicture}

	\hfil\pgfplotslegendfromname{leg:merge:starting_runs}\hfil
	\caption{
		Comparison of \MS*{}, which need an auxiliary array of length either \(n\) (\enquote{No Write-back} / \enquote{Write-back}) or \(\sfrac{n}{2}\) (\enquote{Half Space}), for different lengths of the starting runs.
		The \MS*{} use a \ShS{} with the step sizes \(\stepsizes = (1)\) for length 16, \(\stepsizes = (6, 1)\) for lengths 24 to 48, and \(\stepsizes = (12, 5, 1)\) for lengths 64 and 96, respectively.
	}
	\label{fig:merge:starting_runs}
\end{figure}

Three implementations have been tested:
full space \MS{} without write-backs, full space \MS{} with write-backs, and half space \MS{}.
\Cref{fig:merge:starting_runs,fig:merge:starting_runs_uint32sorted,fig:merge:starting_runs_uint32uniform,fig:merge:starting_runs_uint64sorted,fig:merge:starting_runs_uint64uniform} show their performance for various starting run lengths.
Please note that the plots are smoothed:
Whenever the number of rounds increments, the runtimes hike, making the zigzagging plots cross each other unswervingly and, thereby, hard to read.
Thence, the figures contain marks for select measurements only in such a way that the resulting plots act as an upper bound on the runtime.

The measurements show that the \MS*{} guarantee a runtime of \(\bigoh{n \lb n}\) as expected.
The differences in runtime between the different input distributions are small compared to \QS{} and are ascribable to \ShS{} and to the differing suitability of the unrolling;
cases where the usage of \ShS{} worsened the runtime are unbeknown.

Even though the tested starting run lengths range from 16 to 96 elements, the mean runtime differences are surprisingly small.
Notwithstanding that the optimal choice depends on the specific input length because of the zigzagging, a starting run length of 32 elements fares decidedly well on average across all tested scenarios.

The half-space \MS{} delivers a strong performance despite its vastly lower memory footprint.
With 32-bit integers, it beats the full-space \MS{} without write-backs by 11\% on sorted inputs and effectively ties on all other inputs but the reverse sorted ones where it narrowly falls behind.
Naturally, the full-space \MS{} with write-backs is consistently (with the exception of reverse sorted inputs) at a disadvantage, despite seeing some light with inferior starting run lengths.
With 64-bit integers, the full-space \MS{} without write-backs manages to turn the ties into scant leads in the range from 1\% to 3\%.
Using the \MS{} with write-backs is still unprofitable.

In summary, a proper implementation of half-space \MS{} with deferred copying and fine-tuned unrolling would require some work but has the potential to be the overall best stable sorting algorithm.


\begin{figure}
	\tikzsetnextfilename{wram_sorts}
	\begin{tikzpicture}[plot]
		\begin{groupplot}[
			horizontal sep for labels,
			adaptive group=1 by 2,
			groupplot xlabel={Input Length \(n\)},
			xmode=log,
			xtick={20, 32, 64, 128, 256, 512, 1024},
			xticklabels={\(20\), \(32\), \(64\), \(128\), \(256\), \(512\), \(1024\)},
			legend columns=-1,
		]
			\nextgroupplot[ylabel=Cycles / \((n \lb n)\), ymin=0, ymax=150, legend to name=leg:wram_sorts]
			\legend{\QS{}, \ShS, \HS{}, \MS{}}
%			\plotpernlogn{Merge}{tableWramSorts}
			\plotpernlogn{Quick}{tableWramSorts}
			\plotpernlogn{Shell}{tableWramSorts}
%			\plotpernlogn{MergeWriteBack}{tableWramSorts}
			\plotpernlogn{Heap}{tableWramSorts}
			\plotpernlogn{MergeHalfSpace}{tableWramSorts}
			%
			\nextgroupplot[ylabel=Speed-up, ymin=0.3, ymax=1, extra y ticks={0.3}]
%			\plotspeedup{Merge}{Quick}{tableWramSorts}
			\pgfplotsset{cycle list shift=1}
			\plotspeedup{Shell}{Quick}{tableWramSorts}
%			\plotspeedup{MergeWriteBack}{Quick}{tableWramSorts}
			\plotspeedup{Heap}{Quick}{tableWramSorts}
			\plotspeedup{MergeHalfSpace}{Quick}{tableWramSorts}
		\end{groupplot}
	\end{tikzpicture}

	\hfil\pgfplotslegendfromname{leg:wram_sorts}\hfil
	\caption{
		Comparison of \MS{}, \HS{}, \ShS{}, and \QS{}.
		Due to \MS{}'s increased space requirements, its runtime was measured only for up to 768 elements.
		The \ShS{} uses the step sizes from \cref{fig:shell:against_others}, which are unoptimised for large input sizes.
		The speed-ups are with respect to the \QS{}.
	}
	\label{fig:wram_sorts}
\end{figure}

\subsection{\texorpdfstring{\HS{}}{HeapSort}}
\label{subsec:tasklet:heap}

\pgfplotsinvokeforeach{sorted,reverse,almost,uniform,zipf,normal}{
	\pgfplotstablereadnamed{data/heap/uint32/#1.txt}{tableHeap_32#1}
	\pgfplotstablereadnamed{data/heap/uint64/#1.txt}{tableHeap_64#1}
}

\paragraph{Indexing}
With a zero-based indexing, the sons of a vertex \(i\) can be calculated with the well-known formulas \(2i + 1\) and \(2n + 2\).
With a one-based indexing, the formulas turn into \(2i\) and \(2i + 1\).
The compiler automatically turns the multiplication by two into a left-shift by one.
Since DPUs can execute an instruction called \texttt{lsl\_add} which first shifts leftwards and then adds an offset (useful \eg{} for array indexing), the formulas \(2i + 1\) and \(2i\) take the same amount of time to compute.

Nevertheless, the zero-based indexing is about 7\% slower despite \texttt{lsl\_add} being indeed in use.
The reason is that only the number of bits to shift can be passed as immediate value as plain number but not the offset, which must be passed via a register.
While DPUs have a read-only register storing the number \(1\) at disposal, read-only registers can only ever be the first register argument, not the second one, which, for \texttt{lsl\_add}, would be the offset.
As a consequence, the compiler moves the number \(1\) to a register whenever \(2i + 1\) is to be computed, only to immediately overwrite the \(1\) with the result from \texttt{lsl\_add}.
Hence, the calculation of \(2i + 1\) does take one more instruction than \(2n\) after all.

\paragraph{Sentinel Values}


\begin{itemize}
	\item
	code duplication:
	savings around the 7\% mark

	\item
	sentinel leafs:
	savings around the 2\% mark

	\item
	if n even: vor T\_MIN: -5000 -- +15000 Takte

	\item
	Var 1:
	leichter trend 1 nach oben

	Var 2:
	leichter downwards trend

	Effekte stärker für 64 Bit
\end{itemize}

\pgfplotsset{
	heap/.style={
		adaptive group=1 by 2,
		groupplot xlabel={Input Length \(n\)},
		groupplot ylabel={Cycles / \(n \lb n\)},
		xmode=log,
		xtick={16, 32, 64, 128, 256, 512, 1024},
		xticklabels={\(16\), \(32\), \(64\), \(128\), \(256\), \(512\), \(1024\)},
		legend columns=-1,
	},
}

\begin{figure}
	\def\algos{HeapOnlyDown,HeapUpDown}
	\tikzsetnextfilename{heap_runtime}
	\begin{tikzpicture}[plot]
		\begin{groupplot}[heap]
			\nextgroupplot[title={32-bit\strut}, ymin=130, ymax=145, legend to name=leg:heap:runtime]
			\legend{\HS{}, \HS{}}
			\expandafter\pgfplotsinvokeforeach\expandafter{\algos}{
%			\expandafter\pgfplotsinvokeforeach\expandafter\pgfkeysvalueof{heap algos}{
				\plotpernlogn{#1}{tableHeap_32uniform}
			}
			%
			\nextgroupplot[title={64-bit\strut}, ymin=155, ymax=175]
			\expandafter\pgfplotsinvokeforeach\expandafter{\algos}{
				\plotpernlogn{#1}{tableHeap_64uniform}
			}
		\end{groupplot}
	\end{tikzpicture}

	\hfil\pgfplotslegendfromname{leg:heap:runtime}\hfil
	\caption{
		Comparison of the runtimes of the two different \HS{} implementations on uniformly distributed 32-bit integers and 64-bit integers, respectively.
	}
	\label{fig:heap:runtime}
\end{figure}



%	\clearpage

	\mybibliography

	\begin{description}
	\item[r0] start
	\item[r1] end
	\item[r23] return address
\end{description}
\begin{verbatim}
insertion_sort_nosentinel:
    jleu r0, r1, .LBB2_1  // Continue if array of positive length …
.LBB2_8:
    jump r23  // … else leave the function.
.LBB2_1:
    move r2, r0, true, .LBB2_2  // i ← start; Jump to beginning of outer loop.
.LBB2_5:
    move r4, r5  // ?
.LBB2_7:
    add r2, r2, 4  // i++
    sw r4, 0, r3  // *curr ← to_sort
    jgtu r2, r1, .LBB2_8  // If i > end, terminate.
.LBB2_2:  // Beginning of outer loop.
    lw r3, r2, 0  // to_sort ← *i;
    add r5, r2, -4  // prev ← i - 1
    move r4, r2  // curr ← i
    jltu r5, r0, .LBB2_7  // If prev < start, skip to the next iteration of the outer loop.
    move r5, r2  // (prev + 1) ← i
.LBB2_4:
    lw r6, r5, -4  // *prev
    jleu r6, r3, .LBB2_5  // If *prev > to_sort, terminate inner loop.
    add r4, r5, -4  // Store prev.
    add r7, r5, -8  // Store prev--.
    sw r5, 0, r6  // *curr ← *prev
    move r5, r4  // curr ← prev
    jgeu r7, r0, .LBB2_4  // If prev >= start, continue with the next iteration of the inner loop.
    jump .LBB2_7  // Continue with the next iteration of the outer loop.
\end{verbatim}

\begin{verbatim}
insertion_sort_sentinel:
    jleu r0, r1, .LBB3_1  // Continue if array of positive length …
.LBB3_5:
    jump r23  // … else leave the function.
.LBB3_4:
    add r0, r0, 4  // i++
    sw r3, 0, r2  // *curr ← to_sort
    jgtu r0, r1, .LBB3_5  // If i > end, leave the function.
.LBB3_1:
    lw r2, r0, 0  // to_sort ← *i
    lw r4, r0, -4  // *prev
    move r3, r0  // curr ← i
    jleu r4, r2, .LBB3_4  // If *prev > to_sort, terminate inner loop.
    move r3, r0  // ???
.LBB3_3:
    sw r3, 0, r4  // *curr ← *prev
    lw r4, r3, -8  // *(prev - 1)
    add r3, r3, -4  // curr ← prev
    jgtu r4, r2, .LBB3_3  // If *(prev - 1) > to_sort, continue with the next iteration of the inner loop.
    jump .LBB3_4  // Leave inner loop.
\end{verbatim}
\end{document}