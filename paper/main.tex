% !TeX program = lualatex
\documentclass[final]{../garticle}

% Maths & Algorithms.
\DeclareMathOperator{\lb}{lb}

% Figures.
% taken from: https://tex.stackexchange.com/a/199396
\pgfplotsset{
	select coords between index/.style 2 args={
		x filter/.code={
			\ifnum\coordindex<#1\def\pgfmathresult{}\fi
			\ifnum\coordindex>#2\def\pgfmathresult{}\fi
		},
	},
}

% Draws a plot of data in column µ_#2 of the table #3.
% A column n must be present. Applies custom options #1.
\NewDocumentCommand{\plotruntime}{O{} m m}{
	\addplot+
	plot [#1]
	table [x=n, y=µ_#2] {#3};
}

% Draws a plot of column µ_#2 with error bars loaded from σ_#2 of table #3.
% A column n must be present. Applies custom options #1.
\NewDocumentCommand{\plotwithbars}{O{} m m}{
	\addplot+
	plot [#1, error bars/.cd, y dir=both, y explicit]
%	table [x=n, y=µ_#2, y error=σ_#2] {#3};
	table [x=n, y=µ_#2, y error expr={2 * \thisrow{σ_#2}}] {#3};
}

% Draws a plot of column µ_#2 of table #3, divided by n log n.
% A column n must be present. Applies custom options #1.
\NewDocumentCommand{\plotpernlogn}{O{} m m}{
	\addplot+
	plot [#1]
	table [x=n, y expr={\thisrow{µ_#2} / (\thisrow{n} * log2(\thisrow{n}))}] {#3};
}

% Draws a plot of column µ_#2 of table #3, divided by n² (a column n must be present).
% A column n must be present. Applies custom options #1.
\NewDocumentCommand{\plotpernn}{O{} m m}{
	\addplot+
	plot [#1]
	table [x=n, y expr={\thisrow{µ_#2} / \thisrow{n}^2}] {#3};
}

% Draws a plot of column µ_#3 divided by column µ_#2 of table #4.
% A column n must be present. Applies custom options #1.
\NewDocumentCommand{\plotspeedup}{O{} m m m}{
	\addplot+
	plot [#1]
	table [x=n, y expr={\thisrow{µ_#3} / \thisrow{µ_#2}}] {#4};
}

% Texts.
\NewDocumentCommand{\BS}{s}{Bubble\-Sort\IfBooleanT{#1}{s}}
\NewDocumentCommand{\HS}{s}{Heap\-Sort\IfBooleanT{#1}{s}}
\NewDocumentCommand{\IS}{s}{Insertion\-Sort\IfBooleanT{#1}{s}}
\NewDocumentCommand{\QS}{s}{Quick\-Sort\IfBooleanT{#1}{s}}
\NewDocumentCommand{\SelS}{s}{Selection\-Sort\IfBooleanT{#1}{s}}
\NewDocumentCommand{\ShS}{s}{Shell\-Sort\IfBooleanT{#1}{s}}

\makeatletter
\@ifpackagewith{cleveref}{capitalize}{
	\crefname{implementation}{Impl.}{Impl.}
	\Crefname{implementation}{Implementation}{Implementations}
}{
	\crefname{implementation}{impl.}{impl.}
	\Crefname{implementation}{Implementation}{Implementations}
}

\makeatother

\RequirePackage{pgfplotstable}

\titlehead{Group of Algorithm Engineering\hfill Summer Semester 2024 \\ Institute for Computer Science \\ Goethe University Frankfurt}
\subject{Master's Thesis}
%\title{Sorting \\ for a \\ Processing-in-Memory \\ Architecture}
%\title{Sorting \\ on a \\ Processing-in-Memory \\ Architecture}
%\title{On Sorting for Processing-in-Memory}
\title{On Efficient Sorting Through In-Memory Processing}
%\title{Engineering Sorting Algorithms \\ for a Processing-in-Memory Architecture}
%\title{Engineering Sorting Algorithms \\ for Processing-in-Memory}
%\subtitle{Some cool subtitle if need be}
\subtitle{Implementation and Evaluation \dots{} / Engineering \dots{} / Exploring \dots{}}
%\subtitle{Engineering Algorithms for UPMEM-based DRAM Processing Units}
\author{\texorpdfstring{Ƶ}{Z}eno Adrian \texorpdfstring{\Lss05{W\kern-1.5pt}}{W}eil}
\publishers{\begin{tabular}{r @{~}l}
	Supervisor: & Dr Manuel Penschuck
\end{tabular}}

\makeatletter
\hypersetup{
	pdfauthor=\@author,
	pdftitle=\@title,
	pdfsubject=\@subject,
}
\makeatother

\tikzset{
	legend 3 cols/.style = {
		/pgfplots/legend columns=3,
		/pgfplots/legend style={
			/tikz/column 2/.style={
				column sep=5pt,
			},
			/tikz/column 4/.style={
				column sep=5pt,
			},
		},
	},
}

\begin{document}
	\pagenumbering{gobble}  % turn off pagenumbering

%	\maketitle

%	\begin{abstract}
	\noindent
	The growing disparity between processing and memory speed, coupled with increasing data demands, has led to memory accesses being a bottleneck for many modern workflows.
	An example are sorting algorithms, which are often designed around the constraints set by memory subsytems.
	\Acl*{PIM} (also known as processing in memory, \acs*{PIM}) is an umbrella term encompassing several approaches which offload computational tasks to accelerators in or near the memory itself.
	In \acs*{PIM} systems designed and manufactured by \upmem{}, traditional dynamic random-access memory (\acs*{DRAM}) modules are augmented with general-purpose processors called \acfp*{DPU}.
	These are located next to the memory banks themselves, whereby high memory access speed is accomplished.
	An \upmem{}-based \acs*{PIM} system may contain thousands of \acsp*{DPU}, each capable of additional thread-level parallelism.
	Although designed for general use, the \acs*{DPU} architecture does come with limitations to its computational prowess.

	The scope of this thesis is the design, implementation, and evaluation of sorting algorithms which run on a single \acs*{DPU}.
	We investigate several sequential and parallel sorting algorithms, documenting the engineering process and adaptations to the merits and shortcomings of the \acs*{DPU} architecture.
	We find that sorting is a suitable task for a \acs*{DPU}, which can be sped up nearly ideally through multithreading.
	This paves the way for more large-scale sorting algorithms which run on multiple \acsp*{DPU}.
\end{abstract}


%	\tableofcontents

%	\listoftodos

	\begingroup
	\clearpage
	\pagenumbering{arabic}  % turn on pagenumbering
	\endgroup

	\pgfplotstableread{data/test2.dat}{\tabletest}
	\begin{tikzpicture}
		\begin{loglogaxis}[
			xlabel=Input Length \(n\),
			ylabel=Cycles,
			width=\linewidth,
			height=5cm,
			xtick=data,
			xticklabels from table={\tabletest}{n},
			log ticks with fixed point,
			grid,
			legend 3 cols,
			legend pos=south east,
		]
			\pgfplotsinvokeforeach{1,...,9}{
				\linewithbars[select coords between index={1}{5}]{#1}{\tabletest}
				\addlegendentry{\(#1\)}
			}
		\end{loglogaxis}
	\end{tikzpicture}

	\begin{tikzpicture}
		\begin{loglogaxis}[
			xlabel=Input Length \(n\),
			ylabel=Cycles / \((n \lb n )\),
			width=\linewidth,
			height=5cm,
			xtick=data,
			xticklabels from table={\tabletest}{n},
			log ticks with fixed point,
			grid,
			legend 3 cols,
			legend pos=north west,
		]
			\pgfplotsinvokeforeach{1,...,9}{
				\linepernlogn{#1}{\tabletest}
				\addlegendentry{\(#1\)}
			}
		\end{loglogaxis}
	\end{tikzpicture}

	\begin{tikzpicture}
		\begin{loglogaxis}[
			xlabel=Input Length \(n\),
			ylabel=Cycles / \(n\)²,
			width=\linewidth,
			height=5cm,
			xtick=data,
			xticklabels from table={\tabletest}{n},
			log ticks with fixed point,
			grid,
			legend 3 cols,
			legend pos=south west,
		]
			\pgfplotsinvokeforeach{1,...,9}{
				\linepernn{#1}{\tabletest}
				\addlegendentry{\(#1\)}
			}
		\end{loglogaxis}
	\end{tikzpicture}

	\begin{tikzpicture}
		\pgfplotsset{cycle list shift=1}  % skip over colour of first plot
		\begin{semilogxaxis}[
			xlabel=Input Length \(n\),
			ylabel=Speed-Up,
			width=\linewidth,
			height=5cm,
			xtick=data,
			xticklabels from table={\tabletest}{n},
			log ticks with fixed point,
			grid,
			legend 3 cols,
			legend pos=north west,
		]
			\pgfplotsinvokeforeach{2,...,9}{
				\linespeedup[select coords between index={1}{5}]{#1}{1}{\tabletest}
				\addlegendentry{\(#1\)}
			}
		\end{semilogxaxis}
	\end{tikzpicture}

%	\clearpage

%	\mybibliography
\end{document}