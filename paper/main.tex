\documentclass[draft, british]{../garticle}

\RequirePackage{mfirstuc}  % used for capitalising names of sorting algorithms
% Maths & Algorithms.
\DeclareMathOperator{\lb}{lb}

% Figures.
% taken from: https://tex.stackexchange.com/a/199396
\pgfplotsset{
	select coords between index/.style 2 args={
		x filter/.code={
			\ifnum\coordindex<#1\def\pgfmathresult{}\fi
			\ifnum\coordindex>#2\def\pgfmathresult{}\fi
		},
	},
}

\NewDocumentCommand{\plotruntime}{O{} m m}{
	\addplot+
	plot [#1]
	table [x=n, y=µ_#2] {#3};
}

\NewDocumentCommand{\plotwithbars}{O{} m m}{
	\addplot+
	plot [#1, error bars/.cd, y dir=both, y explicit]
%	table [x=n, y=µ_#2, y error=σ_#2] {#3};
	table [x=n, y=µ_#2, y error expr={2 * \thisrow{σ_#2}}] {#3};
}

\NewDocumentCommand{\plotpernlogn}{O{} m m}{
	\addplot+
	plot [#1]
	table [x=n, y expr={\thisrow{µ_#2} / (\thisrow{n} * log2(\thisrow{n}))}] {#3};
}

\NewDocumentCommand{\plotpernn}{O{} m m}{
	\addplot+
	plot [#1]
	table [x=n, y expr={\thisrow{µ_#2} / \thisrow{n}^2}] {#3};
}

\NewDocumentCommand{\plotspeedup}{O{} m m m}{
	\addplot+
	plot [#1]
	table [x=n, y expr={\thisrow{µ_#3} / \thisrow{µ_#2}}] {#4};
}

% Texts.

\titlehead{Group of Algorithm Engineering\hfill Summer Semester 2024 \\ Institute for Computer Science \\ Goethe University Frankfurt}
\subject{Master's Thesis}
%\title{Sorting \\ for a \\ Processing-in-Memory \\ Architecture}
%\title{Sorting \\ on a \\ Processing-in-Memory \\ Architecture}
%\title{On Sorting for Processing-in-Memory}
\title{On Efficient Sorting Through In-Memory Processing}
%\title{Engineering Sorting Algorithms \\ for a Processing-in-Memory Architecture}
%\title{Engineering Sorting Algorithms \\ for Processing-in-Memory}
%\subtitle{Some cool subtitle if need be}
\subtitle{Implementation and Evaluation \dots{} / Engineering \dots{} / Exploring \dots{}}
%\subtitle{Engineering Algorithms for UPMEM-based DRAM Processing Units}
\author{\texorpdfstring{Ƶ}{Z}eno Adrian \texorpdfstring{\Lss05{W\kern-1.5pt}}{W}eil}
\publishers{\begin{tabular}{r @{~}l}
	Supervisor: & Dr Manuel Penschuck
\end{tabular}}

\makeatletter
\hypersetup{
	pdfauthor=\@author,
	pdftitle=\@title,
	pdfsubject=\@subject,
}
\makeatother

\usepgfplotslibrary{groupplots}
\tikzset{
	plot/.style={  % Not set globally lest other packages break.
		trim axis group left, trim axis group right,  % Only axes define the bounding box. Thus, lables can extend into the margins.
	},
}
\pgfplotsset{
	height=4cm,
	cycle multiindex list={exotic \nextlist mark list*},  % automatic colouring of plots
	scale only axis,  % height and width apply only to the axes, not the labels
	enlargelimits={abs=3mm, auto},  % for axes without specified limits, the limits are a bit bigger than neeeded (→ padding)
	log ticks with fixed point,  % uses 0.1, 0.001, … on logarithmix axes instead of 10^-1, 10^-2, …
	ymajorgrids,  % grey, horizontal background lines at each y-tick
	legend style={  % adds padding to legends with multiple columns
		cells={anchor=west},  % left aligned labels
		/tikz/column 2/.style={column sep=5pt}, /tikz/column 4/.style={column sep=5pt}, /tikz/column 6/.style={column sep=5pt}, /tikz/column 8/.style={column sep=5pt}, /tikz/column 10/.style={column sep=5pt}, /tikz/column 12/.style={column sep=5pt}, /tikz/column 14/.style={column sep=5pt}, /tikz/column 16/.style={column sep=5pt},
	},
	group style={horizontal sep=5em, vertical sep=5em},
}

% Adds groupplot xlabel and groupplot ylabel, i.e. common axis labels for groupplots.
% groupplot xlabel should also be used if there is only one plot in the group plot for margin reasons.
% taken from: https://tex.stackexchange.com/a/117935
\makeatletter
\pgfplotsset{
	groupplot xlabel/.initial={},
	every groupplot x label/.style={
		at={($({\pgfplots@group@name\space c1r\pgfplots@group@rows.west}|-{\pgfplots@group@name\space c1r\pgfplots@group@rows.outer south})!0.5!({\pgfplots@group@name\space c\pgfplots@group@columns r\pgfplots@group@rows.east}|-{\pgfplots@group@name\space c\pgfplots@group@columns r\pgfplots@group@rows.outer south})$)},
		anchor=north,
	},
	groupplot ylabel/.initial={},
	every groupplot y label/.style={
		rotate=90,
		at={($({\pgfplots@group@name\space c1r1.north}-|{\pgfplots@group@name\space c1r1.outer
				west})!0.5!({\pgfplots@group@name\space c1r\pgfplots@group@rows.south}-|{\pgfplots@group@name\space c1r\pgfplots@group@rows.outer west})$)},
		anchor=south
	},
	execute at end groupplot/.code={%
		\node [/pgfplots/every groupplot x label]
		{\pgfkeysvalueof{/pgfplots/groupplot xlabel}};
		\node [/pgfplots/every groupplot y label]
		{\pgfkeysvalueof{/pgfplots/groupplot ylabel}};
	}
}

\def\endpgfplots@environment@groupplot{%
	\endpgfplots@environment@opt%
	\pgfkeys{/pgfplots/execute at end groupplot}%
	\endgroup%
}
\makeatother


\begin{document}
	\pagenumbering{gobble}  % turn off pagenumbering

%	\maketitle

%	\begin{abstract}
	\noindent
	The growing disparity between processing and memory speed, coupled with increasing data demands, has led to memory accesses being a bottleneck for many modern workflows.
	An example are sorting algorithms, which are often designed around the constraints set by memory subsytems.
	\Acl*{PIM} (also known as processing in memory, \acs*{PIM}) is an umbrella term encompassing several approaches which offload computational tasks to accelerators in or near the memory itself.
	In \acs*{PIM} systems designed and manufactured by \upmem{}, traditional dynamic random-access memory (\acs*{DRAM}) modules are augmented with general-purpose processors called \acfp*{DPU}.
	These are located next to the memory banks themselves, whereby high memory access speed is accomplished.
	An \upmem{}-based \acs*{PIM} system may contain thousands of \acsp*{DPU}, each capable of additional thread-level parallelism.
	Although designed for general use, the \acs*{DPU} architecture does come with limitations to its computational prowess.

	The scope of this thesis is the design, implementation, and evaluation of sorting algorithms which run on a single \acs*{DPU}.
	For several sequential and parallel sorting algorithms, we document the engineering process and adaptations to the merits and shortcomings of the \acs*{DPU} architecture.
	We find that sorting is a suitable task for a \acs*{DPU}, which can be sped up nearly ideally through multithreading.
	This paves the way for more large-scale sorting algorithms which run on multiple \acsp*{DPU}.
\end{abstract}


	\tableofcontents

	\listoftodos

	\begingroup
%	\clearpage
	\pagenumbering{arabic}  % turn on pagenumbering
	\endgroup

	\bigskip
	\todo[inline]{Architektur}
	\todo[inline]{Speicherzugriffe (memcpy, mram\_read \dots)}
	\todo[inline]{triple buffer}

	\pgfplotstableread{data/small_sorts.txt}{\tablesmallsorts}
\begin{figure}
	\begin{tikzpicture}
		\begin{groupplot}[
			width=\linewidth,
			groupplot xlabel={Input Length \(n\)},
			ylabel=Speed-Up,
			xtick=data,
			ymin=0.2,
			ymax=1,
			legend columns=2,
			legend pos=north east,
			]
			\nextgroupplot
			\plotspeedup{1NoSentinel}{1}{\tablesmallsorts}
			\addlegendentry{Insertion (no sentinel)}
			\plotspeedup{BubbleNonAdapt}{1}{\tablesmallsorts}
			\addlegendentry{Bubble}
			\plotspeedup{BubbleAdapt}{1}{\tablesmallsorts}
			\addlegendentry{Bubble (adaptive)}
			\plotspeedup{Selection}{1}{\tablesmallsorts}
			\addlegendentry{Selection}
		\end{groupplot}
	\end{tikzpicture}
	\caption{
		The speed-ups of some sorting algorithms over an InsertionSort using sentinel values with uniform input distribution.
		Compared to normal BubbleSort, the adaptive version terminates prematurely if no changes were made to the input array during an iteration.
		Speed-ups below \(1\) indicate slow-downs.
	}
	\label{fig:speed-up_over_is}
\end{figure}

\paragraph{InsertionSort}
This stable sorting algorithm works by moving the \(i\)th element to the left as long as its left neighbour is bigger, assuming that the elements \(0\) to \(i - 1\) are already sorted.
Even though in both the average case and the worst case, InsertionSort has a runtime of \(\bigomicron{n^2}\)\todo{Beleg?}, it displays quite some advantages:
\begin{enumerate*}
	\item
	It works in-place, needing only \(\bigomicron{1}\) additional space.

	\item
	It is inherently adaptive:
	If the input array is mostly or even fully sorted, the runtime drops down to \(\bigomicron{n}\).

	\item
	Its program code is short, lending itself to inlining.

	\item
	The overhead is small.
\end{enumerate*}
Especially the last two points make InsertionSort a good base algorithm for asymptotically better sorting algorithms to use on very small subarrays.

When moving an element to the left, two checks are needed:
Does the left neighbour exist and is it smaller than the element?
The first check can be omitted through the use of \emph{sentinel values}:
If the element at index \(-1\) is at least as small as any value in the input array, the leftwards motion stops there at the latest.
Since a DPU has no branch predictor, the slowdown from performing twice as many checks as needed is quite high and lies between 20\% and 40\% (\cref{fig:speed-up_over_is}).


Other known simple sorting algorithm with similar runtimes are SelectionSort and BubbleSort.
The asymptoticity, however, hides much higher constant factors such that even for as little as three elements InsertionSort is superior (\cref{fig:speed-up_over_is}).



\begin{figure}
	\begin{tikzpicture}%[trim axis group left, trim axis group right]
		\begin{groupplot}[
			width=0.4358\linewidth,
			group/group size=2 by 1,
			groupplot xlabel={Input Length \(n\)},
			xtick distance=3,
			minor xtick=data,
			legend columns=-1,
			legend entries={\(1\), \(...\), \(9\)},
			legend to name={leg:shell_sort},
			]
			\nextgroupplot[ylabel=Cycles / \(n^2\), ymin=0, ymax=80]
			\pgfplotsinvokeforeach{1,...,9}{
				\plotpernn{#1}{\tablesmallsorts}
			}
			%
			\nextgroupplot[ylabel=Speed-Up, ymin=0.6, ymax=1.2]
			\plotspeedup[no markers, draw=none]{1}{1}{\tablesmallsorts}  % invisible but needed to *properly* skip over first colour (even in the legend!)
			\pgfplotsinvokeforeach{2,...,9}{
				\plotspeedup{#1}{1}{\tablesmallsorts}
			}
		\end{groupplot}
	\end{tikzpicture}

	\hfil\pgfplotslegendfromname{leg:shell_sort}\hfil

	\caption{
		The runtime of InsertionSort and various ShellSorts as well as the speed-ups of the ShellSorts over InsertionSort with uniform input distribution.
		Each ShellSort does one InsertionSort pass with a step size between 2 and 9.
		Speed-ups below \(1\) indicate slow-downs.
	}
\end{figure}

	\begin{figure}
	\begin{tikzpicture}[trim axis left, trim axis right]
		\begin{axis}[
			xlabel=Input Length \(n\),
			ylabel=Speed-Up,
			enlargelimits=auto,
			width=\linewidth,
			xtick=data,
			xticklabels from table={\tablesmallsorts}{n},
			ymax=1,
			ytick={0.3333, 0.6666, 1},
			yticklabels={\(0.\bar{3}\), \(0.\bar{6}\), \(1\)},
			legend columns=-1,
			legend pos=north east,
		]
			\plotspeedup{BubbleNonAdapt}{1}{\tablesmallsorts}
			\addlegendentry{Bubble}
			\plotspeedup{BubbleAdapt}{1}{\tablesmallsorts}
			\addlegendentry{Bubble (adaptive)}
			\plotspeedup{Selection}{1}{\tablesmallsorts}
			\addlegendentry{Selection}
		\end{axis}
	\end{tikzpicture}
	\caption{
		The speed-up of BubbleSort and SelectionSort over regular InsertionSort with uniform input distribution.
		The adaptive BubbleSort terminates prematurely if no changes were made to the input array during an iteration.
	}
\end{figure}

%	\clearpage

	\mybibliography
\end{document}