\section{\texorpdfstring{\MS{}}{MergeSort}}
\label{sec:tasklet:merge}

Two-way bottom-up \MS{}~\cites{katajainen1997meticulous}[85\psq]{maurer1974datenstrukturen}[Chapter~2.3.1]{wirth1975algorithmen} repeatedly compares two elements and \emph{merges} them to form sorted pairs.
Once only pairs (and perhaps one single element) remain, the pairs are merged into quadruplets, the quadruples into octuplets and so on until a single sorted sequence remains.
Sorted sequences are also referred to as \emph{runs}.
\MS{} has a guaranteed runtime of \(\bigoh{n \log n}\) and is the only stable sorting algorithm with subquadratic runtime presented in this \lcnamecref{sec:tasklet}.
Its biggest drawback is that additional space of size \(\bigtheta{n}\) is needed.


\paragraph{Starting Runs}
Instead of starting by merging runs of length 1, that is individual elements, it is beneficial to first subdivide the input and use either \IS{} or \ShS{} on the individual subdivisions.
These larger \emph{starting runs} allow to skip a few of the early rounds of merging.
For simplicity, all starting runs have the same, predefined length with possible exception of the last one which can be shorter.
A substantial downside to \ShS{} is that whilst it does allow to sort bigger starting runs quicker, it is not stable unlike \IS{}.
If \MS{} is supposed to sort stably, then \IS{} has to be used.

Unlike \QS{}, where each partition naturally acts as sentinel for the subsequent one, it is necessary to temporarily place sentinels values in front of each starting run and later restore the overwritten values of the preceding run.
The step sizes used for \ShS{} \Dash namely \(\stepsizes = (1, 6)\) for up to 48 elements, and \(\stepsizes = (1, 5, 12)\) for any more \Dash have been chosen based on the findings in \cref{sec:tasklet:shell}, according to which these step sizes offer top performance for uniformly distributed inputs and medial performance for reverse sorted inputs.
Spot-check inspection suggest no deterioration of \IS{}'s and \ShS{}'s compilation due to inlining.


\paragraph{Memory Footprint}
A simple implementation of \MS{} (denoted by \emph{full-space}) writes all merged runs to an auxiliary array, raising the need for space for \(n\) additional elements.
After a round is finished and all pairs of runs have been merged, the input array and the auxiliary array switch roles, and the merging begins anew.
Are the final sorted elements supposed to be saved in the original input array, a final round with a write-back from the auxiliary array to the input array is needed if the number of rounds is odd.

A slightly more sophisticated implementation (denoted by \emph{half-space}) needs space for only \(n/2\) additional elements:
When two adjacent runs are to be merged, the first one is copied to an auxiliary array.
Then, the copy and the second run are merged to the beginning of the first run.
As a side effect, no write-back is ever needed and, additionally, the merging of two runs can be terminated prematurely once the last element of the copied run is merged, since the last elements of the other run are already in place.
Strictly speaking, the auxiliary array holds~\(n-1\) elements in the worst-case.
By way of example, if the starting run length is \(n - 1\), there would be two runs of lengths \(n - 1\) and \(1\), respectively, and the first one would be copied away.
However, both the maximum number of elements to sort and the starting run length are predetermined, so the memory footprint can indeed be halved compared to the full-space \MS{}.

Further optimised, half-space \MS{} would not need to copy the first runs immediately.
It suffices to search for the foremost element of the first run which is greater than the first element of the second element.
All previous elements are already in the correct position so only the following elements need to be copied to the auxiliary array.
This \emph{deferred} copying, although examined during development, was not in use when measuring runtimes since it unfortunately complicates the following optimisation.


\paragraph{Unrolling}
There are four common reasons for \emph{flushing}, that is, writing \Dash many oftwhiles \Dash consecutive elements:
\begin{enumerate}
	\item
	When two runs are merged and the end of one of them is reached, the remaining elements of the other one can be moved safely to the output location.
	Especially with the sorted, reverse sorted, and almost sorted input distributions, the number of remaining elements will be high.

	\item
	The number of runs is odd, so the full-space \MS{} moves the last run to the output location unconditionally.

	\item
	The full-space \MS{} may write all elements from the auxiliary array back to the input array if the former contains the final sorted sequence.

	\item
	Before each merger of a pair of runs, the half-space \MS{} copies the first run to the auxiliary array.
\end{enumerate}
Therefore, flushing accounts for a considerable part of the runtime, and reducing the loop overhead (variable incrementation and bounds checking) is helpful.
This can be done via \emph{unrolling}:
As long as at least, let us say, \(x\) elements still need to be flushed, a loop with step size \(x\) is executed, and in each iteration, \(x\) elements are moved.
Is \(x\) a compile-time constant, the compiler implements the moving of the elements through \(x\) instruction which use constant, pre-calculated offsets.
Once less than \(x\) elements remain, an ordinary loop with step size \(1\), which moves elements individually, is used.
In good cases, this approach reduces the loop overhead to an \(x\)th, whilst in bad cases, where less than \(x\) elements are to be flushed, the overhead is increased by one additional check.

Due to time reasons, we refrained from doing automatic and extensive tests and relied on manual and exploratory tests to come up with the following strategy:
When the full-space \MS{} performs a write-back or when the half-space \MS{} copies the first run, \(x\) is set to the starting run length.
In all other cases, \(x\) is set to 24.
This strategy, albeit not optimal, makes the \MS*{} significantly faster:
Sorting sorted, reverse sorted, and almost sorted inputs gets faster by up to 30\%, while sorting more random inputs still get faster for the most part and slower by low single-digits at worst, depending on the starting run length.

\subsection{Investigation of the Compilation}
\label{sec:tasklet:merge:compilation}

\begin{figure}[p]
	\lstset{basicstyle=\ttfamily\small}
	\def\codewidth{0.34\linewidth}
	\newlength\assemblerwidth \setlength\assemblerwidth{0.595\linewidth}
	\begin{subfigure}{\textwidth}
		\begin{minipage}{\codewidth}
			\begin{lstlisting}[belowskip=3\baselineskip+\medskipamount]
#pragma unroll
for (int k = 0; k < 16; k++) {
	if (*i <= *j) {
		out[k] = *i++;
	} else {
		out[k] = *j++;
	}
}
			\end{lstlisting}
		\end{minipage}
		\hfill
		\begin{minipage}{\assemblerwidth}
			\begin{minipage}{ \widthof{\lstinline|	move rj, rtmp, true, .LABEL_k_out|} }
				\begin{lstlisting}[language={[DPU]Assembler}, mathescape, keepspaces]
$\textnormal{\textit{// iteration \texttt{k}}}$
	lw ri${}_{*}$, ri, 0
	lw rj${}_{*}$, rj, 0
	add rtmp, rj, 4
	jle ri${}_{*}$, rj${}_{*}$, .LABEL_k_i
	move ri${}_{*}$, rj${}_{*}$
	move rj, rtmp, true, .LABEL_k_out
.LABEL_k_i:
	add ri, ri, 4
.LABEL_k_out:
	sw rout, 4${}×{}$k, ri${}_{*}$
\end{lstlisting}
			\end{minipage}
			\hfill
			\begin{minipage}{ \widthof{\itshape// jump if \lstinline|*i| ≤ \lstinline|*j|} }
				\itshape\small
				\phantom{lg}

				// load \lstinline|*(i + 0)|

				// load \lstinline|*(j + 0)|

				// \lstinline|tmp| ← \lstinline|j| + \lstinline|1|

				// jump if \lstinline|*i| ≤ \lstinline|*j|

				// overwrite \lstinline[mathescape]|ri${}_{*}$|

				// \lstinline|j| ← \lstinline|tmp|; jump

				\phantom{lg}

				// \lstinline|i| ← \lstinline|i| + \lstinline|1|

				\phantom{lg}

				// \lstinline|out[k]| ← \lstinline|*i|
			\end{minipage}
		\end{minipage}
		\caption[]{
			This code takes 8 instructions per iteration.
			First, the pointers are dereferenced (lns.~2, 3).
			Then, the resulting address from incrementing pointer \lstinline|j| is calculated (ln.~4).
			If the first run contains the less current element, it is jumped to line~9, where pointer \lstinline|i| is incremented.
			Lastly, the less element \lstinline|*i| is written to the output (ln.~11).
			If the second run contains the less current element, the register holding \lstinline|*i| is overwritten with~\lstinline|*j| (ln.~7).
			Then, a combo operation (ln.~7) finally applies the result from incrementing pointer \lstinline|j| and jumps to the line where the output is set.

			\hspace*{1em}
			We do not know why pointer \lstinline|j| gets temporarily incremented.
			According to the documentation, an \lstinline|add| instruction is compatible with the \lstinline|true| flag, meaning the \lstinline|add| instruction in line~4 and the \lstinline|move| instruction in line~7 could be fused.
		}
		\label{fig:merge:load:twice}
	\end{subfigure}

	\begin{subfigure}{\textwidth}
		\begin{minipage}{\codewidth}
			\begin{lstlisting}[belowskip=2\baselineskip+\medskipamount+\smallskipamount]
int val_i = *i, val_j = *j;
#pragma unroll
for (int k = 0; k < 16; k++) {
	if (val_i <= val_j) {
		out[k] = val_i;
		val_i = *++i;
	} else {
		out[k] = val_j;
		val_j = *++j;
	}
}
			\end{lstlisting}
		\end{minipage}
		\hfill
		\begin{minipage}{\assemblerwidth}
			\begin{minipage}{ \widthof{\lstinline|	jgt ri*, rj*, .LABEL_k_j|~} }
				\begin{lstlisting}[language={[DPU]Assembler}, mathescape, keepspaces]
$\textnormal{\textit{// iteration \texttt{k} (\texttt{val\_i} ≤ \texttt{val\_j})}}$
	jgt ri${}_{*}$, rj${}_{*}$, .LABEL_k_j
.LABEL_k_i:
	sw rout, 4${}×{}$k, ri${}_{*}$
	add ri, ri, 4
	lw ri${}_{*}$, ri, 0
\end{lstlisting}
			\end{minipage}
			\hfill
			\begin{minipage}{ \widthof{\itshape// jump if \lstinline|val_i| > \lstinline|val_j|} }
				\itshape\small
				\phantom{lg}

				// jump if \lstinline|val_i| > \lstinline|val_j|

				\phantom{lg}

				// \lstinline|out[k]| ← \lstinline|val_i|

				// \lstinline|i| ← \lstinline|i| + \lstinline|1|

				// \lstinline|val_i| ← \lstinline|*(i + 0)|
			\end{minipage}
			\smallskip
			\begin{minipage}{ \widthof{\lstinline|	jgt ri*, rj*, .LABEL_k_j|~} }
				\begin{lstlisting}[language={[DPU]Assembler}, mathescape, keepspaces]
$\textnormal{\textit{// iteration \texttt{k} (\texttt{val\_i} > \texttt{val\_j})}}$
	jle ri${}_{*}$, rj${}_{*}$, .LABEL_k_i
.LABEL_k_j:
	sw rout, 4${}×{}$k, rj${}_{*}$
	add rj, rj, 4
	lw rj${}_{*}$, rj, 0
\end{lstlisting}
			\end{minipage}
			\hfill
			\begin{minipage}{ \widthof{\itshape// jump if \lstinline|val_i| > \lstinline|val_j|} }
				\itshape\small
				\phantom{lg}

				// jump if \lstinline|val_i| ≤ \lstinline|val_j|

				\phantom{lg}

				// \lstinline|out[k]| ← \lstinline|val_j|

				// \lstinline|j| ← \lstinline|j| + \lstinline|1|

				// \lstinline|val_j| ← \lstinline|*(j + 0)|
			\end{minipage}
		\end{minipage}
		\caption{
			This code takes 4 instructions per iteration.
			There are 16 cascaded iterations in the assembler code, all of them writing the elements of the first run to the output (top).
			There is an analogue cascade writing only elements of the second run to the output (bottom).
			Labels allow to switch between the cascades.
			First, it is checked whether the cascade should be changed (ln.~2).
			Then, the output is set (ln.~4), the respective pointer incremented (ln.~5), and the new value from dereferencing loaded (ln.~6).
		}
		\label{fig:merge:load:once}
	\end{subfigure}
	\caption{
		Two \langC{} implementations of an unrolled loop which merges 16 elements, contrasted with their compilations.
		Only the assembler codes of one iteration are shown, as all iterations follow the same scheme;
		a sixteenfold cascade of the given assembler codes yields the whole assembler codes of the loops.
		The pointers \lstinline|i| and \lstinline|j| point initially to the first elements of the runs.
		The serially numbered registers (\enquote{\lstinline|r|\dots}) and jump labels (\enquote{\lstinline|.LABEL|\dots}) were renamed to aid understanding.
		Note that the data type \lstinline[keywords={}]|int| is \qty{4}{\byte} large, which is why all offsets are multiples of four.
	}
	\label{fig:merge:load}
\end{figure}

\noindent
A significant portion of the runtime is spent on the repeated comparison of elements in a pair of runs, followed by a write of the less element to the output.
\Cref{fig:merge:load:twice} shows a straightforward implementation of an unrolled loop performing such comparisons and writes.
The code makes use of two pointers \lstinline|i| and \lstinline|j| which are initially set to the first elements of the runs.
To get their values, they are simply dereferenced.
After the output \lstinline|out[k]| has been set in iteration \lstinline|k|, the respective pointer of the less element is incremented.

Despite the succinctness of the \langC{} code, the resulting assembler code is of subpar quality.
Depending on the run from which an element got merged in the previous iteration, an iteration takes either 7 or 8 instructions.
This is a consequence of loading the values of both dereferenced pointers at the beginning of each iteration despite one of the values not having changed since the last iteration.
\Cref{fig:merge:load:once} shows an alternative implementation, whose compilation results in four instructions per iteration.
This was achieved by dereferencing the pointers \lstinline|i| and \lstinline|j| before the loop begins and storing the values in dedicated variables.
The comparisons and writes use only these dedicated variables, of which only one gets updated per iteration.
A more detailed description of the compilations is given in the caption of \cref{fig:merge:load}.

It can only be speculated as to the reason for the poor compilation of the simpler implementation.
Perhaps the constant reloads are related to the ability of tasklets to write to any \ac{WRAM} address.
Theoretically, it could be that the value obtained by dereferencing the non-incremented pointer changes between two subsequent iterations.
This explanation is not fully satisfactory as it would imply yet another load instruction before writing \lstinline|out[k]|.

Unluckily, only with the full-space \MS{} does this change to dedicated variables make iterations take 4 instructions unanimously.
With the half-space \MS{}, some merge iterations take 5 instructions.
The reason is that the second pointer \lstinline|j| is never incremented directly.
Instead, whenever an element of the second run is merged, a counter is incremented and, then, the new address of pointer \lstinline|j| is calculated by taking the address of the first element of the second run and adding the counter.
The fix is to change the loop which iterates over the pairs of runs to merge.
Rather than using the ends of the second runs as natural loop index, it has to be iterated over the ends of the first runs.

A last mention shall be given to the merge function used by the half-space \MS{}.
Passing the copied run as second argument and the uncopied run as the first one nets a noticeably speedup over an implementation with flipped arguments and, of course, flipped logic.
Sadly, we could not pinpoint the fundamental cause for this phenomenon.


\subsubsection*{Evaluation of the Performance}
\label{subsubsec:tasklet:merge:performance}

\def\mergealgos{16,24,32,48,64,96}

\pgfplotstablereadnamed{data/wram_sorts.txt}{tableWramSorts}
\expandafter\pgfplotsinvokeforeach\expandafter{\mergealgos}{
	\pgfplotstablereadnamed{data/merge/threshold=#1/uint32/sorted.txt}{tableMergeStart#1_32sorted}
	\pgfplotstablereadnamed{data/merge/threshold=#1/uint32/reverse.txt}{tableMergeStart#1_32reverse}
	\pgfplotstablereadnamed{data/merge/threshold=#1/uint32/almost.txt}{tableMergeStart#1_32almost}
	\pgfplotstablereadnamed{data/merge/threshold=#1/uint32/uniform.txt}{tableMergeStart#1_32uniform}
	\pgfplotstablereadnamed{data/merge/threshold=#1/uint32/zipf.txt}{tableMergeStart#1_32zipf}
	\pgfplotstablereadnamed{data/merge/threshold=#1/uint32/normal.txt}{tableMergeStart#1_32normal}
}

\pgfplotsset{
	merge/.style={
		horizontal sep for ticks,
		adaptive group=1 by 3,
		groupplot ylabel={Cycles / \((n \lb n)\)},
		x from 16 to 1024 minor,
		xmax=1024,
		enlarge x limits={abs=3mm, true},
		every legend image post={mark=none},
	},
	merge filter 16/.style={x filter/.expression={(\thisrow{n} == 16) || (\thisrow{n} ==  24) || (\thisrow{n} ==  96) || (\thisrow{n} == 384) || (\thisrow{n} == 1536) ? \pgfmathresult : nan}},
	merge filter 24/.style={x filter/.expression={(\thisrow{n} == 16) || (\thisrow{n} ==  32) || (\thisrow{n} == 128) || (\thisrow{n} == 512) || (\thisrow{n} == 2048) ? \pgfmathresult : nan}},
	merge filter 32/.style={x filter/.expression={(\thisrow{n} == 16) || (\thisrow{n} ==  48) || (\thisrow{n} == 192) || (\thisrow{n} == 768) || (\thisrow{n} == 3072) ? \pgfmathresult : nan}},
	merge filter 48/.style={x filter/.expression={(\thisrow{n} == 16) || (\thisrow{n} ==  64) || (\thisrow{n} == 256) || (\thisrow{n} == 1024) ? \pgfmathresult : nan}},
	merge filter 64/.style={x filter/.expression={(\thisrow{n} == 16) || (\thisrow{n} ==  96) || (\thisrow{n} == 384) || (\thisrow{n} == 1536) ? \pgfmathresult : nan}},
	merge filter 96/.style={x filter/.expression={(\thisrow{n} == 16) || (\thisrow{n} == 128) || (\thisrow{n} == 512) || (\thisrow{n} == 2048) ? \pgfmathresult : nan}},
}

\begin{figure}
	\tikzsetnextfilename{merge_starting_runs}
	\begin{tikzpicture}[plot]
		\begin{groupplot}[
			merge,
			ymin=65,
			ymax=90,
			ytick distance=5,
		]
			\nextgroupplot[title={No Write-back\strut}, legend to name=leg:merge:starting_runs]
			\expandafter\legend\expandafter{\mergealgos}
			\clip (0, 0) rectangle (1024, 200);
			\expandafter\pgfplotsinvokeforeach\expandafter{\mergealgos}{
				\plotpernlogn[merge filter #1]{Merge}{tableMergeStart#1_32uniform}
			}
			%
			\nextgroupplot[title={Write-back\strut}]
			\clip (0, 0) rectangle (1024, 200);
			\expandafter\pgfplotsinvokeforeach\expandafter{\mergealgos}{
				\plotpernlogn[merge filter #1]{MergeWriteBack}{tableMergeStart#1_32uniform}
			}
			%
			\nextgroupplot[title={Half Space}]
			\clip (0, 0) rectangle (1024, 200);
			\expandafter\pgfplotsinvokeforeach\expandafter{\mergealgos}{
				\plotpernlogn[merge filter #1]{MergeHalfSpace}{tableMergeStart#1_32uniform}
			}
		\end{groupplot}
	\end{tikzpicture}

	\hfil\pgfplotslegendfromname{leg:merge:starting_runs}\hfil
	\caption{
		Comparison of \MS*{}, which need an auxiliary array of length either \(n\) (\enquote{No Write-back} / \enquote{Write-back}) or \(\sfrac{n}{2}\) (\enquote{Half Space}), for different lengths of the starting runs.
		The \MS*{} use a \ShS{} with the step sizes \(\stepsizes = (1)\) for length 16, \(\stepsizes = (6, 1)\) for lengths 24 to 48, and \(\stepsizes = (12, 5, 1)\) for lengths 64 and 96, respectively.
	}
	\label{fig:merge:starting_runs}
\end{figure}

Three implementations have been tested:
full space \MS{} without write-backs, full space \MS{} with write-backs, and half space \MS{}.
\Cref{fig:merge:starting_runs,fig:merge:starting_runs_uint32sorted,fig:merge:starting_runs_uint32uniform,fig:merge:starting_runs_uint64sorted,fig:merge:starting_runs_uint64uniform} show their performance for various starting run lengths.
Please note that the plots are smoothed:
Whenever the number of rounds increments, the runtimes hike, making the zigzagging plots cross each other unswervingly and, thereby, hard to read.
Thence, the figures contain marks for select measurements only in such a way that the resulting plots act as an upper bound on the runtime.

The measurements show that the \MS*{} guarantee a runtime of \(\bigoh{n \lb n}\) as expected.
The differences in runtime between the different input distributions are small compared to \QS{} and are ascribable to \ShS{} and to the differing suitability of the unrolling;
cases where the usage of \ShS{} worsened the runtime are unbeknown.

Even though the tested starting run lengths range from 16 to 96 elements, the mean runtime differences are surprisingly small.
Notwithstanding that the optimal choice depends on the specific input length because of the zigzagging, a starting run length of 32 elements fares decidedly well on average across all tested scenarios.

The half-space \MS{} delivers a strong performance despite its vastly lower memory footprint.
With 32-bit integers, it beats the full-space \MS{} without write-backs by 11\% on sorted inputs and effectively ties on all other inputs but the reverse sorted ones where it narrowly falls behind.
Naturally, the full-space \MS{} with write-backs is consistently (with the exception of reverse sorted inputs) at a disadvantage, despite seeing some light with inferior starting run lengths.
With 64-bit integers, the full-space \MS{} without write-backs manages to turn the ties into scant leads in the range from 1\% to 3\%.
Using the \MS{} with write-backs is still unprofitable.

In summary, a proper implementation of half-space \MS{} with deferred copying and fine-tuned unrolling would require some work but has the potential to be the overall best stable sorting algorithm.

